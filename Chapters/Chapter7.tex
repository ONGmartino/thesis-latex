% Chapter 7

\Chapter{Case Study}{Fondazione Bruno Kessler}

\label{Chapter7}

In this chapter will be presented the case study of a research fundation, the Bruno Kessler Foundation (henceforth, FBK), to test the hypotesis made into the previous chapter. Specifically, FBK is a private research institute, largely financed by the local public institutions, a case particularly fit to present real world application of many of the consideration previously made. In fact, it includes the problems from which suffer both private and publicly-funded institution, while its history carried some issue of an entirely public institution. 

at the same time, it represent a significant case in the european research landscape, with its importance tested with social network analysis tools. the entity of its revenue flows, about 28-30M in euro, together with the number of researchers and other personnel it employ, over 400, classify it as one of the largest private research institutes in italy. its localization in the trento province, one of the most technological advanced areas of the nation, a region in which local policies have a greater impact on the local context due to the special institution, makes it fitted also for the analysis and evaluation of tech transfer policies, which however will not be discussed here.

the chapter will begin with a short presentation of the institution, including an historical perspective, a brief examination of the local government's influence on mission and operativeness; after a fast presentation of its organizational structure, a specific picture will be presented for the structure and the organization of its TTO.

Secondly, will be described the channel used, taking an esternal perspective focused on services provided to firms, research organizations and spinoffs. the fourth perspective, services for the local government and society, will be left aside?? Later, under and internal perspective, will be analyzed the processes and mechanisms corresponding to those channels; this will be an effort of process modeling, in an organizational structure that otherwise have grown and evloved over time as a series of adjustements and modifications that almost completely covered the original design. in fact, the institution has adapt itself to the specific requirement progressively made by the market and the local institutions. 

Lastly, will be dicussed the evaluation of this organization. on one hand, will be assessed the efficacy of its research  through instruments of social network analysis, trying to provide a truly quantitative evaluaton of its relative performance. on the other hand, will be discussed two methodologies to evaluate the efficiency performance, investigating the issues involved. differently from the previous framework of analysis, policies will be threated marginally, being the rest of the chapter an explanation of what policies established and how them has been implemented. 

\section{The institution}

get it form the social report and possibly other

\subsection{Historical perspective}

and context!

\subsection{Public interference}
\subsection{Organizational structure}
\subsection{TTO structure}

\section{Channels}

in this section, will be presented channels and processes related to the technology transfer activity of FBK. Firstly, an external perspective can be taken to describe the used channels as the kind of services and products that the organization provide. lately, the internal channels will be described through the analysis of internal processes that lead to the development and delivery of the product or service. this way it is possible to describe both the external apparence and the internal mechanisms of the activity. 

\subsection{External perspective}

the external perspective on channels can consider each of those as single products or services that the organization provide. caution: this perspective is more market-pull than the process in the real world: considering only external, independent actors, we are considering the kind of requests that the firm can make to FBK, momentarely leaving aside a more proactive, technology push approach in which the commercial relation with external entities is led by the deliberate actions and intentions of the research organization, rather than the ones of external entities.

\subsubsection{For firms}

for firms, the main service that fbk can provide is the development of a technology or an investigation, i.e. a feasibility study, on a specific topic. the main contractual forms that allow the external to acquire such services are contract research, cooperative research, consulting, licensing. 

the contract researches are contracts in which the research organization (FBK) "si impegna" to develop a technology, investigate an issue, study the feasibility of an idea or project, deliver some knowledge, to another organization in exchange for a "contribution", which usually assume a financial flow but could be also in nature, as the right to access and use a protected idea or technology. contracting firms can also provide resources in kind, i.e. employees, laboratories and other assets, but the core idea of a research contract is the commissioning of an activity, as in outsourcing, of something that cannot be or won't be performed internally.

cooperative research is similar to reseach contracts, but it starts from a different assumption: the idea is that neither the research and the industry firm have, single handed, the entire knowledge base and resources (in any kind) to succesfully complete the project. in this case, in fact, the organizations agree to cooperate on a specific activity, that can range - again - from technology development and knowledge generation and whatsoever. therefore, in this category fall every contract that has r\&d as object and both organization as researchers or providers

consulting usually refer to the transfer of knowledge rather than technology. the primary example is for firms to request the "help" in ending or further the developing of their technology, evaluatin something etc. the main corpus of knowledge and technology for the ultimate product already resides in the requesting firms, and any futher activity should take places in their locations and with their assets. a specific case is the provisioning of training activities for HRs, by far a minoritary activity

again, as previosuly stated, licensing is a contract in which the industry organization acquire the legal ability to make economic use of (slash exploit) a technology or knowledge already developed by the contracting research organization and patented or otherwise protected. other tools are expecially used in different knowledge sectors, i.e. software in italy cannot be patented, but the copyright can provide some protection and excludability. licence agreements can widely differ for the utilization, payment and other legal clauses, therefore providing a fair degree of personalizaiton without other specific contractual forms

\subsubsection{For research organizations}

FBK can also establish contractual relationships with other research organizations, as in exchange or the provision of a service or licence. specifically, exchange contracts between research institutions can take any of the form previously desribed; however, some forms acquire a higher importance, especially research cooperation.

another contractual form more specific to this case is the framework agreement. this contract states the interest of organization to collaborate whith each other, and the contract states the forms and the topic in which these further researches and collaboration will be performed, possibly including resources, decision making rules, propriety of the results. in fact, this agreement constitute a ground for any further development of the relationship.

another specific contractual form is the grant agreement, which however do not take place directly between research orgnaizations. with these, organizations states an arrangement of resources, tasks, objectives relative to a research project financed by some public institution, especially the EU commission. the most important type of grant, for organizations like FBK, is constituted by the FP7 programme and grants. to make a request for these funds, the organization has to present a proposal for a research question/topic decided by the EU institutions. it will state mainly the modalities and resources (especially human resoures) through which the research organization means to achieve the prescribed resources. in latest years, fp7 grants are more and more competitive: given the increase in number of organizations that apply for these funds, and their being fixed to a relatively stable amount, these grants are becoming harder and harder to win. one specific movement is taking place: to assure the presence of the needed expertise, skills and capabilities, and generally increasing the probablity of a successful research, many grant applications are made by consortia of research organization, stating the collective willingness to achieve the result. in the case this proposal will be selected, the consortia has to present a grant agreement, a form of contract between the organization that describe the modalities through which the research will be performed: who will seek for which sub-results, resources, shares of appropriation of the grand funds, responsabilities, the selection of a central coordinator etc. 

\subsubsection{For spinoff}

FBK differentiate its supporting services for spinoffs, for "would be" and already spinned (already a separate legal entity). in the first case, the organization provide services centered on the provision of a favorable entrepreneurial organizational environment, scouting of spinoff opportunities, and supporting services in shaping the would be venture, its business model and its appetibility for external investors

in the first case, such activities mainly encompass entrepreneurial courses, the structuring of understandable and supporting policies, conferences with successful scientists entrepreneurs, meetings, the provision of usable relationships into the industry etc.

in the second, the scouting can be proactive, institutional-pushed, similar to the process of disclosure eliciting in an academic setting; this is the simpliest case, and in fact is the outcome of a normal choice in "this is a tech, how do we exploit it?". the phenomenon can be also emergent, meaning the providing of a self selecting environment. example is the provision of business plan competitions: it is based on the emergent, individual willingness to start a new venture, while the selection activity is provided by the necessary evaluation included in this activity.

supporting activities instead reange from the supporting in business modeling, including specific training activities, suggestions and checks with competent human resources internal to the TTO or other offices, to business development activities, i.e. the help in establishing contacts among firms, the access to the parent organization network, help in hr practices. other supports are financial, as in grants and equity, favorable licences (under the parity, in example)

the supporting activities should last also for already spinned new ventures, especially networking activities. in the first period, i.e. 1-2 years, the parent organization can also be part of the equity, thus directly and legally influencing the choices of the spinoff. however, a safe policy should requre the parent organization to exit the property in a couple of years

\subsection{Internal perspective}

the process through which these channels form and be exploited has arise from both the policies designed by the organizaton itself (with the influence of the public instututions) and as emergent from the experience of involved human resources that helped in further shaping the process. eventually, the process appears as unique and unified, where different offices and HR will be activated on the basis of the ongoing need. 

\subsubsection{Ignition}
\subsubsection{Evaluation}
\subsubsection{Development}
\subsubsection{Finalization}

\section{Evaluation}

\subsection{Effectiveness}

\subsection{Efficiency}

To evaluate the efficiency of a process or an organization, the economic literature usually refer to two different methodologies. The first refer to the tracking over time of the changes in its inputs and outputs, that are the available resources and the resulting products. The second approach require the analyst to compare the organization's performance to the one of other, compatible, instutitons, assessing its efficiency through comparison. Both these methodologies however present some issues.

\subsubsection{Performance comparison over time}

In the first case, the first necessary condition for the usability of this method is the presence of an effective information system, especially for "contabilita analitica", which keeps track of the resources in input and the poducts or services in output. Secondly, to evaluate the very process, it is implicitly assumed that it will not change between periods of evaluation, at least not in an extensive way; otherwise, the analysis will not be able to distinguish between a difference in performance alone and the difference in the context and institutional framework. In the case of FBK, these issues represent clearly recognizable problems with a great impact. 

First, it exist in fact an informative system that could keep track of the performances over time, but it is not able to for a set of reasons. the main issue is to evaluate the amount of resources that end in each process, at an istitutional level: in a tech transfer process, resources that end up in the product include, but are not limited to, the time and effort researchers put into the process; the time of supporting personnel, among which TTO, legal office, administration,and others; materials; machinery time; external services. specifically, the most relevant issue is to keep track of the time researchers and other employees dedicated to the project. -> contabilita dei costi

the most representative example is time of TTO's officers, if the entire office is divided into different units specialized per thematic sets of activities, i.e. marketing and communication, business development, legal counseling, contracts, etc. in this case, projects are concurrent in the time they take to the TTO's employees, being him involved in any project that share the particular activity. At the same time, while impossible to estimate the amount of work hours for the single, identified project, extensive differences among projects make also difficult to evaluate the mean time spent on each project: no average measure to apply to the cost calculation. This issue come up many times during interviews, with no employee being able to estimate the average time spent on each project, with lapses that range betwen hours and days.

moreover, if the evaluation is based on a time serie approach, the danger is to evaluate the impact of external factors that change the process instead of the incremantal endogenous improvement of the process. the point is that the performance itself depend on a serious of factors, both internal and external to the organization. the first case includes the process flows, the quality of involved human resources as in skills and capabilities, financial resources dedicated, a change in inputs, the overall institutional setting and policies. the second case includes governmental policies, changes in markets, industries, technologies, industrial context, national and sovra-national policies. 

The analysis should be based on a model complex enough to include the external factors, to not imputate wrongfully changes to the process or the organization. the model should also distinguish between the change for which the entire organization is responsible, i.e. the research input to the transfer process, the marketing devel program and office etc, from the resposabilities of the TTO and the researchers directly included in the development of the selected technology. with such approach, the administration or who is interested into the evaluation, can discover any enhancement due to the learning process that is supposed to take place. 

However, in the specific case of FBK, the internal accountability seems to not include detailed informations on the amount of resources involved. while the overall costs and resources involved are known, due to the national accountability laws, the actual information system can not in fact connect the single cost to the project. "in the same way, as previously stated, the price estimation for each technology to transfer is assessed on the basis of the presumable costs for matierials, researchers' hours and a flat rate to cover the structure costs"

\subsubsection{Performance comparison between competitor}

the economic literature pointed at the comparison of competitors' performance as an effective method for evaluating the individual performance in absence of the necessary deep knowledge of the cost structure of the original organization. in fact, maybe the process is to complex or contains too many variables to get a relable evaluation, but if another, similar organization perform the same activity, the two can be compared to establish which is the most performant, eventually leading to the discover of potential best practices.

However, this methodology relies on a necessary condition: the availability of a similar organization. While the degree of similarity is somehow a personal evaluation, some condition should be matched nevertheless. the organizations should share the context, or at least be located in similar contexts, includind the degree of technology advancement, r\&d expenditures, academic level/environment, local policies, emplyable workforce, neighbour firms and, to some extent, their level of specialization and economic performance, as well as the relative importance of knowledge for their business (even if it is a requirement needed in this specific case)

similarly, the organizations should share some trait: the scale of operations, that is needed for the level of economies of scale at play; the organizational mission; the main type of activity; the scale of incomes. otherwise, the analysis will highlight which local, institutional and organizational setting can achieve the best performance, rather than which process performs better, as in its structure, personnel, resources, activity etc.

the problem starts with the peculiarities of the italian context in general. the average size of firms is smaller, which impact their innovation capabilities and processes; italian universities perform worse than the EU average ("NOMEAUTORE); the limited number of private research organizations; the tendency of these research organization to be public, as in for local and national government funded and financed. a secondary problem is the trentino context itself, which represent quite an ecception in the italian landscape, for legislative and historical conditions. 

thsu, the availability of a "paragonabile" competitor, in the case of FBK, is more of an issue. in its context, R\&D is performed by the local university, which is not directly comparable due to its main mission imposed by the local governement; and the edmund mach foundation (henceforth, FEM). This last institution, in fact, is the "sister" foundation to FBK. 

similarly to FBK, FEM is a publicly-funded research foundation. it conduct teaching and training activities, having an internal second grade school (istitutio secondario, approfondisci) and doctoral programs. it conduct r\&d activities in high-tech fields characterized by a large rely on codified knowledge, especially in the field of biotechnologies. it performs production activities, through the connected wine "factory", and provides specialized services to the agricoltural industry, both local and not. lastly, it operates technology transfer activities and possess a TTO.

however, differences among these organizations are very extensive. firstly, the size: FEM employ twice the personnel, but only a small part of them are researchers (indicatively, around 150 scientists), while the total income is more the double (90M against 35M). activities are centered over the provision of formation activities and services to the local agricultural sector, as in environmental analysis, study of fito-deseases and their diffusion, genomics. moreover, the FEM organizational structure is far more heavier than the FBK's one: at its institution, the Trento province institution actually grouped a series of different public institutions, maintaining a separate administration for each new established organiational unit. 

lastly, and even more important, the approach through which FEM interpret, concept, and perform technology transfer activities is significantly different. FEM perform some basic research which is hardly applicable to the local industry, thus the necessary shift away of TT activities from the basic research. the applied research, instead, is limited in its extent, thus the tt transfer of newly developed technologies. what's left is the provision of specific services, which might include some research from FEM. the TTO department is actually involved into this kind of service provision.

thus, the extensive differences between FEM and FBK make the former not a suitable organization for a comparative analysis of performances. however, comparable organizations could be found in different contexts. this research should start from the comparison of local innovation contexts and local government policies; if a similar context is individuated, the researcher could investigate the internal actor to this environment seeking a similar organization. 

considering the focus, objective of this thesis, this procedure and other further developments are left for later research. what was significant was to give the reader a perspective on the evaluation of the organizational performance both in a qualitative and quantitative setting. at least, the quantitative evaluation seems useful.

\section{Policies}

FBK has 4 main policies, already described before while discussing the internal process. however, to be precise, the policies are for "ELENCO PUNTATO"