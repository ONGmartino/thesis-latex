% Chapter 7

\Chapter{Case Study}{Fondazione Bruno Kessler}

\label{Chapter7}

In this chapter, will be presented the case study of a research institution, the Bruno Kessler Foundation (henceforth FBK); the hypothesis made in the previous \hyperref[Chapter6]{Chapter 6} will be later tested through the case discussion and the conclusion in \hyperref[Chapter8]{Chapter 8}. Specifically, FBK is a non-profit research institute, largely financed by the local public government, a case particularly fitted to present real world applications of many of the previous considerations. In fact, it includes problems from both a private and publicly-funded organization, while its historical luggage offers the space for some consideration on an entirely public institution. 

At the same time, FBK represents a significant case in the European research landscape, where its importance will be tested with the aid of social network analysis tools. At the same time, its relevance can also be assessed by the proxy of its dimension, both in revenue flows (about 30M euros) and the number of researchers and other personnel it employs (over 450), which both classify this institution as one of the largest private research institutes in Italy. Its localization in the Trento province, one of the most technological advanced areas of the country, and a region in which peculiar institutional settings positively influence the power and impact of local policies, makes it fitted also for the analysis and evaluation of government technology transfer policies, which however will be demanded to a more proper literature.

Due to the specific objectives of this thesis, the case study will point toward the organizational structure and the process perspective. The chapter will begin with a short presentation of the organization, including a historical perspective and a brief description of the local government's influence on mission and operativeness. After a short presentation of the organizational structure, a specific section will provide specific information about the structure and the organization of its TTO.

Later, the chapter will describe channels and processes: firstly, an external perspective will be taken to focus on the various products and services provided to firms, research organizations, and spin-offs, through a generic perspective on channels. These will be later decomposed and analyzed, following an internal perspective on the corresponding processes and mechanisms. This analysis will be an effort of process modeling, in an organizational structure, that actually has grown and evolved over time as an emergent auto-adaptation that shaped the structure concurrently with the organization designs provided by policies. Eventually, the institution has indeed adapted itself to the specific requirement progressively made by the market and the local institutions. 

Lastly, will be discussed the evaluation perspective. On one hand, will be assessed the efficacy of its research activities, through instruments of social network analysis, with the objective of providing a quantitative evaluation of a typically qualitative topic. On the other hand, will be discussed two methodologies to evaluate the efficiency performance, investigating the involved issues. Will follow a brief examination of the main organizational policies on technology transfer activities.

A LAST FUNDAMENTAL NOTE 

\section{The institution}

Nowadays, the Bruno Kessler Foundation is a nonprofit research institute of \enquote{public interest}, funded in 2007 by a law of the local government but autonomous and regulated by the private legislation.\footnote{For \enquote{public interest} is intended an organization that is subjected to the legislation dedicated to private entities, while one or more conditions (in this case the presence of a public institution among the founders) entails the respect of secondarly obbligations for public institutions, i.e.\ higher requirements for trasparency.} It has two major objectives: the scientific excellence and the local economic development.

The history of the institute begins in 1962. In this year, Bruno Kessler, the president of the Autonomous Province of Trento, founded the predecessor of FBK: the Trento Institute of Culture (ITC).\footnote{
	\href{http://www.consiglio.provincia.tn.it/leggi-e-archivi/codice-provinciale/archivio/Pages/Legge\%20provinciale\%2029\%20agosto\%201962,\%20n.\%2011_565.aspx}
	{Provincial law n.11, 29 August 1962}
	[accessed: February 2, 2017]
} At the time, Trento had not a local public university, and the ITC was instituted with the long-term objective of providing a favorable, scientific and innovative context for the later creation of a university. 

In 1972, the Free University of Trento was finally founded, thus requiring a change in the objective and future of the ITC. With a functional university, the institution finally could invest its energy in the knowledge- and research-based kind of activities and services that the context still needed, but the university could not provide. Therefore, the focus shifted to a more applied research, utilizable by the context and characterized by a long-term development perspective, capable also of maintaining and develop of the local culture.

A remarkable milestone, in fact, was the institution of a research center dedicated to more technical research and applied science: the Institute for Technological and Scientific Research (IRST). Along with other institutes, i.e.\ the Italian-German Historical Institute (ISIG), the Center for Religious Studies, and later centers, the ITC could cope with both the need for a technological development of the region and the need for preserving the local culture. 

A more recent, major change in the strategic trajectory of the institution started in 2005 when the local government decided to reorganize the local system of research and innovation. This action ended with the dissolution of the ITC in the form of a public institution, with its transformation into the Bruno Kessler Foundation, officially funded in 2007.\footnote{
	\href{http://www.consiglio.provincia.tn.it/leggi-e-archivi/codice-provinciale/archivio/Pages/Legge\%20provinciale\%202\%20agosto\%202005,\%20n.\%2014_12567.aspx?zid=6003d625-228e-4e5d-820d-d6cf459dfc36}
	{Provincial law n.14, 1 March 2005}
	[accessed: February 2, 2017]
} The foundation has, in fact, the local government (in the form of the Trento Autonomous Province) among its founders, but the institution is mainly regulated by the private legislation, the same that regulate the activities of foundations and other non-profit organizations. 

While FBK may have lost its public status, the strong interference of the local government deeply influenced and still influence both the mission and the methodologies of the foundation. First, the financial impact: the local government accounts for two-thirds of the total income of the foundation, with about of 30M in a 45M euro total income. By all means, without the public sustain and support, the foundation would not be able to cope with the financial requirements needed to perform both applied and basic research, necessary to the accomplishment of its organizational missions.

A second major influence is on the mission and business model. Differently from other private research organizations, FBK explicitly stated among its main objectives the positive influence for the local society, equal for importance to the scientific excellence and the economic survival. This set of objectives, in fact, reflect the pure identity of a non-profit organization, but it is not clear if the actual interests and activities of FBK do descend from the influence of the public institutions or its history and legacy.

Eventually, the two primary objectives of FBK are the scientific excellence and the local development, suggesting the first as a necessary condition, a tool in achieving the second. The underlining idea is to hire star scientists, include them in a favoring structure, and employ them in (1) producing excellent basic research and (2) perform industry-led, research-based activities, i.e.\ contract development. Implicitly, by furthering the knowledge and technology bases of the local industry, FBK should be able to help the local economic growth by improving the competitiveness of local firms, attracting high-tech finances and other star scientists.

\subsection{Organizational structure}

A brief description of the organizational structure should help to understand this continuous duality, between technological progress and historical luggage, basic science and development services, local development and economic survival. Moreover, a blended framework of the organization will be useful in understanding the internal processes. 

Main entities in the administration are the President, the Board of Directors, and the Secretary General. Notably, the Autonomous Province of Trento, as a founding member, nominate both the President and 6 of the 8 members of the Board, therefore the extent of the influence of the local government. In staff to the Board of Directors, the Scientific Committee, tasked with the ex-ante evaluation of annual and long-term plans, i.e.\ the long-term Program for Research Activities and Investments (PPARI), and the Budget and Annual Plan of Activities (B\&PAA).

Under the administration, the structure includes the research structure and various support offices in staff. The research body is firstly divided in two different hubs: the (1) scientific and technological, and the (2) human and social science hub. Each hub is further divided into research centers, specialized for knowledge and technology topics, led by different directors:

\begin{itemize}

\item Scientific and technological hub:
	\begin{itemize}
	\item Information Technology Center (ITC)
	\item Center for Materials and Microsystems (CMM)
	\item International Center for Mathematic Research (CIRM)
	\item European center of Theoretical Physics (ETC*)
	\end{itemize}

\item Human and social science hub:
	\begin{itemize}
	\item Research Institute for the Evaluation of Public Policies (IRVAPP)
	\item Italian-German Historical Institute (ISIG)
	\item Religious Sciences Center (ISR)
	\end{itemize}

\end{itemize}

Main centers are ICT and CMM, which account for the largest part of researchers. Given their size, these centers are further divided into different research units: 6 for CMM and 23 for ICT. While they employ in fact a similar number of researchers, CMM's projects tend to be larger, structured and demanding, requesting substantially larger teams; an example is the management and maintenance of the internal clean room. The structure also comprehends independent research units, that arise from special projects; i.e.\ the framework agreement with the Italian National Research Council (CNR) which led to the creation of a dedicated micro-center, with 3 internal units.

Other offices are in staff to the Secretary General and the research subsystem; the most important are Human Resources, ICT support, Infrastructure and Corporate Assets, AIRT (the organizational name for the TTO), Legal Office, Communication office. These are autonomous and independent from the research structure; they can activate or be activated by the research centers or units while performing activities on the request of the administration and other operations of maintenance. 

The entire organization employs more than 450 human resources, clearly inclined toward the research structure rather  than administration and support offices. According to the 2011 Integrated Report\footnote{
	\href
	{http://airt.fbk.eu/it/report-sociale-2011}
	{Integrated Report 2011}
	[accessed: July 31, 2016]
}\, the organization counts 462 employees, 347 researchers and 115 resources among administration and support personnel. This ratio describes the organizational effort to maintain its flexibility in an otherwise massive structure: a lightweight support structure, with a network of relatively independent and agile research units.    

\subsection{TTO structure}

In this organizational setting, the office in charge for technology transfer activities is called \enquote{Innovation and Territorial Relationships Area} (AIRT), an institutional name for the Technology Transfer Office. It is tasked with two main objectives: to maintain and manage the relationships with external entities and seek for external financial support to the research. In the widest perspective, the office employs more than 12 peoples, both with specific tasks or more generic supporting functions. 

The head of the Area is directly involved in building, developing and maintaining relationships with relevant external actors, i.e.\ spin-offs and relevant entities in the local context; on the other hand, the resource manages the internal connection with other prominent positions, as the President, the Secretary General, and the Centers' Directors. Along with these networking activities, the employee administrates the office and his subordinate, yet mostly on matter of strategic importance. In fact, his employees might have a greater knowledge on the specific topics and issues, while they may also lack the larger, strategic perspective to manage individual key relationships.

The organizational role and position are also greatly shaped by the individual characteristics of the resource: prior to his engagement in FBK, the employee had acquired a relevant experience both as a Ph.D.\ researcher and in various, major companies. His double background is believed to be fundamental in understanding and cope with the needs and requirements of both the research and the industry side of his activity. Moreover, significant personal relationships in the local context should ease and foster the local institutional networking activities.

Next to him, a second organizational position is specialized in the management of industrial contacts. More specifically, the employee performs any activity related to: (1) developing an initial relation of mutual interest and trust, instrumental for the later development of the relationship; (2) manage the relationship and maintaining contact for already established and active exchanges, especially bridging the communication between researchers and firms. Examples of day-by-day activities are the participation to acquaintance and technical meetings, economic and contractual arrangements and alike.

Again, the resources gained a background in both the scientific research, mainly internally to FBK as a former Head of a research unit, and in the management of industrial relationships, mainly through his previous transfer projects. In this case, the availability of technical know-how is essential to participate and support researchers in the most technical exchanges, while his personal network of industry contacts and the understanding of industry needs help in increasing the probability of a successful transfer.

A third specific position is dedicated to business development activities, including the scouting of potential industry partners and customer, and the support to researchers in delineating technology-push commercial actions. Daily activities include the elaboration of value propositions wrapping research products, individuate potentially interested industry segments, contact firms, gain their initial interest. Other activities include the strategic management of the patent portfolio, counseling to potential spin-offs, marketing operations. The related employee, differently from the previous examples, do not have a deep understanding of the technical content and the research processes, but he possesses a relevant experience in firms, business development and technology transfer activities.

Two distinct offices follow, each one operating in a more traditional fashion in support to research: a legal support office, for contractual matters, and the Research Funding office, which offers support for research grants.

The legal support offers internal services as the analysis and evaluation of prior contracts, as well as the literal writing and the negotiation of any research-based contract, including research contracts, cooperative research, licenses, grants and framework agreements. The office employs two human resources, combining both the experience and the knowledge needed, in economic and legal issues: the negotiation process, part of their daily operations, surely require a legal knowledge and training, but also the skills and competencies needed to evaluate the economic value of any project and contract. 

The Research Funding office has instead the objective of providing a specialized support for the application and management of research grants. The employees actively seek for research grants from local, national and European institutions, screening the calls for projects that match the internal competencies and knowledge. Later, the project will be forwarded to the best-suited research Unit, while providing support in contacting potential partners, writing the grant application and the eventual consortia agreement. These activities clearly require the awareness of the research topic of every internal unit, deeper than the simple field of research, while personal contacts and relationships with researhers can offer a distinctive advantage in the successfulness of their activities. For this purpose, both the involved human resources have a long-term, decennial experience in these activities, in this very organization.

The last core office of AIRT, the Territorial Relationship Office, has the main purpose of managing the relations that FBK established with other institutions in the local context. In fact, one of the channels available and chosen, for a positive impact on the Trento province, is the active involvement of academic institutions and schools, i.e.\ in the form of curricular and formative internships; similar are the hosting and organization of conferences, meetings and alike. While this office might have not a direct impact on the foundation's incomes, it is necessary for accomplishing the second of the two main missions and purpose of FBK.

A separate mention should be made for an independent employee in staff to the Director of the CMM, due to the continuous and structured collaboration among him and AIRT. More specifically, the peculiar organizational role of this employee places him in the best-suited position to support researchers in a more informal setting. His daily activities comprehend the evaluation of patent proposals, the management of technical and informal aspects of industry relationships, and technology scouting. Peculiar to his position is the required experience and credibility that is required among researchers, in order to be spontaneously involved in their work, developed in over 20 years of experience in FBK.

\section{Channels}

In this section will be presented channels and processes related to the technology transfer activity of FBK. Firstly, an external perspective will be taken to describe the various channels as in the kind of services and products that the organization provides. Secondly, the channels will be described through the analysis of internal processes that constitute the development and delivery of the product or service. This methodology will also expose both the external appearence and the internal mechanisms of the activity. 

\subsection{External perspective}

The external perspective considers each channel as a single product or service that the organization provides. While this perspective might be useful in exemplifying the entire offer, it may picture the process as more market-pull than the reality. In fact, considering external, independent actors alone, the analysis will proceed through the various products that a firm can request to FBK, momentarily leaving aside a more proactive, technology-push approach in which the commercial activity is led by the emergent, deliberate actions of the research organization, rather than the interests of external entities.

\subsubsection{For firms}

The main service that FBK can provide to firms is the development of a technology or investigations and feasibility studies, on rather specific topics. The main contractual forms for this kind of service, that allow external entities to acquire technologies and knowledge, are contract research, cooperative research, consulting and licensing. 

The contract research is a legal contract in which the research organization undertakes the development of a technology, the investigation of an issue, the feasibility study of an idea or project, the delivery of a specific knowledge to another organization in exchange for a \enquote{contribution}. While the compensation usually assumes the form of a financial flow, it could also contain contributions in nature, as in the right to access and use a protected idea or technology and alike; contracting firms can also provide resources in kind, i.e.\ employees, laboratories and other assets. Despite the form of the compensation, the core idea of a research contract is the commissioning of an activity, not unlike the outsourcing, of something that cannot or will not be performed internally.

Cooperative research shares most of its legal traits with research contracts, but it starts with a different assumption: neither the research organization and the firm have, individually, the entire knowledge base and resources (in any kind) to successfully complete the project. In this case, the organizations agree to cooperate on a specific activity, that can range - similarly to the research contract - from technology development, knowledge generation and others previously cited. Therefore, in this category falls every contract that has the R\&D as object and both the organizations as active researchers.

Consulting usually refers to the transfer of knowledge rather than the technology. The primary example is for firms to request a support in ending or further the developing of their technology, product etc. The main corpus of knowledge and technology for the ultimate successful development already resides in the requesting firms, but its exploitation requires an external intervention; exemplary is the routine of performing any further activity at the firm location and with their assets. A specific case is the provisioning of training activities for human resources, but these are clearly minor activities among the services provided by FBK.

As previously stated, licensing is a contract in which an organization acquires the legal ability to exploit or make economic use of a technology or knowledge already developed by the licensing organization, an ability otherwise forbidden by the patent or other forms of protection. Tools tend to be relatively specific to the knowledge sector: in example, in many countries, a software cannot be patented, while the copyright can provide a similar degree of protection and excludability. License agreements can widely differ, according to the type of utilization, contribution and other legal clauses, thus providing a degree of personalization for the exchange without other specific contractual forms.

\subsubsection{For research organizations}

FBK also establishes contractual relationships with other research organizations. Specifically, exchange contracts between research institutions can assume any of the form previously described, i.e.\ the provision of a service or a license; however, some forms acquire a greater importance, especially research cooperation, with the two organizations collaborating on a shared topic.

Another example of a contractual form more specific to the case is the framework agreement. This contract describes the mutual interest of the organizations to collaborate with each other, framing forms and topics in which the further collaborative research will be performed. The contract may include resources, rules for decision-making processes, propriety of the results and similar clauses. Eventually, this contractual agreement constitutes the basis for any further development of the relationship.

A separate mention should be made for a very specific contractual form: the grant agreement. Similarly to the previous forms, while it may involve both research organization and production companies, it is usually signed between research organizations, both public or private in nature, as a necessary step in applying for a public call for research grants. The funding source may be a generic public institution, but the most relevant case involve the EU Commission and the FP7 or H2020 grants. Briefly, to apply for these funds, the organization must deliver a proposal for the call's research question or topic, assigned by the Commission. The proposal will state the modalities and resources (including material, immaterial, financial and human capital) through which the research organizations mean to achieve the prescribed objectives.

Since the begin of the H2020 program, applications for these grants have become more and more competitive: given the rising number of organizations that apply for these funds, the increasing specialization of calls and applications, and the relatively fixed amount of funds, these grants are becoming harder and harder to win. To overcome this issue, and increasing the probability of a successful application, the usual solution is to secure the presence of the needed expertise, skills and capabilities, and resources through a collective application, made by consortia of research organizations. If the proposal will be selected, the organizations will deliver a consortium agreement, a contract between the participants that describe the modalities through which the research will be performed: individual tasks, results, resources, share of the grant funds, responsibilities, the central coordinator and alike. 

\subsubsection{For spin-off}

For spin-offs, FBK differentiates its support services among potential, in development projects and already founded new ventures. In the first case, the organization provides a set of services centered on the provision of (1) a favorable entrepreneurial organizational environment, (2) scouting for spin-off opportunities, and (3) supporting services in shaping potential projects, their business models and their appetibility for external investors. The first two cases share extensive similarities with the generic foster of a commercial-friendly organizational culture, as common, basic and multipurpose policies.

In the first case, activities mainly encompass the structuring of comprehensible and supportive policies,  entrepreneurial courses, training, conferences with successful scientists entrepreneurs, meetings, and the provision of useful and exploitable relationships into the industry. In respect to the standard activities for the organizational climate, these activities are more focused on the main spin-off-related topics: business modeling, seed and venture capital, legal aspects of the creation of a new venture, and alike. 

As any other commercialization activity, the opportunity scouting can be intended and performed as proactive, institutional, not dissimilar to the disclosure eliciting in an academic setting; in the simplest case, the technologies available are known, and the question is on the modalities through which exploit it. In the most complex case, AIRT officers must ask researchers for their current activities, then individuate and suggest a commercialization path. The phenomenon can be also emergent, provided by a self-selecting environment. The most significative example is the provision of business plan competitions: the spin-off process will begin with the emergent, individual willingness to start a new venture, while the evaluation and selection activity will be provided by the contest itself.

Support activities, instead, range from the aid in business modeling to business development activities. Examples include specific training activities, suggestions, and validation of the business model and business plan, the introduction of the new entrepreneurs in the network of the parent organization, aid in developing new personal contacts, support in human resources practices. Other forms of support are financial, both in grants or equity, favorable licenses, access to machinery and workspaces. FBK offers its support also to already founded spin-offs, especially networking activities. If the process involved the acquisition of an equity share, in the first 1-2 years the parent organization can directly influence the choices of the spin-off.

\subsection{Internal perspective}

The inner processes behind these channels have been initially structured by the statue and other policies, but the actual design has also bene refined by the experience of the involved human resources, who helped in further shaping the processes' flows in an emergent fashion. Eventually, the processes appear as unique and unified, where different offices and employees will be activated on the basis of the ongoing needs. 

From a comprehensive perspective, the most part of technology transfer activities are performed by the Innovation and Territorial Relationship Area (AIRT). Apart from the advisory role for the various economic aspects of the research, AIRT holds five main processes: patenting and licensing, support for spin-offs, grants, and contracts, relationships management.

\subsubsection{Patenting process}

The sole Research Centers responsible for the origination of patentable technologies are the Center for Materials and Microsystems (CMM) and the Center for Information and Communication Technologies (ICT). The process of patenting, in its beginning, differ among the Center of origin, due to the presence of a specific organizational position in staff to the CMM Director, which act as facilitator through informal preparatory activities

Within the CMM, the process usually begins with researchers delivering a patentable technology, or a commercial idea, to the facilitator employee, through informal meetings or presentations. Specifically, researchers may directly suggest the patent channel; however, the decision is usually postponed to the examination and assessment of the commercial potential value of the technology, while comparing concurrent exploitation alternatives. The researcher is expected to expose key facts of the new technology, how it differs from the previous state-of-the-art, and how to extract economic and social value from it. 

After a first evaluation from the facilitator employee, especially for the anteriority research, a basic template will be written to clarify the most common and important contractual aspects. It contains information and clauses about the proprietary structure, financial incentives for the inventor, the legal process, and alike. Lately, the patent attorney will be contacted to better assess the feasibility and potential value of the endeavor. 

Otherwise, if the invention originates within the ICT, researchers should directly contact AIRT, to ask for an initial evaluation of the project, replacing the support of the CMM facilitator. In both cases, after different initial phases, the process flows will converge in a more formal procedure that begins with the so-called \enquote{invention notice}. 

It consists in a notice, similar in spirit to a dossier, which notifies AIRT and the Center Director of the reaching of a potentially patentable research product; it should include information about the discovery, its applicability to industry, identified market segment, novelty, and alike. The invention notice will be therefore evaluated, to be eventually accepted or refused. A necessary condition for a positive evaluation is the recognition of a clear potential customer or customer segment, already identified.

If the notice is positively evaluated, the patent attorney will be involved to support researchers in converting the invention notice into an inventor declaration, which will be deposited initially as an Italian patent. After twelve months, the application will be converted in a PCT or EU patent, in order to gain additional time to invest in commercialization activities before the nationalization process and its relative costs.

Valorization activities will begin immediately after the first deposit. Since the characteristic early stage of the patenting technology, AIRT and the CMM facilitator will start seeking for a potential investor, specifically interested in further developing the technology (1), apart from the usual organizations interested in licensing (2) or acquire (3) the pending patent. In the nearest future, a specific partner will be engaged in this phase: a firm specialized in patent valorization, who will perform activities like market analysis, partner scouting and customer seeking. 

While the described process matches a typical academic case, it represents only a fraction of the potential patents: the largest part of patent proposals seems arises from EU-granted projects and direct assignments. In the former case, EU funds (especially H2020 funds) are reported to be assigned preferentially to consortia of organizations which include at lease one firm; according to the interviewee, the industrial partners usually participates EU projects because interested in any child patent that can be originated. Therefore, apart from the possible direct request of patenting of participant firms, even their sole presence will influence FBK and participant research organizations to patent the results, simplifying the decision process and the seek for a licensee. 

For the latter case, instead, the direct research contract can anticipate the need for the patenting of research results: FBK already establishes and agrees, at the signing, to patent the research output. However, the property of the patent may be shared or entirely of the firm, therefore influencing the licensing process.

In both these last cases, the patenting process eventually differs for the absence of the invention disclosure and the initial evaluation of feasibility and profitability, resulting in a simplified seek for a licensee.

\subsubsection{Relationships management}

One of the most important tasks for AIRT is to manage the relationships with firms: the source for contract and cooperative research, the main technology transfer channel for FBK. The inner activities behind the task can be decomposed in three different processes, due to the involvement of different employees with different objectives, which eventually constitute different points of contact between the organization and the market.

In a chronological perspective, the first is the business developer employed by AIRT. Along with marketing activities, he actively elicits information from researchers on their research projects, to later seek for a match between these and potential markets and customers. Differently from other employees involved in this process, his approach usually points at new, unexplored markets, in a disruptive rather than incremental perspective. 

More specifically, once gained information about a project, the employee will model any business opportunity that may originate from internal research, seeking any market niche or segment in which the technology may be deployed. After the identification of the best competitors in these markets, the employee will contact them and try to gain an initial interest, in order to establish a relationship. After this first contact, the potential customer will be redirected to other specialized AIRT's employees. 

The main process of relationship development, in fact, is held by a different resource, whose sets of activities includes: the identification and contact of potential customers, focusing on territorial promotion based on FBK competencies; the reception of interested firms; the maintenance of relationships. 

The identification and contact of potential customers and partners begin with the seek for technological issues in entire industrial sectors related to FBK internal research areas. A necessary condition, which will be discussed later, is the knowledge of internal strengths and the applicability fields of any internal research activity. Technological issues mainly refer to a limitation in current processes, being qualitative, quantitative or in terms of efficiency, which can be overcome by a technological leap. Once spotted a similar issue, if an FBK's technology can achieve, or help to achieve this leap, will be compiled a list of potential customers or partners, along with a brief documental analysis for every actor. 

Then, a first contact will be made. Even in initial stages, the approach is relational, direct and informal: usually, a meeting which involves a presentation on the entire organization and its projects. The main objective of this stage is the understanding of the firm, its activities and industrial context, to better adapt the technological offer to the firm’s peculiarities: the approach shift from a resource-based view to a market-based approach.
 
In this phase, the tendency is to rely on personal, local and previous networks of collaboration, rather than different contexts and industries, which can be seen as an effect of the local development mission of the organization. In example, the first contact will be made through a common acquaintance, if available. However, this approach may have a negative impact, such as a self-limitation in the market scope and auto-financing opportunities.

Direct firms’ requests, instead, can be made directly to the office, to a researcher, or to external partner organization, i.e.\ Hub Innovazione Trentino and Trentino Sviluppo; in any case, the request will be forwarded to the office, for its analysis and development. The necessary condition for the acceptance is the innovativeness of the project and the final objective of an advanced demonstration of the newly developed technologies. Otherwise, the interviewee suggests that extended projects could compromise the FBK's role of research institute, shifting from a knowledge generator to a \enquote{supervisor}, distracting resources from core activities of research.

In this scenario, the relationship evolves as follow:

\begin{enumerate}

\item A preparatory phase dedicated to the assessment of the counterpart;
\item The exchange of specific technical information about the potential project, under a non-disclosure agreement;
\item Contract drafting and negotiation;
\item Additional bureaucratic procedures;\footnote{The most significant example is for local firms: the \enquote{Legge 6}, a law promulged by the Autonomous Province of Trento, act as a financial aid for research collaborations; if the contract is eligible, the process will include an additional precontractual agreement and the submission of the proposal.}
\item Sign of the definitive contract and its technical annexes;
\item The actual research phase, based on stages defined by contract;
\item Continuous follow-up;
\item Final demonstration;

\end{enumerate}

A secondary share of this process is held directly by the CMM Director’s staff. In this case, external requests usually involve the development and production of specific hardware, which constitutes the main competency of the CMM and its researchers. Secondly, the seek for customers is performed as a secondary and residual activity, mainly through the participation in various exhibitions and fairs.

\subsubsection{Spin-off process}

In respect to the patenting and licensing process, the spin-off process is less structured and standardized. Even if a specific policy does exist, and it anticipate differently, the process usually begins in an informal fashion. In fact, the standard exploitation channels for FBK are the research contracts and licenses: the opportunity to found a new venture is more of an exception, with the chance being usually considered during informal meetings and alike.

The initial proposal may come directly from researchers, or driven by AIRT; which explicitly suggest the opportunity to the scientist. The proposal may take the form of a communication of intent or more informal, in a simple inter-office meeting; after a first analysis and discussion between researchers and the TTO, the proposal will be articulated in a business plan, which will be officially submitted to the Entrepreneurship Evaluation Committee. 

The committee consists of 6 members, aiming at gaining a complete evaluation by including perspectives from different professional backgrounds and areas of expertise. Half of the committee is composed by FBK members, representing the mission and scientific knowledge of FBK; the other three members are representatives of institutional partners, involved in different industries: startups and spin-offs, venture capitalists, patent attorneys. Together, they should be able to provide a complete assessment, including the mere market value and the strategic value the proposal may yield to FBK and the local context.

After this preparatory phase, a more complete report will be redacted by the proponent, including a business plan and the draft of the new venture's statue. Eventually, the report will be presented to the Board of Directors for a more formal and complete assessment. The main objective of the entire process is clearly to extensively assess the robustness of the proposal and to ensure its alignment with the FBK mission. If the proposal receives a positive evaluation, an investment can be made, both in equity, grant or loan.

A final note on this process must be made: the process is currently being redesigned at its root. No startup has been spun off in the last years, to describe the recent form of the process flow, and no final design of the process has been implemented yet. This analysis is based on previous projects and different spin-off opportunities arisen in previous years.	

\subsubsection{Grant support}

The Research Funding Office can provide a separate support for EU grants, especially in the form of H2020 calls, and local, less known and participated calls.

In the first case, the support for H2020 projects is mostly limited to the administrative and bureaucratic assistance for the application. This restriction is due to the scientific specificity embedded in each project: the researcher himself, over years of experience and activity in a specific field and its relative network, has a deeper knowledge regarding available calls, their financial entity, and feasibility, as well as other external individuals or firms to involve. The single office can not outperform the researcher in such activities. The focus of its activity, in fact, is sharper: to relieve researchers from any bureaucratic, organizational and legal affair, while exploiting their personal networks and scientific capabilities, enabling them to focus on the scientific aspects of the application.

The office performs activities such as budgeting, documental management, control and package of the proposal, submitting the application. If many actors are involved, the office will coordinate them, at least on the overall administrative affairs instead of scientific matters. Later, if the application is granted, the office will manage the later negotiation phase, especially the contractual phase for any Consortium agreement and the introduction of amendments. Lastly, no scouting activity will be performed. 

In the second case, for local, smaller and less known calls, the office will provide a more extensive support. As a preparatory activity, the scouting for potentially interesting calls will be made through institutional communications, alerts, and channel such as official EU publications. After an initial evaluation on the eligibility of FBK researchers, these last will be contacted for any call related to their research area. 

More specifically, the documentation will be sent to any scientist possibly interested: the assumption here, as for many other activities in the Area, is to know the actual interests of every researcher. In this specific case, the main tool is the experience of the employee involved, gathered through direct interviews and the development of personal contacts in almost any research unit. An example of the importance of these relationships is the continuous and spontaneous follow-up that researchers send to the office: in this scenario, no eliciting is needed.

Later, the willing researcher will be supported in every aspect of the proposal procedure. As usual,  the office may also be activated by a direct request from scientists. 

An interesting insight gained from interviewees is about the characteristics of researchers this support activity is aimed to, both on the organizational and individual level. Specifically, smaller units tend to be less competitive, negatively influencing their probability to successfully apply for major grants, i.e.\ H2020 calls. At the individual level instead, apart from the personal ability and talent, great emphasis has been placed on the personal network of the researcher. 

\subsubsection{Contractual support}

The legal support office performs activities of contract drafting and negotiation; it can be activated by the individual researcher, for single collaboration with firms and grants, and by other FBK positions, i.e.\ Center Directors and the Secretary General, for institutional agreements with a variety of actors. Types of contracts include:

\begin{itemize}

\item Collaboration Agreements; 
\item R\&D Agreements; 
\item Patent, know-how and software licenses; 
\item Services, for the prototypation and production of hardware;
\item Feasibility Studies; 
\item Program Agreement, frameworks for the development of relationships with public institutions; 
\item Grant agreements for European projects. 
\item Non-Disclosure Agreements; 

\end{itemize}

The contractual support typically begins with a formal internal request. The initial phase requires the employees to discuss with the researchers the basis and the evolutionary pattern of the contractual relationships. Later, a basic contractual model will be customized to the specific circumstance, which will be sent to the counterpart to begin the negotiation of inner clauses. 
 
This process can easily require several weeks and requires a continuous strategic evaluation of the economic, financial and strategic means of every change. Eventually, the process ends with the reaching of a satisfactory combination of FBK’s and the counterpart’s interests, as the compromise the contract represent. This process is performed alongside with researchers, to ensure the desirability of the technical contents of the contract.

Among common clauses, a standard contractual model includes three significant elements: intellectual property, publication, and the clauses of licence-back and best-effort.

The intellectual property describes the property structure of knowledge and technologies involved in the process. It differentiates between: the background, known at the beginning of the process: the sideground, generated during the project, but not directly linked to the research objective; and the foreground, the actual results. The first two are usually property of the organization which developed it, while the third can be conjoint or individually assigned to one organization. 

The publication is another important element for any research contract. The typical clause states that any potential publication will be sent to the counterpart for a formal control prior to the public release; the organization may deny the publication, presenting a report that clearly defines the motivation. In any case, publications must not jeopardize patenting activities, i.e.\ by postponing every external communication to the patent application, as well as no confidential information disclosed.

Two minor but relevant clauses are the licence-back and the best-effort clauses. Firstly, FBK usually requires a clause of license-back for the non-exclusive, unlimited use of the foreground generated, in the case in which the property is entirely assigned to the counterpart. The best-effort clause instead protects the Foundation from the potential absence or incompleteness of research results, due to the intrinsic uncertainty and risks of these activities.

In any case, researchers have the control over, and responsibility for the resulting contract: they lead the negotiation process and decide which form will be the accepted for final; the office act only as a support, providing skills and competencies and having little control over the economic and scientific content of the contract.

\subsubsection{Additional insights}

Interviewees involved in this process highlight several interesting topic.

Firstly, the need for a correct identification of internal competencies and research areas. In similar, highly-structured organizations with clear boundaries between research teams, inter-unit and inter-area communications become harder: the TTO may encounter difficulties in gathering this knowledge, especially considering the esteem and trust needed for a continuous and spontaneous communication. In such scenario, it might be fundamental to have a relevant experience in research, in order to gain a reputation before researchers, to be seen as a peer, therefore improving the communication performance. Formal and institutional moments of exchange may be useful but also potentially insufficient.

Moreover, disclosure eliciting is harder in the case of a research unit led by a researcher whose vision for its group is a \enquote{strongly independent, autonomous isle}. These units have usually a low auto-financing ratio, due to few cooperation contracts and little cooperation with the TTO and external firms. On the other hand, more proactive and propositional units show a positive attitude in collaborating with other entities, leading to a higher auto-financing ratio. This difference can be due both to the leadership, in a top-down approach, but also the individual behavior, as in a bottom-up approach. 

Secondly, interviews highlighted the importance of the perceived credibility, both for FBK as institution, its representatives, the researchers, and the patent attorney (if the contract anticipate the patent application). Specifically, the interviewees consider helpful to appears, as FBK, more application oriented, lean and flexible than a fully public institution; the representative credibility instead seems to be directly linked to the experience in industrial contexts and previous relationships with other firms. 

Another topic is the researchers’ perspective in the technology transfer activities. According to the interviewees, one of the main reasons for researchers to participate in these activities is actually the pecuniary incentive, both as potential royalties and generic rewards. However, these incentives may be for the individual benefit, but also financial resources for consecutive research projects.  

Elsewhere, interviews have highlighted the need for researchers to participate in every phase of the relationship development, in an active and propositive fashion. Specifically, the initial stages can be used to build a direct channel of communication between the researcher and the firm's technical office, improving the flexibility and efficiency of exchanges in the prototypation phase.

Regarding the contract objectives, instead, has been highlighted the strong preference of firms for incremental R\&D instead of a more disruptive innovation, which influences the innovativeness of projects. In fact, interviewees reported that at the initial request, during the very same meeting, the researcher is usually capable of immediately assess the feasibility of the project, indicating that the scientific complexity of the proposal might be less challenging than expected.

Lastly, a perspective on the origin and localization of FBK relationships. According to the interviewees, about 70\% of the external relationships arise from the personal network of employees, reflecting the importance of the involved social capital. Moreover, external direct requests seem to origin prevalently from firms in the Trento local context. Apart from the local development mission of FBK, the phenomenon can be directly linked to the Autonomous Province of Trento and its legislation, especially for financial incentives, i.e.\ the previously cited \enquote{Legge 6}.



\section{Evaluation}

As previously stated, the performance of an organization can be evaluated for its effectiveness or its efficiency. Especially in the case of research organization, which processes are not as clear, known, and standardizable as the ones of a production firm, these evaluations require different perspectives and approaches. In the first section, an attempt will be made to evaluate the efficacy of the research activities; later, will be discussed two different methodologies to evaluate the process efficiency. 

\subsection{Effectiveness}

Partially following the example from \citet{Giuliani2005} and \citet{Cantner2006}, this section will provide an attempt to evaluate the effectiveness of the FBK performance through the usage of Social Network Analysis instruments. 

The Social Network Analysis (SNA) aims to study an ensemble of actors, or nodes, characterized by simultaneously operating in a certain context, economy, supply chain, or simply the same system. These actors may be or not connected each other with a relationship, a tie, that can be evaluated for its presence (in a boolean fashion) or its strength, whenever this information is available. The social network analysis uses this nodes-edges representation, called graph, as the object for a quantitative, statistical analysis; the analysis can aim to describe it, as in descriptive statistic, or to provide some information about its dynamics and evolutionary patterns, similarly to a predictive approach.

In the specific case of a research organization as FBK, the social network analysis is particularly fitted to evaluate the effectiveness of its performance. Assuming that the most fundamental of its two main missions is the scientific excellence, and assuming that the organization operates in an open science approach, it derives that the performance can be evaluated through the comparative analysis of its relative status in the scientific community.

This approach will require a suitable data source, which can be easily found among the publications of the European Commission, namely the data relative to the EU Framework for Research and Technological Development, a phenomenon that is similarly based on the same open science principles. Briefly, it is a funding program aimed to support and foster the extent and quality of the research activities of the European Research Area, by the provision of grants for research-based and innovative projects through open and competitive calls for proposal; the FP7 cover the 2007-2013 period, while the H2020 span between the 2014 and the 2020

The general tendency for the selection process of these calls is to favor projects presented by ensembles of research organizations and private firms, usually organized as consortia. The \enquote{aggregativeness} of these applications suggests that the participation in these consortia is directly linked to the scientific reputation. It will influence the willingness of other entities to contact, relate and cooperate with it for the purpose of these research grants: the higher the reputation of an organization, the more and more valuable the projects an institution is involved in.

Assuming that the co-participation to a common project is a favorable condition for the establishment of a continuous and proficient relation between organizations, it can be used as a proxy for the presence of a relationship, leading to a method for building a social network. The graph will be composed of the individual institutions, tied by the co-participation to a common project. In this analysis, the graph will reflect the FP7 network, because of its closeness and the larger dataset with respect to the H2020. The dataset will also be purged of useless data, especially grants entitled to a single organization. 

Following this procedure, the network has been generated on the basis of 12.126 projects, about half of the FP7 grants, resulting in 851.655 relationships among 29.717 organizations. 

Once represented the European research network, the graph can be used to evaluate the relative position and importance of FBK. The literature regarding SNA tools for the descriptive analysis refer to a wide landscape of standard metrics, all differently useful in evaluating the individual position. Among them, the most important are the centrality measures; the most common and simple are the degree centrality, closeness centrality, and the betweenness centrality, which will be used in this case.

The degree centrality is by far the widest used and simplest measure. It represents the connectedness of a node as its total amount of relationships; in fact, it is calculated as the number of ties that connect the node to the graph. Specifically, FBK scored 909 ties, filling the 257th position in the entire FP7. In other words, FBK is among the top 1st percentile of actors by degree centrality. It can be considered a remarkable result, given the presence and competition of significantly larger institutions, among which top-tier academic institutions and massive public research organizations. 

Closeness centrality represents instead the average distance of a node from every other actor in the graph. Basically, this index assumes that the higher the centrality of a node, the more directed will be its linkages, instead of being mediated by other nodes. It is calculated as the reciprocal of the total distance, or length $d$ of the shortest path, between the actor $v$ in analysis and every other node $t$ in the graph; it is usually normalized for the total number of nodes not in analysis. 

\[
	C_C (v) =
		\frac 
			{n-1}
			{
				\sum_t d(v, t)
			} 
\]

The closeness centrality of FBK has been measured in 0.41493, ranking 1448th in the top 5th percentile. The difference between the degree and the closeness centrality indexes is attributable to the presence in the network of a sizeable number of independent sub-graphs, isolated from the network, which members present a relatively higher score in a smaller environment, thus affecting the evaluation. Nevertheless, it can be considered a positive result.  

Betweenness centrality instead evaluates the position of the node as an intermediary, a broker in the graph: the higher the number of shortest paths that cross the node, the higher its index. Its determination require, for the node $v$ in analysis, to compute every shortest path $j,k$ between every possible pair of nodes $\sigma$ ; for each of them, will be calculated the fraction of shortest paths $\sigma_{j,k} (v)$ which pass through the node in analysis; finally, the results will be aggregated. 

\[
	C_B (v) =
		\sum_{v \neq j \neq k}
			\frac {\sigma_{j,k} (v)}{\sigma_{j,k}}
\]

The betweenness centrality of FBK has been evaluated in 1644274, 168th in the graph, in the 1st percentile. Again, a remarkable result, which places the organization in a truly central position in the graph, where it assume the role of bridging different sub-graph, realities, or local contexts. 

Eventually, this analysis suggests that FBK has gained a position of primary importance in the European network of research and innovation, a result that would be hard to achieve without the performance of an \enquote{excellent} research, as prescribed by the organizational mission. Thus, relatively to the period 2007-2013, during the FP7 project which refer the data, it can be stated that the organization has in fact performed in according to, and reached, its objectives. 

\subsection{Efficiency}

To evaluate the efficiency of an organization or its processes, the economic literature usually refers to two different methodologies. The first contemplates the tracking over time of the changes in its inputs and outputs, as in the available resources and the resulting products. The second approach requires the comparison of the organization's performance to other similar institutions, assessing its efficiency through the comparison. Both these methodologies, however, present some issues.

\subsubsection{Performance comparison over time}

In the first case, the basic necessary condition for the applicability of this method is the presence of an effective information system, especially for the analytical accounting, which should keep track of the input resources and the products or services in output. Secondly, to evaluate the very process, it is implicitly assumed the absence of major, uncontrollable change between periods of evaluation; otherwise, the analysis may not be able to distinguish between a difference in performance and the change in context or institutional framework. In the specific case of FBK, these issues do represent a clearly recognizable limitation. 

In regard to the first issue, it does exist an informative system, but its effectiveness is limited by a set of issues. The main reason is the need for this evaluation to track the current amount of resources involved in each process, at an institutional level. However, in a technology transfer process, relevant resources include, but are not limited to: the time and effort of researchers and supporting personnel, among which TTO, legal office, administrative, and alike; materials, acquisition or use of machinery; external services unavailable internally. More specifically, the most relevant issue is the ability to keep track of the working time dedicated to each project. 

The most representative example regards TTO's officers, especially when the entire office is divided into different units specialized per thematic sets of activities, i.e.\ marketing and communication, business development, legal counseling, etc. In this case, projects compete for the time of TTO's employees, being them involved in any project that requires their specific competencies. Therefore, the impossibility of estimating the cost in human resources for each single project. Moreover, extensive differences among projects preclude the alternative evaluation through the mean time devoted to each project: the average measure will not be a reliable proxy. In fact, the issue at hand has been cited many times during the interviews of FBK TTO's officers: no employee, from any office, has been able to estimate the average time spent on each project, with lapses that ranged between hours and days.

In the second case, relative to the time series approach, the issue is to distinguish between the impact of external factors and the incremental, endogenous improvement of the process. The first set of factors includes local policies, changes in markets, industries, technologies, industrial context, national and supranational policies. The latter refer instead to the process flows, skills and capabilities of human resources, dedicated financial resources, different inputs, the overall institutional setting, and policies.

The temporal analysis should be based on an economic model complex enough to account for the impact of external factors, in order to not wrongfully ascribe positive or negative trends that can not be controlled by the organization. More specifically, assuming that this kind of organizations has a manageable impact on its context, the model should also distinguish between controllable factors, i.e.\ the impact on the local workforce training and the market development due to collaborative, government level policies, from truly exogenous factors.

The problem that FBK faces in this specific case, instead, is an ever-changing environment. The continuous refinement, but still a change, in the local government's policy, as well as in the organizational leadership by the Autonomous Province of Trento as founder, blur the distinction between exogenous and endogenous factors. Unitedly to the limited accountability - and process standardization - possibilities, these issues make the evaluation the FBK efficiency over time clearly complex and unreliable. Any further discussion is left to a more competent literature.

\subsubsection{Performance comparison between competitors}

In the absence of the necessary deep knowledge of the organizational cost structure, the economic literature usually points at the comparison between competitors' performance as an effective, alternative method for evaluating the individual performance. The rationale is that while the process itself may be too complex or unstable, similar organizations that perform the same activity should encounter the same issues, opening the possibility for a comparison and the potential discovering of best practices.

Again, this methodology relies on an obvious, necessary condition: the availability of a similar organization. While the degree of similarity is somehow questionable, some condition should be respected nevertheless. In a general perspective, the organizations should share the context, or at least its major traits; examples are: the degree of technology advancement, R\&D intensity and expenditures, the academic environment, local policies, level and competencies of the workforce, neighbour firms and, to some extent, the organization's level of specialization and economic performance, as well as the knowledge and research fields.

At a more operative level, other traits that the organizations should share are: the scale of operations, as in the level of economies of scale at play; the institutional mission; the main type of activity; the income structure. Otherwise, the analysis may highlight the various institutional and economic differences among context or unmanageable organizational factors, rather than discovery the difference in performance that arises from structure, personnel, resources, activity and alike.

This comparative evaluation, however, presents some issues arising the peculiarities of the Italian context in general and the Trentino context more specifically. For the first issue, examples are: the smaller average size of firms, which impact their innovation capabilities and processes; the performance of Italian universities in technology transfer activities, on average lower than the EU mean \citep{Balderi2007}; the limited number of private research organizations to use for a comparison; the related tendency of research organizations to be public, entirely founded and financed by the local or national government. Moreover, the Trentino context itself  represent quite an exception to the Italian landscape, for legislative and historical conditions. 

Thus, the issue in identifying a suitable organization for the comparison. In the local context of FBK, R\&D is performed by the local university, which is not directly comparable due to its public mission and structure; and the Edmund Mach Foundation (FEM), which is, in fact, is the \enquote{sister} foundation of FBK. Other technology transfer entities, i.e.\ the Trentino School of Management and the Trentino Innovation Hub, do not perform internal research activities.

Similarly to FBK, FEM is a publicly-funded research foundation. It performs teaching and training activities, through its internal high school and cooperative doctoral programs. Secondarily, FEM conducts R\&D in technological fields, especially  biotechnologies and genomics. At the same time, the foundation performs production activities, through the connected winery, and provides specialized services to the local agricultural industry. Lastly, it formally operates technology transfer activities and its organizational structures include a TTO.

However, the differences between these organizations outweigh their similarities. Firstly, the size: FEM employ twice the  personnel, of which researchers represent only a secondary component: about 150 scientists. Similarly, the total income is also doubled, while activities focus on the provision of teaching and training activities, and services to the local agricultural industry. Moreover, its organizational structure is significantly heavier and less flexible than FBK: at its foundation, the Autonomous Province of Trento gathered different public institutions, merging them together while maintaining internal separate administrative units for each of them; the result can be seen in the high number of administration personnel, which account for almost half of the employees. 

Lastly, and more importantly, the approach through which FEM conceive, design, and perform technology transfer activities is significantly different. FEM perform basic research on topics that are less appliable to the local industry, thus the necessary change in the object of transfer activities. The applied research instead is limited in its extent, as well as its relative impact on the TTO input resources. Last is the provision of specific services, which is actually the focus of the FEM's TTO. This mix, in fact, inverts the FBK primary attitude toward basic and applied research.

The extensive differences between FEM and FBK indicate that the former is not a suitable alternative for a comparative analysis of performances; yet, comparable organizations could be fund in different contexts. The seek should start from the comparison of local innovation contexts and government policies; whenever a similar context is individuated, the researcher could investigate the internal actor to this environment seeking a similar organization. However, considering the focus and the objectives of this thesis, this procedure, as well as other further developments are left to a more competent literature.

\section{Policies}

FBK addresses its commercial activities through five main policies. While the results of these policies have already been described by analyzing the various internal processes, this section will describe the letter of the individual policies. Firstly will be described the major, central policy, which describes the organizational attitude toward commercial activities; later, will be described other 4 minor policies related.

\subsubsection{Policy for the valorization of FBK research}

This is the most generic policy, which describes the attitude and orientation of the research organization. It delineates the general policies for the commercial exploitation, spin-offs, and patents while leaving the description of the inner processes and bureaucratic affairs to successive, specific policies. The same policy instituted the internal TTO and declares its functions and objectives.

\begin{itemize}

\item \textbf{The Innovation and Territorial Relations Area} is a TTO invested with two main tasks: to promote the sensibilization of internal human resources for the direct commercial exploitation, especially spin-offs; and to manage the intellectual property, spin-offs and other commercial activities, including negotiation and contractual activities. 

\item \textbf{The policy for the creation of spin-offs and subsidiaries.} This policy grants FBK the ability to participate in the foundation of spin-offs, by investments in equity, and to sustain the new venture through other means of economic and financial support. FBK may also provide supporting activities as tutoring, networking activities and alike. The main requirement for the potential spin-off is to have as main objective the valorization of research and know-how internally generated; in later stages, at least one member of the Board of Directors must be nominated by FBK.

The initial proposal must contain a business plan, a draft of the statue, business partners and the composition of institutional bodies. The project will be firstly evaluated by the Entrepreneurship Evaluation Committee (CVI), composed by members with significant experience and knowledge; later, the Board of Directors will deliberate on the endorsement and financial aid to the project. Equity shares will be released after 5 years, other financial investment instead will be withdrawn after 2 years.

\item \textbf{Exploitation through patenting.} This short policy describes the decision to outsource a significant share of the activities surrounding patents to external entities (i.e.\ patent attorneys and patent valorization firms). The policy, however, requires the first application to the Italian Patent office, as the first step for the patenting process; at the same time, researchers and AIRT will start seeking for potential licensees. In the successive twelve months, patent fees and legal costs will be sustained by the TTO, but if the patent has gained no interest in this lapse, it will be abandoned, sold back to the inventor, or maintained directly by the research unit. 

\item \textbf{Commercial exploitation of research results.} This last short policy states the limit of commercial activities, which do not comprehend research projects funded through research grants or direct research contracts from private firms. Therefore, the origination of commercial activities is left to the internal, independent research, and collateral discoveries in the excluded activities. If a financial income is gained, the amount will be shared with the researcher and its research unit.

\end{itemize}

\subsubsection{Child policies}

\begin{itemize}

\item \textbf{Policy for the management of the intellectual property.}\footnote{
	\href
	{https://airt.fbk.eu/sites/airt.fbk.eu/files/policy_gestione_proprieta_intellettuale_fbk.pdf}
	{FBK Policy for the management of Intellectual Property}
	[accessed: Jenuary 31, 2017]
} The policy describes the willingness to protect and valorize the intellectual property, mainly as an instrument for the local economic development. To achieve this objective, researchers are required to disclose the results of their research to the TTO and other dedicated employees; it also anticipates the requirement for researchers to transfer the entire intellectual property relative to the project. 

To foster the individual willingness to participate in commercial activities, the clause for the \enquote{fair compensation and protection} of the inventor states that for economic results outside the normal contractual research, results will be aknowledged and fairly compensated. The policy also describes the decisional process: the Center Directors, Unit heads, AIRT and the CVI are in charge of the evaluation of the project and the decision to pursue it, including the requirement for a preparatory a market analysis. AIRT can also act proactively, by discovering the market needs and contacting the researchers.

\item \textbf{Procedure for patent exploitation.} This policy describes the general process flow for patenting and licensing FBK research. The policy firstly describes a preliminary phase, in which AIRT will provide training activities surrounding the major topics related to patents, and explicitly mentions institutional moments for the internal scouting of commercializable research, through periodical meetings between research representatives and TTO's officers.

On the procedure itself, the policy divides the process into three stages. The process begins with the proposal, through a formal communication that includes: the description of the technology, its competitive advantages, an estimation of the potential market value and impact. Successively, the CVI will the assessment of the proposal. Lastly, an external organization will be appointed for the legal procedure. Again, contemporary to the first, national application, the valorization activity should start. On the licensing process, is explicitly described the preference for non-exclusive licenses, then the exclusive ones, then the sell of the entire patent.

\item \textbf{Procedure for the creation of startup - spin-offs.} The policy describes the internal procedure for the creation of spin-offs; this activity is, in fact, more of an institutional objective, with its roots in the original statue and mission which describe the willingness to further and support new entrepreneurship projects based on internal research. Similarly to patenting and licensing, the policy anticipate a preliminary phase and the actual process.

The preliminary phase includes the evergreen training activities, the internal scouting for ideas and projects suited for this channel, the external scouting, i.e.\ matchmaking tools and events, and the organizational support in the basic concept, as in design, of the project. The procedure itself is structured in seven different stages: the formal communication of intention (1) and the first, generic evaluation by the CVI (2) on its potential. Later, the preparatory stage (3) includes the structuring of the business plan, the statute, and other documents; then, the formal evaluation stage, firstly by a delegate (4) then by the Board of Directors itself (5). Eventually, the NewCo is activated (6) and the formal contractual arrangement between the spin-off and FBK can be made (7).

\item \textbf{Procedure for the monitoring of subsidiaries.} The statute of FBK, along with granting the ability to spin off new ventures and participate other economic realities, requires the continuous monitoring of these relations. Specifically, external organizations are required to periodically report their financial and economic situation and position to the parent organization. The policy describes two different types of reports: a qualitative one, bimestrial, which should include an analysis of the ongoing performance, and its comparison with the forecasted values presented with the business plan; and a quantitative, semestral communication that should include patrimonial, economic and financial information.

\end{itemize} 