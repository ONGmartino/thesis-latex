% Chapter 7

\Chapter{Case Study}{Fondazione Bruno Kessler}

\label{Chapter7}

In this chapter, will be presented the case study of a research institution, the Bruno Kessler Foundation (henceforth FBK), to test the hypothesis made in the previous chapter. Specifically, FBK is a non-profit research institute, largely financed by the local public government, a case particularly fitted to present real world applications of many of the previous considerations. In fact, it includes problems from both a private and publicly-funded organization, while its historical luggage offers the space for some consideration on an entirely public institution. 

At the same time, FBK represents a significant case in the European research landscape, where its importance will be tested with the aid of social network analysis tools. At the same time, its relevance can also be assessed by the proxy of its dimension, both in revenue flows (about 30M euros) or the number of researchers and other personnel it employs (over 450), which both classify this institution as one of the largest private research institutes in Italy. Its localization in the Trento province, one of the most technological advanced areas of the country and a region in which peculiar institutional settings positively influence the power and impact of local policies, makes it fitted also for the analysis and evaluation of government technology transfer policies, which however will be demanded to a more proper literature.

Due to the specific objectives of this thesis, the case study discussion will point toward the organizational structure and the process perspective. The chapter will begin with a short presentation of the organization, including an historical perspective and a brief description of the local government's influence on mission and operativeness. After a short presentation of the organizational structure, a specific section will provide more specific information about the structure and the organization of its TTO.

Later, the chapter will describe channels and processes: firstly, an external perspective will be taken to focus on the various products and services provided to firms, research organizations and spinoffs, providing a generic perspective on channels. These channels will be decomposed and analyzed later, following an internal perspective on the corresponding processes and mechanisms. This analysis will be an effort of process modelling, in an organizational structure that actually have grown and evolved over time as an emergent auto-adaptation that shaped the structure concurrently with the organization designs provided by policies. Eventually, the institution has indeed adapted itself to the specific requirement progressively made by the market and the local institutions. 

Lastly, will be discussed the evaluation perspective. On one hand, will be assessed the efficacy of its research activities, through instruments of social network analysis, with the objective of providing a quantitative evaluation of a typically qualitative topic. On the other hand, will be discussed two methodologies to evaluate the efficiency performance, investigating the involved issues. Will follow a brief examination of the main organizational policies on technology transfer activities.

\section{The institution}

Nowadays, the Bruno Kessler Foundation is a nonprofit research institute of "public interest", funded in the 2007 with a law of the local government but autonomous and regulated by the private legislation\footnote{For "public interest" is intended an organization that is subjected to the legislation dedicated to private entities, while one or more conditions (in this case the presence of a public institution among the founders) entails the respect of secondarly obbligations for public institutions, i.e. higher requirements for trasparency.}. It has two major objectives: the scientific excellence and the local economic development.

The history of the institute begins in the 1962. In this year, Bruno Kessler, the president of the Autonomous Province of Trento, funded the predecessor of FBK: the Trento Institute of Culture (ITC)\footnote{
	\href{http://www.consiglio.provincia.tn.it/leggi-e-archivi/codice-provinciale/archivio/Pages/Legge\%20provinciale\%2029\%20agosto\%201962,\%20n.\%2011_565.aspx}
	{Provincial law n.11, 29 August 1962}
	[accessed: February 2, 2017]
}. At the time, Trento had not a local public university, and the ITC was instituted with the long-term objective of providing a favorable, scientific and innovative context for the later creation of a university. 

In the 1972, the Free University of Trento was finally funded, thus requiring a change in the objective and future of the ITC. With a functional university, the institution finally could invest its energy in the knowledge- and research-based kind of activities and services that the context still needed, but the university could not provide. Therefore, the focus shifted to a more applied research, utilizable by the context and characterized by a long-term development perspective, capable also to maintain and development of the local culture.

A remarkable milestone, in fact, was the institution of a research center dedicated to more technical research and applied science: the Institute for Technological and Scientific Research (IRST). Along with other institutes, i.e. the Italian-German Historical Institute (ISIG), the Center for Religious Studies, and later institutes, the ITC could cope with both the need for a technological development of the region, and the need for preserving the local culture. 

A more recent, major change in the strategic trajectory of this institution started in the 2005, when the local government decided to reorganize the local system of research and innovation. This action ended with the dissolution of the ITC in the form of a public institution, with its transformation into the Bruno Kessler Foundation, officially funded in the 2007\footnote{
	\href{http://www.consiglio.provincia.tn.it/leggi-e-archivi/codice-provinciale/archivio/Pages/Legge\%20provinciale\%202\%20agosto\%202005,\%20n.\%2014_12567.aspx?zid=6003d625-228e-4e5d-820d-d6cf459dfc36}
	{Provincial law n.14, 1 March 2005}
	[accessed: February 2, 2017]
}. The foundation has in fact the local government (in the form of the Trento Autonomous Province) among its founders, but the institution is mainly regulated by the private legislation, the same that regulate the activities of foundations and other non-profit organizations. 

However, while FBK has lost its public status, the strong interference of the local government has deeply influenced both the mission and the methodologies of the foundation. First, the financial impact: the local government account for two thirds of the total income of the foundation, with about of 30M in a 45M euro total income. By all means, without the public sustain and support, the foundation would not be able to cope with the financial requirements needed to perform both applied and basic research, necessary to the accomplishment of its organizational mission.

A second profound influence refer to the mission and business model. Differently from other private research organizations, FBK explicitly stated among its main objectives the positive influence for the local society, equal for importance to the scientific excellence and the economic survival. This set of objectives in fact reflect the pure identity of non-profit organization, but it is not clear if the actual set of interests and activities of FBK do descend from the influence of the public institutions or its history and legacy.

Eventually, the two primary objectives of FBK are the scientific excellence and the local development, suggesting that the first a necessary condition, a tool to achieve the second. The underlining idea is to hire star scientists, include them in a favoring structure, and employ them in (1) producing excellent basic research and (2) perform industry-led, research-based activities, i.e. contract development. Implicitly, by furthering the knowledge and technology bases of the local industry, FBK should be able to help the local economic growth by improving the competitiveness of local firms, attracting high-tech finances and star scientists.

\subsection{Organizational structure}

A brief description of the organizational structure should help understanding this continuous duality, between technological progress and historical luggage, between basic science and development services, between local develompent and institutional survival. Moreover, a blended organizational framework will be useful in understanding the internal processes. 

Main entities in the administration are the President, the Board of Directors, and the Secretary General. Notably, the Autonomous Province of Trento, as founding member, nominate both the President and 6 of the 8 members of the Board, therefore the extent of the influence of the local government. In staff to the Board of Directors there is a Scientific Committee, tasked with the ex-ante evaluation of annual and long-term plans, i.e. the long-term Program for Research Activities and Investments (PPARI), and the Budget and Annual Plan of Activities (B\&PAA).

Under the administration, the structure includes the research structure and various support offices in staff. The research body is firstly divided in two different hub: (1) scientific and technological, and (2) human and social science hub. Each hub is further divided divided into research centers, specialized for knowledge and technology topics, led by different directors:

\begin{itemize}

\item Scientific and technological hub:
	\begin{itemize}
	\item Information Technology Center (ITC)
	\item Center of Materials and Microsystems (CMM)
	\item International Center for Mathematic Research (CIRM)
	\item European center of Theoretical Physics (ETC*)
	\end{itemize}

\item Human and social science hub:
	\begin{itemize}
	\item Institution for the valutative research of public policies (IRVAPP)
	\item Italian-German Historical Institute (ISIG)
	\item Religious Sciences Center (ISR)
	\end{itemize}

\end{itemize}

Main centers are ICT and CMM, which account for the largest part of researchers. Given their size, these centers are further divided into different research units: 6 for CMM and 23 for ICT. While they employ in fact a similar number of researchers, CMM's projects tend to be larger, structured and demanding, requesting substantially larger teams; an example is the management of the internal clean room. The structure also comprehend independent research units, that arise form special projects; i.e. the framework agreement with the Italian National Research Center (CNR) led to the creation of a dedicated micro-center, with 3 internal units.

Other offices are in staff to the Secretary General and the research subsystem; the most important are Human Resources, ICT support, Infrastructure and Corporate Assets, AIRT (the organizational name for the TTO), Legal Office, Communication office. These offices are autonomous and indepentent from the research structure; they can activate or be activated by the research centers or units, while performing actitivies on the request of the administration and other normal maintenance activites. 

The entire organization employ more than 450 human resources, clearly inclined toward the research structure rather than than administration and support offices. According to the 2011 Integrated Report\footnote{
	\href
	{http://airt.fbk.eu/it/report-sociale-2011}
	{Integrated Report 2011}
	[accessed: July 31, 2016]
}, the organization counts 462 employees, 347 researchers and 115 resources among administration and support personnel. This ratio describes the organizational effort to maintain its flexibility in an otherwise massive structure: a lightweight support structure, with a network of relatively independent and agile research units.    

\subsection{TTO structure}

In this organizational setting, the office that is in charge for technology transfer activities is called "Innovation and Territorial Relationships Area" (AIRT), an institutional name for the Technology Transfer Office. It is tasked with two main objectives: to maintain and manage the relationships with external entities and to seek for external financial support to the research. In the most wide perspective, the office employs more than 12 peoples, both with specific tasks or for supporting functions. 

The head of the Area is directly involved in building, develop and maintain relationships with relevant external actors, i.e. spinoffs and relevant entities in the local context; on the other hand, the resource manage the internal connection with other prominent positions, as the President, the Secretary General and the Centers' Directors. Along with these networking activities, the employee administrate the office and his subordinate, yet mostly on matter of strategic importance. In fact, his employees might have a greater knowledge on the specific topics and issues, while they may also lack the larger, strategic prospective to manage individual key relationships.

The organizative role and position is also greatly shaped by the individual characteristics of the resource: prior to his engagement in FBK, the employee had acquired a relevant experience both as PhD researcher and an extensive experience in various, major companies. His double background is believed to be fundamental in understanding and cope with the needs and requirements of both the research and the industry side of his activity. Moreover, significant personal relationships in the local context should ease and foster the local institutional networking activities.

Next to him, the second organizational position is specialized in the management of industrial contacts. More specifically, the employee performs any activity related to: (1) developing an intial relation of mutual interest and trust, instrumental for the later development of the relationship; (2) manage the relationship and maintaining contact for already established and active exchanges, especially bridging the communication between researchers and firms. Examples of day-by-day activities are the participation to acquaintance and technical meetings, economic and contractual arrangements and alike.

Again, the resources gained a background in both the scientific research, mainly internally to FBK as a former Head of a Research Unit, and in the management of industrial relationships, mainly through his previous transfer projects. In this case, the availability of technical knowhow is essential to participate and support researchers in the most technical exchanges, while his personal network of industry contacts and the understanding of industry need help increasing the probability of a succesful transfer.

A third specific position is dedicated to business development activities, including the scouting of potential industry partners and customer, and the support to researchers in delineating technology-push commercial activities. Day-by-day activities include the elaboration of value proposition wrapping research products, individuate potential interested industry segments, contact firms, gain their initial interest. Other strategic activities include the strategic management of the patent portfolio, counseling to potential spinoffs, marketing operations. The related employee, differently from the previous examples, do not have a deep understanding of the technical content and the research processes, but he possess a relevant experience in firms, business development and technology transfer activities.

Two distinct offices follow, each one operating in a more traditional fashion in support to research: a legal support office, for contractual matters, and the Research Funding office, which offer support for research grants.

The legal support offers internal services as the analysis and evaluation of prior contracts, as well as the literal writing and the negotiation of any research-based contract, including research contracts, cooperative research, licenses, grant and framework agreements. The office employs two human resources, combining both the experience and the specific knowledge needed, in economic and legal issues: the negotiation iter of their daily operations surely require a legal knowledge and training, but also the skills and competences needed to evaluate the economic value of any project and cotnract. 

The Research Funding office has intead the objective of providing a specialized support for the application and manage of research grants. The employees actively seek for research grants from local, national and European institutions, screening the calls for projects that match the internal competencies and knowledge. Later, the project will be forwarded to the best suited research Unit, while providing support in contacting potential partners, writing the grant application and the consortia agreement. These activities clearly require the awareness of the research topic of every internal Unit, deeper than the simple field of research, while personal contacts and relationships with Units members can offer a distinctive advantage in the successfulness of their activities. For this purpose, both the human resources involved have a long-term, decennial experience in these activities, in this very organization.

The last core office of AIRT, the Territorial Relationship Office, has the main purpose of managing the relations that FBK has established with other institutions in the local context. In fact, one of the channels available, and chosen, for a positive impact on the Trento province is the active involvement of academic institutions and schools, i.e. in the form of curricular and formative internships; similar is the hosting and organization of conferences, meetings and alike. While this office might have not a direct impact on the foundation's incomes, it is necessary in accomplishing the second of the two main missions and purpose of FBK.

A separate mention should be made for an independent employee in staff to the Director of the CMM, due to the continuous and structured collaboration among him and AIRT. More specifically, the peculiar organizational role of this employee place him in the best-suited position to support researchers in a more informal setting. His daily activities comprehend the evaluation of patents proposal, the management of technical and informal acpects of industry relationships, and technology scouting. Peculiar to his position is the required experience and credibility that is required among researchers, in order to be spontaneously involved in their work, developed in over 20 years of experience in FBK.

\section{Channels}

In this section will be presented channels and processes related to the technology transfer activity of FBK. Firstly, an external perspective will be taken to describe the used channels as in the kind of services and products that the organization provide. Secondly, the channels will be described through the analysis of internal processes that constitute the development and delivery of the product or service. This methodology will also expose both the external apparence and the internal mechanisms of the activity. 

\subsection{External perspective}

The external perspective consider each channel as a single product or service that the organization provides. While this perspective might be useful in exemplifying the entire offer, it might induce in viewing the process as more market-pull than the actual real-world. In fact, considering external, independent actors alone, the analysis will procede through the various products that a firm can request to FBK, momentarely leaving aside a more proactive, technology-push approach in which the commercial activity is led by the emergent, deliberate actions of the research organization, rather than the interests of external entities.

\subsubsection{For firms}

The main service that FBK can provide to firms is the development of a technology or investigations and feasibility studies, on rather specific topics. The main contractual forms for this kind of service, that allow external entities to acquire technologies and knowledge are contract research, cooperative research, consulting and licensing. 

The contract research is a legal contract in which the research organization (FBK) undertake the development of a technology, the investigation of an issue, the feasibility study of an idea or project, deliver a specific knowledge, to another organization in exchange for a "contribution". While the compensation usually assumes the form of a financial flow, it could be also in nature, as in the right to access and use a protected idea or technology and alike; contracting firms can also provide resources in kind, i.e. employees, laboratories and other assets. Despite the form of the compensation, the core idea of a research contract is the commissioning of an activity, not unlike the outsourcing, of something that cannot or will not be performed internally.

Cooperative research shares most of its legal traits with reseach contracts, but it starts with a different assumption: neither the research organization and the firm have, individually, the entire knowledge base and resources (in any kind) to successfully complete the project. In this case, the organizations agree to cooperate on a specific activity, that can range - similarly to the research contract - from technology development, knowledge generation and whatsoever. Therefore, in this category falls every contract that has R\&D as object and both the organizations as active researchers.

Consulting usually refers to the transfer of knowledge rather than the technology. The primary example is for firms to request a support in ending or further the developing of their technology, product etc. The main corpus of knowledge and technology for the ultimate successful development already resides in the requesting firms, but its exploitation require an external intervent; exemplary is the routine of perform any futher activity at the firm location and with their assets. A specific case is the provisioning of training activities for human resources, but a minoritary activity among the services provided by FBK.

As previosuly stated, licensing is a contract in which an organization acquires the legal ability to exploit or make economic use of a technology or knowledge already developed by the licensing organization, an ability otherwise forbidden by the patent or other forms of protection. Tools tend to be relatively specific to the knowledge sector: in example, in many countries a software cannot be patented, while the copyright can provide a similar degree of protection and excludability. Licence agreements can widely differ, according to the type of utilization, contribution and other legal clauses, thus providing a degree of personalization for the exchange without other specific contractual forms.

\subsubsection{For research organizations}

FBK also establishes contractual relationships with other research organizations. Specifically, exchange contracts between research institutions can take any of the form previously desribed, i.e. the provision of a service or a license; however, some forms acquire a greater importance, especially research cooperation, with the two organizations collaborating on a shared topic.

Another example of a contractual form more specific to the case is the framework agreement. This contract describes the mutual interest of the organizations to collaborate whith each other, framing the forms and the topic in which the further collaborative research will be performed. The contract may include resources, rules for decision making processes, propriety of the results and similar clauses. Eventually, this contractual agreement constitutes the basis for any further development of the relationship.

A separate mention should be made for a very specific contractual form: the grant agreement. Similarly to the previous forms, while it may inolve both research organization and production companies, it is usually signed between research organizations, both public or private in nature, as a necessary step in applying for a public call for research grants. The funding source may be a generic public institution, but the most relevant case involve the EU Commission and the FP7 or H2020 grants. Briefly, to apply for these funds, the organization must deliver a proposal for the call's research question or topic, assigned by the Commission. The proposal will state the modalities and resources (including material, immaterial, financial and human capital) through which the research organizations mean to achieve the prescribed objectives.

Since the begin of the H2020 program, applications for these grants have become more and more competitive: given the raising number of organizations that apply for these funds, the increasing specialization of calls and applications, and the reatively fixed amount of funds, these grants are becoming harder and harder to win. To overcome this issue, and increasing the probabilty of a successful application, a specific movement is taking place: to secure the presence of the needed expertise, skills and capabilities, and resources through a collective applicatioin, made by consortia of research organizations. If the proposal will be selected, the organizations will present a consortium agreement, a contract between the participants that describe the modalities through which the research will be performed: individual tasks, results, resources, share of the grant funds, responsabilities, the central coordinator and alike. 

\subsubsection{For spinoff}

For spinoffs, FBK differentiates its supporting services among potenial, in-itere projects and already founded new ventures. In the first case, the organization provide a set of services centered on the provision of (1) a favorable entrepreneurial organizational environment, (2) scouting of spinoff opportunities, and (3) supporting services in shaping potential project, its business model and its appetibility for external investors.The first two cases share extensive similarities with the generic push toward a commercial-friendly organizational culture, as common, basic and multipurpose policies.

In the first case, activities mainly encompass the structuring of comprehensible and supportive policies,  entrepreneurial courses, training, conferences with successful scientists entrepreneurs, meetings, and the provision of useful and exploitable relationships into the industry. In respect to the standard activities for the organizational climate, these activities are more focused on the main spinoff-related topics: business modeling, seed and venture capital, legal aspects of the creation of a new venture, and alike. 

Similarly to any other commercialization activity, the opportunity scouting can be intended and performed as proactive, institutional, not dissimilar to the disclosure eliciting in an academic setting; in the simpliest case, the technologies availabe are known, and the question is on the modalities through which exploit it. In the most complex case, AIRT officers must ask researchers for their current activites, then individuate and suggest a commercialization path. The phenomenon can be also emergent, as in provided by a self-selecting environment. The most significative example is the provision of business plan competitions: the spinoff process will begin with the emergent, individual willingness to start a new venture, while the evaluation and selection activity will be provided by the contest itself.

Supporting activities, instead, range from the support in business modelling to business development activities. Examples include specific training activities, suggestions and validation of the business model and business plan, the introduction of the new entrepreneurs in the network of the parent organization, aid in developing new personal contacts, support in human resources practices. Other forms of support are financial, both in grants or equity, favorable licences, access to machinery and local work spaces. FBK offer its support also to already founded spinoffs, especially networking activities. If the process involved the acquisition of an equity share, in the first 1-2 years the parent organization can directly influence the choices of the spinoff.

\subsection{Internal perspective}

the process through which these channels form and be exploited has arise from both the policies designed by the organizaton itself (with the influence of the public instututions) and as emergent from the experience of involved human resources that helped in further shaping the process. eventually, the process appears as unique and unified, where different offices and HR will be activated on the basis of the ongoing need. 

\subsubsection{Ignition}
\subsubsection{Evaluation}
\subsubsection{Development}
\subsubsection{Finalization}

\section{Evaluation}

As previously stated, the performance of an organization can be evaluated along its effectiveness or its efficiency. Especially in the case of research organization, which processes are not as clear, known, and standardizable as the ones of a production firm, these evaluations require different perspectives and approaches. In the first section, an attempt will be made to evaluate the efficacy of the research activities; later, will be discussed two different methodologies to evaluate the process efficiency. 

\subsection{Effectiveness}

Partially following the example from \citet{Giuliani2005} and \citet{Cantner2006}, this section will provide an attempt to evaluate the effectiveness of the FBK performance through the usage of Social Network Analysis instriments. 

The Social Network Analysis (SNA) aims to study an ensemble of actors, or nodes, characterized by simultaneously operating in a certain context, economy, supply chain, or simply the same system. These actors may be or not connected each other with a relationship, a tie, that can be evaluated for its presence (in a boolean fashion) or its strenght, whenever this information is available. The social network analysis uses this nodes-edges representation, called graph, as the object for a quantitative, statistical analysis; it can aim to describe it, as in descriptive statistics, or to provide some informations about its dynamics and evolutionary patterns, similarly to a predictive approach.

In the specific case of a research organization as FBK, this social network analysis is particularly fitted to evaluate the effectiveness of its performance. Assuming that the most fundamental of its two main missions is the scientific excellence, and assuming that the organization operate in an open science approach, it derives that the performance can be evaluated through the comparative analysis of its relative status in the scientific community.

This approach will require a suitable data source, which can be easily found among the publications of the European Commission, among the data relative to the EU Framework for Research and Technological Development, a phenomenon that is similarly based on the same open science principles. Briefly, it is a funding program aimed to support and foster the extent and quality of the research activities of the European Research Area, by the provision of grants for research-based and innovative projects through open and competitive calls for proposal; the FP7 cover the 2007-2013 period, while the H2020 span between the 2014 and the 2020

The general tendency for the selection process of these calls is to favor projects presented by ensembles of research organizations and private firms, usually organized as consortia. The "aggregativeness" of these applications suggests that the participation to these consortia is directly linked to the scientific reputation. It will influence the willingness of other entities to contact, relate and cooperate with it for the purpose of these research grants: the higher the reputation of an organization, the more and more valuable the projects an institution is involved in.

Assuming that the co-participation to a common project is a favourable condition for the establishment of a continuous and proficient relation between organizations, it can be used as a proxy for the presence of a relationships, leading to a method for building a social network. The graph will be composed by the individual institutions, tied by the co-participation to a common project. In this analysis, the graph will reflect the FP7 network, because of it closeness and the larger dataset in respect to the H2020. The dataset will also be purged of useles data, in this case grants entitled to a single organization. 

Following this proceudre, the network has been generated on the basis of 12.126 projects, about half of the FP7 grants, resulting in 851.655 relationships among 29.717 organizations. 

Once represented the European research network, the graph can be used to evaluate the relative position and importance of FBK. The literature regarding SNA tools for the descriptive analysis refer to a wide landscape of standard metrics, all differently useful in evaluating the individual position. Among them, the most important are the centrality measures; the most common and simple are the degree centrality, closeness centrality and the betweenness centrality, which will be used in this case.

The degree centrality is by far the widest used and simplest measure. It represent the connectedness of a node as its total amount of relationships; in fact, it is calculated as the number of ties that connect the node to the graph. Specifically, FBK scored 909 ties, filling the 257th position in the entire FP7. In other words, FBK is among the top 1st percentile of actors by degree centrality. It can be considered a remarkable result, given the presence and competition of significanlty larger instutions, among which top-tier academic institutions and massive public research organizations. 

Closeness centrality represent instead the average distance of a node from every other actor in the graph. Basically, this index assumes that the higher the centrality of a node, the more directed will be its linkages, instead of being mediated by other nodes. It is calculated as the reciprocal of the total distance, or lenght $d$ of the shortest path, between the actor $v$ in analysis and every other node $t$ in the graph; it is usually normalized for the total number of nodes not in analysis. 

\[
	C_C (v) =
		\frac 
			{n-1}
			{
				\sum_t d(v, t)
			} 
\]

The closeness centrality of FBK has been measured in 0.41493, ranking 1448th in the the top 5th percentile. The difference between the degree and the closeness centrality indexes is attributable to the presence in the network of a sizeable number of independent sub-graphs, isolated from the network, which members present a relatively higher score in a smaller environment, thus affecting the evaluation. Nevertheless, it can be considered a positive result.  

Betweeness centrality evaluate instead the position of the node as an intermediary, a broker in the graph: the higher the number of shortest paths that necessarily cross the node, the higher its index. Its determination require, for the node $v$ in analysis, to compute every shortest path between every possible pair of nodes $\sigma_{j,k}$ ; for each of them, will be calculated the fraction of shortest paths $\sigma_{j,k} (v)$ which pass through the node in analysis; finally, the results will be aggregated. 

\[
	C_B (v) =
		\sum_{v \neq j \neq k}
			\frac {\sigma_{j,k} (v)}{\sigma_{j,k}}
\]

The betweeness centrality of FBK has been evaluated in 1644274, 168th in the graph, in the 1st percentile. Again, a remarkable result, which place the organization in a truly central position in the graph, where it assume the role of bridging different sub-graph, realities, or local contexts. 

Eventually, this analysis suggests that FBK have gained a position of primary importance in the European network of research and innovation, a result that would be hard to achieve without the performance of an "excellent" research, as prescribed by the organizational mission. Thus, relatively to the period 2007-2013, during the FP7 project which refer the data, it can be stated that the organization has in fact perfomed in according to, and reached its objectives. 

\subsection{Efficiency}

To evaluate the efficiency of an organization or its processes, the economic literature usually refers to two different methodologies. The first contemplateso the tracking over time of the changes in its inputs and outputs, as in the available resources and the resulting products. The second approach require the analyst to compare the organization's performance to other compatible instutitons, assessing its efficiency through the comparison. Both these methodologies however present some issues.

\subsubsection{Performance comparison over time}

In the first case, the basic necessary condition for the applicability of this method is the presence of an effective information system, especially for the analytical accounting, which should keep track of the input resources and the products or services in output. Secondly, to evaluate the very process, it is implicitly assumed the absence of major, uncontrollable change between periods of evaluation; otherwise, the analysis may not be able to distinguish between a difference in performance and the change in context or institutional framework. In the specific case of FBK, these issues do represent a clearly recognizable limitation. 

In regard to the first issue, it does exist an informative system, but its effectiveness is limited by a set of issues. The main reason is the need for this evaluation to track the current amount of resources involved in each process, at an istitutional level. However, in a technology transfer process, relevant resources include, but are not limited to: the time and effort of researchers and supporting personnel, among which TTO, legal office, administrative, and alike; materials, acquisition or use of machinery; external services unavailable internally. More specifically, the most relevant issue is the ability to accunt the working time dedicated to the project. 

The most representative example regards TTO's officers, especially when the entire office is divided into different units specialized per thematic sets of activities, i.e. marketing and communication, business development, legal counseling, etc. In this case, projects compete for the time of TTO's employees, being them involved in any project that require their specific competences. Therefore, the impossiblity of estimating the cost in human resources for each single project. Moreover, extensive differences among projects preclude the alternative evaluation through the mean time devoted to each project: the average measure will not be a reliable proxy. In fact, the issue at hand has been cited many times during the interviews of FBK TTO's officers: no employee, from any office, has been able to estimate the average time spent on each project, with lapses that ranged between hours and days.

In the second case, relative to the time series approach, the issue is to distinguish between the impact of external factors and the incremantal, endogenous improvement of the process. The first set of factors includes local policies, changes in markets, industries, technologies, industrial context, national and sovra-national policies. The latter refer instead to the process flows, skills and capabilities of human resources, dediated financial resources, different inputs, the overall institutional setting and policies.

The temporal analysis should be based on an economic model complex enough to account for the impact of external factors, in order to not wrongfully imputate positive or negative trends that can not be controlled by the organization. More specifically, assuming that this kind of organization has a manageable impact on its context, the model should also distinguish between controllable factors, i.e. the impact on the local workforce training and the market development due to collaborative, government level policies, from truly exogenous factors.

The problem that FBK faces in this specific case, instead, is an ever-changing environment. The continuous refinement, but still a change, in the local government's policy, as well as in the organizational leadership by the Autonomous Province of Trento as founder, blur the distinction between exogenous and endogenous factors. Unitedly to the limited accountability - and process standardization - possibilities, these issues make the evaluation the FBK performance efficiency clearly complex and unreliable. Any furter discussion is left to a more competent literature.

\subsubsection{Performance comparison between competitor}

In the absence of the necessary deep knowledge of the organizational cost structure, the economic literature usually point at the comparison between competitors' performance as an effective, alternative method for evaluating the individual performance. The rationale is that while the process itself may be to complex or unstable, similar organizations that perform the same activity should encounter the same issues, opening the possibility for a comparrisone and the potential discovering of best practices.

Again, this methodology relies on an obvious, necessary condition: the availability of a similar organization. While the degree of similarity is somehow questionable, some condition should be respected nevertheless. In a general perspective, the organizations should share the context, or at least its major traits; examples are: the degree of technology advancement, R\&D intensity and expenditures, the academic environment, local policies, level and competence of the workforce, neighbour firms and, to some extent, the organization's level of specialization and economic performance, as well as the knowledge and research fields.

At a more operative level, other traits that the organizations should share are: the scale of operations,as in the level of economies of scale at play; the institutional mission; the main type of activity; the income structure. Otherwise, the analysis may highlight the various institutional and economic differences among context or unmanageable organizational factors, rather than discovery the difference in performance that arise from structure, personnel, resources, activity and alike.

This comparative evaluation, however, presents some issues arising the peculiarities of the Italian context in general and the Trentino context more specifically. For the first issue, examples are: the smaller average size of firms, which impact their innovation capabilities and processes; the performance of Italian universities in technology transfer activities, on average lower than the EU mean ("NOMEAUTORE); the limited number of private research organizations to use for a comparison; the related tendency of research organization to be public, entirely founded and financed by the local or national government. Moreover, the Trentino context itself  represent quite an exception to the Italian landscape, for legislative and historical conditions. 

Thus, the issue in identifying a suitable organization for the comparison. In the local context of FBK, R\&D is performed by the local university, which is not directly comparable due to its public mission and structure; and the Edmund Mach Foundation, which is, in fact, is the "sister" foundation to FBK. Other technology transfer entities, i.e. the Trentino School of Management and the Trentino Innovation Hub, do not perform internal research activities.

Similarly to FBK, FEM is a publicly-funded research foundation. It performs teaching and training activities, thruough its internal high school and cooperative doctoral programs. Secondarily, FEM conduct R\&D in technological fields, especially  biotechnologies and genomics. At the same time, the foundation performs production activities, through the connected winery, and provides specialized services to the local agricoltural industry. Lastly, it formally operates technology transfer activities and its organizational structures includes a TTO.

However, differences among these organizations outweight their similarities. Firstly, the size: FEM employ twice the  personnel, of which researchers represent only a secondary component: about 150 scientists. Similarly, the total income is also doubled, while activities focus on the provision of teaching and training activities, and services to the local agricultural industry. Moreover, its organizational structure is significantly heavier and less flexible than FBK: at its foundation, the Autonomous Province of Trento gathered different public institutions, merging them together while maintaining an internal separate administrational unit for each of them; the result can be seen in the high number of administration personnel, which account for almost half of the employees. 

Lastly, and more importantly, the approach through which FEM conceive, desing, and perform technology transfer activities is significantly different. FEM perform basic research on topics that are hardly applicable to the local industry, thus the necessary change in the object of transfer activities. The applied research instead is limited in its extent, as well as its relative impact on the TTO input sources. Last is the provision of specific services, which is actually the focus of the FEM TTO. This mix in fact invert the FBK primary attitude toward basic and applied research.

The extensive differences between FEM and FBK indicate that the former is not a suitable alternative for a comparative analysis of performances; yet, comparable organizations could be fund in different contexts. The seek should start from the comparison of local innovation contexts and government policies; whenever a similar context is individuated, the researcher could investigate the internal actor to this environment seeking a similar organization. However, considering the focus and the objectives of this thesis, this procedure and other further developments are left for later research.

\section{Policies}

FBK address the commercial activities through 5 main policies. while the results of these policies has already been described while analyzing the various internal processes, it could be useful to describe the individual policies. In the first section will be described the major policy, which in fact describes the most the organizational attitude toward commercialization activities; the secon section will describe other 4 minor policies related.

\subsubsection{Policy for the valorization of FBK research}

this is the most generic policy, which describes the attitude and "orientamento" of the research organization. it delineate the general policies for spinoffs, patents and commercial exploitation, while leaving for specific policies the description of the various bureaucratic procedures. the same policy declare the functions and objectives of AIRT, the TTO.

\begin{itemize}

\item AIRT is invested with 2 main tasks: to promote the internal sensibilaztion on the direct commercial exploitatin, especially spinoffs, and to manage the intellectual property, spinoffs and other commercial activities, including negotiation and contractual activities. 

\item Policy for the creation of spinoffs and "compartecipate". The policy describe the ability of FBK to participate in the foundation of spinoffs, in equity, and to sustain the new venture through economic and financial support. the main requirement for spinoff is to have as main objective the valorizaiton of knowhow and research generated in FBK. the proposal for the spinoff project must contain a BP, proposal of the "statuto", "soci" and the composition of the instituitonal organs; at least one board member has to be nominated by fbk. the project is evaluated by the CVI in a first stage, which is composed by HR with a significant experience and knowledge in the knowledge topic and the economic matter; later, the project will be evaluated, thus financed, by the CDA. the limit for the way out form the social capital is 5 years, 3 for other financial aids; FBK can also provide tutoring and other supporting services, but the economic affairs seem to be governed by the market logic: no discount on the licensing of techonologies, or contract research etc. 

\item Exploitation through patenting. This short policy describe the outsourcing of legal affairs surrounding patents to external entities (i.e. patent attorneys). The policy require the first italian deposit as first step on the patenting process, starting a 12month period in which the patent fees and requirements will be sustained by the TTO while the same office, the researchers and other offices will seek for potential licensee or buyers. after 12 months, if nothigh came up, the patent can be sold to the researcher or maintained by the research unit. for licensing, is preferred the non-exlusive license, then the exclusive, then selling.

\item Commercial exploitation of research results. ANother short policy that states that the commercial activities do not refer to research projects funded by institution, i.e. local (pat) or european (i.e. fp7), including  direct research contracts from private firms. thus, for the remaining commercial activities (patenting on internal, independent r\&d) which produce a financial income, the amount will be devided between the researcher, its research unit, and eventually the administration.

\end{itemize}

\subsubsection{Child policies}

\begin{itemize}

\item \textbf{Policy for the management of the intellectual property.}
\footnote{
	\href
	{https://airt.fbk.eu/sites/airt.fbk.eu/files/policy_gestione_proprieta_intellettuale_fbk.pdf}
	{FBK Policy for the management of Intellectual Property}
	[accessed: Jenuary 31, 2017]
} 	
firstly, it describes the willingness to protect and valorize the intellectual property, mainly for the purpose of being as institution a "" for the local economic development. To achieve this objective, researchers are required to disclose to other internal personnel and offices the results of their research, as well as "cedere" the relative intellectual property in its entirety. to further the individual willingness to participate in such projects, the policy "prevede" also a "fair compensation and protection for the author" which basically states that for economic results outside the normal contractual research, results will be aknowledge and fairly compensated. the policy also "decisional" chain: the Center Director, Unit head, AIRT, CVI, including the requirement for them to consider the ideas of the individual researhcers and continue the valorization project only after a market analysis; the TTO can also act proactively, by discovering the market needs and contact the research units/center.

\item \textbf{Procedure for patent exploitaton.} This policy describes as an overview the process for patenting and licensing FBK researches. The policy "contempla" a preliminar phase, in which the TTO will provide training and formation surrounding the major topics surrounding the patent, and esplicitally includes institutional moments for the internal scouting - as in disclosure eliciting - through periodical meetings between research representatives and TTO officers. for the procedure itself, the policy divide the process into: proposal, through a formal comunication that includes the description of the technology, its competitive advantages and the potential market value/impact; the assessment of the proposal, made by the CVI; "istruttoria", a phase in which external organizations will be appointed for the bureaucratic procedure, in which is based anyway on a first standardized step (the first deposit in italy) and economically backed by the TTO, and a contemprary start of valorization activities. again, after 12 months with no interests from external firms, the patent should be transferred to the resarcher or paid by its unit. Lastly, the policy includes the payment of a compensation to involved researchers, at the moment of the financial income.

\item \textbf{Procedure for the creation of startup - spinoffs.} Thsi policy describes the internal procedure for the creation of spinoffs, an organizational objective with its roots in the original statue and mission of further and support new entrepreneurship projects based on internal research. as a preliminar phase, the policy describe the evergreen training and formation; the internal scouting for ideas and projects bettwe suited for this channel; external scouting, meaning the matchmaking of external needs and internal ideas; and the concept, as in design, of the basic ideas as aided by the TTO. the procedure itself includes 7 different stages: a formal communication of intention; a first, quite generic evaluation of the project by the CVI, to assess its potentiality; "istruttoria" which includes the structuring and perfection of a business template, statute for the newco, and other institutional slash legal thingies; then, the formal evaluation, firstly by a delegate of the CDA then by the CDA itself; if things didn't go south, the newco can be activated (step 6) and the formal contractual arrangement between the spinoff and FBK can be made.

\item \textbf{Procedure for the monitoring of "collegate".} The statute of FBK, along with the possibility of spinning off new ventures and participating other external economic realities, require these external organizations to periodically report their situation and results to the parent organization. the policy describe two different reports: a qualitative one, to present every two months, which should include the analysis of the ongoing performance with the forecast presented into the BP; and a quantitative, semestral communication that should include patrimonial, economic and financial informations.

\end{itemize} 