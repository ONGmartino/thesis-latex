% Chapter 7

\Chapter{Case Study}{Fondazione Bruno Kessler}

\label{Chapter7}

In this chapter, will be presented the case study of a research institution, the Bruno Kessler Foundation (henceforth FBK), to test the hypothesis made in the previous chapter. Specifically, FBK is a non-profit research institute, largely financed by the local public government, a case particularly fitted to present real world applications of many of the previous considerations. In fact, it includes problems from both a private and publicly-funded organization, while its historical luggage offers the space for some consideration on an entirely public institution. 

At the same time, FBK represents a significant case in the European research landscape, where its importance will be tested with the aid of social network analysis tools. At the same time, its relevance can also be assessed by the proxy of its dimension, both in revenue flows (about 30M euros) or the number of researchers and other personnel it employs (over 400), which both classify this institution as one of the largest private research institutes in Italy. Its localization in the Trento province, one of the most technological advanced areas of the country and a region in which peculiar institutional settings positively influence the power and impact of local policies, makes it fitted also for the analysis and evaluation of government technology transfer policies, which however will be demanded to a more proper literature.

Due to the specific objectives of this thesis, the case study discussion will point toward the organizational structure and the process perspective. The chapter will begin with a short presentation of the organization, including an historical perspective and a brief description of the local government's influence on mission and operativeness. After a short presentation of the organizational structure, a specific section will provide more specific information about the structure and the organization of its TTO.

Later, the chapter will describe channels and processes: firstly, an external perspective will be taken to focus on the various products and services provided to firms, research organizations and spinoffs, providing a generic perspective on channels. These channels will be decomposed and analyzed later, following an internal perspective on the corresponding processes and mechanisms. This analysis will be an effort of process modelling, in an organizational structure that actually have grown and evolved over time as an emergent auto-adaptation that shaped the structure concurrently with the organization designs provided by policies. Eventually, the institution has indeed adapted itself to the specific requirement progressively made by the market and the local institutions. 

Lastly, will be discussed the evaluation perspective. On one hand, will be assessed the efficacy of its research activities, through instruments of social network analysis, with the objective of providing a quantitative evaluation of a typically qualitative topic. On the other hand, will be discussed two methodologies to evaluate the efficiency performance, investigating the involved issues. Will follow a brief examination of the main organizational policies on technology transfer activities.

\section{The institution}

Nowadays, the Bruno Kessler Foundation is a nonprofit research institute of "public interest", funded in the 2007 with a law of the local government but autonomous and regulated by the private legislation. It has two major objectives: the scientific excellence and the local economic development.

The history of the institute begins in the 1962. In this year, Bruno Kessler, the president of the Autonomous Province of Trento, funded the predecessor of FBK: the Trento Institute of Culture (ITC) \footnotetext{Provincial low n. 11, 29 August 1962}. At the time, Trento had not a local public university, and the ITC was instituted with the long-term objective of providing a favorable, scientific and innovative context for the later creation of a university. 

In the 1972, the Free University of Trento was finally funded, thus requiring a change in the objective and future of the ITC. With a functional university, the institution finally could invest its energy in the knowledge- and research-based kind of activities and services that the context still needed, but the university could not provide. Therefore, the focus shifted to a more applied research, utilizable by the context and characterized by a long-term development perspective, capable also to maintain and development of the local culture.

A remarkable milestone, in fact, was the institution of a research center dedicated to more technical research and applied science: the Institute for Technological and Scientific Research (IRST). Along with other institutes, i.e. the Italo-German Historical Institute, Religious Science, and later institutes, the ITC could cope with both the need for a technological development of the region, and the need for preserving the local culture. 

A more recent, major change in the strategic trajectory of this institution started in the 2005, when the local government decided to reorganize the local system of research and innovation. This action ended with the dissolution of the ITC in the form of a public institution, with its transformation into the Bruno Kessler Foundation, officially funded in the 2007\footnotetext{Provincial law n.14, 1 March 2007}. The foundation has in fact the local government (in the form of the Trento Autonomous Province) among its founders, but the institution is completely regulated by the private legislation, the same that regulate the activities of foundations and other non-profit organizations. 

However, while FBK has lost its public status, the strong interference of the local government has deeply influenced both the mission and the methodologies of the foundation. First, the financial impact: the local government account for two thirds of the total income of the foundation, with about of 30M in a 45M euro total income. By all means, without the public sustain and support, the foundation would not be able to cope with the financial requirements needed to perform both applied and basic research, necessary to the accomplishment of its organizational mission.

A second profound influence refer to the mission and BM. Differently from other private research organizations, FBK explicitly stated among its main objectives the positive influence for the local society, equal for importance to the scientific excellence and the economic survival. This set of objectives in fact reflect the pure identity of non-profit organization, but it is not clear if the actual set of interests and activities of FBK do descend from the influence of the public institutions or its history and legacy.

Eventually, the two primary objectives of FBK are the scientific excellence and the local development, suggesting that the first a necessary condition, a tool to achieve the second. The underlining idea is to hire star scientists, include them in a favoring structure, and employ them in (1) producing excellent basic research and (2) perform industry-led, research-based activities, i.e. contract development. Implicitly, by furthering the knowledge and technology bases of the local industry, FBK should be able to help the local economic growth by improving the competitiveness of local firms, attracting high-tech finances and star scientists.

\subsection{Organizational structure}

The organizational structure help understanding this continuous duality, between technological progress and historical luggage, between basic science and development services, between local develompent and institutional survival.

after the obvious CDA, president and segretary genera, researchers are organized in two different hub: scientific and technological, and human and social science. again they are divided into research centers:

\begin{itemize}

\item Information Technology Center (ITC)
\item Center of Materials and Microsystems (CMM)
\item International Center for Mathematic Research (CIRM)
\item European center of Theoretical Physics (ETC*)
\item Institution for the valutative research of public policies (IRVAPP)
\item Italo-German Historic Institute (ISIG)
\item Religious Sciences Center (ISR)

\end{itemize}

THe main centers are the ITC and CMM, accounting for the largest part of researchers. centers are lead by a unique director, but these two, given their size, are further divided into different research units: 6 for CMM and 25 for ICT. while they employ a not so different number of researchers, CMM projects tend to be larger, structured and demanding, requesting substantial human resources. the presence of a clean room is an example. The organization also comprehend other research units that arise form special projects, i.e. the framework agreement with the CNR which led to the institution of an apposit micro-center with 3 internal units.

Other offices are in staff to the secretary general and the research subsystem: the most important are the HR department, ICT support and "patrimonio", AIRT (a fancy name for the TTO), Legal office, Communication office. These offices are completely separated from the research structure, and can both be activated or activate the research, apart form internal autonomous work. 

The entire organization employ more than 450 HR, clearly pending toward the research part rather than the administration and support; according to the 2011 Integrated Report ("link grazie"), in the total amount of 462 employees, 347 were researchers and 115 supporting and administration personnel. this ratio can describe the effort to give flexibility to an otherwise large organization: a "lightweight" supporting structure, with a network of relatively independent and agile research units.    

\subsection{TTO structure}

In this organizational structure, technology transfer activities are demanded to a unit called Innovation and Territorial Relationships Area, an institutional name for the Technology Transfer Office. It performs all the main task related to the relations with external entities and the seek for external financial support to the research.

The Area is led by Giuliano Muzio, who gain a PhD in Physics prior to an extensive experience in important firms in various industry. His background is believed to be fundamental to the understanding of identities and requirements of both researchers, research institutions and firms. He is involved in keeping relationships with relevant external actors, as in spinoffs and the public institutional context, as well as with internal prominent positions, such as President, Secretary General and centers' directors. Along with these networking activities, he also acts as a counselor/decision maker for his employees: there is basically a great degree of freedom within the staff in day-by-day operations, as long as these do not involve any strategic or prominent counterpart. Examples can be meetings with important firms and the evaluation committee for spinoffs. However, the actual control and lead of the area is not his main activity.

Next to him, two specific positions should be highlighted.

The first is covered by Alessandro Bozzoli, previously researcher and head of the green techs unit. He performs any activity related to already established relationships with firms, specifically when FBK researchers are involved. Again, his background of scientific research and cooperation with industry constitute a valuable asset for the Area. Briefly, he usually takes part in any technical meeting between researchers and firms, as well as some of the early acquaintance visit. An interview has been planned in order to get an insight on processes involving this position.

The second position closely linked to the Area head is covered by Samuele Morales, who perform commercialization activities. Experienced in extremely relevant contexts, he is formally employed as Business Developer. His main task is to help researchers in the definition of their results’ applicability to the market. This includes: elaborate value propositions surrounding research results, determine potential interested industry segments, contact firms, gain their interests. Other activities comprise strategic management of the patent portfolio, counseling to potential spinoffs, marketing operations.

Two distinct offices follow, each one operating in a more traditional fashion in support to research.

The first is the Legal office, which employ two human resources, one with a specific education background and one with a decennial experience in the field. They act both as legal support, as in the analysis of prior contracts and the evaluation of any potential contractual legal bonds, as well as the literal writing of general agreements, service contracts and grant agreements. The least is known to be time-intensive, because of the number of iteration required by a contract to be eventually signed. Moreover, both the human resources employed in this office possess specific and unique set of skills fundamental to any external economic initiative. An interview has been planned to uncover differences in processes they get involved in, both for research cooperation and research contracts.

The other is the Office for research funding. It employs two resources, greatly experienced, who actively seek for possible research grants from national and European institutions. More precisely, one employee has been professionalized firstly on FP7, then in H2020 grants, and occupies herself analyzing the various call for proposal, forward them to the interested research units, help them contact potential partners and lastly to write grant consortia agreement and application. The other resource cover other European call, such as linked to Marie Curie action and EIT, and from local entities and foundation like Cariplo and Caritro. For this office, has been planned a fourth and a fifth interview.

The core part of the TTO involve one more office, who’s main purpose is to keep relationships with the local context and institutions. The main channel chosen for a positive impact on the Trento province is the active involvement of schools and other minor academic institution. Particular effort has recently been put in the management of curricular and formative internships, involving hundreds of 16- and 17-years-old students with the aim of getting them interested in research activities. This office has proved himself of great importance, even in absence of connected financial flows, for the ability of office workers to get acquainted with any FBK researcher and possibly bridge them to other Area’s personnel. 

Apart from those employees, highly skilled and experienced, the Area involve another 4-6 workers, devolved to specific projects, or in assistance for high time-consuming tasks. Frequently, university trainee has been involved, mainly from economics and in international relationships areas.

Separate mention should be made for the unique organizational role of Mario Zen, now in staff to the director of the Materials and Microsystems Center. His experience in FBK starts as a researcher, to achieve in early 2000 the position of Director of IRST – the entity that precedes FBK, to finally end up in a tailored position. His activities comprehend patents, relationships with firms and researchers and technology scouting. Peculiar to his position is the required experience and credibility that is needed against both researchers, to get spontaneous update on their work, and external companies, who want to face a stable organization capable of engagement, represented by a unique point of contact and without getting involved in any of its complexity.


\section{Channels}

in this section, will be presented channels and processes related to the technology transfer activity of FBK. Firstly, an external perspective can be taken to describe the used channels as the kind of services and products that the organization provide. lately, the internal channels will be described through the analysis of internal processes that lead to the development and delivery of the product or service. this way it is possible to describe both the external apparence and the internal mechanisms of the activity. 

\subsection{External perspective}

the external perspective on channels can consider each of those as single products or services that the organization provide. caution: this perspective is more market-pull than the process in the real world: considering only external, independent actors, we are considering the kind of requests that the firm can make to FBK, momentarely leaving aside a more proactive, technology push approach in which the commercial relation with external entities is led by the deliberate actions and intentions of the research organization, rather than the ones of external entities.

\subsubsection{For firms}

for firms, the main service that fbk can provide is the development of a technology or an investigation, i.e. a feasibility study, on a specific topic. the main contractual forms that allow the external to acquire such services are contract research, cooperative research, consulting, licensing. 

the contract researches are contracts in which the research organization (FBK) "si impegna" to develop a technology, investigate an issue, study the feasibility of an idea or project, deliver some knowledge, to another organization in exchange for a "contribution", which usually assume a financial flow but could be also in nature, as the right to access and use a protected idea or technology. contracting firms can also provide resources in kind, i.e. employees, laboratories and other assets, but the core idea of a research contract is the commissioning of an activity, as in outsourcing, of something that cannot be or won't be performed internally.

cooperative research is similar to reseach contracts, but it starts from a different assumption: the idea is that neither the research and the industry firm have, single handed, the entire knowledge base and resources (in any kind) to succesfully complete the project. in this case, in fact, the organizations agree to cooperate on a specific activity, that can range - again - from technology development and knowledge generation and whatsoever. therefore, in this category fall every contract that has r\&d as object and both organization as researchers or providers

consulting usually refer to the transfer of knowledge rather than technology. the primary example is for firms to request the "help" in ending or further the developing of their technology, evaluatin something etc. the main corpus of knowledge and technology for the ultimate product already resides in the requesting firms, and any futher activity should take places in their locations and with their assets. a specific case is the provisioning of training activities for HRs, by far a minoritary activity

again, as previosuly stated, licensing is a contract in which the industry organization acquire the legal ability to make economic use of (slash exploit) a technology or knowledge already developed by the contracting research organization and patented or otherwise protected. other tools are expecially used in different knowledge sectors, i.e. software in italy cannot be patented, but the copyright can provide some protection and excludability. licence agreements can widely differ for the utilization, payment and other legal clauses, therefore providing a fair degree of personalizaiton without other specific contractual forms

\subsubsection{For research organizations}

FBK can also establish contractual relationships with other research organizations, as in exchange or the provision of a service or licence. specifically, exchange contracts between research institutions can take any of the form previously desribed; however, some forms acquire a higher importance, especially research cooperation.

another contractual form more specific to this case is the framework agreement. this contract states the interest of organization to collaborate whith each other, and the contract states the forms and the topic in which these further researches and collaboration will be performed, possibly including resources, decision making rules, propriety of the results. in fact, this agreement constitute a ground for any further development of the relationship.

another specific contractual form is the grant agreement, which however do not take place directly between research orgnaizations. with these, organizations states an arrangement of resources, tasks, objectives relative to a research project financed by some public institution, especially the EU commission. the most important type of grant, for organizations like FBK, is constituted by the FP7 programme and grants. to make a request for these funds, the organization has to present a proposal for a research question/topic decided by the EU institutions. it will state mainly the modalities and resources (especially human resoures) through which the research organization means to achieve the prescribed resources. in latest years, fp7 grants are more and more competitive: given the increase in number of organizations that apply for these funds, and their being fixed to a relatively stable amount, these grants are becoming harder and harder to win. one specific movement is taking place: to assure the presence of the needed expertise, skills and capabilities, and generally increasing the probablity of a successful research, many grant applications are made by consortia of research organization, stating the collective willingness to achieve the result. in the case this proposal will be selected, the consortia has to present a grant agreement, a form of contract between the organization that describe the modalities through which the research will be performed: who will seek for which sub-results, resources, shares of appropriation of the grand funds, responsabilities, the selection of a central coordinator etc. 

\subsubsection{For spinoff}

FBK differentiate its supporting services for spinoffs, for "would be" and already spinned (already a separate legal entity). in the first case, the organization provide services centered on the provision of a favorable entrepreneurial organizational environment, scouting of spinoff opportunities, and supporting services in shaping the would be venture, its business model and its appetibility for external investors

in the first case, such activities mainly encompass entrepreneurial courses, the structuring of understandable and supporting policies, conferences with successful scientists entrepreneurs, meetings, the provision of usable relationships into the industry etc.

in the second, the scouting can be proactive, institutional-pushed, similar to the process of disclosure eliciting in an academic setting; this is the simpliest case, and in fact is the outcome of a normal choice in "this is a tech, how do we exploit it?". the phenomenon can be also emergent, meaning the providing of a self selecting environment. example is the provision of business plan competitions: it is based on the emergent, individual willingness to start a new venture, while the selection activity is provided by the necessary evaluation included in this activity.

supporting activities instead reange from the supporting in business modeling, including specific training activities, suggestions and checks with competent human resources internal to the TTO or other offices, to business development activities, i.e. the help in establishing contacts among firms, the access to the parent organization network, help in hr practices. other supports are financial, as in grants and equity, favorable licences (under the parity, in example)

the supporting activities should last also for already spinned new ventures, especially networking activities. in the first period, i.e. 1-2 years, the parent organization can also be part of the equity, thus directly and legally influencing the choices of the spinoff. however, a safe policy should requre the parent organization to exit the property in a couple of years

\subsection{Internal perspective}

the process through which these channels form and be exploited has arise from both the policies designed by the organizaton itself (with the influence of the public instututions) and as emergent from the experience of involved human resources that helped in further shaping the process. eventually, the process appears as unique and unified, where different offices and HR will be activated on the basis of the ongoing need. 

\subsubsection{Ignition}
\subsubsection{Evaluation}
\subsubsection{Development}
\subsubsection{Finalization}

\section{Evaluation}

As previously state, the performance of an organization can be evaluated for its effectiveness or its efficiency. Especially in the case of research organization, which processes are not as clear, known, standardizable as a production firm, these evaluations require different perspectives, thus approaches. In the first section, an attempt will be made to evaluate the efficacy of the research activities; later, will be discussed two different methodologies to evaluate the process efficiency. 

\subsection{Effectiveness}

Partially following the example from \citet{Giuliani2005} and \citet{Cantner2006}, this section will make an attempt to evaluate the effectiveness of the FBK performance through the usage of social network analysis instriments. 

The social network analysis uses as object of analysis the ensemble of actors, or nodes, that operate in a certain context, economy, supply chain, or simply the same system. These actors might be or not connected each other with a relationship, a tie, that can be evaluated for its presence (boolean like) or its strenght, if this information is available. The social network analysis uses this nodes-edge representation as the object for a quantitative, statistical analysis, both for describing it, as in the descriptive statistic, or to provide some informations about its dynamics and evlutionary patterns (more of a predictive approach).

In the specific case of a research organization as FBK, this social network analysis is particularly suitable to evaluate the effectiveness of its performance. assuming that the most basic of the two main missions of the foundation is its scientific excellence, and assuming that the institution operate in an open science approach, derives that its performance can be evaluated through the relative status, credibility, opinion that the scientific environment has of FBK.

However, this approach will open a quest for a suitable data source, which can be found among the publications of the EC: the the EU Framework for Research and Technological Development is mainly based on the same open science principles. it is a funding program aimed to support and foster the research activities and quality of the European Research Area, by the provision of grants for research and innovation projects through calls for proposal open and competitive in nature; the FP7 cover the 2007-2013, while the H2020 span between the 2014 and the 2020

the general tendency for the selection process in these calls is to give preference to projects presented by ensembles of research organizations and private firms, usually organized as consortia. The "aggregativeness" of these applications suggests that the total participation of an organization is directly linked to its scientific reputation, which will influence be the willingness of other entities to contact, relate and cooperate with it for the purpose of these research grants. thus, the more and more valuable the projects an institution is part of, the higher its reputation.

assuming that the co-participation to a common project is a favourable condition for the establishment of a continuous and proficient relation, it can be used as a proxy for the presence of a relationships, or tie, leading to a method for building a social network. this will be composed by the individual organizations, tied by the coparticipation to the same project. more specifically, it will be used the fp7 data, freely available in cordis, because of the availability of a largest dataset spanning more years. moreover, will be excluded the projects that have only 1 participant, because they generate no usable ties.

Thererfore, the number of projects used for generating the network ties is 12.126, resulting in 851.655 relationships among 29.717 organizations. 

Once generated this representation of the european research network, the graph can be used to evaluate the relative position and importance of FBK. The literature concerning the usage of SNA tools for the descriptive analysis usually makes use of different standard metrics to evaluate the individual position, among which the most important are the centrality measures; the most commons and simples, Degree centrality, closeness centrality, betweenness centrality and clustering coefficient, will be used in this case.

The most used centrality measure is also the simpliest one: the degree centrality. it represent the connectedness of a node as the amount of relationships it has in the graph; in fact, it is calculated as the number of ties that connect the node to other actors.  

FBK scored 909 ties, filling the 257' position in the entire graph. In other words, FBK is among the top 1\% of actors by the number of ties. This can be considered a remarkable result, given the presence and "competition" of far more larger instutions, as in top-tier academic institutions and monumental public research organizations. 

closeness centrality represent instead the average distance of a node from every other actor in the graph. basically, this index assumes that the higher the centrality of a node, the more will be its direct linkages toward other members, instead of being mediated by other nodes. it is calculated as the reciprocal of the total distance between the node in analysis and every other node in the graph, usually normalized (multiplied) for the total number of nodes -minus one 


\[
	C_C (v) =
		\frac 
			{n-1}
			{
				\sum_t d(v, t)
			} 
\]


The closeness centrality of FBK has been  calculated in 0.41493, ranking 1448, down to the top 5\%. The difference between the degree and the closeness centrality indexes is attributable to the presence in the graph of a number of independent subgraphs, isolated in respect to the other part of the network, which present a relatively higher score in a much smaller environment, slightly affecting the calculation. Nevertheless, it can be considered a positive result.  

betweeness centrality refer to the centrality as the position of the node as a "mediatore" in the graph: the higher the number of shortest paths that cross the node, the higher its index. its determination require, for the node in analysis, to compute every shortest path between every other possible pair of nodes, and for each of them calculate the fraction of shortest paths that pass through that node, then summed up. 

\[
	C_B (v) =
		\sum_{v \neq j \neq k}
			\frac {\sigma_{j,k} (v)}{\sigma_{j,k}}
\]

FBK's betweeness centrality has been evaluated in 1644274, 168" in the graph, in the first percentile. Again, a remarkable result, which place the organization in a truly central position in the graph, where it can assume the role of bridging different realities. 

Eventually, this analysis suggests that FBK have a position of primary importance in the european network of research and innovation, a result that would be hard to achieve without the performance of an "excellent" research, as wanted by the organizational mission. THus, relatively to the period 2007-2013, during the FP7 project which refer the data, it can be stated that the organization has perfomed well and reached its objectives. 

\subsection{Efficiency}

To evaluate the efficiency of a process or an organization, the economic literature usually refer to two different methodologies. The first refer to the tracking over time of the changes in its inputs and outputs, that are the available resources and the resulting products. The second approach require the analyst to compare the organization's performance to the one of other, compatible, instutitons, assessing its efficiency through comparison. Both these methodologies however present some issues.

\subsubsection{Performance comparison over time}

In the first case, the first necessary condition for the usability of this method is the presence of an effective information system, especially for "contabilita analitica", which keeps track of the resources in input and the poducts or services in output. Secondly, to evaluate the very process, it is implicitly assumed that it will not change between periods of evaluation, at least not in an extensive way; otherwise, the analysis will not be able to distinguish between a difference in performance alone and the difference in the context and institutional framework. In the case of FBK, these issues represent clearly recognizable problems with a great impact. 

First, it exist in fact an informative system that could keep track of the performances over time, but it is not able to for a set of reasons. the main issue is to evaluate the amount of resources that end in each process, at an istitutional level: in a tech transfer process, resources that end up in the product include, but are not limited to, the time and effort researchers put into the process; the time of supporting personnel, among which TTO, legal office, administration,and others; materials; machinery time; external services. specifically, the most relevant issue is to keep track of the time researchers and other employees dedicated to the project. -> contabilita dei costi

the most representative example is time of TTO's officers, if the entire office is divided into different units specialized per thematic sets of activities, i.e. marketing and communication, business development, legal counseling, contracts, etc. in this case, projects are concurrent in the time they take to the TTO's employees, being him involved in any project that share the particular activity. At the same time, while impossible to estimate the amount of work hours for the single, identified project, extensive differences among projects make also difficult to evaluate the mean time spent on each project: no average measure to apply to the cost calculation. This issue come up many times during interviews, with no employee being able to estimate the average time spent on each project, with lapses that range betwen hours and days.

moreover, if the evaluation is based on a time serie approach, the danger is to evaluate the impact of external factors that change the process instead of the incremantal endogenous improvement of the process. the point is that the performance itself depend on a serious of factors, both internal and external to the organization. the first case includes the process flows, the quality of involved human resources as in skills and capabilities, financial resources dedicated, a change in inputs, the overall institutional setting and policies. the second case includes governmental policies, changes in markets, industries, technologies, industrial context, national and sovra-national policies. 

The analysis should be based on a model complex enough to include the external factors, to not imputate wrongfully changes to the process or the organization. the model should also distinguish between the change for which the entire organization is responsible, i.e. the research input to the transfer process, the marketing devel program and office etc, from the resposabilities of the TTO and the researchers directly included in the development of the selected technology. with such approach, the administration or who is interested into the evaluation, can discover any enhancement due to the learning process that is supposed to take place. 

However, in the specific case of FBK, the internal accountability seems to not include detailed informations on the amount of resources involved. while the overall costs and resources involved are known, due to the national accountability laws, the actual information system can not in fact connect the single cost to the project. "in the same way, as previously stated, the price estimation for each technology to transfer is assessed on the basis of the presumable costs for matierials, researchers' hours and a flat rate to cover the structure costs"

\subsubsection{Performance comparison between competitor}

the economic literature pointed at the comparison of competitors' performance as an effective method for evaluating the individual performance in absence of the necessary deep knowledge of the cost structure of the original organization. in fact, maybe the process is to complex or contains too many variables to get a relable evaluation, but if another, similar organization perform the same activity, the two can be compared to establish which is the most performant, eventually leading to the discover of potential best practices.

However, this methodology relies on a necessary condition: the availability of a similar organization. While the degree of similarity is somehow a personal evaluation, some condition should be matched nevertheless. the organizations should share the context, or at least be located in similar contexts, includind the degree of technology advancement, r\&d expenditures, academic level/environment, local policies, emplyable workforce, neighbour firms and, to some extent, their level of specialization and economic performance, as well as the relative importance of knowledge for their business (even if it is a requirement needed in this specific case)

similarly, the organizations should share some trait: the scale of operations, that is needed for the level of economies of scale at play; the organizational mission; the main type of activity; the scale of incomes. otherwise, the analysis will highlight which local, institutional and organizational setting can achieve the best performance, rather than which process performs better, as in its structure, personnel, resources, activity etc.

the problem starts with the peculiarities of the italian context in general. the average size of firms is smaller, which impact their innovation capabilities and processes; italian universities perform worse than the EU average ("NOMEAUTORE); the limited number of private research organizations; the tendency of these research organization to be public, as in for local and national government funded and financed. a secondary problem is the trentino context itself, which represent quite an ecception in the italian landscape, for legislative and historical conditions. 

thsu, the availability of a "paragonabile" competitor, in the case of FBK, is more of an issue. in its context, R\&D is performed by the local university, which is not directly comparable due to its main mission imposed by the local governement; and the edmund mach foundation (henceforth, FEM). This last institution, in fact, is the "sister" foundation to FBK. 

similarly to FBK, FEM is a publicly-funded research foundation. it conduct teaching and training activities, having an internal second grade school (istitutio secondario, approfondisci) and doctoral programs. it conduct r\&d activities in high-tech fields characterized by a large rely on codified knowledge, especially in the field of biotechnologies. it performs production activities, through the connected wine "factory", and provides specialized services to the agricoltural industry, both local and not. lastly, it operates technology transfer activities and possess a TTO.

however, differences among these organizations are very extensive. firstly, the size: FEM employ twice the personnel, but only a small part of them are researchers (indicatively, around 150 scientists), while the total income is more the double (90M against 35M). activities are centered over the provision of formation activities and services to the local agricultural sector, as in environmental analysis, study of fito-deseases and their diffusion, genomics. moreover, the FEM organizational structure is far more heavier than the FBK's one: at its institution, the Trento province institution actually grouped a series of different public institutions, maintaining a separate administration for each new established organiational unit. 

lastly, and even more important, the approach through which FEM interpret, concept, and perform technology transfer activities is significantly different. FEM perform some basic research which is hardly applicable to the local industry, thus the necessary shift away of TT activities from the basic research. the applied research, instead, is limited in its extent, thus the tt transfer of newly developed technologies. what's left is the provision of specific services, which might include some research from FEM. the TTO department is actually involved into this kind of service provision.

thus, the extensive differences between FEM and FBK make the former not a suitable organization for a comparative analysis of performances. however, comparable organizations could be found in different contexts. this research should start from the comparison of local innovation contexts and local government policies; if a similar context is individuated, the researcher could investigate the internal actor to this environment seeking a similar organization. 

considering the focus, objective of this thesis, this procedure and other further developments are left for later research. what was significant was to give the reader a perspective on the evaluation of the organizational performance both in a qualitative and quantitative setting. at least, the quantitative evaluation seems useful.

\section{Policies}

FBK address the commercial activities through 5 main policies. while the results of these policies has already been described while analyzing the various internal processes, it could be useful to describe the individual policies. In the first section will be described the major policy, which in fact describes the most the organizational attitude toward commercialization activities; the secon section will describe other 4 minor policies related.

\subsubsection{Policy for the valorization of FBK research}

this is the most generic policy, which describes the attitude and "orientamento" of the research organization. it delineate the general policies for spinoffs, patents and commercial exploitation, while leaving for specific policies the description of the various bureaucratic procedures. the same policy declare the functions and objectives of AIRT, the TTO.

\begin{itemize}

\item AIRT is invested with 2 main tasks: to promote the internal sensibilaztion on the direct commercial exploitatin, especially spinoffs, and to manage the intellectual property, spinoffs and other commercial activities, including negotiation and contractual activities. 

\item Policy for the creation of spinoffs and "compartecipate". The policy describe the ability of FBK to participate in the foundation of spinoffs, in equity, and to sustain the new venture through economic and financial support. the main requirement for spinoff is to have as main objective the valorizaiton of knowhow and research generated in FBK. the proposal for the spinoff project must contain a BP, proposal of the "statuto", "soci" and the composition of the instituitonal organs; at least one board member has to be nominated by fbk. the project is evaluated by the CVI in a first stage, which is composed by HR with a significant experience and knowledge in the knowledge topic and the economic matter; later, the project will be evaluated, thus financed, by the CDA. the limit for the way out form the social capital is 5 years, 3 for other financial aids; FBK can also provide tutoring and other supporting services, but the economic affairs seem to be governed by the market logic: no discount on the licensing of techonologies, or contract research etc. 

\item Exploitation through patenting. This short policy describe the outsourcing of legal affairs surrounding patents to external entities (i.e. patent attorneys). The policy require the first italian deposit as first step on the patenting process, starting a 12month period in which the patent fees and requirements will be sustained by the TTO while the same office, the researchers and other offices will seek for potential licensee or buyers. after 12 months, if nothigh came up, the patent can be sold to the researcher or maintained by the research unit. for licensing, is preferred the non-exlusive license, then the exclusive, then selling.

\item Commercial exploitation of research results. ANother short policy that states that the commercial activities do not refer to research projects funded by institution, i.e. local (pat) or european (i.e. fp7), including  direct research contracts from private firms. thus, for the remaining commercial activities (patenting on internal, independent r\&d) which produce a financial income, the amount will be devided between the researcher, its research unit, and eventually the administration.

\end{itemize}

\subsubsection{Child policies}

\begin{itemize}

\item \textbf{Policy for the management of the intellectual property.} firstly, it describes the willingness to protect and valorize the intellectual property, mainly for the purpose of being as institution a "" for the local economic development. To achieve this objective, researchers are required to disclose to other internal personnel and offices the results of their research, as well as "cedere" the relative intellectual property in its entirety. to further the individual willingness to participate in such projects, the policy "prevede" also a "fair compensation and protection for the author" which basically states that for economic results outside the normal contractual research, results will be aknowledge and fairly compensated. the policy also "decisional" chain: the Center Director, Unit head, AIRT, CVI, including the requirement for them to consider the ideas of the individual researhcers and continue the valorization project only after a market analysis; the TTO can also act proactively, by discovering the market needs and contact the research units/center.

\item \textbf{Procedure for patent exploitaton.} This policy describes as an overview the process for patenting and licensing FBK researches. The policy "contempla" a preliminar phase, in which the TTO will provide training and formation surrounding the major topics surrounding the patent, and esplicitally includes institutional moments for the internal scouting - as in disclosure eliciting - through periodical meetings between research representatives and TTO officers. for the procedure itself, the policy divide the process into: proposal, through a formal comunication that includes the description of the technology, its competitive advantages and the potential market value/impact; the assessment of the proposal, made by the CVI; "istruttoria", a phase in which external organizations will be appointed for the bureaucratic procedure, in which is based anyway on a first standardized step (the first deposit in italy) and economically backed by the TTO, and a contemprary start of valorization activities. again, after 12 months with no interests from external firms, the patent should be transferred to the resarcher or paid by its unit. Lastly, the policy includes the payment of a compensation to involved researchers, at the moment of the financial income.

\item \textbf{Procedure for the creation of startup - spinoffs.} Thsi policy describes the internal procedure for the creation of spinoffs, an organizational objective with its roots in the original statue and mission of further and support new entrepreneurship projects based on internal research. as a preliminar phase, the policy describe the evergreen training and formation; the internal scouting for ideas and projects bettwe suited for this channel; external scouting, meaning the matchmaking of external needs and internal ideas; and the concept, as in design, of the basic ideas as aided by the TTO. the procedure itself includes 7 different stages: a formal communication of intention; a first, quite generic evaluation of the project by the CVI, to assess its potentiality; "istruttoria" which includes the structuring and perfection of a business template, statute for the newco, and other institutional slash legal thingies; then, the formal evaluation, firstly by a delegate of the CDA then by the CDA itself; if things didn't go south, the newco can be activated (step 6) and the formal contractual arrangement between the spinoff and FBK can be made.

\item \textbf{Procedure for the monitoring of "collegate".} The statute of FBK, along with the possibility of spinning off new ventures and participating other external economic realities, require these external organizations to periodically report their situation and results to the parent organization. the policy describe two different reports: a qualitative one, to present every two months, which should include the analysis of the ongoing performance with the forecast presented into the BP; and a quantitative, semestral communication that should include patrimonial, economic and financial informations.

\end{itemize} 