% Chapter 1

\Chapter{The University Perspective}{Institutions, academics, and TTOs} % Main chapter title

\label{Chapter2} % For referencing the chapter elsewhere, use \ref{Chapter1} 

The academic environment and the university organizational structure are the most influencing factors in determining the modalities through which the technology transfer takes place. The environment shapes the individual attitude toward commercial activities, while the organizational structure provides instruments and paths for the development of commercial ideas into successful endeavors. Moreover, these two factors constitute the strongest difference between public universities and private research organizations. In this chapter, firstly will be presented the culture that marks these institutions, and the individual perspective of academic researchers. Later, will be described the role of TTOs and faculties in the development process and the individual attitude toward technology transfer. The difference between the public and private environments and structures will be analyzed in \hyperref[Chapter6]{Chapter 6}.

\section{University as institution}

The first university was founded in 1088, but the \enquote{scientific} approach and role that is usually acknowledged them was shaped only in the 17th century. The \enquote{open science}, that has largely characterized their activity until now, is a \enquote{widespread system of exchange based on the value of scientific priority and prestige} \citep{Murray2005}. It relies on a series of norms that facilitate full disclosure and diffusion of research results, where a set of economic incentives push toward a cumulative knowledge production, i.e.\ public research funding that recognize and reward the priority of the research outcome.

Lately, in the previous century, a new system has arisen oppositely to the open science: the \enquote{patent system}, also called the \enquote{economy of inventions}. In this regime, property rights on scientific discoveries ensure the potential stream of financial rewards, based on the ability to exclude others from the appropriation of newly created value and the compensation for commercial exploitation in exchange for the disclosure. 

These two models clearly collide, a conflict that is enforced by the increasing focus of universities and public policies on commercialization activities, challenging the culture of open science. In fact, academic scientists' fundamental devotion to the open science paradigm shaped a third alternative to these systems, a hybrid economy with mixed elements from both parties. In this perspective, both firms and academic inventors use patents to protect and exchange new knowledge – even if it is believed to be \enquote{less efficient than the previous reliance on 'pure science'} \citep{Geuna2009}. Similarly, \citet{OwenSmith2001} found a convergence toward a hybrid system of scientific and technological success.

An example is the phenomena of \enquote{patent-paper pairs}: starting from the same knowledge, a patent will be firstly applied for, then a paper will be published after the delay required by the patent process \citep{Murray2005}. In this model, both objectives will be satisfied: the publication, for academic purpose, and the IP protection, for commercialization. In fact, even scientists themselves are interested in protecting their ideas, since patents have become major bargain chips, a currency in the knowledge economy.

A different terminology is used by \citet{Stern2004} who referred to a \enquote{science approach} quite similar to the open science approach. He defined it also (and most interestingly) through the peculiar reward system based on the scientific priority, which drives researchers to publicize their findings as quickly as possible, without retaining any right on the intellectual property. 

There is a widespread debate on which paradigm should be adopted by publicly-funded institutions like universities, especially when considering the type of knowledge involved in these different approaches: basic science and applied research. Many organizations mixed these elements, but the tension between them lead to wide differences in objectives and administration, including significant difficulties in managing the conflicting goals of curiosity-driven research and commercialization activities, left alone the various shades between these extreme points \citep{Rasmussen2006}.

A first argument could be the one of \citet{Geuna2009}, in favor to the new commercialization model, who stated that the mission of creating and disseminating \enquote{knowledge for its own sake} could drive university scientists away from real-world, practical problems that need to be solved. \citet{Rosenberg1994} and \citet{Nelson1998} also observed that the large part of disciplines that take part in the academic curricula was developed specifically to meet requirements of firms. \citet{Balconi2006} noted that universities must be \enquote{intimately familiar} with industrial technology, to identify and perform useful research. Moreover, the technology transfer is a two-way knowledge flow, where academics can benefit too.

A moderate approach has been taken from \citet{Beath2000} who reported a trade off in the effort an academic scientist can devote to the creation and dissemination of knowledge and its commercialization, starting from a limited amount of time and energy. Therefore, the author seems to justify these last activities only if they have an actual financial impact on the university budget. He argued that due to its public good nature \citep{Muscio2013}, academic knowledge should be treated carefully: he referred to the case of the pharmaceutical industry, even if this topic relates more to the question of patent legitimacy in \enquote{greater good} matters.

\citet{Siegel2003a} instead defended the open science system. Firstly, he reported several interviews where the commercialization mission was seen as inconsistent, with the traditional public domain philosophy that should be endorsed by these institutions. Then, they referred to a trade-off from \citet{Nelson2001} where \enquote{the shift away from open science might slow down technological diffusion}. Also \citet{Stern2004} stated that the priority system embedded into the open science approach seems relatively efficient, by discouraging shirking and encourages maximal knowledge diffusion.

\citet{Rasmussen2006} reported various changes in the government's control and administration of universities, nowadays widely aiming to greater autonomy and competitiveness through a performance-based funding. However, even if the shift itself has no positive or negative connotation, they seem to agree with others, arguing that academic freedom and basicness of university research may be treated by commercial activities. Others exposed their concern about the shorter time horizon, various tensions, and conflict of interests arising from a more applied research. 

\citet{Murray2005} took a case study as the exemplification of the real attitude and reaction of academic and industrial scientists to the new patent system: the Oncomouse case. Briefly, a company funded and patented a research that eventually led to one of the most recognized innovation in its field. However, the firm imposes strong limitations in the acquisition and usage of oncomice, the informal exchange of the related knowledge, and the appropriability of \enquote{reach-through} discoveries, leading to \enquote{a widespread infringement} from scientists. 

Similarly, \citet{Stern2004} empirically investigated the scientists' willingness to perform basic or applied research within these different systems, to get a better understanding of their actual preferences. He studied the relationship between wages and the scientific orientation of R\&D organizations, founding a significant and positive correlation between the closeness of the employers' to the science approach and the premium they must pay.

Elsewhere, other authors criticized more directly the way universities are currently interpreting the phenomenon. \citet{Siegel2003a} considered the academic management of IP rights too aggressive and reported a conflict between the overall willingness to commercialize and tenure and promotion policies, which promote scientific publications and grants. \citet{Murray2005} instead observed that a broad protection of intellectual property requires the negotiation of every exchange relationship, slowing down the transfer process.

Historically, the exploitation of research results has not been the first concern for most academic institutions. As reported by \citet{Balconi2006}, in the previous century the technology brought to the market was merely an application of previously and independently developed scientific knowledge. However, since the early 1980s, universities have witnessed the emergence and consolidation of a \enquote{third mission}, alongside the research and dissemination objectives. This further aim requires them to bring research results into the market, both for fundraising purposes (even if it is usually pointed as the main reason) as well as the economic development, technical advancement, and wellbeing. 

This setting led to the development of the concept of \enquote{entrepreneurial university}. The term was originally coined by \citet{Etzkowitz1998} to describe institutions that have proven themselves critical to the economic development of their region. His idea was based on the academic entrepreneurship, defined in \citet{Louis1989} as \enquote{the attempt to increase individual or institutional profits, influence, or prestige through the development and marketing of research ideas or research based products}.

In this new economy landscape, this knowledge-based entrepreneurship has become one of the driving force of economic growth. Entrepreneurial universities act as a knowledge producer and disseminator, but also as a natural incubator where scientists and entrepreneur can explore, assess and exploit ideas, through the provision of an \enquote{adequate atmosphere} and a network of relationships necessary to the entrepreneurial activity \citep{Guerrero2014}.

To better understand this approach, it can be compared to the \enquote{research university}, whose primary purposes are to (1) conduct research and (2) train graduate students to perform research. While these institutions have been considered more effective in the knowledge transfer, relatively to federal laboratories \citep{Rogers2001}, they still lack the acknowledgement of the growing role of know-how and technology advancement in this era, jeopardizing their opportunity to become fundamental actors in the economic development of their regions \citep{OShea2004}.

Nowadays, the importance of entrepreneurial universities can be evaluated through the extent of actions taken by governments and other institutions in administrating them, actively fostering the direct engage in technology transfer activities, i.e.\ contract and collaborative research. At the same time, it can be seen as the recognition of the significant endowment of knowledge-based assets owned by universities, that is believed to be economically underexploited \citep{Tijssen2006}. 

\citet{Balconi2006} seen this recognition also in the trend of putting \enquote{institution of science} in charge for riskier explorations: their though was that the \enquote{ultimate sense} of these organizations should be to set the foundations for new developments. It is implicit that the necessary condition is for universities to possess the necessary knowledge and organizational climate.

\subsection{The new millennia}

What changed over time in a more operative perspective? Early in the days, when the third mission was not so firmly established, the technology transfer heavily relied on personal relationships of academics, and only a few institutions have units or offices to support these activities, usually with few non-specialized employees. New technologies for the market was developed prevalently in large companies' laboratories, while small and medium enterprises have other choices, i.e.\ cooperative research centers. What \citet{Geuna2009} recognized later, instead, was the broad trend to professionalize the technology transfer and the third mission of universities, whereas industrial R\&D was getting more and more directed toward commercial ends \citep{Fritsch2007}.

What they acknowledged as the source of this discontinuity was a change in the context, especially in firms' involvement. Specifically, the use of knowledge as a competitive advantage; the increasing demand for skilled employees; the availability of higher educational levels; the usage of universities as policy tools for economic development by public institutions; greater budget constraints and diminishing incomes from governments. Another perspective refers to the \enquote{scientification} of the industrial production process, in which universities and public research organizations represent the main knowledge producers, fundamentals to the economic development \citep{Balderi2007}. Lastly, already \citet{Thursby2002} found evidence to further support the importance of changes in faculty orientation toward commercialization and collaboration, in addition to a shift in industry R\&D.

\citet{Baldini2007} reinforced this view stating some non-obvious factors that have driven to a greater involvement of universities. First, innovative technologies reduce the need for concentrating research activities in large facilities that can afford expensive machinery, allowing smaller centers to be competitive in performance and effectiveness. Secondly, the policy shift in the allocation of intellectual property rights acts as a motivation for researchers to get more involved. Third, they observed how these factors, unitedly to the declining amount of public funds, lead to increasing competition for any source of funding. In this perspective, the technology transfer process has become of \enquote{vital importance} for universities \citep{Muscio2008}; however, its use as a funding source is a secondary objective \citep{Jensen1998}.

A relevant non-financial aim, for example, can be taken from \citet{Tijssen2006}: the willingness to achieve and maintain the leadership in scientific-related, high-tech fields. \citet{Leitch2005}, instead, cited as a driver the desire to develop, retain and acquire skilled graduates. \citet{Baldini2006} reported the importance of technology transfer to reinforce the university reputation. 

In this setting, \citet{Wong2010} investigated the impact of technological commercialization on publication activities: they found a significant and positive correlation between patent applications and publication output, suggesting a mutually reinforcing relationship. \citet{Lee2000} listed other potential individual motivations: to test hypotheses, gain insights and knowledge on practical problems, for teaching purpose, students internships and placement, to look for business opportunities. 

Closing the stakeholder perspective, governments use universities to accomplish various tasks, especially in referring to the regional innovation system framework \citep{Fritsch2007,Balderi2007}. Examples are: to generate, accumulate and transfer knowledge to local actors, i.e.\ through R\&D cooperation and innovation-related services; develop the local labor force; strength the absorptive capacity of the region and firms; incubate new technology-based companies. \citet{AzagraCaro2010} specifically cited as advantages, among these, the develop of new technical instruments and methodologies, the enhancement of problem-solving capabilities in the local context, the formation and stimulation of networking activities and social knowledge. Another reported advantage is the spin-off formation, a highly prioritized area for governments.

In fact, universities have a \enquote{special attention} for topics of national interest: \citet{Rasmussen2006} observed that they \enquote{seem quite eager to satisfy public expectations, which in turn generates goodwill from research councils and ministries}, referring to the issue of decreasing funding. Moreover, it exposes the need to show actual, short term results both in traditional missions (publications and students) and the direct contribution to the national economy, as described in \hyperref[Chapter5]{Chapter 5}.

These forces have led to a genuine change in the attitude of universities; \citet{Tijssen2006} described this shift through 3 different phases of development. The first is the \enquote{application-oriented/science-driven} approach, that coincides with the rising awareness of a possible link between universities and companies for the resolution of industry-related problems. Second is the \enquote{product-oriented/utility-driven} approach, which consists in the emergence of activities such as prototyping, development services, etc. specifically aimed to the discovery and exploitation of opportunities. Third and last is the \enquote{business oriented - market driven}, where IP rights are secured, relations with firms are established, and first products are sold.

In this perspective, great distance has been put between the traditional Ivory Tower image - usually associated with the 80s and 90s typical university - and the actual industry-friendly university involved in the solution of real-world problems \citep{Baldini2006}. However, must be mentioned \citet{Rasmussen2006}, whose empirical findings still indicated \enquote{soft emphasis on commercialization}.

What this historical perspective cannot illustrate are the challenges that universities are facing in this new system; elsewhere, instead, \citet{Rasmussen2006} divided them into three different groups. First, to increase the extent of commercialization activities and outcomes, to produce a greater impact on the innovation system and gain a better financial perform. Secondly, to visualize their contribution to the economic development, both at a local and a national level, to justify the university's funding requirements with the economic role and social impact. Lastly, to overcome the various difficulties in the management of the new fundamental activities: teaching, the exogenous (as driven by curiosity) and endogenous (market-driven) research \citep{Debackere2005}. 

Summing up, universities must cope with multiple, considerably different mission required by various stakeholders for contrasting purposes. What is needed, to overcome these challenges, is ambidexterity: nowadays, they should not draw a clear line between exploration and exploitation activities. Instead, they should manage their relative tension, integrate them and take advantage of every potential synergy.

\citet{Chang2016} developed this perspective by suggesting two levels, or types, of ambidexterity. The first is the top-down, \enquote{structural} ambidexterity, which uses organizational tools to promote synergy exploitation, such as task partitioning, separation of tasks and units, an appropriate leadership. The second is the bottom-up, \enquote{contextual} approach, that aims to develop an appropriate organizational climate to encourage individuals in directly managing the conflict.

Referring to this ambidexterity capability, many authors assumed a historical perspective to identify, and evaluate, the most common constraints for its development: the very scientific method of universities. These institutions have developed over time a series of values, beliefs, norms of conduct and organizational practices devoted to the open science paradigm that can actually impede an effective push toward commercialization activities. 

\citet{Argyres1998} referred to this problem as the \enquote{social-contractual commitment of universities to the intellectual commons}, which constrains their ability to exploit economic opportunities. \citet{Muscio2013} instead observed the tension between two extreme points of a continuum: the ideal norms of open science and the Mertonian ethos, which have long driven activities and evolution of the academia (the modern commercial approach), filled with business values.

In this perspective, every university has its peculiar historical path. In fact, many authors found them to be one of the most relevant factors in determining the technology transfer performance. As an example, \citet{OShea2005} identified the \enquote{uniqueness of historical conditions} as the basis for a sustained competitive advantage, especially when evaluating spin-off performances. 

Empirical research support this point, confirming the greater importance of the previous performance in technology transfer in promoting engagement. Another relevant research on this topic is the one of \citet{Baldini2006}, which linked the behavior of potential inventors and entrepreneurs to the historical performance, observing that processes as patenting can be \enquote{cumulative} in their performance, due to learning effects. Finally, past competitive performance can reinforce the institutional (and government) attention to the commercialization activities, further pushing them.

However, has to be remembered that there is also empirical evidence to discredit this correlation, at least in particular perspectives. An example is the results from \citet{Thursby2002} which suggest a little effect of past (licensing) performance on the propensity of the university administration in further promoting (patenting) activities.

%----------------------------------------------------------------------------------------

\subsection{Performance and attitude}

Other factors make universities differ in the degree they engage with industry, and many authors have built formal models to uncover which of them are significant and evaluate their impact. While the econometric results will be reported later in \hyperref[Chapter4]{Chapter 4}, it is relevant to introduce now which academic characteristics play a fundamental role in determining the technology transfer performance.

As a starting point, the most recognized factor that influences the academic inclination to engage in such activities is its commercial orientation \citep{DEste2007}, or the \enquote{catch-all phrase entrepreneurial culture} in \citet{OwenSmith2001}. It is intended as the sum of its cultural legacy, its history and past performance, and to what extent these factors influence the nature of the academic research. In fact, the commercial orientation in its definition is so close to the willingness to engage in technology transfer, that it can be interpreted both as the dependent and independent variable. However, it is mainly recognized as an enabling factor that sets a fertile ground for commercial and business activities. 

Strictly related to this attitude, are a series of factors that refers the supportiveness of the university environment. Examples are various policies for academic entrepreneurship: leave of absence, access to laboratories and other infrastructures, entrepreneurial programs, senior members championing for entrepreneurial projects, on-campus incubators, investments in startups, earlier and easier access to venture capital \citep{Baldini2007}. \citet{Guerrero2014} used similar variables in his model, such as entrepreneurship education programs, the attitudes toward entrepreneurship of key actors in the hierarchy, a strong top-down oriented leadership, the availability of an entrepreneurial model.

Similar, but loosely coupled factors are variables linked to the perception of the university orientation, rather than influencing the context themselves as the environment supportiveness. An example is the contribution of \citet{DEste2007}, on whether the founding mission specifically includes the university support to regional development; in fact, they found a significant and positive relationship with commercial engagement.

The literature has considered other university-level factors that refer to the actual, processual capacity to support the technology transfer process, rather than its ability to motivates academics and grow a supportive environment. Examples are the size of the research staff, the quality of the research production, a sizeable budget for entrepreneurial-related activities \citep{Colombo2010}, the university own financial, technological and social capital, the institution status and prestige \citep{Guerrero2014}.

Among these, many researchers aimed to uncover the influence of governance and organizational structures on the engagement probability, which has been found to have a significant impact on the ability to manage internal conflict between missions, thus on the overall performance. \citet{Siegel2003a} found in their performance analysis a deviation from the best performance that cannot be entirely explained by environmental and institutional factors, which should be attributed to organizational factors. Specifically, two level of organizational configurations has been studied.

The first is the internal organization of academics and research units. A specific structure that has been found improving the commercialization performance is the so-called research division \citep{VanLooy2004}. In it, research personnel can decide to organize themselves and their research, alongside their contract research and other exploitation activities, in autonomous groups called research divisions, usually professionalized on a single topic or a particular industry. The performance of academics involved in such divisions has been found to be superior, and the gap widens over time.

The latter configuration refers to the presence, position, and role of a dedicated office to technology transfer, henceforth TTO (Technology Transfer Office), which effectiveness has been found to depend on organizational practices \citep{Siegel2003a}. Generically, the institution of dedicated units and the importance vested in them are found to be relevant; similar internal offices and external organizations are the Technology Licensing Offices, Industry Liaison Office, incubators, science parks, joint ventures, spin-offs \citep{Tijssen2006}. More information on the topic will be provided later.

A related topic that has gained an increasing attention refer to policies for the evaluation of academic researchers as personnel. Specifically, the reward systems, both monetary and not \citep{Guerrero2014}, criteria for career advancement, tenure policies, royalties allocation and alike. As a general rule, incentive policies should be promoted to raise awareness around the benefit in protecting and exploiting intellectual properties, to ease the perception of the patent process and increase the esteem that scientists have of TTO's competence. This either will be discussed later, but it should be recognized from the beginning that a major issue is whether researchers have sufficient incentives to disclose \citep{Debackere2005}.

Lastly, local and national contexts in which universities are located play a relevant role too. Examples are the relationship between annual public expenditure in university R\&D \citep{OShea2005}, the national economic growth \citep{Siegel2003a}, national culture and academic socialization \citep{Bercovitz2006}, the extent of R\&D activities of local firms \citep{Siegel2003a}, and the strength of ties between universities and industries developed in past projects (Powers, 2005). 

\section{Academic scientists}

The development of new innovative knowledge and cutting-edge technology critically depend on the quality of human resources involved. Star scientists and academics from top-tier universities can reach discoveries faster, through the easier access to the personnel with the needed specializations rather than the necessary technical resources \citep{OShea2005}. Apart from the ability to reach innovations, academic inventors can also play a positive role in commercialization activities, i.e.\ providing surrounding knowledge or leading themselves the process through a spin-off.

However, scientists differ in their willingness to get involved in technology transfer. Various research has been performed to identify which factors can foster academics involvement, and what are their basic characteristics and traits. The basic framework for an appropriate categorization and understanding of these differences come from \citet{Stokes1997}. He divided academics into three different categories, according to their interests and orientations. 

The first is what he called the Edison scientist: an academic who is interested only in purely applied research, oriented to the development of new products to meet people's needs. At the other end, there is the Bohr scientist that was intended to be a pure, basic scientist who most strictly adheres to the open science approach. Between these extremes, Strokes identified the Pasteur scientists as the most relevant for the technology transfer process, whose main trait is the desire to advance scientific knowledge, but only in fields or applications that can have real-world, useful applications. Ideally, this academic employee authored many high-quality scientific papers, as well as applied for many patents; he should be a recognized expert in its fields, and both inclined to develop inventions and to gain a strong reputation in the scientific community through research activities \citep{Baba2009}.

Later studies use a binary categorization, useful even if simpler, in which scientific occupations have been divided into two distinct types \citep{Beath2000}. On the one hand, university scientists perform fundamental and basic research, motivated by the desire and the reward of establishing their priority on the discovery. On the other, private sector researchers who have no opportunity for fundamental research; they are financially motivated in keeping up with the scientific literature and transfer it into marketable products. 

Both these perspectives match what previously was referred to as the difference between the open science approach and the IP protection system. Bohr scientists, \enquote{university} scientists \citep{Beath2000}, and partially Pasteur scientists share a common orientation toward the open science; Edison and again Pasteur scientists tend to commercialization activities. Therefore the peculiar and extremely relevant role of the Pasteur scientist, who is in the position to bridge interests, knowledge, and activities.

Specific to the dual perspective is the fundamental study of \citet{Stern2004}, who investigated the impact of the organizational orientation on scientists' wages, comparing public research organizations and private firms. He started from two different assumptions: researchers may have an intrinsic preference for the open science approach, while firms may have economic interests and benefit from participating to the same paradigm. He based his research on the tightness of control over IP, non-disclosure clause and related variables.

His empirical results shown that once controlled for the relative ability of researchers, there is an actual premium wage paid by \enquote{closed} firms devoted to the protection of ideas, and that it is considerable in its extent, evaluated in a difference of about 27\% in wages. In fact, scientists pay a compensating differential for participating in the open science system. 

What seems to motivate and drive the academics in this perspective is their unique sensitivity to reputational awards: an increase in reputation among colleagues, in an open science system, open the path to better research projects, teams, and memberships, ultimately leading to employments in more prestigious institutions and access to relevant resources. This effect, as in other examples of the Matthew effect, seems to increase over time.

\citet{OwenSmith2001} recognized in their work this difference in a slightly different perspective, addressing two type of scientists from two main areas: life science and physical scientists. These labels might correspond to basic and private researchers, respectively. The authors found that both types of scientists express concerns for various constraints due to commercial exploitations, and they still may engage with a similar probability but for very different objectives. The private scientist (including commercial-oriented academic) seek IP protection mainly to pursue commercialization activities while maintaining the original, unexploited value of the new technology. The latter category, representative of a purer researcher, may also seek IP protection, but with the very different intent of \enquote{shielding the environment of his lab from encroachment by commercial interests} and ultimately as leverage to attract investments.

The open science approach, the most diffused today, shapes various factors that influence the academic willingness to engage in commercial activities. The main group of variables refers to career concerns: the system heavily relies on reputational awards, based on priority, novelty, and quality of the research, rewarding scientists with a greater access to resources and better personal networks. This drive scientists to construct their research agenda freely from external influences that might carry unwanted constraints in time and effort. It is partially due to the \enquote{ambiguous relationship of researchers to money} \citep{OShea2004}, that refer to the disinterested nature of university research and the seek for public funding that enables academics and teams to work completely unrelated and free from industry-related constrains. 

A well-recognized factor related to this perspective is the unwillingness to delay publication of many researchers \citep{Thursby2002, OShea2004, Baldini2007}. The primary driver is the \enquote{publish or perish} rule, which requires academics to continuously publish in order to gain, and not lose, reputation. These delays may arise from: bureaucratic issues, i.e.\ time and effort required in disclosing the invention to the TTO; commercial issues, i.e.\ establishing a relation and to reach an agreement with firms; legal issues, i.e.\ the time required by the patenting process. Moreover, agreements with firms may include secrecy or a fixed-length publishing delay to allow the industrial counterpart to gain a first mover advantage; these requirements clearly threat the publication, but also networking activities. 

These factors ultimately drive faculty member to not disclose internally their invention, thus not engaging technology transfer, because of \enquote{their believe that commercial activity is not appropriate for academic scientists} \citep{Bercovitz2006}, referring to the norms of open academic science.  In this perspective, the university administration acquires a relevant role: empirical findings suggest that a relevant part of the faculty resistance in disclosing is due to university-level policies focused on scholarly works, while commercial activities are believed to be perceived as mere, non-relevant \enquote{services} \citep{Markman2005}. Examples are factors considered for tenure and promotion decision.

\citet{Muscio2013} expressed this duality of imperatives, the open science paradigm, and the commercialization need, as a cognitive dissonance that academic scientists may directly experience when trying to reconcile these conflicting aims and their own research agenda. They noted that even if the cultural environment and the organizational context do matter, scientists' decision on which research to perform may largely depend on their personal propensity toward a particular approach.

These factors may keep scientists from commercialization activities, but there are actual benefits that should positively influence the willingness to engage. In fact, even the impact of commercial activities on publication has been largely discussed in the literature. A fundamental study on this topic is the one from \citet{Lee2005}, which showed that researchers who collaborate with firms are generally more productive in terms of publication, and the quality of journals not diminished. \citet{Lebeau2008} found empirical support for the hypothesis that industry collaboration increases academics' citation indexes when normalized by field, while at first glance the average relative impact factor could be diminished. Oppositely to Lee and Bozeman, however, they found that journals published in have a lower impact factor, while articles receive higher citation counts.

Many authors agree on this view. \citet{Meyer2006} found that academic inventors tend to perform better in publications than pure academic scientists; \citet{Wong2010} found a positive relationship between research performances and patenting, used as a proxy of commercial activities. Elsewhere has been found a positive relationship between the quality of research, measured through publications and citations, and the extent of collaboration with industries \citep{AzagraCaro2010} or the patent productivity both at individual and university levels \citep{Baldini2007}.

How to reconcile these views? On the one hand, academics are afraid of constraints that commercial activities may impose, and the negative impact that could have on their career; these worries may arise from inadequate information and a scarce awareness. On the other hand, industry collaborations do have a positive impact on research activities and academic careers, even when it implicates some limits.

\subsection{Individual characteristics}

As previously stated, the open science culture may have a negative effect on scientists' engagement with industry; however, many factors have been found to influence their willingness. Apart from the institutional impact presented earlier, the literature analyzed many individual characteristics, while seeking for the determinants of researchers' entrepreneurship.

The first is a sort of cliché, which suggests that in order to have a good performance in commercialization activities, the academic should be \enquote{the entrepreneurial type}, identified as \enquote{who have always wanted to start companies and who use their university inventions as a way of achieving their entrepreneurial goals} \citep{Fini2009}. In fact, \citet{DEste2007} found that the most significant individual characteristic is the previous experience in collaborative research – a proxy for past behaviors – suggesting the presence of a Matthew effect and thus the importance of the initial attitude toward industry.

Individual attributes usually associated with the academic entrepreneurship are: outgoing, extrovert personalities; a strong need for achievement; the desire for research independence \citep{OShea2004}; tenure and occupational skill level \citep{Roberts1991}; age and scientific experience \citep{Audretsch2000}; higher citation rates and publication performance \citep{Zucker2001}.

Elsewhere, academic life cycle models suggest that academics are more prone to technology transfer activities later in careers: in the early stages, academics work on increasing their reputation and building more extensive and valuable social networks; later, once reached a satisfying level, they try to build legitimacy for their invention \citet{Fini2009}. However, empirical findings from \citet{DEste2007} suggest a higher probability of engagement for younger academics, possibly due to the minor devotion to the open science system and a different need for network development \citep{Bercovitz2006}.

Obviously, an extremely relevant factor is the environment researchers work in; as previously noted, a supportive university organization and culture is fundamental to improve the commercial involvement. As an example, \citet{OShea2004} found that the decision to start a spin-off is socially conditioned by the consensus that commercialization activities gain and the relative behavior of colleagues, as proxies for social norms, expectations, and the presence of former academic entrepreneurs. 

\subsection{Academic motivations}

These characteristics describe what empirical research have found as a typical academic scientist prone to technology transfer activities. But what motivates them? 

Many authors agree that the primary motive for engaging with firms is not the entrepreneurial attitude, as might appear from the individual-level characteristics stated before. Instead, the main driver should be the expectation of a positive influence on their academic position and their research \citep{Fini2009, DEste2011}, according to the large adhesion to the open science approach.

Factors that directly influence individuals can be the gain in visibility, network development, prestige and reputation, recognition by peers \citep{Baldini2007, Fini2009, Rizzo2015}. In fact, a fundamental motivation for the academic scientists seems to be the recognition of the community and the reputation gain: collaborating with industry can help in publishing in top-tier journals, conferences, federal research grants.

As widely recognized, technology transfer positively influences the research performance through the bi-directionality of the information flow \citep{Geuna2009}, even if often neglected \citep{DEste2007}. Examples are the access to industry skills and facilities, opportunity for hypothesis testing, verifying the applicability of the research, keeping up with industry problems; new stimuli and new topics, etc. Generically, these reasons refer to the opportunity to interact with the external, real world \citep{Baldini2007}. 

Other motivations relate to the environment. Leadership, for example, can be effective as guidance, as in altering what is perceived as socially desired through its actions. The cohort effect \citep{Bercovitz2006} refers instead to the presence of a positive, previous experience of those in the same position, that act as a \enquote{case study} and affecting the willingness to engage. The same authors found relevant a previous experience, especially a formal academic training (i.e.\ the Ph.D. itself) in an institution which presents a positive performance in technology transfer. Similarly, \citet{Murray2004} and \citet{Link2007} demonstrated the importance of social capital in enhancing the probability of successful cooperation with industry, therefore the importance of a supportive social network surrounding the scientist.

Different are the motivations that work at the individual level but refer to the overall university. Cooperating with industry can bring more research funds, laboratory equipment, federal and European funds and grants, even attract star scientists to the institute \citep{OShea2004, Baldini2007, DEste2007}. However, the effectiveness of these motivations largely depends on the university reward system, i.e.\ tenure and promotion policies and royalty distribution formula; if not managed properly, this system could lead to an institutional conflict between basic and applied research. 

This conflict is greater when disclosure is elicited and researchers are driven by exogenous motivations, rather than endogenous and indirect ones. Many authors addressed the issue by modeling the technology transfer process as a game between academics, TTOs, university administration, and firms, and found a significant moral hazard problem for inventors \citep{Jensen1998}. Their effort cannot be adequately monitored or enforced, which means that motivations for scientists should come from fine-tuned policies and the overall environment, rather than imposed by some internal regulation or by the hand of TTOs or administrations. 

In this perspective, compensation forms differ in extent and direction of their impact on researchers' motivations; clearly, the preferred form is the sponsored research, which allows scientists to continue work as they prefer, i.e.\ in-house with their own teams, on self-selected topics. Another suggested compensation form is the equity share, which \citet{Jensen1998} found to be less distorting on firms' and academics' decisions. Further attention should be placed on the choice of the form, especially when it comes to the various, unwanted effect of each channel and the conflict between them. An example from \citet{OShea2004} is that higher royalties allocated to academics will decrease the spin-off activity, by modifying opportunity costs, their ratio, thus their appetibility.

Along with compensation forms, it should be noted that even if a primary objective of academics is the recognition within their community, i.e.\ through papers, presentations, conference and research grants, they are also motivated by personal financial reward and additional funding for their research \citep{Siegel2003a, Link2007, Fini2009}. \citet{DEste2007, DEste2011} agreed on the point, stating that fundraising performance can act as a signaling mechanism, positively affecting the reputation. \citet{Rizzo2015} added two financial motives specifics to the spin-off path: funding for research and tax avoidance. Greater financial incomes for the researcher or his unit may also lead to increased independence in research lines.

In fact, \citet{DEste2011} found a positive attitude of researchers to financial ties with industry (74.5\% of interviewees), as long as funds are related to their research topic of choice and the open science paradigm is respected, i.e.\ when disclosure is agreed upfront and ideas are freely publicized. This study also confirms the priority of research-related motivations, thus the importance of a compatibility between financial and scientific rewards \citep{Baldini2007, Link2007}, and the need for a match between personal ambitions and the financial and business opportunity \citep{Tijssen2006}.

Finally, even if universities are seen as \enquote{professional bureaucracies whose members are relatively free to pursue activities that they believe are in the overall interests of the organization} \citep{DEste2011}, there is a clear need for pushing inventors to actively engage in the commercialization process \citep{Jensen1998}. Apart from the intimate knowledge they can provide on the discovery and the surrounding scientific context, their passion in their work can make them assume the role of project champions, as well as helping TTOs in identifying and contacting companies, ultimately maintaining the relationship among the institution and the firm \citep{Markman2005}.

Summing up, similarly to the overal organization, individual academics should demonstrate ambidexterity, as the attitude to simultaneously achieve publication and commercialization goals, starting with the ability to recognize exploitation opportunity from their research results \citep{Chang2016}. Individual motivation seems fundamental to accomplish both academic and entrepreneurial results, led by the comprehension of synergies and the mutuality of benefits for academics and firms. Similarly important appears to be the environment, which should balance the relative emphasis of activities and provide social support. 

\section{Technology Transfer Offices}

Intermediaries between suppliers and users of knowledge progressively emerged as central in bringing academic research to market \citep{Landry2013}. Universities and firms belong to different communities and react to different incentives, establishing a social and cognitive distance that requires a specialized intermediator that can comprehend and effectively relates with both sides. The generalized acknowledgment and recognition of this problem led to a recent and substantial increase in investments, both from universities, firms, and governments \citep{Muscio2010}.

The primary source of their importance is the potential market failure involved in the transaction of newly developed technologies and knowledge: firstly, it is hard to identify which of the disclosed invention is actually marketable, due to the intrinsic innovativeness of the research output. The fundamental question, at this stage, is whether or not the idea has enough appeal for the market.

Secondly, if the invention matches a potential market, there is no guaranty that the knowledge or the idea will grow in a functional prototype, due to high uncertainties embedded in the research and development process; even if this is available, it may not satisfy the industry requirement for a profitable production. In this case, the question to answer regard the feasibility of the project.

After these first, necessary leaps, the technology can be transferred to the industry. Two main issues may arise in this phase: to identify a potential, interested partner, and to establish a value for the transaction. Peculiar to the second, is the problem of estimating a new, unknown technology, which involves uncertainties and opportunisms and may lead to true market failures, i.e.\ great differences between social values and contractual prices, investments that fail to take places, failures in starting the transfer process.

These issues arise from the general problem of exchange and communication between research organizations and industry exponents, and a specialized organization can undertake any of these matters. However, the literature has mainly seen the resolution of the first two issues as \enquote{second order} topics, in researching the impact of intermediaries in the latest problem: the focus is on their ability in, and the positive effect of, reconciling university and industry perspective, and driving them to a mutually beneficial agreement. The implicit hypothesis is that the same factors that may foster the competitive performance on this topic, will be the same that allows the overcoming of all issues.

The particular interest of the literature arises from the recognition that the main and greater market failure come from the difficulties in, thus the imperfection of, estimating technology \citep{Hoppe2005}. This is mainly due to a problem arising from the asymmetric information: firms usually cannot assess a priori the invention potential, while researchers have difficulties in identifying exploitation opportunities and evaluating their potential \citep{Debackere2005}. 

In fact, researchers have the best knowledge of the technology, and firms about the market; both are required, to a successful and profitable transfer, but their match is constrained by limits typical of the agency theory. Similarly, \citet{Bercovitz2006} observed that differences in estimation might arise from subjective expectations on the knowledge value, while \citet{Hoppe2005} cited the uncertainty about the technology profitability. \citet{Debackere2005} further identified the source of these constraints in the high-uncertain and non-codifiable nature of the scientific knowledge.

Within this perspective, intermediaries can enhance the overall system performance by reducing the asymmetric information issue: their professionalization, therefore their expertise in successfully locating and screening new ideas, can actively reduce the risks perceived by firms while providing indications to scientists on which idea develop \citep{Debackere2005}. Moreover, intermediaries can balance the low bargaining power of individual researchers \citep{Bercovitz2006}, and overcome the opportunism problem.

To uncover this last benefit, some researchers modeled the activity of intermediaries (mostly TTOs) as repeated games. An example is the one of \citet{Hoppe2005} who ultimately found that in a repeated licensing game, TTOs fully benefit only from high-quality inventions, inducing them to push toward an equilibrium in which intermediaries sell only profitable technologies – if the disclosure frequency is high enough. In this setting, rewards for scientists are strongly suggested to be success-based, because fixed payments will not sustain the equilibrium. 

\citet{Macho-Stadler2007} come to the same conclusion: in a repeated game, intermediaries have the incentive to \enquote{behave honestly} to build a valuable reputation, while if the disclosing rate is insufficient or if the game is single and non-repeated, technology brokers may prefer to take advantage of the information asymmetry. Elsewhere, individual agents and small intermediaries may have the incentive to offer low-quality inventions, whenever potential investors outnumber profitable innovations: scale is an issue \citep{Hoppe2005, Macho-Stadler2007}. 

However, there is a natural market force that pushed toward the concentration of intermediaries: first, a critical mass of inventions, thus the related research activity, is required to achieve a relatively good performance, therefore the survival of the intermediary. Secondly, the economic performance of these actors is tightly linked to the ability and the relative cost of personnel; the bigger the organization, the greater the opportunity to attract valuable human resources, reducing the marginal cost of this critical element.

Third, even if specialization is common in markets with intermediaries, the same force may lead to inefficiencies. In this particular case, and limited to the specialization by thematic, its low effectiveness in offering a balanced share of opportunities among intermediaries will eventually lead to the waste of high-quality invention, thus a market failure, and the underutilization of other intermediaries' resources, thus a market inefficiency \citep{Hoppe2005}.

Moreover, intermediaries with a larger innovation pool may have incentives to further invest in experienced professional, more capable of locating new and profitable inventions. Apart from the obvious, immediate comparative advantage they could generate, their deployment may be necessary and their costs more effective due to the numerosity of inventions \citep{Debackere2005}. In fact, economies on the costs of expertise generate one of the economic raison d'être of intermediation \citep{Hoppe2005}.

For specific forms of intermediaries, i.e.\ internal to the organization or participated by public research institutions, they are also required to manage the patent portfolio, entailing the need for additional personnel like patent attorneys. Empirical findings demonstrated the importance of these professionals, to protect and market inventions and secure additional funds \citep{Siegel2003a}; moreover, the patent portfolio comes with management costs, which require a proper valorization activity to make it profitable \citep{Balderi2010}.

Lastly, intermediaries can also support the creation of spin-offs, through their industrial linkages and by building a synergistic network between academics and venture capitalists, advisors and managers \citep{OShea2004}. Through these ties, they also have the opportunity to identify business needs and to forward them to scientists; however, this activity will further increase the need for specialized personnel for additional marketing competencies \citep{Geuna2009,Muscio2008}. 

Summing up, intermediaries can overcome the fundamental problem of universities: their bureaucratic organizational culture makes them inflexible in structuring deals \citep{Siegel2003a}, thus keeping them at a great cognitive distance from the market. In fact, intermediaries are becoming central, even if they may present some lack in effectiveness for most research institutions \citep{Geuna2009}, being them universities, research institutions or private research organization.

\subsection{Organizational forms}

Intermediaries may assume different organizational forms, due to specific purposes, property settings, internal configurations and other factors. Examples are universities' TTOs, including TLO, ILO, and other nomenclatures; colleges; public research organizations, publicly-founded agencies, knowledge intensive firms, professional associations, knowledge workers. Many researchers have investigated this issue, tackling it from various perspective, with the aim of establishing an ideal, effective form for every type of intermediaries. For the purpose of this work, apart from a general introduction, the focus will be centered on intermediaries strictly related to universities.

For a general framework, the starting point may be the work of \citet{Yusuf2008} who divided intermediaries in: general purpose, including universities, which aims to disseminate new knowledge; specialized in helping various actors in the technology transfer process, as TTOs and alike; financial intermediaries, specialized in providing financial resources to projects, as venture capitalists and angel investors; institutional agencies, which provide incentives in order to promote the general economic development. Another useful classification comes from \citet{Landry2013}, which identified as \enquote{emblematic types} university TTOs, community college TTOs, public research organizations and non-profit organizations.

Both examples have their roots in the same idea: different institutions need specialized services, therefore intermediaries must assume different forms based on specific resources and capabilities. To gain a proper insight on these differences, \citet{Landry2013} turned to the knowledge value chain construct: the development of an idea can be divided into three non-linearly stages, namely exploration, technical validation, and exploitation. Each stage requires a different support from intermediaries, i.e.\ consulting on market needs, prototyping and patenting, commercialization activities, etc. In fact, the author found significant differences in the level of engagement among institutions and intermediaries in the knowledge value chain stages: public research organization in exploration, community college TTOs on validation, non-profit organizations on exploitation, while university TTOs provide less customized solutions.

In the case of universities' dedicated offices, the previously cited Technology Transfer Office is, in fact, an oversimplification of the phenomenon. Other real-world examples are the Knowledge Transfer Office, Industrial Liaison Office, Office of Technology Licensing, the University Technology Transfer Office and many other. Regardless of their specific labels, these offices all perform variations of the same activity, with minor changes in focus and processes \citep{Brescia2016}; the only significant difference is the relative tendency for the suport of the licensing process or more general in purpose. However, the main tendency is to \enquote{resize the importance of licensing income, to rather increase the effort in maximize industry-funded research} \citep{Balderi2010}. In this perspective, this difference may not last long. 

Instead, a significant difference among universities' TTOs is the organizational form, role, and position they assume in different institutional context. More specifically, the organizational form is the first factor that has been found by many to be fundamental in determining the TTO performance. In the case of a public entity like a university, in which academics' devotion goes to the open science paradigm, an appropriate structure should allow the provision of both competitive mechanisms and adequate incentives to disclose. 

In fact, \citet{Bercovitz2001} proposed that any transfer activity is \enquote{shaped by resources, relationships, autonomy and incentives of the TTO}; therefore, the process outcomes may largely depend on organizational practices. This view is reinforced by \citet{Debackere2005}, who stated that different organizational arrangements might lead to different propensities of academics to commercial activities, especially for the \enquote{professional bureaucracy} previously cited. 

This traditional approach is based on a single, centralized office which employs a variable number of professionals, which can be organized by activities, projects, or none. In larger institutions, or where the attitude toward the market is greater, the office can grow enough to justify an internal divisional form, usually based on tasks' similarities. However, empirical research suggests that most universities employ only a few officers with almost identical functions. This traditional model can be seen as a divisional structure in which a single unit, the TTO, is in charge of the exploitation strategy. 

The second most common organizational form is the decentralized model, which is usually linked to a higher engagement in university-industry cooperation. Although there is no formal, shared definition for this approach, it is mainly recognized as an ensemble of technology transfer officers located in the various research units, teams, faculty and alike. In this case, officers will be in charge for every aspect of the technology transfer, with respect to their research unit. This model refers to a matrix structure \citep{Debackere2005}, where the exploitation division became integrated within research groups while responding to the central administration for their research activity.

Both models have some limitations. On the one hand, the centralized structure may induce the TTO to behave as a gatekeeper, to constitute a bureaucratic step toward the market instead of a supportive actor. The complex internal communication flows that come with a central structure, moreover, can slow down the process and make the university unresponsive to business and market needs \citep{Litan2008}. On the other hand, specialization and decentralization can ensure a higher responsiveness, positively influencing the academics' propensity through proximity and avoiding possible conflict of interests \citep{Debackere2005}. However, this last solution compromises the exploitation of synergies among activities and topics, other than the economies of scale embedded in a single office.

Anyway, an organizational form must be chosen. A comparative evaluation can be taken from \citet{Bercovitz2001}, who analyzed the potential impact of various Chandler's organizational forms on the office performance, both theoretically and on a case study basis. He found that the matrix form (roughly corresponding to the decentralized form) provides a better coordination among internal actors while ensuring the best incentives alignment. 

Later, a third possible model has been described by \citet{Brescia2016}: the semi-centralized model, as a compromise between the previous extremes. In this case, activities are distributed among a central TTO, typically in charge of intellectual property and spin-off activities, and decentralized officers employed for grants, collaborations, and contracts. Theoretically, this configuration should allow the exploitation of advantages of both the previous models. 

It should be noted that the TTO may not be internal to the university. \citet{Fisher2002} differentiated between the fully integrated, internal model, and external entities which may assume the most different forms, ranging from a non-profit, participated organization to a consortial form of networked, interconnected TTOs \citep{Brescia2016}. These external entities are a particularly useful solution in the case of smaller universities and fragmented environments \citep{Debackere2005}.

\subsection{Models and objectives}

Even more important than the structure, is the organizational role of TTOs: these offices should foster researchers to engage in technology transfer activities, through their impact on the academic environment. Starting from \citet{Jensen1998}, many authors modeled the TTO role as an agent of both the university administration, faculties, and researchers, involved in balancing the needs of every actor. In this perspective, the orientation and the activity of the office both improve the outcome of the technology transfer process and the raw material the office has to work with, the invention disclosure \citep{Siegel2007}. 

More specifically, apart from the impact of university-level regulation and incentives, TTOs may also leverage their reputation to enhance the attitude of scientists to disclose. By demonstrating a deep understanding of both academics and firms, and by building a history of successes, TTOs can be perceived as professionalized and useful, acquiring credibility and gaining the trust of academics \citep{OwenSmith2001}. Specific to the ability to understand firms and markets and its importance, \citet{Muscio2010} empirically demonstrate that a non-academic background of the TTO leaders is linked to a greater use of the TTOs services.

Otherwise, faculty and academics may prefer to circumvent the TTO by establishing a direct relationship with the industry or refusing to disclose inventions, thus leading to a waste of resources and a possible underperformance driven from the non-use of professionalized services. The same result may be reached if previous collaborations with the TTO  led to poor outcomes, or if the TTO cannot offer a valuable service due to lack or inadequateness of personnel. 

The way TTOs relate with academics largely depend on the goals set by the university administration. Two main models are recognized by the literature: the revenue maximization and the diffusion maximization models. The first, the most common, will make TTOs focus on short-term cash maximization, where a high risk aversion may lead to \enquote{suboptimal licensing strategies} \citep{Markman2005}. Moreover, \citet{Siegel2003} identify a mismatch between this commercialization model and the motivations that drive the involvement of faculty and academics, thus a lower performance.

The second model, preferred by scientists, would be the natural path for technology transfer officers: a survey by \citet{Jensen1998}, demonstrated that the main objectives of technology managers are the number of inventions commercialized and of licenses executed. However, this model does not fit with the university's actual need for research funds from other sources than the public institutions. This may be the main factor explaining why the technology manager interviewees see themselves as \enquote{juggling} between different interests, trying to maximize a weighted average of the administration's and the inventor's utility.

\citet{Fitzgerald2015} used a quite different approach to uncovering drivers and models beneath TTOs activities: their mission statements, an instrument to signal to stakeholders the main, long-term mission of the organizations. In concept, they should be useful also as a guide for decision making, as a tool for the formulation and implementation of a strategic plan, eventually influencing the personnel behavior and their performance.

Their empirical research highlighted that even if quite every mission statement stated the identification of main products, services, customers and markets, almost no one reported the desired public image, self-concept, expression of commitment or any element of the organizational philosophy. These results may suggest that the impact of the instituttional role of TTOs, especially as motivator for academics, is still underestimated, or worse, poorly understood. 

\subsection{Activities and personnel}

The office's objectives and model directly influence the type of activities it will perform, as well as the extent of the effort dedicated to each activity. In earlier days, \citet{Siegel2003a} recognized as main objectives licenses and royalties, from the patent activities, and the support to firms and scientists for economic and product development. Similarly, \citet{OwenSmith2001} reported that the resource constraint makes TTOs concentrate in the core activity, identified in the management of the IP portfolio.

More recent research reported a shift and enlargement of TTOs' principal activities: \citet{Geuna2009} reported as main focuses patent, licensing, and the creation of spin-off. While they mentioned also contract research and consultancy, it is important to note that in his view the focus shift from the evergreen, well-known patent affair, to more general activities that imply a commercial exploitation of university knowledge directly performed by internal personnel. This perspective was reinforced by \citet{Balderi2010} who identified as main activities patenting and licensing, spin-off and other activities complementary to the first ones.

A recent and useful framework can be taken from \citet{Alexander2013} who linked TTO's core activities to four key objectives: 
\begin{itemize}
\item Facilitate the management of activities and projects, which this require a continue assistance to academics and firms from the setup to the control of key stages and follow-up;
\item Enabling the transfer of intellectual property and facilitate entrepreneurial activities, i.e.\ the management of the patent process, administrative and bureaucratic support for spin-offs;
\item Promote and develop knowledge-based support services, ranging from contract and cooperative research to training and personal development;
\item Establish knowledge-based boundary-spanning activities, both in terms of stand-alone knowledge and through mobility and networking. 
\end{itemize}

A more debated topic refers to the disclosure eliciting activity. It is true that nowadays quite all academics engage technology transfer activities only through the TTO, letting the office managing and controlling the relation, but some researchers still not disclose inventions. It may be required by the university regulation or other external sources, i.e.\ the case of the Bayh-Dole Act, but these rules are rarely enforced. On the one hand, there is an evident difficulty in monitoring research activities and comprehend which results may be susceptible of commercialization, while on the other hand TTOs usually have to face constraints in resources and time, which impede the eliciting activity. 

This scenario highlights the importance for TTOs of having highly skilled and trained professionals. Every activity requires specific knowledge and often interdisciplinary competencies: examples are IP and contract lawyers, business developers, business analysts, marketing personnel. They must comprehend both the academic and the business world, preferably with experience in both, as well as be able to evaluate the technology embedded in each proposal. Skilled personnel may gain a good reputation and enhance the disclosure, while dissatisfaction may lead academic to circumvent the TTO. 

To ensure a good performance, TTO may also consider the provision of specific services from external providers, both specialized in a single activity, i.e.\ patent attorneys and technology consultants, as well as specialized in the technology transfer itself, i.e.\ TT centers, consortia, etc. Another alternative is to develop dedicated external support facilities, as business incubators and science parks.

\section{Faculty}

Faculty and departments may have a significant role in the technology transfer process, both as a group of academics that share a common research theme and as an organizational intermediary between the central administration and scientists. 

A first, relevant factor is the faculty leadership: as stated before, an experienced, supportive leader positively influence the academics perception of the university orientation toward commercial activities. \citet{Guerrero2014} reinforced this perspective by analyzing the importance of the attitudes of key actors, like faculty member and leaders. 

Similarly, \citet{Muscio2010} found significant the faculty leadership's (the department director) trust in the TTO, which increase the probability of a successful exploitation – assuming a positive impact of the office's professionalization. He also found an inverse correlation between the leader age, as a proxy of experience and previous involvement with industry, and the probability of trusting and contacting the TTO. However, faculty leadership is expected to execute policies in compliance with the university mission and objectives \citep{Chang2016}, therefore linking its supportiveness to general policies and the administration. 

A fundamental study on this topic is the one authored by \citet{OwenSmith2001}. They found three significant factors for the institutional success of patenting, as one of the technology transfer channels: (1) the faculty perception on potential benefits; (2) the perception of time and resource required to interact with the TTO, as well as its perceived quality; (3) the general opinion of the university on the technology transfer topic. Along with perceptions, faculty behavior is influenced by the institutional environment and its organizational structure. 

Firstly, promotion and tenure policies may be shaped by the university mission and orientation, but are enforced by faculties and departments. Moreover, they are in charge for the evaluation of research personnel, both on scientific outcomes and commercial exploitation \citep{Chang2016}. Secondly, social norms and practices established at the university level may strongly influence the faculty attitude; should be remembered that, according to \citet{Bercovitz2006}, a shift can be made by setting the right incentives, particularly at the faculty level, regardless the university history.

Another relevant faculty variable is its scale, both in term of human and technical resources; as an example, empirical findings from \citet{OwenSmith2001} suggest that ceteris paribus, best performers have \enquote{more researchers and more resources devoted to research}. Similarly, the size of the department is belived to have a U-shaped relationship with the volume of industry interaction \citep{DEste2007}, where small structures may have not enough resources and the large ones tend to engage a more basic research, far from industry problematics.

Past performance is found to be useful in predicting future ones. A useful proxy is the departmental research income received from industry, per member, which has been found to be correlated with future engagement in commercialization activities \citep{DEste2007}. Furthermore, \citet{Blumenthal1996} in earlier days found evidence that industry-funded faculty members are commercially more productive, suggesting a Matthew effect. Similarly, empirical findings from \citet{Thursby2002} indicate that the growth in faculty propensity to disclose inventions, itself one of the most significant predictors of technology transfer performance, is clearly linked to its past licensing success. 

Of similar impact is the overall past scientific performance of the faculty \citep{OShea2005}. \citet{DEste2011} indicated that, apart from its effect on the volume of interaction, it influences also the preferred channel; specifically, they found that researchers in lower rated departments tend to prefer consulting, whereas evaluations and incomes are more likely to indicate more frequent contract research.

Other authors studied the effect of the faculty's scientific field. Firstly, cultural norms across fields may be highly significant \citep{DEste2007}; secondly, the nature of the research – its applicability to industries – is critical in determining the extent and volume of commercialization activities. In fact, some research fields easily meet specific market conditions, and \citet{OShea2005} found that faculties and departments involved in such research usually receive more funding, either from industry, government or the university administration.

Oppositely, faculties specialized in basic research may prefer not to disclose inventions, afraid or unwilling to spend time and resources on the successive applied research needed for commercial exploitation \citep{Bercovitz2006}. Basic research also refers more strictly to the open science paradigm, leading such faculties to avoid commercialization due to possible delays in publication for patenting and marketing purposes. 

Inside the basic and applied research areas, the literature is more divided around the effect of specific disciplines on the commercial engagement; different authors provide contradictory findings on whether the research topic, as in biotech, medical school or engineering, increase the attituDe toward technology transfer. However, \citet{OwenSmith2001} observed that in any case, the value of patents may vary significantly across areas, in terms of the protection extent, for leverage potential and as a source of incomes. 

Another significant research on this topic refers to the impact of research areas and the cognitive distance from the related industry: \citet{Muscio2010} found that departments with greater cognitive distance from industrialists do collaborate more. This finding supports the theoretical literature that suggests an inverted U-shaped relationship between cognitive distance and university-industry collaboration.

\citet{Wong2010} also found significant the impact of faculty internationalization on patenting performance, but results are ambiguous: in North America it has a negative effect, while positive elsewhere. These findings suggest further investigations on which faculty factors, among different institutional university systems, can lead to this contradictory result. 

Finally, faculty's benefits from technology transfer usually refer to the individual benefits and the impact that these could have on the research performance of the personnel. \citet{Baldini2007}, for example, indicated as main advantages the direct access to industry knowledge, laboratories, and funds, the positive influence on researchers' career and earnings, new and different ways to exploit the researchers' abilities, factors previously seen as motivations for academics.

Again, also departments should demonstrate ambidexterity in order to gain a good overall performance. It is also fundamental to note that this ambidexterity is widely influenced by the department perception of the institutional flexibility, and may have a positive, significant impact on the individual ambidexterity. This setting helps clarify the vast extent of environmental and institutional impact on the probability to engaging in technology transfer.

\section{Recap}

In this chapter has been described the academic sphere of the technology transfer. 

In the first section, the historical perspective highlighted the two major cultures that characterize universities, the open science system and the knowledge economy and their impact on transfer activities, exposing authors' opinions to line out significant pros and cons. The second section describes the attitudes and preferences of academic scientists, including the research they may favor, incentives to foster and attributes linked to a greater commercial performance. 

The third section provided information on the technology transfer office, starting from its role of intermediator in the knowledge economy. Has been exposed the two main organizational structures and objectives that shape their transfer model, a view of their activities and the fundamental importance of employees. Finally, the fourth section outlined the influence of faculties and departments in shaping the individual attitude toward and performance in transfer activities.

Each of these elements will be later compared with private research institutions in \hyperref[Chapter6]{Chapter 6}, which instead are characterized by an economic purpose, flexible and lean organizations, larger commercial structures, in which scientists are more prone to applied research and development for the market needs.