% Chapter 6

\Chapter{Different approaches}{Technology transfer: U-I and B2B}

To describe the university-industry technology transfer, the previous chapters reviewed selected topic from the literature, including: the academic environment (\hyperref[Chapter2]{Chapter 2}), the industrial counterparts and other receivers (\hyperref[Chapter3]{Chapter 3}), the various channels that bridge the two realities (\hyperref[Chapter4]{Chapter 4}), the evaluation and policy management of the process (\hyperref[Chapter5]{Chapter 5}). Following the same outline, this chapter will decline the university-industry framework in a business-to-business approach, pointing out the differences in perspectives. This analysis will be later used in \hyperref[Chapter8]{Chapter 8}, to discuss the case study and draw the conclusions. 

\label{Chapter6}

\section{Motivation}

As previously stated, the largest part of the aforementioned authors focused on the technology transfer between university and industry. In example, many of them longer discuss the university open science approach, the scientists' willingness to engage transfer activities, the best management practices for the transfer performance of a university. Thus, the literature covers almost any topic of an otherwise very specific topic: a scenario in which the only knowledge producer, repository and distributor is the university, a scenario that is clearly and largely characterized by the various characteristic of this kind of institutions.

In the modern knowledge economy, however, universities are not anymore the only sources of knowledge and technologies: other research organizations have emerged as relevant and competitive alternatives, i.e. public and national research centers, publicly-funded and non-profit research organizations, private development firms etc. These are only some of the large ensemble of organizations that the complex economic landscape has generated, but helpful in exemplifying how restrictive can be to consider only universities.

Therefore, the question is whether this narrow focus of the literature jeopardizes its effectiveness when analyzing and modelling other types of private entities. More specifically, the issue does not refer to the applicability itself of the technology transfer literature to private agents; it relies instead on the identification of factors and constructs that should be changed and what instead can be held constant, to successfully use the previous literature in a different environment.

Has to be noted, in fact, that universities and private research organizations do share a major behavioral trait: even if with different extent and modalities, they both perform research and development activities, and both try to valorize their results through commercial activities. By and large, academic and private scientists are driven by similar motivations and objectives; exploitation activities follow similar paths; their management may encounter the same issues and make use of similar tools. However, private organizations do have very different and narrower priorities, and are able to take greater advantage of various instruments not available to public organizations. 

In this chapter the topic will be analyzed by focusing on the differences between paradigms. Following the same structure of the literature review, in each section will be drawn a comparison of actors, channels, evaluation and policies. For practical purpose, except where explicitly mentioned otherwise, the basis for comparison in the private environment will be a completely private organization, mainly involved in applied research. At the same time, the university side of the comparison will be expected to show at least a minimum entrepreneurial activity among all channels. 

\section{Actors}

\subsection{The institution}

Private organizations and universities do share some traits: both are structured institutions, which make use of units or teams of researchers to generate new knowledge. However, the activities they perform might be very different: universities, especially the public ones, mainly focus on basic research with an open-science approach; private firms instead may need to focus on more practical research, centering their business models on requested, on-demand R\&D services.

In fact, private research organizations are characterized by a different organizational culture. They may borrow a few traits from the open-science approach, but they also rely on the excludability of their research results, through instruments as patents, NDAs and alike. While they may perform basic research and publication activities, these activities will not overcome the relative importance of knowledge exchanges for a financial reward. Private organizations have not the historical baggage of universities, thus the acceptance of an economic model for knowledge production and share comes with less difficulties in respect to the academic environment.

Research firms also lack one of the three academic missions: teaching. Specifically, teaching and training activities are still present, i.e. the offer of collaborative PhD programs, trainings and apprenticeship. However, these are not a core element of the business model; students might be seen as a human resource, with significant specialization but no experience. Moreover, these activities might be a requirement to access to local or European research grants, i.e. through policies that aim to improve the skill and competencies set of the local workforce. 

The research activity, as in the basic science approach, may also be needed by the context or be included in the main business model. In the first case, related research grants and networking activities may be necessary to sustain the organization, through funding and relationships in various kind. In the second case, the firm might recognize the basic research as a distinctive and advantageous activity, to perform in order to build a stronger knowledge base, thus a greater competitivity in the research market. In this case, the objective is to contemporary manage short- and long-term objectives; potential ground-breaking patents might also be an incentive.

However, the largest part of research activities should be expected to be tightly linked to the commercial endeavour, represented by the third mission in an academic environment. In this latter case, the economic literature has pictured it as a distinct activity from the research, but in the case of a private organization that survive by selling its research outcomes, the boundaries are less clear. In the traditional academic approach in fact, the commercial activities were profitable, but accessory to the research, pillar in which where autonomously and independently generated ideas and prototypes to bring into the market. 

In a private environment, the relative importance is inverted: leading activities are commercial in nature, while research is more of an internal service, functional to the ability of realizing and delivery a product or service. This difference can be seen in the driving force for the research activity: academics choose to investigate a topic according to their interests, personal agenda and funding opportunity. Firms' researches, instead, are driven by the needs that the organization spotted from the market. These institutions usually work on commission, or sell discoveries to the highest bidder. 

This different perspective push the private organization to focus mainly on applicable knowledge fields, transferring the choice from the individual to the institutional level. However, the duality of basic and applied research, and their relative approaches, has to be manage also in the private environment. Eventually, researchers have the strongest expertise, thus the ability to successfully predict the next shift in the industry they refer to: it might be useful to weights the market pull and the pushing force from the technology and knowledge field, therefore the research and the commercialization pillars. 

This issue is strongly connected to the differences in the funding sources among types of organizations. Universities mainly relies on public funds, block grants and research grants, from local, regional governmental and European institutions; the usage of industry funds instead is limited in its extent. These funds are mainly directed toward research and teaching activities. 

Private organizations instead invert the relative importance of their funding sources. They might also use public funds and grants, but their amount implicates the need to raise substantial funds elsewhere: the market. Private research firms are required to gain funds through research and development services, as in research contracts, cooperative research, consulting. Therefore, the need for a competitive business models, a flexible organization, the responsiveness to the market needs.

Also the organizational structure and the internal processes should be properly configured. Firstly, commercial exploitation of research outcomes is no longer an emergent activity, relying on the individual attitude and willingness: it should be institutionalized, mandatory for research positions. The organizational structure instead should include a commercial structure, among which a TTO, legal and financing office, a marketing function etc.

The degree of complexity increases in the case of a public-funded research organization. Along with the financial support, public institutions can improve in various forms the local relationships, cooperation and networking activities of the private firm. In example, greater and easier participation to the local context, driven by public institutions, may bring the organization new technologies, development partners, customers, suppliers.  

However, the public interference in private matters can influence the mission and the modalities in which a private organization act. In example, it may push the organization toward low competitive choices, in matters of technologies, products, partners. The issue arises from a divergence in interests: the growth of the context for public institutions, the survival for firms.  

\subsection{Researchers}

Along with the type of research performed, also researchers and their characteristics should change. In the collective imagination, universities should seek for scientists with a preference for theoretical research and hypothesis testing, with little interest in individual financial gain. Private organizations should prefer instead researchers with a strong preference for applied research, its benefits and outcomes: possibly less recognized by the scientific community, but nevertheless ground-breaking and with a powerful impact on the society. At the same time, publications should not be important for private as academic researchers, especially when it comes to the need to postpone publications for patent applications.  

In fact, research firms should need human resources for short-term developing projects, i.e. stage-gate commissioned projects with an outcome designed by the customer: the focus of the activity shift from the objective of the research, the \enquote{what}, to the process itself, the method and approach to achieve the required objective, the \enquote{how}. Private researchers might lose the control over their short-term tasks, while eventually maintaining the control over the long-term choices.

More specifically, they should understand and value the economic protection of their ideas, not only in the form of patents, but also as secrecy and excludability of economic outcomes, for competitive purposes. These reasons should not be an exogenous factor, imposed by the organization, but an endogenous driving force: as long as researchers have the best knowledge of their knowledge and technological fields, and decide on the prosecution of their research, they should exhibit the economic attitude to choose the most competitive alternatives, while comparing the choice of which technologies and knowledge fields to further develop.

Another significant change between environments involve mechanisms and processes for the evaluation and management of these researchers. In the organizational structure of universities, faculties or departments are the best suited for the assessment of their personnel; significantly, researchers are evaluated by peer academics, with a similar area of expertise and specialization, i.e. the head of department. Other offices provide accessory services, for mobility and visiting professors, research funding and grants and alike.

The organizational structure of a firms, instead, might include a central office for the human resource management, which oversees their evaluation, reward and promotion. However, this office has no competencies on the research topics, thus it lacks the necessary resources to properly evaluate the scientists. To overcome this issue, the HR office might base its assessment on the evaluation that the heads of units make on their subordinates, as a reference for the individual activity.

Therefore, in the case of a university setting, the evaluation is mostly direct and research-related. In the latter case, in a private environment, the evaluation can be split: each scientist can be evaluated along multiple dimensions of its work, i.e. quality and complexity of his research, the effort required, the income he was able to generate, the number of contract he was involved in, further projects that derives from his work. The evaluation can also take advantage from classic firms' instruments as the 360-degree feedback, therefore including the concurrent assessment of other offices, i.e. TTOs and legal offices, for the same human resource. Eventually, private research organizations have the tools for a more precise and effective evaluation.

The same argument holds for promotion decisions and reward policies. In the first case, career development patterns can be borrowed from other public-funded or public research institutions, while the decision on whether and how to promote can be made in concert between the administration, HR office and the head of unit or research center. For the reward policies, private organizations should provide both pecuniary and non-pecuniary incentives, i.e. preferences for mobility programs. In fact, the reward policy might be fundamental in acquiring the best human resources: as previously stated on the work of \citet{Stern2004}, scientists do prefer universities and public research institutions, while private organizations must provide a premium to compete in the labor market.

Related is the topic of the legal framework organizations are subjected to. For public entities, available contractual forms usually provide a stronger protection for the human resource, depending on the country and its legislation. Firms instead have at their disposal a lager extent of employment contracts and any configuration of legal protection, reward and promotion clauses, length of the contract etc. Again, contract forms can be used as a competitive element in the labor market. 

\section{Channels}

Differences in channels between private organizations and universities are mainly limited to the preference toward specific channels, and minor adjustments in their usage. In fact, has to be notices that choices and management on this topic are not entirely up to the transferring organization. 

On one hand, knowledge producers might follow a technology-push approach, autonomously deciding to perform R\&D activities in topics of their own choices, produceing outcomes to place in the market. In this case, these organizations might choose themselves channels and terms for the transfer.

The market-pull approach instead requires these organization to act on the basis of the market needs, being them unexpressed desires or specific requests from the industry. In this latter case, the industry representative has a significant influence on the characteristics of the requested transfer: the specific technology to develop or the issue to investigate, channels, prices, legal agreements etc. Moreover, firms tend to have a higher bargaining power on the research organization, mainly due to the characteristics of the research market: highly specialized, with strongly customized services, limited in its dimension both for demand and supplies representatives.

On the other hand, also researchers concur in determining the characteristics of the transfer: their competencies and interest in the selected knowledge field, their willingness to participate in future development, the time they could devote to the project. They also shape the technology trajectory of the research institution, influencing the field of expertise, objectives for future developments and fieldd of applicability.

\subsection{Preference}

The traditional economic literature suggests that private institutions do have a preference for easily controllable outcomes and less uncertain activities; in the case of research organizations, this risk aversion should influence the channel preference, thus its choices. Therefore, there might be the preference for contractual forms of transfer, short term and less risky projects.

The primary effect of this preference affect the usage of informal channels, especially direct transfer between researchers and firms, and more indirect ones, as publications, conference and alike. Considering the process perspective alone, the impact of the organizational private status might be negative; however, strategic issues come into play. Participation to the scientific community, mainly made through publication activities and the contribution to the literature, strongly influences the reputation of the research center, thus the ability to absorb new scientific advancement, hire skilled researchers, collaborate with prestigious institutions. Moreover, this participation might be necessary to build the right company image for the market.

Informal channels might be fundamental also for research organization embedded in favoring local contexts, especially high-tech clusters. In fact, the informal participation in the local community might bring to the organization various advantages, among which customers, suppliers, new technologies, new ideas for their development, technology validation and alike, that can easily outweigh the short-term loss in incomes and profits. The decision should also consider the potential effect on local and national governments, including the influence on their policies, legislative activities, funding strategies and decisions. 

A very specific example, but not uncommon in these high-tech contexts and industries, is the usage of informal channels as market development tools. In example, scientific publications surrounding a newly developed technology, as well as the open sourcing of a patented technology, may in fact help other firms to get on board. While it might seem counter-intuitive, markets for new and immature technologies are usually small, and might encounter difficulties in reaching the momentum needed for an autonomous expansion. In similar markets, producers do not compete over each other's market share, but in expanding the market frontier: as the cumulative investment in marketing and communication arise, the total market size increases, thus the economies of scale, let alone opportunities for product testing and technology validation. Disclosing information, the core of informal transfer mechanisms, might be the basis for a successful long-term strategy, while sacrificing potential profits in the short-term.

Spinoff might also not be the preferred channel for private organizations. The creation of a new venture eventually requires for the spinning organization to left a significant part of its management and control to the entrepreneur and other equity partners, resulting in the mere ability to influence the project, rather than drive it. Moreover, this activity might require the loss of valuable human resources in favor to the spinoff. 

At the same time, their financial profile might negatively influence the extent of spin-off activities: they require considerable investments, prolonged in time. With the increase of the number of spinoffs, this issue growth exponentially, but extended activities might justify the institution and usage of specific instruments. Examples are the creation of, or the relationships with external venture funds and incubators, creation of business development consortia, specific agreement with the local government etc. On the other hand, these instruments might lower the financial need of the activity, but will require substantial efforts in networking activities and contractual management.

Moreover, spinoffs might be the one of the less risky channels: as previously stated, these ventures have a significantly low ratio of failures in respect to common firms. In respect to other channels, especially informal transfer and licensing of patents, spinoffs can be considered a safer choice, but their effectiveness largely rely on the availability of skilled and trained human resources in supporting the project, i.e. the incubator personnel. Eventually, other processes as licensing can be modelled to include similarly specialized employees, bridging the gap in the relative risk perceived.

Licensing requires similarly high costs and time: apart from the development requirements, the granting process itself might easily need years. The licensing process can be difficult, especially in the case of immature technologies hard to evaluate. Specific institutional-level strategies and policies might be required to make this process successful and profitable. In example, focusing the research activity on a narrow knowledge field, or even on a single technology, may result in an ensemble of results that can generate a thematic stack of patents, significantly more effective and valuable than a the single one.

For specific choices and tasks among the process, it is also strongly advice the usage of external, specialized firms: patent attorney or firms specialized in their valorization or the protection from patent infringement. In fact, the ultimate value of a patent largely relies on how the application - and especially the inventor's claims - are written, which require professionalized and specialized skills. These external agents might also reinforce the bargaining power of the research organization and make use of their networks in seeking potential licensees. 

Eventually, patenting and licensing activities might be considered safer, while their incrementality allow a greater control over resources and costs involved. Their profitability instead is reported to be limited \citep{Balderi2010}: only a small number of patents will become \enquote{blockbusters}, but their incomes might be able to sustain the entire organization, alone. Thus, this channel may in fact be preferred over spinoffs and informal mechanisms.

Consulting, contract and cooperative research instead are expected to be the preferred channels. Firstly, these process starts from a formal contract with other firms, therefore their profitability is known from the beginning while institutional effort and investments required are relatively smaller. The contractual form also allows the transferring organization a larger control over the process, its outcomes and the resources to dedicate, while being compatible with other contemporary research activities. On the other hand, these channels might generate a sizeable amount of incomes, and under the proper management they could sustain the entire organization. 

The downside of these contractual forms are the legal costs associated with the writing and management of contractual arrangement. In example, an internal legal office or specialized attorneys might be required. At the same time, the bargain power of the research organization might be lower than the firm's, limiting the profitability of these projects. 

\subsection{Mechanisms}

Another relevant question is whether, and how, the private status of an organization influence the modalities through which each channel is managed or used. 

Informal transfers, if performed, should acquire an institutional trait rather than being managed at the individual level. As previously stated, informal knowledge transfer might be essential for the creation and development of a network of exploitable relationships; thus, the organization might establish different institutional moments for informal exchanges, in order to take advantage from these mechanisms while retaining some level of control over them. 

Examples are the hosting of conferences and contests, the participation to fairs and other events, or the provision of mobility programs for employees. Depending on the institutional focus on local or larger contexts, scientific publications might still be discouraged, while preferring more personal forms of exchange for tacit knowledge, i.e. co-development activities that not include public disclosures. 

The spinoff channel and its characteristics largely depend on the specific policy chosen by the parent organization: at the two extremes, the question is whether or not to support these activities. On one hand, a spinoff-friendly organization might provide instruments as business plan competitions, business development services, access to their industry networks, seed funds, favoring licenses and alike. In respect to a university, in a private environment should be expected a stronger pressure and larger responsibilities and accountability over new entrepreneurs, at least in an informal way. On the other hand, non-supporting organizations might choose to not provide any assistance, or even to take legal precautions toward spinoff establishment.

Moreover, if the institution is willing to support spinoffs, a major effect can be foreseen. In respect to a university or a public institution, a private organization should achieve higher performance in supporting spinoffs, especially in network development activities. The roots of this effect reside in the connections that a private organization should have into its economic context, that are expected to be larger in number and stronger. Thus, business and network development activities should be facilitated.

Similarly, differences in the patenting and licensing process mainly resides in the relative importance of this channel in the business model of the private organization, especially if this last contemplates basic research along with the on-demand R\&D. In fact, basic research may be performed in order to acquire the knowledge necessary to future research contracts, but also specifically patent applications and license incomes. If not, patents might be required by research contracts, as a clause, or arise as by-product of the required development activities.

These approaches differ in the extent of resources dedicated to the patenting process, as in financial and employees. In fact, if patents arise only as a by-product, the need for specific skills and competences might require the organization to continuously outsource the process, increasing the costs and progressively lowering the institutional attitude toward licensing. In the case of institutional basic research activities, internal professionalized services may be provided.

A relevant issue on this topic refer to the ability to grant exclusive licenses. Public institutions are usually advised against this practice: according to the larges part of the economic literature, public funds should lead to public-domain knowledge and technologies. Private organizations instead might be preferred by firms because of their ability to exclude other parties from the appropriation of economic and financial results of research outcomes.

For consulting, contractual and cooperative research, the main difference among institutions should reside in the larger contractual freedom of private organizations. Examples are funding options, accountability, NDAs, exclusivity and the generic assignment of research outcomes. Moreover, contracting firms may exhibit a higher willingness to engage with private organizations, rather than public ones, because of their relative flexibility, accountability and other factors. The examination of legal differences is left to the competent literature.

\section{Evaluation}

The evaluation of technology transfer activities in a completely private environment should be more straightforward, once compared with the university case. Among other differences, one of the most outstanding is that private research organization have a clear primary mission: sell technologies and knowledge, thus generate incomes and profits. This chain of events arises from a unique purpose: to survive in a competitive environment. Eventually, the evaluation of the entire organization along a unique objective, instead of the university's triple mission, simplify the process. 

Moreover, private organizations can make use of all the standard economic tools for evaluation and management; these include HR practices, project management tools, specific and oriented contractual forms, hierarchy and many others. These tools are in fact available also for public institutions, but they are designed and best suited for a private environment, enabling a selective but greater relative performance. However, these newly available instruments have to face some of the same issues that may be experienced by public institution, arising from the kind of activities performed rather than the type of organization.

The first, major issue still present in a private organization surround the personnel, including its motivations, orientation and evaluation. While the private contractual forms of employment give the organization more control over the personnel, their effectiveness will be useless if they push toward misleading objectives incompatibles with the organizational mission. The issue of identifying the right objectives and evaluate their achievement arise from the characteristics of research and development activities.

Firstly, the outcome of the research process is not certain: evaluate and incentive the personnel on the basis of the process results may be counter-productive. The outcome, in fact, may or may not be the one expected, or even available to be assessed: inconclusive researches, unreachable objectives, etc. If researchers are evaluated only on the fitting of their achievement to the original project's objectives aims, their willingness to embrace technology transfer activities, especially the riskiest projects, may be fundamentally diminished, as well as the relative performance and the choices on the allocation of work time among concurrent researches. They may even refuse to engage fundamental, ground-breaking proposal because of their relative risks, the kind of projects from which the firm can profit the most. 

This issue refers to the larger concept of risk-aversion, and it affects researchers as well as other employees, especially TTOs' officers. More specifically, the problem has its roots in the individual evaluation along concurrent projects, projects that all have different risk factors. Therefore, the individual will choose considering the value of the outcome, moderated by the probability of achieving it. 

Secondly, research projects may require different degrees of effort, capabilities and skills. Evaluating researchers only according to the presence of an outcome or its fit to the project requirement may discourage the most competent and valuable human resources, by jeopardizing their ability to distinguish themselves from other scientists. In an era in which the competency-based human resource management is more of a necessity then a fashion accessory, it is necessary the use of another, different approach. To balance the choices of scientists, the organization might consider to offer an insurance, as seen in \citet{Panagopoulos2013}, or to use different approaches to evaluation, as in a mix of factors each one with a different weight.

At the same time, the evaluation of research personnel along task-related factors instead of the outcome is not safer. In this case the problem is more on the identification of indicators that can universally represent the participation of an employee to the project. These indicators should be applicable to every researcher in every project, in order to be perceived as fair, thus shared and trusted. Examples might be the number of hours spent on the project, financial income, number of partners involved, of patent generated, of subsequent child projects. 

However, a set of quantitative indicators cannot represent the research process in its entirety: qualitative indicators should also be considered. On the other hand, a complete different set of problems arise in the establishment and usage of qualitative indicators: subjectivity, reliability, trustworthiness etc. If the organization makes use of both the types of indicators, an additional issue might be the understandability of the evaluation process: to be effective, it should be simple, understandable and perceived as righteous, and complete at the same time. In fact, the process-based evaluation does not resolve opportunism issues and the deviations of behaviors that may arise form an incomplete set of factors or non-fundamental variables: allocate more working time than needed, generate useless patents, grant agreements with low contributing partners etc. 

Eventually, the evaluation of the daily R\&D operations exhibit several issues regardless the type of institution. However, private organization might be better equipped for this task. On one hand, the activities researchers are involved in are narrower in scope, more focused and centered on the development: in respect to a university environment, private researchers have not to deal with the teaching activity, simplifying the evaluation. Secondly, applied research which characterize private institutions is easier to evaluate than the basic research of universities. Thirdly, private researchers might also be involved in a smaller number of researches, avoiding the need to evaluate them over more teams, research groups and projects. It is also easier to manage the individual responsibility and accountability, once considered the impact of private organizational structures, hierarchy and clearer assignments. Lastly, contractual forms of employment might contain specific objectives and metrics, easing the evaluation process and reinforcing the institutional ability to drive the individual efforts. 

Apart from the evaluation of researchers, significant differences arise also in the performance evaluation of administrative and supporting personnel, i.e. TTO's employees, marketing and communication, lawyers and patent attorneys. Agaiin, the different type of organization should easier the evaluation, control and management these human resources.

Among the supporting structure of the organization, the most significant differences arise in the management of TTO's employees. In a university environment, many authors reported the activity of this office as contingent, on a case basis; officers try their best on an ensemble of projects that are uneven, hardly structured and disorganized \citep{Jensen2003}. In a private organization instead, specialization may dictate the activities and processes of the office: skilled and trained personnel, larger in number, competent in different functions. The institution, its structure, mission and policies, should also push officers toward a unique methodology, i.e. preferred channels or mechanisms, further the structuring of the TTO's activities. 

On the other hand, if the input is planned, structured and expected, TTO's activities can be further standardized, easing the management of transfer activities, thus the evaluation the office performance. In fact, if the input value is known and fixed, the relationship between research input and commercial output will be easier to evaluate, along its efficiency, efficacy and added value. For activities that have no direct relation with the research results, i.e. marketing and business development, more standard approaches can be taken from the traditional management literature, except for the need to decline these tools in a research environment, where uncertain results and knowledge asymmetries might represent additional issue. 

Similarly, employees belonging to other support functions, as HR, legal office, administration, etc. may be evaluated with traditional methodologies. In example, HR officers can be assimilated to their colleagues in other knowledge producing industries dominated by creative employees: designer, writer, advertiser, and alike. Best practices can be borrowed from successful organizations involved in those industries.

The evaluation of management positions should be similarly influenced, and simplified, by the type of organizational structure. Specifically, the relative strength of institutional policies and the hierarchy structure, in addition to methodic internal processes, should clarify the impact of these roles, and ease their assessment. As example, the performance of the head of research and support units might be evaluated through the comparison of the ongoing and previous achievements, weighted for the known variation in office's inputs. 

A particularly useful insight on the performance of the top-level management might arise from the analysis of inter-unit indicators, as in number of collaborative projects, shared patents and alike, and of the networking activities, i.e. through the number of signed framework agreement, variation in public funds etc.

\section{Policies: examples} 

The easier process design and evaluation influence also on the identification and implementation of effective policies:  methodological and structured processes make the evaluation clearer, enabling a faster and more precise discovering of ineffective or inefficient tasks along the process of transfer. Thus, these issues should be objectives of policies, improving their speed and efficacy. In other words, a better representation of the process lead to a better analysis, thus discovering failures to address and the best method to correct them. What follow are examples of policies that a private organization should consider; general purpose policies at first, then specific for process or office involved.

Firstly, in a private organization technology transfer activities should not rely on the individual attitude, propensity or willingness. In a similar environment, the relation between researchers' activities and results and the TTO's activity should be structured and organized, substituting the spontaneous trait of universities with institutionalization. These activities are no longer additional and voluntary: they are a fundamental task in the researchers' job description. Thus, the need for the provisioning of structured services from the TTO and other supporting offices, and the structuring of the connection between these different organizational units. 

Even more, in private research organizations transfer activities should emerge from institutional activities, as in marketing and business development activities: planned, as part of the normal daily operations. Surely, researchers might still be activated by the request of the TTO and vice versa, depending on who receive a request from external entities; again, scientists might still propose commercial activities based on unforeseen new technologies developed. However, the organization's economic survival should not rely on these spontaneous triggers, but on more institutional, structured and planned mechanisms.

In this perspective, a continuous communication and a structured relationship between research and transfer units is necessary for the success of a development-based organization. There is the need for policies that describe which actions should be taken, and by whom, when a request comes from a potential customer, the consequent line of actions, responsibilities, channel choices, mechanisms for market development, and alike. Even in the case of an \enquote{emergent} transfer, it is essential to picture it immediately into the landscape of the entire organization's activities and external relations, in order to take full advantage from the synergies that may arise. 

In fact, the management of different units and offices as relatively independent might be misleading and jeopardize the profitability of the firm. The organization may still survive if processes are managed in complete isolation, but it will generate inefficiencies that in the long run could affect the competitivity of the research center. Apart from the suggestion the management literature make toward the complete exploitation of internal resources, another important reason resides in the type of industrial knowledge that is mostly profitable nowadays: in an era in which single knowledge fields have grown exponentially, the cross-contamination of ideas has become more and more important. 

Markets that relies on the convergence of multiple fields have become more and more profitable. Examples are the diffused need for (ergonomic) design, the economic value of hybrid projects that includes both hardware and software, the competitivity of high-tech products based on traditional knowledge and new technologies. Similarly, a research center should focus on distinctive capabilities and knowledge, including fields which require two or more knowledge fields that the organization may master. 
 
Another general policy worth considering is the writing or improvement of the mission statement; as found by \citet{Fitzgerald2015}, they are often underestimated. In the case of a private environment, a clear and useful mission statement should include the description of the orientation toward the market, fields of interests, vision and main strategy, and alike. Private research organizations, still an economic movement limited in its extent, should never be confused with a public or academic environment, but the public understanding of the difference should not be taken for granted. 

Lastly, specific policies might aim to single teams and office (research units, TTO, HR, administration, legal office etc.) or a process, which encompass more units along the same channel or transfer mechanism. In the first case, useful policies might consider to implemented HR policies, especially evaluation, reward and promotion practices tailored for a research organization. Other examples may be directed toward the internal structure of the TTO, its process flows and its human resources.

Other policies might regulate the single transfer channel or process, i.e. spinoff, patents and licensing, contract research. In this perspective, the private organization and the university do not differ: they both need clear, understandable policies to manage each channel; any misunderstanding from researchers may decrease their motivation and effectiveness. Again, policies should establish which internal actors are involved, their tasks, responsibilities, process flows, legal management etc.
