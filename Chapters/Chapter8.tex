% Chapter 1

\Chapter{Discussion and Conclusion}{} % Main chapter title

\label{Chapter8} % For referencing the chapter elsewhere, use \ref{Chapter1} 

After the literature review, \hyperref[Chapter6]{Chapter 6} presented the most significant differences between the Univerity-Industry and the Business-to-Business approaches to the technology transfer. The previous \hyperref[Chapter7]{Chapter 7}, instead, reported the significant case of a private research organization. Bearing in mind the theoretical differences and the real-world phenomenon, this chapter will initially discuss the case study, by highlighting three major positive and negative arguments. Lastly, will be drawn the final conclusions, extending the case discussion to an entirely private environment.

\section{Case study discussion}

The previous chapter exposed the case study of the Bruno Kessler Foundation, highlighting the main organizational structure, the Technology Transfer Office, research products, and the transfer processes. This section will provide a qualitative evaluation of the FBK overall administration, management, and performances, focussing on the areas in which the Foundation excels (the pros), and the weakness of their business model (the cons).

\subsection{Pros}

By all means, the most evident and positive result of FBK is the research performance. As seen in \hyperref[Chapter6]{Chapter 6}, the social network analysis of the European system of innovation shown a significant and positive performance, easily framing the institution as one of the top 200 research organizations participating in FP7 projects, even considering the competition from research centers of national and European levels. According to \hyperref[Chapter2]{Chapter 2} and following the \hyperref[Chapter6]{Chapter 6}, the source of the positive performance can be traced back to two main factors.

The first, and most obvious, is the quality of the human resources employed. The performance level heavily relies on the competencies and skills of the researchers, their personal networks, and reputation. Eventually, the organizational success can be seen in the efficacy of the scouting and selection process, among which stands out the ability to attract new researchers. This may constitute one of the most significant and defendable competitive advantages of the organization. 

The second factor refers instead to the ability of the Foundation, in its entirety, to constitute an organizational environment and culture that actively support researchers in achieving their best performance. Therefore, the overall evaluation of the structure, hierarchy, reward and promotion policies and alike is, in fact, positive. However, should be noted immediately that the organizational factors that foster the research performance may influence negatively transfer and commercial activities, as later discuss.

Another positive result that should be highlighted is the impact of the institution on its local context, through various modalities. The first and most evident is the impact of a highly skilled workforce, the creative class of \citet{Florida2002}. Effects range between informal and involuntary knowledge spillovers, attractiveness for other valuable human resources, higher payroll and their impact on the local economy. 

THe second element of local impact is the number of spin-offs generated over time. Specifically, FBK spun off 17 new ventures in the last decade, each of them exploiting internal research.\footnote{The list can be found at \href{https://techtransfer.fbk.eu/startup/}{techtransfer.fbk.eu/startup}}
Again, the spin-off impact can be evaluated in a number of mechanisms: the diffusion of technologies in the industry first and the market later, the knowledge impact on the local context, the number of employees, financial flows brought to the Trentino economy, the ability to attract external companies or furthering the economic performance of pre-existent local firms. 

As for the research, also the spin-off performance should be considered positive, since the Foundation outperforms many Italian public universities and research centers. However, the quantitative economic assessment of their impact is not currently available, nor a comprehensive evaluation model has been developed, and further studies should be performed on the topic.

A third area in which the Foundation excels is in balancing three levels of networking activities: local, national and international. Firstly, an example can be extracted directly from the data on the various contracts (including grant consortia) that FBK signed with companies and research organizations.\footnote{Examples can be found at \href{https://techtransfer.fbk.eu/story/}{techtransfer.fbk.eu/story}} Among the first 44 partners, selected by the monetary value of the relationships are:

\begin{itemize}

\item In the local context: OptoElettronica Italia, Informatica Trentina, Spaziodati, PerVoice, AdvanSiD, Edmund Mach Foundation, Muse Science Museum, CaRiTRo Foundation, Create-Net, EIT ICT, Trentino Innovation Hub (former TrentoRISE), University of Trento, Marconi Institute and the local Health Board;

\item In the Italian context: SunEdison, Engineering Ingegneria Informatica, Environment Park, RF MicroTech, Thales Group, LFoundry, Telecom, Cariplo Foundation, National Institute of Nuclear Physics, National Research Council, University of Padova, Verona, Milano, Milano-Bicocca, Pisa, Rome "Tor Vergata", Politecnico di Milano, di Torino;

\item In the European context: ST Microelectronics, University of Leuven, RWTH Aachen University, Delft University of Technology, German Research Center for Artificial Intelligence, SAP, Fraunhofer Society for the Advancement of Applied Research, European Space Agency, French Alternative Energies and Atomic Energy Commission,  CERN, French National Center for Scientific Research, University of Catalunya, University of Valencia.

\end{itemize}

While this balance may be seen as trivial, it acquires relevance once considering the main modality through which FBK aims to achieve its mission of local development: by connecting the main local innovators to the national and European research network. The nature of research activities greatly influence the business model of research organization, especially for the key partners, and the scale of operations; thus, the largest part of the relationships maintained by these institutions may be expected to be on an international level. 

FBK instead demonstrated to possess the necessary ambidexterity to handle the research side of its activities at the national and international levels, while managing the national and local relations with firms and other institutions. More interestingly, this balance seems to have emerged autonomously from the daily operations, without the intervention a specific and structured policy. Thus, the natural conclusion is that, given the original mission, the organizational climate and culture, autonomously and without the influence of internal and external policy-makers, precisely fit the objectives of the organization, and successfully shaping the daily processes.  

\subsection{Cons}

While the research performance can be evaluated as positive, the first issue that arises from the study of the FBK's business model is financial in nature, recognizable as a threat in a SWOT analysis. This issue can be recognized in the latest income statement, which reports a total income of 44.5M euro: 30M are provided by the Autonomous Province of Trento (70\%), the research grants account for 9.4M (21\%), while industry funds represent the 5\%, with 2.3M.\footnote{The report can be found in \href{https://trasparenza.fbk.eu/Bilanci/Bilancio-preventivo-e-consuntivo}{the institutional website}}

While the statement reports a positive balance, it also expresses a major concern: the financial dependency on the funding from the local government. In neutral perspective, the dependence arises from the historical luggage of the organization, previously a public institution; the actual situation can also be considered an advancement from the initial starting point, as well as a natural condition for a non-profit foundation. However, regardless its internal strengths, FBK truly depends on a financial flow granted from an external entity: without the public funds, the institution will face a considerable resize of its structure, both in term of activities and workforce, including a fundamental change in its business model.

The nature and extent of the financial risks embedded in the external dependency could be evaluated through the assessment of the local policies and government. While this evaluation is left to the proper literature, a significant observation can be made. In \hyperref[Chapter2]{Chapter 2} has been described the government approach to the universities' administration, mostly pointing toward a greater responsibilization for their incomes, the self-financing and the economic valorization of public-financed research. Since this shift, the same approach may be expected to take place in the relationship between the Foundation and the Autonomous Province of Trento.

In fact, the change has already begun: the public funds exhibit a slight decrease from 31.5M (2011) and 30.5M (2012-2013) to 29.5M (2014), and the perspective is of further and generalized reduction. Therefore, the safest choice may be to foster internally the research commercialization and the self-financing to anticipate the decrease in external funds. In other words, to reshape autonomously the research orientation toward a more commercial approach, according to the internal culture and environment, rather than being forcibly driven by the lack of funds and struggling in the path. Eventually, a suitable role for FBK in the most long-term perspective on the local innovation system, may be the one of a non-profit organization that primarily perform research for the market, with a positive influence on its context arising as a positive externality of its activities.

This last perspective exposes the next issue: the influence of the local government on the mission and activities of the Foundation. Today, the two main objectives are the scientific excellence and a positive impact on the local context, the result of both its history combined with the newest local policy. This public mission may be a perfect fit for a truly public institution, but it should be re-interpreted for the particular case of an organization that, regardless its legacy and the non-profit form, has clear financial and economic requirements for its survival in the open market. 

The original mission required FBK to perform both basic research and applied research. The first, fundamental to the business model, dedicated to gain a reputation in the European research network; the last, as a secondary or residuary activity. This perspective has been further shaped from several interviews inside FBK, which highlighted the perception of employees of the Foundation as a research organization that works within a pure open science system - as in a truly public institution. 

While the predecessor ITC was, in fact, a public institution that can afford this model, FBK is a private organization, that must bear with the economic requirements for its survival, needs that acquire a stronger relevance in the perspective of reducing public funds. Therefore, may be advisable an organizational shift, additional and incremental in its evolution, toward a greater orientation to the market; the main objective should be the increase of industry funds, while decreasing the dependency on public funds, eventually fostering the probability of economic survival of the foundation.

The modalities through which implement this shift lead us to the last argument. The question is on what, and how, needs to be modified in order to gain a better commercial performance. In the previous section, has been highlighted how the organization is capable of handling both the research and the industry side of its external relationships; however, the ability does not ensure a greater performance, nor the lack of space for improvements. Specifically, a positive outcome of this ability is the need for a development and evolution that is incremental in nature, rather than disruptive, possibly easier.

The first area for possible interventions arises from interviews among TTO officers, who exposed various difficulties in the disclosure eliciting and, more generally, in gathering information on the actual research interests and topics of internal scientists. These activities actually rely on the pre-existent network of relationships that officers have already developed in the organization, an alternative which exhibits a degree of inefficiency and unreliability. Three levels of solutions may be suggested: the improvement of the organizational culture, the enforcement of internal policies for disclosure eliciting, and the restructure of the TTO.

On the organizational culture, the main objective of any change should be to clarify, communicate and foster, both internally and externally, the institutional attitude and orientation toward commercial activities rather than the open science system. Specifically, the nature of private foundation should be clearly communicated, including the need and willingness to perform research for the market, while the basic research should be seen mainly as a competitive advantage, a tool for building new knowledge to later transfer to customer firms. The positive impact on the local context, indeed, will be the strong but still an externality rather than the primary focus. 

Intervention on policies may start from the mission itself, to give institutional relevance to the message stated above. Additional and more specific policies may involve the obligatoriness of disclosure, the structuring of a system to codify and share information about current research projects, the establishment of more formal moments of exchange between researchers, and between them and TTO officers.

Thirdly, policy interventions may consider a major change in the organization and structure of the TTO, especially toward a decentralized model. This shift should entail the establishment of bridging position in every research unit, which act as a TTO representative: being this officer already embedded into the research unit, the disclosure should be ensured, while main tasks should include the structuring of commercial ideas, the management of the technical content of ongoing relationships, and alike. The decentralized structure implies the ability to take advantage of the central TTO's specialization for legal activities, marketing operations, and similar processes that entail scale economies.

In fact, the second issue highlighted by interviewees is the need for a more extensive and effective commercial structure, especially for marketing activities and any process related to the seeking for customers and the relationship management. Examples are market analyses, the contact of potential clients, commercial communications and advertisement, and alike. 

The obvious solution is to hire specialized employees, but this solution is incomplete: the inclusion of new personnel will entail the need for reconfiguring the organizational structure and the process flows, especially to support the efficacy of the new activities and a suitable connection between them and the original organizational structure. The enlargement of the commercial structure may be coupled with the shift toward a decentralized model, by operating on both levels, fostering the effectiveness of the policy intervention.

\section{Conclusions}

The discussion of the case study leads to conclusions in three main areas: the organizational mission, culture, and structure. The original question was on whether, and how, the technology transfer between private organizations differ from an university-industry setting. From \hyperref[Chapter2]{Chapter 2} to \hyperref[Chapter5]{Chapter 5} has been reviewed the literature concerning the university-industry transfer, and in \hyperref[Chapter6]{Chapter 6} an attempt has been made to draw some differences between systems, which however constitute only a preparatory work for the following conclusions. Finally, in \hyperref[Chapter7]{Chapter 7} has been presented the case of a private research organization, publicly-funded but still a private organization closer to the market than a public institution as the university. At the beginning of this chapter, has been discussed the case study, highlighting different positive and negative factors, laying the ground for some conclusion on a larger scale, here presented.

\subsection{Organizational mission}

The first conclusion regards the centrality of the organizational mission and its perception. Discussing the case study, the mission emerged as the factor that influences the most the approach to the technology transfer and its relative performance, the largest differences between public and private research institutions. While it may seem trivial, this idea can be further decomposed into more interesting elements.

The starting point is the consideration that the various kind of research organizations can be represented on a continuum, between a public university with a complete adhesion to the open science approach, and a research organization that performs exclusively commissioned research. In different ways, both extremes deliver only a few technologies to the market: while this university focus only on non-commercial research, an organization that performs applied research exclusively on contract does not have the ability to build through basic research a sustainable and competitive knowledge base. In other words, if a private research company does not perform at least a minimum basic research, in the mid- and long-term it will not have the knowledge necessary to perform the commissioned research.\footnote{A solution may be the continuous hiring of researchers with new, cutting-edge knowledge; however, this practice will lead to a turnover that suggests to consider this organization as an intermediary between the researcher and the market, rather than a research organization. While these organizations should be studied, they are not the subject of this thesis.}

In the continuum between these two extremes, lie a discrete number of alternatives in the modalities through which an organization can perform and deliver research to the market. Two specific factors should distinguish them the most: the original mission and the organizational structure. Therefore, the most influencing individual factor should be expected to be the mission, as in the overall objective that guides direction and activities, while the structure represents the "supporting environment" that will allow the organization to reach these aims. 

The case study enforces this perspective. Similarly to a public university, FBK does have a public mission: the development of the local economy to further the local society. Similarly to a private research organization, the mission also concerns the delivery of new technologies to the market. The original statue seems to put more emphasis in the first group of objectives, therefore the organizational effort is greater in accomplishing them. On the other hand, following a historical perspective, the balance of these activities has changed over time, toward a more market-oriented mission.

Bringing this set of objectives in continuum previously stated, the Foundation exhibit a distance both from the \enquote{public mission} and a \enquote{private mission}, therefore indicating the discrete number of shades between the two extremes. In a historical perspective, can also be observed how the position of FBK in this line has changed over time, following the need for different approaches for different times and level of local development. The direct consequence is that among alternatives, the mission or group of objectives may seamlessly float to fit the market needs and the local context and society, while balancing short-term applied with long-term basic research for a sustainable advantage.

Mapping to the same space the picture that the economic literature made of public universities, they seem less flexible in changing and adapting to the newest needs, both in time and scope. The public mission of a university, mostly aimed at the historical objectives of basic research and teaching, obstacles the implementation of the third mission, the commercialization of research products, resulting in a limited capability in adapting to the ever-changing economic context, especially in a timely fashion.

Therefore, the factor that differentiates the most universities and private research organizations is the organizational mission. Beyond the obvious differences in their relative objectives, private research institutions exhibit a greater flexibility and adaptability to the economic contest. Eventually, this difference influence also the approach to and the performance in technology transfer. Faster and lighter private organizations achieve the best in rapidly changing contexts, as in the role of applied researcher in the knowledge economy, while massive and slower institutions like universities perform the best in research that requires more time, effort and a more stable environment.

Different kind of organizations are best suited for different tasks. The suggestion is that the contents of the mission, the historical condition that led to it, the ability to change and adapt it, should be a fundamental element in discriminating the activities and the research that should be performed by a private or a public institution, thus the available technology to transfer and the commercial performance. The outcome is the need to further investigate how the differences in content, importance, and adaptability of missions influence the performance in technology transfer among types of research activities.

\subsection{Organizational culture}

The case study highlighted how the ensemble of researchers and officers inside FBK, as in an organic view of the organization, seamlessly and emergently adapted to the change in the organizational mission. As previously stated, despite the presence of specific policies, i.e.\ for the patent and spin-off process, they are usually completed by the emergence of organizational practices tacit in nature, surrounded by a network of informal relationships and sustained by flows of knowledge, information, suggestions, opinions. 

In this perspective, the organizational culture of private research firms seems to differ greatly from the academic culture and environment. \hyperref[Chapter2]{Chapter 2} reported how universities might struggle in shaping and spreading an entrepreneurial culture toward a more positive attitude for commercial activities. Private research organizations, instead, do not suffer from this issue: given a commercial-friendly mission, the FBK's case demonstrated how in private organization the culture autonomously embrace, to some extent, commercial activities and technology transfer. 

Clearly, the cultural difference entails distinct management practices: whereas universities may need to undertake prolonged and widespread activities for changing the academic culture, the private organization will need more subtle and narrow actions. However, while the culture may be already commercial-friendly, firms are still required to govern its development, in order to employ it as a positive driving force. Universities, instead, may be more involved in initiatives suitable to start the culture shift. Therefore, different backgrounds and cultures entail a change in focus for initiatives and management practices, including objectives, mechanisms, and tools.

The main implication of this perspective is the need for the top management of private entities to (1) establish a clear and proper mission, compatible with the market conditions and the local context, and (2) set up and lead activities for shaping and orienting the evolution of the organizational culture. Specifically, the mission sets the foundations, policies sketch out processes and mechanisms, while the fine tuning, which eventually actively contributes in determining the final performance in both effectiveness and efficiency, may be the organizational culture.

The consequence is a greater importance of the human resource management, especially in the selection process for new hires. Specifically, they must be selected not only for their academic achievements, the crude knowledge, but also and foremost for other basic competencies: the attitude toward the market, the ability to recombine previous knowledge, the willingness to work with and for external firms, the attitude to work with colleagues in different fields, and eventually to promote an informal, continuous flow of information, opinions, ideas, projects. Later, these competencies must be fostered, through training, exchange and alike, through the provision of organizational tools suitable for promoting the further development of the entrepreneurial culture. 

By and large, the management of a private organizational culture may borrow some practices from the fostering of an entrepreneurial culture in universities, once adapted for a starting point that greatly differs. Similarly, a research organization may adopt several instruments from the literature of innovation management in private (production) firms, once reconfigured for the difference in activities and processes. Further research is needed to precisely identify the tools that these different realities may share.

\subsection{Organizational structure}

After the differences in organizational missions and cultures, the next step is the organizational structure, which is required to follow this new pattern. The \enquote{force} brought in the process by the culture need to be reined and oriented toward the aims stated in the mission. The most suited tool is the structure: it must be compatible with the culture and objectives, efficient and effective, able to drive and foster the organizational momentum. 

Needless to say, the first difference between the university and private organization structures is the presence of a structure for teaching activities, and a different size and hierarchy of the basic and long-term research structure. Instead, a research firm will need a larger and structured department for applied research and the various commercial support offices. In fact, in a private environment, an obvious suggestion is to expand the commercial component of the structure over the single TTO, including marketing and communication, market and business developer, patent attorneys and lawyers.

Another significant change may be the decentralization of the TTO. This model requires the presence of an officer in every research team or unit, in charge for: keeping up with researchers' projects; provide them consulting services; maintain the communication with the central TTO; seek for potential customers; manage the ongoing relations for technical contents. At the same time, the central office should retain only the tasks that entail scale economies or specific competencies and knowledge, i.e.\ contractual and patent support, marketing and alike.

Significant differences also arise from the hierarchy and the leadership positions in the research structure. The leadership should be stronger, advocating for commercial activities, with a comprehensive perspective both on the internal competencies and the external technology trend and market needs. Leaders should also foster the usage of institutional moments of exchange and cross-contamination of ideas, as well as endorse researchers in the development of their own projects of technology transfer.

A specific topic is the evaluation of private researchers, which should be performed through other means with respect to the university environment. Indicators may include the number of active and successfully achieved projects, weighted for their challenges and the relative difficulty, the number of projects proposed to the TTO, and alike. The quality of the produced research should also be included, but it should be considered secondary, in order to urge scientists to commercial activities. Private organizations should also be easier to take advantage of a 360-degree evaluation, including team leaders, colleagues, customers, and especially the TTO. 

\section{Concluding remarks}

The chapter began with the discussion of the case study. Firstly, has been reported three positive arguments: the scientific performance in the European innovation network, the economic impact on the local context and innovation system; and the ability to balance scientific and commercial activities at the international, national and local level. Later, has been discussed three major negative traits of the Foundation: the high financial dependency on public funds; the relative higher importance of basic research over commercial activities; the need for fostering the entrepreneurial organizational culture. 

The second section proposed various conclusions along three areas. Firstly, the institutional mission seems to have the greatest influence on the performance in technology transfer activities; specifically, private firms appear to be more competitive, due to their flexibility in adapting to the newest needs of the market and the local context. Subsequently, the organizational culture seems to be the second most influential factor in shaping the overall performance, and differentiating the public and private environment; while private organizations may have a fundamental advantage, the internal culture still needs to be managed and directed toward a higher entrepreneurialism. Thirdly, the organizational structure, already one of the largest difference between the institutions, follows as a fundamental enabling factor for a competitive performance; thus, the motivation for its proper configuration and management, aimed to foster the entrepreneurial culture and direct the organizational efforts toward sought objectives.

With respect to the differences in approaches described in \hyperref[Chapter6]{Chapter 6}, the case study reinforces the emerging need to balance basic and applied research activities in both the environments, while it acquires a new significance among private organizations. It also seems to support the hypothesis of different requirements in the kind of researchers that institutions need the most, but the particular case of FBK cannot provide definitive indications due to the peculiar organizational mission and the interference of the local government.

However, within the institutional factors, the case study provided insights on the relative importance of the organizational mission, rather than the ensemble of activities that the institution may perform. Moreover, the ability of the mission to shape the internal culture and provide an entrepreneurial culture seems to be the single most influencing factor on the transfer performance. The case study eventually exposed the need for a proper communication of the mission, in order to gain the maximum impact on the performance. 

Among the channels, the case study did not reveal significant or drastic changes in the mechanisms and processes, apart from the obvious impact of the organizational structure. However, it exposed a fundamental difference in the preference and choice of channels: while the economic literature on the technology transfer mostly described licensing and spin-offs as channels, FBK demonstrated the relative importance of research contracts and cooperative research, which constitute the primary mechanisms for the transfer of both knowledge and technologies.

Lastly, the evaluation has been proven to be as challenging as expected. While the case study provided the opportunity to apply the social network analysis to the assessment of the performance effectiveness and test its applicability, the results seem to be partial and incapable of fulfilling the need for a comprehensive evaluation of the entire impact of a research organization in its context. 

At a glance, this work has provided several hypotheses on the differences that may occur between the university-industry and the business-to-business technology transfer; the hypotheses have been compared with a case study, which exposed similar, but not identical insights. Eventually, a sizeable difference between approaches seems to exists, in both the individual organization and the relative usage of the various transfer tools. At the same time, the conclusions suggest that a few topics, among which the very mechanisms and processes, do not exhibit a wide difference as may be expected. 

This work cannot be considered complete, nor the topic depleted. The objective was to provide an initial insight on the question at hand, aimed to understand if the issue was worth the effort. Indeed, given the rising knowledge economy, and the existence of sizeable difference, further research is advised. Specifically, areas for further development are: the measurement of the impact of the organizational mission on the technology transfer, in driving its undertake and performance; the most significant factors of the organizational culture in shaping the attitude and performance of private research firms; the seek for a performant organizational structure for R\&D activities in research- and transfer-based firms; the build of an evaluation model capable of capturing the overall effectiveness and the impact of the individual organization on the local context. 