% Chapter 1

\Chapter{Discussion and conclusion}{} % Main chapter title

\label{Chapter8} % For referencing the chapter elsewhere, use \ref{Chapter1} 

After the literature review, \hyperref[Chapter6]{Chapter 6} presented the major differences between a Univerity-Industry and a business-to-business approach to the technology transfer. The previous \hyperref[Chapter7]{Chapter 7} instead reported the significant case of a private research organization, including the analysis of its organizational structure, processess and performance. "Bearing" in mind the theoretical differences and the real-world phenomenon, this chapter will initially discuss case study, by highlighting the three major pros and cons. Lastly, the chapter will draw the final conclusions, "espandendo" the publicly-funded case discussion to a fully private environment.

\section{Case study discussion}

The previous chapter exposed the case study of FBK, highlighting the organizational structure, expecially regarding the Technology Transfer Office, research products and transfer processes. This section will provide a qualitative evaluation of the FBK overall administration, management and performances, focussing in the areas in which the Foundation excels (the pros), and the weakness of their business model (the cons).

\subsection{Pros}

The most "evident" and positive result of FBK is by all means its research performance. As seen in \hyperref[Chapter6]{Chapter 6}, the social network analysis of the European network of innovation "evidenziato" a significant and positive performance, easily framing the institution as one of the top 200 research organizations participating FP7 grants, even considering the competition from national- and European-level research centers. According to \hyperref[Chapter2]{Chapter 2} and partially \hyperref[Chapter6]{Chapter 6}, the source of the positive performance can be traced back to two main factors.

The first, and most obvious, is the quality of the involved human resources. The performance depends on the competencies and skills of the researchers, as well as their personal networks and reputation. Eventually, the organizational success can be seen in the efficacy of scouting, "recall" and selection of new researchers to employ, among which the ability to "recall and interest" figures as the most significant and defendable competitive advantage. The second factor refers instead to the ability of FBK, in its entiretly, to build an organizational environment that actively support these researchers and enable them to perform the better. Therefore, the overall evaluation of the organizational structure, reward and promotion policies and support activities for research purposes is in fact positive.

Another positive result that should be highlighted in this case study is the impact of the organization on its local context. Apart from the obvious impact on the local society through its workforce, a relevant factor is the number of spinoffs generated over time. Specifically, FBK spinned XX new ventures in the last 5? 10? years\footnote{A list can be found at \hyperref[techtransfer.fbk.eu/spinoff]{techtransfer.fbk.eu}}. As required by the relative policy, every spinoff has as business model the commercial exploitation of internal research, thus furthering the diffusion of these technologies in the market. The spinoffs impact can be evaluated for their employees, the financial flows "portati" to the Trentino context, the ability to attract external ficompaniesrms or furthering the economic performance of pre-existent, local firms. By and large, the spinoff performance is positive, "superando" most of the Italian public universities, and their impact on the local context is worth noting. Further studies should be performed on the topic, to assess the value of this impact.

A third topic in which this case excels is in bilancing three levels of networking activities: local, national and international. An example can be extracted directly from the data on the contracts that FBK signed with companies and research organizations: among the first 53th partners, selected by monetary value of the relationships are:

\begin{itemize}

\item In the local context:

\item In the national context:

\item In the european context:

\end{itemize}

While this balance might be seen as trivial, it acquires importance considering the main modality through which FBK "punta a" achieve its mission of local development: by connecting the main local innovators to the European and national research network. By the nature of research activities, thus the business model of research organization (expecially at this scale of operations), the most part of the relationships maintained by such institutions might be expected to be of international level; instead, FBK demonstrate to possess the necessary ambidexterity to handle the research side of its activities at an international and national levels, while managing the national and local relations with firms and other institutions. Moreover, this balance seems to have emerged autonomously from the daily operations, without requiring a specific and structured policy. this means that the organizational climate and culture, alone and without "external" influences from specific policies, precisely fit what is needed to be done.  

\subsection{Cons}

While the research performance can be evaluated as positive, the first issue that arises from the "analysis" of the FBK's business model is financial in nature, that can be recognize as a threat in a SWOT analysis. This issue can be recognized in the latest "conto economico", which report that on the total 45M income, XXM are provided by the Autonomous Province of Trento (xx\%), the incomes from research grants account for XXM (\%), while industry funds "ammonta" at XX, the XX \%.

While the report "express" a positive balance, it also "rivela" a major issue: the financial dependency on public funds. Taking a neutral perspective on this fact, this dependency arises in fact from the historical luggage of the Foundation, previously a public institution. Following this point of view, the actual situation exhibith a "miglioramento o progresso" from the initial starting point. 

However, FBK truly depends on founds "concessi" from an external actor, instead of "counting" on its internal strengths and market incomes. Without the public funds, the institution will "affrontare" a "great" resize of its organization, both in term of activities and workforce, including a fundamental change in its business model. 

The financial risks embedded in the external dependency could be evaluated through the stability and reliability of the provider, the local government. However, without dig into the evaluation of local institutions and policies, a major observation can be made. in chapter 2 has been described the governmental approach to the universities' administiation, pointing toward a greater responsibilizaiton for their incomes, pushing for the auto-financement and the economic valorization of public-financed research. since this shift in perspectives, how long would it take for the PAT to follow the same path with FBK, and resizing their incomes?

in fact, the shift is already taking place: the public funds decrease from a XX (2000) to XX (2500) to XX (3000). at this point, the safest choice may be to push hard on the research commercialization and auto-financement, anticipating the decrease in external funds, shifting the research orientation toward a more commercial approach rather than being forcebly resized by the lack of funds. In the most long term perspective, the best case scenario may picture FBK as a non-profit organization that makes research for the market, with the positive influence on the local context arising as a positive externalities of its activities, rather than sacrificing economic and financial resources for "getting there now".

This last perspective in fact expose the next issue: the influence of local public institutions on the FBK's mission first and its activities later. remember the two main missions of the Foundation: the scientific excellence (for) a positive impact on the local context, the result of its legacy combined with the newest local policy. perfect fit for a 100\% public institution, but they must be interpretated in the case of an institution that, regardless its form as non-profit, fund, private firm, etc.: a private organization cannot "prescindere" from the economic and financial needs required for its survival in the open market. 

The originial mission pushed FBK to perform both basic research and applied research, with the first primary in importance, dedicated to gain a reputation in the european research network, the last as a secondary or residuary activity. this idea has been mutuated form several interviews inside FBK, with highlighted the perception of employees of the organization: a research organization that works with pure open science - the description of a public institution. 

However, while the old ITC was in fact a public institution that can afford this model, the new FBK is a private organization, that must bear with the economic survival. expecially in the perspective of reducing public funds. therefore, it may be "suggeribile" an additional shift toward a greater orientation to the market, with the aim of increase the industry funds, decrease the dependency on public funds, and evenutally ensuring the survival of the foundation.

The modalities through which implement this shift lead us to the last argument. what, and how, needs to be modified? in the previous section, has been stated how the organization is capable of handle both the research side of its external relationship management, and the industry relationships; however, being capable of does not means that the activities cannot be improved or does not need an improvement. Specifically, if the organization is already capable of, the needed change is incremental rather than disruptive, surely an easier approach.

First, interviews among tto officers highlighted the difficulties in disclosure eliciting and, generally, to gather information on the actual research interests of internal researchers, activities that now relies on the pre-existent network of relationships that an officer have in the organization. two levels of solution are available: the improvement of the organizational culture and the policy enforcement to disclose. a third way involve the restructuring of the tto.

For the organizational culture, the main area is to clarify, even internally, that the organization is not a public research organization that follow exclusively the open science approach. must be clearly communicated that fbk is a private foundation, that perform research for the market, for which the basic reesarch is only a tool for building the knowledge that will be later transferred to the market, and the positive impact on the local context will be the strong but still an "esternalità". 

The starting point for the intervention may be the mission itself. more specific policies instead may involve the obligatoriety of disclosure, i.e. the structuring of an information system that will require, as mandatory, to upload the informations about current research projects to a unique information source, available to tto employees (and eventually to the external industries). More formal moments of exchange between researchers and tto may be institutied.

eventually, can be considered a major change for the tto, expecially the shift toward a decentralized model. this will inlcude the establishment of bridging position in every research unit, which act as a tto representative. being him already into the research unit, the disclosure should be ensured; main tasks should be the structuring of commercial ideas, the management of technical content of ongoing relationships and alike, while taking advantage of the central tto's specialization for legal, marketing and similar works.

the second issue highlighted in interviews is the need for more extensive and useful marketing activities, especially any activity incuded in the seek for customers: market analysis, contacting the potential customer, commercial communication, advertisement and other communication etc. 

the obvious solution is to hire new specialized people. however, this solution is in fact partial: there will be the need to reconfigure a portion of the organizational structure to include and support the efficacy of these new hires. thus, after the institution of these specialized offices, there will be the need for connecting them properly to the organization, especially redesign the process flow internal to the tto to include these activities.

\section{Conclusions}


conclusioni

primo fattore: l'influenza dalla mission e la sua percezione

secondo fattore: la struttura organizzativa, soprattutto la presenza e dimensione di TTO e organi commerciali

terzo fattore: l'impatto pubblico, da public a publicly-funded a private

quarto fattore: cambiamenti nei processi

quinto: cambiamenti nella valutazione 