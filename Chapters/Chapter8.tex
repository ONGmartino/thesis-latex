% Chapter 1

\Chapter{Discussion and conclusion}{} % Main chapter title

\label{Chapter8} % For referencing the chapter elsewhere, use \ref{Chapter1} 

After the literature review, \hyperref[Chapter6]{Chapter 6} presented the major differences between a Univerity-Industry and a Business-to-Business approach to the technology transfer. The previous \hyperref[Chapter7]{Chapter 7}, instead, reported the significant case of a private research organization, including the analysis of its organizational structure, processes, and performance. Bearing in mind the theoretical differences and the real-world phenomenon, this chapter will initially discuss the case study, by highlighting three major positive and negative arguments. Lastly, the chapter will draw the final conclusions, extending the case discussion to a fully private environment.

\section{Case study discussion}

The previous chapter exposed the case study of the Bruno Kessler Foundation, highlighting the main organizational structure, the Technology Transfer Office, research products, and the transfer processes. This section will provide a qualitative evaluation of the FBK overall administration, management, and performances, focussing in the areas in which the Foundation excels (the pros), and the weakness of their business model (the cons).

\subsection{Pros}

By all means, the most evident and positive result of FBK is the research performance. As seen in \hyperref[Chapter6]{Chapter 6}, the social network analysis of the European network of innovation shown a significant and positive performance, easily framing the institution as one of the top 200 research organizations participating in FP7 projects, even considering the competition from research centers of national and European levels. According to \hyperref[Chapter2]{Chapter 2} and following the \hyperref[Chapter6]{Chapter 6}, the source of the positive performance can be traced back to two main factors.

The first, and most obvious, is the quality of the human resources employed. The performance level heavily relies on the competencies and skills of the researchers, their personal networks, and reputation. Eventually, the organizational success can be seen in the efficacy of the scouting and selection process, among which stands out the ability to attract new researchers. This may constitute one of the most significant and defendable competitive advantages of the organization. 

The second factor refers instead to the ability of the Foundation, in its entirety, to build an organizational environment and culture that actively support these researchers in achieving their best performance. Therefore, the overall evaluation of the organizational structure, hierarchy, reward and promotion policies and alike is, in fact, positive. However, should be noted immediately that the organizational factors that foster the research performance may influence negatively transfer and commercial activities.

Another positive result that should be highlighted is the impact of the institution on its local context, through various modalities. The first and most evident is the impact of a highly skilled workforce, the creative class of \citet{Florida2002}. Effects range between informal and involuntary knowledge spillovers, attractiveness for other valuable human resources, higher payroll and their impact on the local economy. 

A second element of local impact is the number of spinoffs generated over time. Specifically, FBK spinned 17 new ventures in the last decade, each of them exploiting internal research.\footnote{The list can be found at \href{https://techtransfer.fbk.eu/startup/}{techtransfer.fbk.eu/startup}}
Again, the spinoff impact can be evaluated in a number of mechanisms: the diffusion of technologies in the industry first and the market later, the knowledge impact on the local context, the number of employees, financial flows brought to the Trentino economy, the ability to attract external companies or furthering the economic performance of pre-existent local firms. 

As the research performance, also the spinoff performance should be considered positive, since the Foundation outperforms many Italian public universities and research centers. However, the quantitative economic assessment of their impact is not currently available, nor a comprehensive evaluation model has been developed, and further studies should be performed on the topic.

A third area in which the Foundation excels is in balancing three levels of networking activities: local, national and international. Firstly, an example can be extracted directly from the data on the various contracts (including grant consortia) that FBK signed with companies and research organizations. Among the first 44 partners, selected by monetary value of the relationships are:

\begin{itemize}

\item In the local context: OptoElettronica Italia, Informatica Trentina, Spaziodati, PerVoice, AdvanSiD, Edmund Mach Foundation, Muse Science Museum, CaRiTRo Foundation, Create-Net, EIT ICT, Innovation Hub Trentino (ex TrentoRISE), University of Trento, Marconi Institute and the local Health Board;

\item In the Italian context: SunEdison, Engineering Ingegneria Informatica, Environment Park, RF MicroTech, Thales Group, LFoundry, Telecom, Cariplo Foundation, National Institute of Nuclear Physics, National Research Council, University of Padova, of Verona, of Milano, of Milano-Bicocca, di Pisa, of Rome "Tor Vergata", Politecnico di Milano, di Torino;

\item In the European context: ST Microelectronics, University of Leuven, RWTH Aachen University, Delft University of Technology, German Research Center for Artificial Intelligence, SAP, Fraunhofer Society for the Advancement of Applied Research, European Space Agency, French Alternative Energies and Atomic Energy Commission,  CERN, French National Center for Scientific Research, University of Catalunya, University of Valencia.

\end{itemize}

While this balance may be seen as trivial, it acquires relevance once considering the main modality through which FBK aim to achieve its mission of local development: by connecting the main local innovators to the national and European research network. The nature of research activities greatly influence the business model of research organization, especially for the key partners, and especially at this scale of operations; thus, the largest part of the relationships maintained by these institutions may be expected to be on an international level. 

Fbk istead demonstrated to possess the necessary ambidexterity to handle the research side of its activities at the national and international levels, while managing the national and local relations with firms and other institutions. More interestingly, this balance seems to have emerged autonomously from the daily operations, without the intervention a specific and structured policy. Thus, the natural conclusion is that the organizational climate and culture, autonomously and without the influence of internal and external policy-makers, precisely fit the objectives of the organization, and successfully shaping the daily processes.  

\subsection{Cons}

While the research performance can be evaluated as positive, the first issue that arises from the study of the FBK's business model is financial in nature, recognizable as a threat in a SWOT analysis. This issue can be recognized in the latest income statement, which reports a total income of 44.5M euro: 30M are provided by the Autonomous Province of Trento (70\%), the incomes from research grants account for 9.4M (21\%), while industry funds represent the 5\%, with 2.3M.\footnote{The report can be found in \href{https://trasparenza.fbk.eu/Bilanci/Bilancio-preventivo-e-consuntivo}{the institutional website}}

While the statement reports a positive balance, it also expresses a major concern: the financial dependency on the funding from the local government. In neutral perspective, the dependency arises from the historical luggage of the organization, previously a public institution; the actual situation can be also considered an advancement from the initial point, as well as considering the dependency as a normal condition for a non-profit foundation. However, independently from its internal strengths, FBK truly depends on a financial flow granted from an external entity: without the public funds, the institution will face a considerable resize of its structure, both in term of activities and workforce, including a fundamental change in its business model.

The nature and extent of the financial risks embedded in the external dependency could be evaluated through the assessment of the local policies and government. While this evaluation is left to the proper literature, a major observation can be made. In \hyperref{[Chapter2]{Chapter 2}} has been described the government approach to the universities' administration, mostly pointing toward a greater responsibilization for their incomes, the self-financing and the economic valorization of public-financed research. Since this shift, the same approach may be expected to take place in the relationship between the Foundation and the Autonomous Province of Trento.

In fact, the change has already begun: the public funds exhibit a slight decrease from 31.5M (2011) to 30.5M (2012-2013) to 29.5M (2014), and the perspective is of further and generalized decrease. Therefore, the safest choice may be to foster internally the research commercialization and the self-financing to anticipate the decrease in external funds. In other words, to reshape autonomously the research orientation toward a more commercial approach, according to the internal culture and environment, rather than being forcibly driven by the lack of funds and struggling in the path. Eventually, a suitable role for FBK in the most long-term perspective on the local innovation system, may be the one of a non-profit organization that primarily perform research for the market, with a positive influence on its context arising as a positive externality of its activities.

This last perspective exposes the next issue: the influence of the local government on the mission and activities of the Foundation. Actually, the two main missions are the scientific excellence (for) a positive impact on the local context, the result of both its history combined with the newest local policy. This public mission may be a perfect fit for a truly public institution, but it should be re-interpreted for the specific case of an institution that, regardless its legacy and the non-profit form, has clear financial and economic requirements for its survival in the open market. 

The original mission required FBK to perform both basic research and applied research. The first, fundamental to the business model, dedicated to gain a reputation in the European research network; the last, secondary or residuary activity. This perspective has been further shaped from several interviews inside FBK, which highlighted the perception of employees of the Foundation as a research organization that works within a pure open science system - as in a truly public institution. 

While the predecessor ITC was, in fact, a public institution that can afford this model, FBK is a private organization, that must bear with the economic requirements for its survival, needs that acquire a stronger relevance in the perspective of reducing public funds. Therefore, may be advisable an organizational shift, additional and incremental in its evolution, toward a greater orientation to the market; the main objective should be the increase of industry funds, while decreasing the dependency on public funds, eventually ensuring the survival of the foundation.

The modalities through which implement this shift lead us to the last argument. The question is on what, and how, needs to be modified in order to gain a better commercial performance. In the previous section, has been highlighted how the organization is capable of handling both the research and the industry side of its external relationships; however, the ability does not ensure a greater performance, nor the lack of space for improvements. Specifically, a positive outcome of this ability is the needed for change that is incremental in nature, rather than disruptive, possibly an easier development.

The first area for possible interventions arises from interviews among TTO officers, who exposed various difficulties in the disclosure eliciting and, more generally, in gathering information on the actual research interests and topics of internal scientists. These activities actually rely on the pre-existent network of relationships that officers have already developed in the organization, an alternative which exhibit a degree of inefficiency and unreliability. Three levels of solutions may be suggested: the improvement of the organizational culture, the enforcement internal policies for disclosure eliciting, and the restructure of the TTO.

On the organizational culture, the main objective of any change should be to clarify, communicate and foster, both internally and externally, the organizational attitude and orientation toward commercial activities rather than the open science approach. Specifically, the nature of private foundation should be clearly communicated, including the need and willingness to perform research for the market, while the basic research should be seen mainly as a competitive advantage, a tool for building new knowledge to later transfer to customer firms; the positive impact on the local context, indeed, will be the strong but still an externality rather than the main focus. 

Intervention on policies may start from the mission itself, to give institutional relevance to the message stated above. Additional and more specific policies may involve the obligatoriness of disclosure, the structuring of a system to codify and share information about current research projects, the establishment of more formal moments of exchange between researchers, and between them and TTO officers.

Thirdly, policy interventions may consider a major change in the organization and structure of the TTO, especially toward a decentralized model. This shift should entail the establishment of bridging position in every research unit, which act as a TTO representative: being this officer already embedded into the research unit, the disclosure should be ensured, while main tasks should include the structuring of commercial ideas, the management of the technical content of ongoing relationships and alike. The decentralized structure imply the ability to take advantage of the central TTO's specialization for legal activities, marketing operations, and similar processes that lead to scale economies.

In fact, the second issue highlighted by interviewees is the need for more extensive and effective commercial structure, especially for marketing activities and any process related to the seek for customers and the relationship management. Examples are market analysis, the contact of potential customers, commercial communications and advertisement, and alike. 

The obvious solution is to hire specialized employees, but this solution is incomplete: the inclusion of new personnel will entail the need for reconfiguring the organizational structure and the process flows, especially to support the efficacy of the new activities and a suitable connection between them and the original organizational structure. The enlargement of the commercial structure may be coupled with the shift toward a decentralized model, by operating on both levels, fostering the effectiveness of the policy intervention.

\section{Conclusions}

The discussion of the case study allow us to draw some conclusions. the original topic was: we know a lot on the technology transfer between U and I, as seen in chap 2, 3, 4, and 5. but do something change in a b2b tech transfer? in the 6th chap an attempt has been made to hypotize some differences, which however are purely theoretic for now. in chap 7 however we presented the case of a private research organizaton, which may be publicly-funded but still an organization closer to the market than a university. at the beginning of chap 8, we seen how the case study highlighted some pros and cons, preparing the ground for some conclusion on a larger scale.

\subsection{Organizational mission}

the first idea is that the mission, and its perception, seem to be the most changing and influencing element. it may be sound obvious but it can be further decomposed.

first, we need to consider the various research organizations as distributed on a continuum between the public university totally on an open science approach, and at the other extreme a research organization that works exclusively on commissioned research. obviously, both extremes do not delivery technologies to the market: let alone the university, an organization that perform applied research exclusively when asked have not the ability to build internally a knowledge and technology base that will provide a sustainable competitive base. that is, is it not perform some basic research, tomorrow the firm will not have the knowledge necessary to perform the research asked. a solution may be to continuously hire new people with the new knowledge, but a similar turnover may suggest to catalogue this firm as an intermediary between the researcher and the market, rather than a research organization. moreover, this firm is not the subject of this thesis.

in the continuum between these two extremes, there should be a discrete number of alternatives in organizations that perform research and deliver to the market. two factors should distinguish them the most: the original mission and the organizational structure. therefore, should be expected that the major influencing factor would the mission, the overall objective that dictate the direction and the activities performed internally, while the organizational structure is in fact a "supporting environment" that allow the organization to reach these objectives. 

the case study enrich this perspective. similarly to a public university, fbk has in fact a public aim: the development of the local society and economy. similarly to a fully private research organizaion, the mission is also to deliver new technologies to the market. the original statue seems to put more emphasis in the first mission, therefore the organizational effort in the first is greater. however, taking an historical perspective, the balance of these activities has in fact shift toward a more market-oriented mission, since the transfromation from itc to fbk. 

Considering that in continuum previously stated, there is not a "public mission" and a "private mission", the case study therefore suggests that the mission may assume a discrete number of shades between the two extremis. direct consequence is that may exists a perfect mission that lead the organization to specifically aim to market needs, combined with the minimum basic research required to gain a competitive advantage in the long-term perspective. 

therefore, the factor that differentiate and influence the technology transfer the most between the U-I paradigm and the B2B paradigm may be the organizational mission. the conclusion in this case is that the perspective on basic open scence approach VS the knowledge economy must be revisited to fully understand the impact on the overall approach, thus the mission, in influencing the performance and the competition modality of a private organization. 

a "traccia" for this further research may be the study of TTO's mission statement from author XXX, while usefull insights may be borrowed from studies on knowledge-based and entrepreneurship and its relation with the innovation management.

\subsection{Organizational culture}

secondly, the case study highlighted how, quite surprisingly, the ensemble of researchers and officers inside fbk, aka the organization, seamlessy and emergently adapted to the organizational mission. as previously stated, "nonstante" the presence of organizational policies on the patent channel, or the spinoff channel, they are usually completed by the emergence of organizational practices tacit in nature, surrounded by a network of informal relationships and flows of knowledge, information, suggestions, opinions: the organizational culture.

in this perspective, the organizational culture of private research firms seem very different from the university culture and environment. in chapter 2, we've seen how universities may struggle in pushing an entrepreneurial culture toward a more positive attitude for commercial activities. in the case of a private research organization, it seems to not be a problem at all: given a commercial friendly original mission, fbk demonstrated how in private realities the organizational culture embrace commercial activities and technology transfer. 

obviously, this cultural difference entail a difference in their management: whereas universities may need to undertake heavy and difficult activities for changing the university culture, the private organization will need more subtle, easier, lighter actions. however, while the culture may be already commercial friendly, its development must be governed nevertheless, to use it as an internal positive force while universities may be more occupied in the initial shape shifting initiatives. thus, a change in focus for organizational culture initiatives and management, including obectives and tools slash instruments.

the main implication of this perspective is that the CEO, top management, founder or who-know-what-acronym, has to establish non only a clear and proper mission, compatible with the local firms, innovation system, government etc, but also to place great attention in shaping and orienting the formation and evolution of the organizational culture. the mission set the "fondamenta", policies shape the overall processes and mechanisms, but the factor that do the fine tuning, and eventually determine the final performance for effectiveness and efficiency, seems to be the organizational culture.

implications are that the hr management becomes of primary importance, especially in the selection process of new hires. they must be selected not only on the academic achievements, their crude knowledge, but also, and especially, for other competencies as the attitude toward the market, the ability to recombine previous knowledge (and fields of knowledge) for the benefit of the market, the willingness to work with and for external firms who may commission the research, the ability and willingness to work with colleagues in different fields, to promote an informal, constinuous flow of informations and opinions, ideas, projects. 

a second order implication is the need for the provisioning of proper organizational environment, especially tools, that promote the development of such entrepreneurial culture. in fact, the mission may represent the "final" objective, the organizational structure and the hr base may constitute the "starting point", policies the channels, the limits for the arrow from the basis to the objectives, but the organizational culture does the heavy work. therefore, the need to provide an environment capable of set the boundaries for the autonomous, emergent development of the organizatonal culture, toward a desiderable form.

further research may be needed on the topic. in fact, the organizational culture may borrow some research from the fostering of an entrepreneurial culture in universities, but the aim is not to convince people to engage economic slash commercial activities: in a private organization, they already are (or should be). the focus should be on promoting cross contamination of ideas, informal exchanges, social support and alike. similarly, something can be taken from the literature on the innovation management in a private firm, but again - the need for reconfiguring it from a 100\% industrial climate toward an organization that may be required to perform basic research and undertake, in a limited extent, an open science approach. 

\subsection{Organizational structure}

after the changes in organizational missions and cultures, eventually the organizational structure will be required to follow these new pattern. using the previous metaphor, the mission is the end of the arrow, the culture the starting point and the general tendency toward the ultimate objective, but the force brought in the process from the culture need to be "imbrigliata" toward the specific point pictured as final objective. the best way to do it is to provide an organizational structure that is not only compatible with objectives and culture, efficient and effective, but also able to foster the original force. to direct, and increase the original organizational force. so, what's the differences?

an obvious starting point is the fact that private research organizations do not need an organizational structure for teaching activities, and the structure for basic and long term research may be greatly reduced in respect to the university. however, a research firm will need a greater appied research structure and the commercial structure. in a private environment, an obvious suggestion is to expand the commercial structure far over the tto alone, including marketing and communication, market developer, business developer, attoreys and lawyers.

another important change may be the decentralization of the TTO, by providing an officer in every research unit or center in charge of (1) keeping up with researchers' projects (1a) provide them consulting services on which projects foster etc (2) talk to the central tto (3) seek for possibly interested external firms (4) manage ongoing relations for technical contents. ideally, the central tto should retain only tasks and processes which involve scale economies or technical and specific competences and knowledge, i.e. patent attorneys, contractual aid, marketing etc.

another topic is the hierarchy and the leadership position in the oganizational strucure of research. the leadership must be strong, oriented toward the commercialization, with a overall perspective both on the internal competencies and the external technology trend and market needs. he must also further moments of exchange and cross contamination of ideas, as well as pushing researchers to grow their own projects for tech transfer and propose them to the tto.

within the research organizational structure, however, the performance may take advantage of a fullly flat structure, with researchers "eventualmente" grouped in thematic teams but with "permeabili" bonduaries between them, even with internal mandatory exchange of personnel between functions. may be useful to have a succesful case study, a succesful entrepreneur or alike, in each team.

researachers must also be evaluated through other means than the university environment. indicators may be the number of successful projects active and recently achieved, weighted for relative difficulty, number of proposal to the tto. obviously also the quality of the research. may be useful a 360 degree evaluation, including their boss, colleagues, customers, tto. 


conclusioni

primo fattore: l'influenza dalla mission e la sua percezione

secondo fattore: la struttura organizzativa, soprattutto la presenza e dimensione di TTO e organi commerciali

terzo fattore: l'impatto pubblico, da public a publicly-funded a private

quarto fattore: cambiamenti nei processi

quinto: cambiamenti nella valutazione 