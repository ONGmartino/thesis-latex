% Chapter 1

\Chapter{Channels}{Mechanisms and processes} % Main chapter title

\label{Chapter4} % For referencing the chapter elsewhere, use \ref{Chapter1} 

Channels constitute the bridge, as in infrastructure, between the academic environment (\hyperref[Chapter2]{Chapter 2}) and the industry (\hyperref[Chapter3]{Chapter 3}). While they all eventually represent a path through which delivery a new knowledge or technology to the market, they may expose various differ between paradigms, for actors and activities involved, contractual forms and bureaucratic issues. Considered their influence on the effectiveness of a transfer and the influence that the originating environment has on them, an overview of the main channels is a necessary condition for a proper evaluation of the differences later described in \hyperref[Chapter6]{Chapter 6}.  

\section{Channels and categories}

A technology transfer channel can be defined as an instrument or a mechanism through which knowledge can flow from the academia to industry \citep{Gilsing2011}. In other words, it is a set of actors and activities specifically configured and aimed to a peculiar objective – the transfer of newly developed innovations, technologies, and knowledge. The specific configuration and the objective, rather than the basic building blocks of actors and activities, are what differentiate channels the most: eventually, the process will involve a research organization and another institution, sharing information through training, meetings, documentation and alike, but with greatly different objectives and organizations. It is important to note, moreover, that channels are not mutually exclusive: they may complement each other, and cover the respective limits.

What is implied is that exists a wide range of possible configurations, and every channel is meant to satisfy a specific need and purpose. According to \citet{DEste2007}, technology transfer processes, in fact, have an idiosyncratic nature: they depend on the specific context, the nature of the knowledge to be transferred, the receiver firm and various other factors. Therefore, different configurations will require different arrangements among organizations; examples of variables to consider are the innovativeness and the stage of development of the technology, frequency, and intensity of exchange, resources involved, the need for contractual rules and arrangements \citep{DEste2007}.

\citet{Alexander2013} gave instead more emphasis to the type of governance. They categorized each channel by five variables: the degree of formalization, the extent of the effort to minimize risks through a contractual approach, the level of previous engagement, knowledge tacitness and media richness. More importantly, they found that no channel is completely relational in nature, due to the need for risk management and knowledge codification. Similarly, \citet{Bercovitz2006} observed that knowledge is both difficult to evaluate and to appropriate, creating the need for market transactions in the form of contractual mechanisms, voluntary and negotiated. As recognized by \citet{Rogers2001}, also the external environment and its characteristics should be considered in modeling the process.

Other authors use a dualistic approach in their perspective: science-based and development-based, formal and informal, contractual and relational. Should be noted that in these perspectives there is never a clear cut between opposites and that these, while significant, are only examples.

The first of these perspectives come from \citet{Gilsing2011}. In their view, the science-based regime relies on the importance of basic knowledge, non-cumulative and universal, which makes firms strongly depend on external sources through publications, consultancy, collaboration. The development-based regime, instead, uses a more applied, systemic and interdisciplinary knowledge, specific to the industrial application; in this case, there is a lower dependency of industry on external sources, shifting toward channels as R\&D projects, collaborations, and contracts, professional networks, the flow of students and PhDs. 

\citet{Link2007} separated formal and informal mechanisms. In this case, formal technology transfer prefers contractual and other legal instruments, based on the allocation of property rights and obligations. Informal processes instead are based on more relaxed and informal flows of knowledge, such as technological assistance, consulting and collaborative research. While the property rights and obligations are still present in this latter perspective, they assume a secondary role, more normative in character.

Lastly, transactional and relational perspectives, or \enquote{buy-sell transactions} and the \enquote{technology transfer at the arm's length} according to the original author \citep{Harmon1997}. In the first case, the process is formal, contractual and usually involves an intermediator; the information flow tends to be unidirectional, linear, driven by contractual rules. In the latter case, instead, the relationship and collaborative aspects acquire a central importance, the exchange is mutual and the barriers to the information flow decrease. The author also considered the existence of a third, hybrid model. 

Other authors sorted the ensemble of mechanisms in different channels and groups, following different perspectives. A first example can be taken from 	\citet{Rogers2001}, who simply listed the main channels: spin-offs, licensing, publications, meeting, cooperative R\&D. \citet{Debackere2005} further differentiated contract research from collaborative research, and includes a residual category for cooperation in graduate education, exchange programs, informal contact and personal networking. \citet{DEste2007} focused on less commonly studied channels, grouping the processes into: industry sponsored meetings and conference, consultancy and contract research, new companies and new physical facilities, training, joint research. 

More recently, \citet{Muscio2013} individuated 12 different types of collaboration, dividing them in: physical facilities, consultancy and contract research, collaborative research agreements, training, meeting, and conferences. \citet{Alexander2013} instead differentiated between: transactional, i.e.\ patents, licenses, spin-offs, joint ventures; mixed governance, including collaborative research, research contracts and consultancy; mainly relational, such as shared facilities, journal publication, training, joint supervision; and relational, meaning joint conference, networks, student placement and alike.

Elsewhere, \citet{Balderi2010} categorized the main channels, licensing, spin-off and research contracts by modeling the choices that public institutions can take. The two main alternatives, in their view, are too diffuse the new technology through publications, congress, and other non-contractual mechanisms, or to protect and privatize the research results. In the latter case, the underlying knowledge can be either codifiable, resulting in a patent, or tacit in nature, thus requesting a more participative channel. Again, in the first case, the choice is between selling or licensing the patent, to another firm or a spin-off, exclusively or not. In the latter, should be preferred mechanisms such as training, consulting, collaborative research and alike. 

Lastly, the literature mainly refers to two main channels: patents and spin-offs. As previously stated, the concentration of research on these topics arise from the relative greater availability and easier access to information and data; however, many of the hypothesis and findings regarding these channels can be successfully transferred to other mechanisms. Therefore, the following review will start from patents and spin-offs, to later describe other contractual and informal mechanisms. 

\section{Patents and licenses}

The technology transfer process has its roots from the research activity performed by scientists. In the case of patents, as for spin-offs and other proactive approaches to the market, the outcomes of this activity are considered as fixed, unchangeable and with any possibility, or space, for intervention. 

The successive step, which marks the entry in the true transfer process, is the invention disclosure. It requires researchers to report a discovery that is believed to have a commercial potential; this disclosure should provide the dedicated office with information regarding the invention itself, various relevant market conditions and other data \citep{Thursby2002}. As previously stated, this leap can be an issue for scientists, involving various factors as incentives, opportunity costs and the personal attitude \citep{OwenSmith2001}. However, these invention disclosures are the \enquote{key intermediate input}, or rather the raw material, for TTOs and their officers. Some authors suggest a need for disclosure eliciting, including proper organizational incentives \citep{Siegel2003a}, but empirical research indicate that this may be unnecessary, and the institution unwilling or unable. 

\citet{Jensen2003} investigated this issue by modeling the behavior of inventors, the TTO and the university administration as a game, which \enquote{rules} are set by the administration, i.e.\ contract terms, incentives and TTO's objectives. The inventor has 3 choices: to disclose the invention, further develop it or switch to another project; more specifically, he may decide to dedicate a greater time and effort to the idea further developing it, thus increasing its probability of success and appetibility for the market. However, his attitude toward this last alternative largely depends on various factors, among which policies and the TTO's ability to execute an attractive license.

\citet{McAdam2005} focused instead on a relevant topic related to the invention disclosure: many scientists initiating a technology transfer process have an \enquote{overly simplified view of business and management issues}. More specifically, they tend to be over optimistic and underperform activities such as specifying the customer target, assessing the market potential, validating the technology and the business project, and other primary business activities for startuppers. The authors suggested as possible solutions a relevant management training for the TTO employees and specific training activities for academics who want to engage the technology transfer.

The next phase requires the TTO to assess the disclosure. The office must comprehend the technology potentiality, if it truly corresponds to a market, if there is an interest from the industry and if the potential revenues can sustain the relative costs. This activity involves the management of uncertainties and the ability to evaluate the new technology, a difficult task in this phase. Most inventions, at this stage, are only proofs of concepts; and only a few companies may be interested in licensing in early stages, providing a market feedback \citep{Jensen2003}. In this phase, the TTO is also required to choose which channel to use for the technology, among which patents are only one viable alternative. The evaluation also depends on the general tendency to a revenue-maximization or a diffusion-maximization model. Lastly, according to \citet{McAdam2005}, a positive decision should be linked to the intention to endorse the project, also financially, in later stages.

If the technology is positively evaluated, other minor decisions must be taken: the choice of which patent type apply for, whether to use external patent attorneys and specialized firms and alike. Another relevant preparative is to write the technology description and claims for the patent application, which will later determine the extent and the degree of protection the patent will grant, thus its value. This phase may require a continuous interaction between inventors, TTO and patent attorneys and other dedicated staff. After these additional steps, the patent can be filled.

At the same time, concurrent with the project evaluation, the TTO usually starts seeking for a potential licensee \citep{Markman2005}. In fact, as reported by \citet{Siegel2003a}, these steps do not occur in a linear fashion, and many firms will license even before the patent will be granted. These attempts require the involvement of the inventor and the faculty, in order to identify potential licensees through experience and knowledge of the field. An alternative is to outsource this activity to a firm specialized in the patent valorization: the patent portfolio has its management costs, which will require proper and professionalized competencies to make it profitable \citep{Balderi2010}. 

If a firm appears interested, the counterparts will start negotiating the license agreement. In the event of a negative result of the negotiation, and in the case of a successful one which comprehends a non-exclusive license, the process may continue by seeking another potential partner. In any case of a successful negotiation, the new relationship among organizations may require maintenance and, occasionally, the re-negotiation of the agreement \citep{Siegel2003a}. 

\subsection{Performances, laws and approaches}

Some authors performed quantitative, empirical analyses. One of the most interesting findings regard the stage of development of the average licensed technology: most university inventions are licensed as proof of concept (45\%), with only a prototype available (47\%) but the largest part requires further development (85\%); only the 12\% are ready for commercial use, and even less have a known manufacture feasibility (8\%) \citep{Thursby2002}. In fact, many TTOs start seeking for potential licensees already before the patent application, and only if the license is expected to be \enquote{easy}: only the 20\% of inventions disclosed will become a patent.

Regarding the probability of a successful exploitation of a patent, incomes are significantly concentrated in a few patents: \citet{Thursby2002} reported on average a 76\% of incomes attributable to the top 5 inventions, little more (78\%) according to \citet{Jensen2003}. The same authors found an inverse correlation between the financial success of a patent and the relative shares or royalties allocated to the faculty. These patents are funded by federal research grants (63\%), industry (17\%) or unsponsored (18\%). Elsewhere, has been found a concentration of inventions in a few areas, specifically arising from engineering, medicine, and nursing. However, this concentration around specific topics has been widely discussed, but the most of the literature agrees on the better performance of applied sciences. 

\citet{Jensen2003} specifically studied the optimal license contract. Firstly, inventors cannot be effectively monitored during the further development; the resulting moral hazard issue, unitedly to the asymmetric information issue, requires the usage of a mixed payment. Examples are: a mix of fixed, up-front, and license-issued fees, royalties, milestone payments and alike, which may positively influence the inventor's effort; equity payments should be preferred. The financial value of these incentives should outweigh the inventor's' disutility from the further effort required and other disadvantages, like the publishing delay. 

In the particular case of a royalty payment, other technology transfer models suggest that any distribution scheme which does not allocate the entire sum to the inventor is suboptimal, and will negatively influence the academic attitude toward patents and licensing. Three other important characteristics of the optimal license agreement are: to grant exclusive rights, as preferred by the firm; to clearly specify the focus and contents of the underlying research project; the provision of equipment and personnel by the industrial counterpart. 

Other authors analyzed the legal perspective, since universities have become more and more aggressive in securing and protect their patents \citep{Wysocki2004}. This point of view, in fact, allow a better understanding of how critical has become the issue over time, and which is the true attitude of universities and their administration toward their knowledge assets.

Basically, the intellectual property protection embedded in patents is intended to be beneficial for both inventors and the society: the former receives the exclusive right to exploit the patented technology and knowledge for a limited period, while the society receives knowledge that might not receive otherwise. 

Earlier in the latest century, governments and courts recognized the unique role of universities as knowledge producers and repositories, giving them a peculiar ability to overcome, or simply ignore, the intellectual property rights of other organizations, at least in specific cases. This rule was known as \enquote{academic exceptionalism}, and allowed universities to use external intellectual property in the case of experimental or fair purposes, without licensing it.

However, with the Madey v. Dukes case \citep{Hayter2016} American courts engage a different line of thinking. As the court stated, \enquote{like other major research institutions of higher learning, is not shy in pursuing an aggressive patent licensing program from which it derives a non insubstantial revenue stream}. The changes in universities'' attitude and behavior in patenting and licensing, in conflict with the open-science approach previously associated with these public institutions, made the non-profit status of the Duke University immaterial to the court. In a legal perspective, it can be considered a milestone in the shift of university objectives, and the relative importance of the third mission.

Other authors focused on the analysis of the Bayh-Dole Act and its effects. Promulgated in 1980, this law focused on the intellectual property arising from federally funded research: at the time, only few research products were being patented by universities, as an underutilization of federal funds, thus paid taxes. The new legislative approach assigned to universities the ability to protect and possess research results, in order to use the potential financial outcome as a secondary incentive to firstly patent, then commercialize. 

The effect of the Bayh-Dole Act and similar legislations in other countries has been profusely studied in the economic literature. One of the most significant results is the one of \citet{Leydesdorff2010}, who found an increase between 250\% and 500\% of patenting activities performed by universities when a Bayh-Dole type legislation is introduced. However, the same authors found a relative decline in their effects since the 2000s: while at a global level the university patenting activities are still increasing, in most advanced economies the impact of such legislations has \enquote{faded away}, possibly due to learning effects or differences in incentives and evaluation policies.

Apart from the positive impact on patenting applications, the Bayh-Dole act influenced other types of patent-related activities, especially their enforcement: \enquote{the act of threatening to sue or actually suing third party companies for patent infringement}. Specifically, even if most interviewees report to be conflicted on the topic and to decide on a contingency base, the number of patent infringement lawsuits which involve universities has significantly increased \citep{Hayter2016}.

One last perspective on patents should be considered when it comes to the evaluation of universities' patenting activities: the home advantage effect, which states that a patent applier tends to fill more patents in their home country than abroad \citep{Criscuolo2005}. Causes can be cognitive in nature but also based on economic evaluations on potential and existing markets, i.e.\ due to the technological specialization of countries.

\section{Spin-offs}

As previously stated, spin-offs are new ventures founded in order to commercialize a technology or a knowledge developed in a research organization, as a university, a federal or government laboratory or a private organization \citep{Rogers2001}. Again, the spin-off can be intended as the firm itself, but also as the process that will lead to the firm formation. Following this last perspective, many authors proposed different models, but the fundamental ratio can be easily understood by comparing them.

A first, simplified stage model is the one proposed by \citet{Druilhe2004}. They divided the generic entrepreneurial process, with no direct reference to the spin-off process, into 3 different stages. The first is the opportunity recognition, from the research result to the identification of a commercial project; the main issue in this stage is the ability to unconsciously and immediately perceive the potentiality of an idea: in the author's view, opportunities are \enquote{objectively identifiable}, but their \enquote{recognition is subjective}. The second stage refers to the mobilization of resources and their (re)combination to achieve the expected outcome; difficulties in this phase arise from the scarce expertise of academic scientist in this kind of entrepreneurial activity. The last stage is the ongoing organization of the resource base, in order to enable and increase the revenue generation.

\citet{Clarysse2005} proposed a different three-stage model for the spin-off process, focusing on the activities that lead to the creation of an independent venture. The first is the invention phase, the act of creating a new technology or knowledge; its main issue is the technology uncertainty. The second stage is the transition phase, in which the entrepreneurial idea is validated through small market experiments. Lastly, the innovation phase refers to the creation of the new venture and the growth of the project. This model, however, places a little attention into the opportunity recognition.

\citet{Degroof2002} instead suggested a six-stage model, which requires a more direct involvement of the university and its dedicated offices. The process starts with the seeking for a technology opportunity, both from TTO's officers and scientists; later, the office will assess and evaluate the intellectual property involved in the idea, and select the projects based on their feasibility. The selected spin-offs will be supported in the development of a business plan, and later in the seek for funding. Lastly, the new venture will be founded.

Another well-recognized approach is the one of \citet{Ndonzuau2002}, who modeled the spin-off process into 4 different stages focusing on the relative issues. Firstly, the business idea generation (post-recognition); it can be inhibited by the academic culture and poor internal competencies in opportunity recognition. The second stage refers to the finalization of the new venture project, structuring a coherent and feasible plan; in this case, issues may arise in the identification of owners and the most suitable method to protecting the idea, how to exploit it, and how to finance it. Thirdly, the spin-off launch: ideally, the creation of a firm which (1) exploit an actual opportunity, (2) managed by a professional team, (3) supported by available resources; the main topic, in this case, is how to gather the necessary resources. Lastly, strengthening the creation of economic value; in this phase, attention should be paid to the relocation risk and a change in the business trajectory.

Similarly, \citet{Lockett2005} used a stage model to uncover the key process issues; as other authors, they found relevant the opportunity recognition, but they focused on two other topics previously overlooked: the decision to commercialize and the choice between channels. For the first, they observed that the decision relies on both the technology and the academic involved, but much reliance is placed in the academic and his motivation. For the latter, the authors recognized the independence of the technology from the channel: in theory, the idea can be exploited in several different ways; however, an exploitation mechanism must be chosen. They used as a milestone the licensing and the spin-off channels, based on the estimated financial returns, academic willingness, the extent of the TTO's involvement and similar factors; however, it should be remembered that these are not the only available channels.

Lastly, one of the most recognized models has been described by \citet{Vohora2004}, who used five stages separated by four critical junctures. The process starts with the research stage, which ends with the opportunity recognition: the authors described this critical juncture as \enquote{the match between an unfulfilled market need and a solution that satisfies the need that most others have overlooked}. Its overcoming requires the ability to synthesize academic knowledge with market and industry knowledge, thus high levels of social capital. The second phase is the opportunity framing stage, in which researchers and technology transfer officers assess and evaluate the identified opportunity, and try to frame it in a commercial project. At the end of these activities, the issue regards the entrepreneurial commitment of researchers; the authors identified four key obstacles: lack of a successful entrepreneurial role model, of prior business experience, of self-awareness over personal limitation and difficulties in accessing surrogate entrepreneur. 

The successive phase regards the pre-organization, in which will be established how to exploit the opportunity and the involved researchers and TTO's officers start implementing the strategic plans. At this stage, the main problem arises from the (lack of) credibility of the new entrepreneur, which jeopardize his ability to access and acquire the necessary initial resources; a solution may come from the relationships established by the TTO during its networking activities. Fourth is the re-orientation phase, the attempt to generate returns from the new technology; this stage is highly characterized by the need of effectively reconfigure the resource base in a competitive setting. If achieved, the new venture can overcome the \enquote{last} issue, and reach the sustainable returns phase. In this last stage, the spin-off finally leave the umbrella of the originating institution and enters a pure commercial environment.

\subsection{Entrepreneurs and teams}

The spin-off process involves a multitude of different individuals and institutions, from the inventors, the parent organization, external and surrogate entrepreneurs, venture investors and many others \citep{Djokovic2008}. Two of the most significant contributions to the technology transfer literature refer to the entrepreneur itself and the composition of his venture team.

Firstly, \citet{Shane2015} investigate the \enquote{typical} academic entrepreneur. More precisely, they began from the hypothesis that TTO officers, venture capitalists, and other actors tend to support spin-off and entrepreneurs that meet specific characteristics, recognized in previous successful cases: \enquote{the representativeness heuristic means people tend to favor those examples which look like the standard case}. In fact, previous studies show the typical inventor-entrepreneur usually as a male immigrant, with industry experience and \enquote{easy to work with}; the authors empirically confirmed that this is actually the kind of inventor that TTO officers tend to favor.

Secondly, \citet{DerFoo2005} studied the impact of the team composition and its internal diversity on the external evaluation from venture capitalists and other investors. The underlying hypothesis is that larger teams may not increase the amount of information and capabilities embedded into to the new venture, rather dependent on the marginal ability of individual employees to bring new experiences and fields of knowledge. They differentiated between task-related and non-task diversities, i.e.\ education, work functions and company tenure versus personal and psychological attributes. They found a little empirical support for a positive correlation of task-related diversity and investors' evaluations, and a negative one for non-task diversities, suggesting the relatively superior performance of a team whose composition include different but overlapping competencies and backgrounds, of individuals otherwise similar.

Related to this topic is the contribution of \citet{Zhou2014}, who investigated the impact of immaterial assets on the external investors' evaluation, specifically the ownership of patents and trademarks. The starting point is the presence of an information asymmetry between the new venture and venture capitalists, a scenario that the latter tend to overcome through the usage of proxy variables for the startup economic performance: patents and trademark. The hypothesis, strongly supported, is that these portfolios signal to VCs the willingness to engage and commit to the new venture, which will positively influence the assessment until the growth of an external, autonomous evaluation capability from VCs. The authors found significant and positive coefficients for both patent applications (B = 0.35) and trademarks (B = 0.40), even greater for the interaction term of these (B = 0.35, up to 0.61 for early rounds).

\section{Contract and cooperative research}

While the economic literature focused on licensing and spin-offs, the sponsored research channel is reported to be the preferred by faculty inventors \citep{Jensen1998}. As stated before, this type of mechanism allows the academic to continue his research activity with a higher degree of freedom in respect to the previous channels, which represents at the same time as a compensation mechanism. It is important to note that the various forms of sponsored research are not mutually exclusive, and may take place alongside other channels, i.e.\ contract research financed by licensee firms. Moreover, this category refers to various mechanisms, including collaborative research, cooperation agreements, and research contracts.

Collaborative joint research is a formal agreement of collaboration between research institutions and firms; it may include several actors, up to a consortium. In this mechanism, different organizations confer their knowledge and backgrounds to a unique research activity, in the form and for the objectives established by the initial contract. The structure and contents of the process itself may vary substantially, i.e.\ the location of research activities, infrastructures, resources and human capital provided by actors, the legal protection adopted. The aim, however, is usually a \enquote{precompetitive} research: to gain a better understanding of a scientific field before any kind of industry application \citep{DEste2011}.

Collaborative research projects are usually based on a cooperation agreement, which constitutes a legal framework for later contractual forms of cooperation. In these contracts are described the legal boundaries of every forthcoming activity, i.e.\ the appropriation of research outcomes, the usage, and propriety of scientific backgrounds, who will oversee patent applications and its fees, and alike. The aim is to reduce later coordination costs and information leakages with partners. A preliminary legal agreement is particularly useful in the case of a great distance in attitudes, approaches, and priorities between organizations, as between an open science academic institution and a firm operating in a highly competitive market. 

A related mechanism is the research contract, in which a company outsources the research activities to an external organization. In this case, the firm usually provides only the financial means for the research activity or a previous protected intellectual property, while the research organization provides human resources, laboratories, and knowledge. The aim of this contract is usually the development of a product or a new specific technology, rather than exploring a scientific field and creating new knowledge. In fact, \citet{DEste2011} referring to its objective as research and development with a direct commercial relevance, more applied than collaborative research.

\citet{Sohn2012} specifically investigated which is the most diffused research contract form and its standard clauses; they focused on publication of outcomes, indemnity responsibility, ownership of results and compensation forms. They found the optimal combination in a contract that (1) allows publication after the firm's consent, (2) in which the results are owned by the firm or together with the research organization, and the former is responsible for patents application and maintenance, (3) the responsibility of indemnification is shared or on the firm only, and (4) compensations to researchers are in the form of incentives and benefit from both parties. 

Many authors seemed to favor cooperation over contracts, due to the better opportunities offered by the complementarity of knowledge and resources between organizations. In an attempt to understand these partnering decisions, authors investigated the characteristics of firms and their impact on the probability of establishment and success of cooperation and research contracts. \citet{Powell1996} found relevant the following attributes: size, position in the value chain, the degree of sophistication, resource constraints and prior experience with alliances. Similarly, \citet{Aristei2016} individuated as firm's relevant characteristics its size, industry affiliation, previous experience, its absorptive capacity and R\&D intensity. More importantly, their empirical results indicate that the typical firm engaging in research cooperation is R\&D intensive, makes a wider use of IP rights, takes benefits from public R\&D support and generically relies more on external sources of finance.

\citet{Cantner2006} instead found that the cooperation between organizations strongly depends on different forms of previous linkages other than the past experience, i.e.\ job mobility and other informal ties. In fact, their results showed the presence of \enquote{definitively more linkages between innovators than documented in patents}. While these informal channels of technology transfer are often overlooked, due to the relative difficulty in gather information and data, it is important to get at least an overall understanding of their extent. 

\section{Informal channels}

The informal exchange of information between organizations has always been a relevant phenomenon, and many authors investigated its extent and impact. One of the most important contributions on the topic comes from \citet{Schrader1991} who investigated the information trading, defined as the \enquote{informal exchange of information between employees working for different, sometimes directly competing firms}. He found that the 83\% of employees has already provided information and a significant positive correlation of this exchange and the overall firm performance.

In fact, flows of tacit knowledge and informal relationships are fundamental to the technology transfer \citep{Geuna2009}. Despite the typically narrowed focus of TTOs on patents and spin-offs, only a part of university knowledge and research can be codified into patents, while spin-offs are limited in number, thus in potential impact. Informal channels, instead, can transport a large amount of information to the industry, independently from its tacit nature: in this perspective, informal channels can be considered as an alternative route, a fallback when contractual mechanisms fail to take place.

Another reason for the existence and usage of informal channels of knowledge exchange is that \enquote{transactional mechanisms do not occur in isolation} \citep{Bercovitz2006}. Licenses, spin-offs, contracts and alike will eventually require the establishment of a relationship between university and industry, in a long-term perspective, to allow more complex interactions, feedbacks and cooperative behaviors. As an example, a cooperative research project or a license agreement expose an information asymmetry, to overcome through trust, thus the building of a long-lasting relationship. If the university is not willing to fulfill this need, firms may be interested in directly contact the scientist and arrange an informal technology transfer \citep{Siegel2003a, Link2007}. 

Examples of informal channels are non-contractual consulting, sabbatical leaves, participation in events, congresses and meetings, publications and alike. Apart from the transfer itself, these mechanisms also allow researchers to perform boundary spanning activities, connecting the various networks they are embedded in, fostering the establishment of new relations or \enquote{bringing together entrepreneurs and professionals for face to face encounters that can forge other university-industry linkages} \citep{Yusuf2008}.

A specific but relevant case is the role of star scientists in cooperative and informal technology transfer. \citet{Zucker2001} studied the impact of these collaborations on the firms' innovation performance in the Japanese context, characterized by two institutional features: the tendency of collaborating through the deploy of a company scientist toward the university laboratory, and the keiretsu membership. Empirical findings indicate that two co-authored articles led to a 77\% increase in patents, 60\% more products in development and 18\% more products on the market, suggesting a great effectiveness and productivity of the collaboration.

This case highlight another important source of knowledge spillovers and technology transfer: the researcher mobility. According to the previously cited Marshall's district theory, a higher workers' mobility, including both academic and private scientists, should increase the web of relationships linking individuals and institutions, allowing greater flows of knowledge and innovations, thus fostering overall technology progress \citep{Cantner2006}.

To better understand this mechanism, \citet{Argote2000} decomposed a systemic source of innovation in three different repositories: people, tools, and tasks; together, they form a network which embeds knowledge and competencies. Moving the entire system could be difficult, due to the internal interdependencies and their linkages with the context they are born in. However, moving only the involved human resources may be an effective, even if partial, solution: individuals can adapt and reconfigure their knowledge, both explicit and tacit, to the new context and the new local group they would refer to; their reallocation may also be a preparatory step to the later transfer of tools or technologies. 

\citet{Slavova2015} conducted an empirical analysis of the effect of inbound mobility on the firm performance: he found a positive, significant effect on the recruiting of high-quality researchers on the performance of the internal incumbent scientists, thus the overall innovativeness. The extent of the effect, however, depends on the social dynamics internal to the organization, knowledge sharing, and collaboration practices. The outbound mobility, especially toward spin-offs, has been found instead to be a side effect of the technological advance of the organization: according to \citet{Franco2000} firms with wider and deeper know-how are more likely to survive the competition and inevitably generate spin-offs and new startups.

Researchers' mobility takes a fundamental role also in the social capital framework: by employing an academic researcher, a firm does acquire his knowledge and capabilities, but also his network of prior relationships that may be translated into the employer's social capital. \citet{Murray2004} identified two different categories of social capital through which the prior academic career of the scientist can affect the new environment. The first is the \enquote{current laboratory affiliation}, that refers to the team of research, i.e.\ colleagues and students; the second instead, the \enquote{cosmopolitan network}, broadly includes all the relations the scientist had built, even in conferences, co-publications, grants, committees and alike. In fact, the acquisition of external human and social capital through the employment of researchers can play a critical role in the competitiveness of firms.

\section{Relative importance}

The literature widely debated on the relative importance of these channels, mainly in order to identify the most effective in which focus the university resources. From the academic perspective, TTOs are usually focused on patenting, licensing and spin-offs activities: \citet{Chapple2005} referred to licensing as the \enquote{most popular mode of university technology transfer}, based on the empirical analysis from \citet{Siegel2003}. However, from the industry perspective, different channels might be more relevant.

Starting points are various contributions which indicate that patents, independently from the TTOs' approaches, are of lesser importance and smaller entity in respect to other channels like contract research and joint research programs \citep{DEste2007}. \citet{Link2005} stated that the firms' willingness to license technologies even before the patent application (if patented at all) suggests that the importance of patents is \enquote{often overstated}. Patents, in fact, represent inventions that may or may not be transformed into a commercially viable innovation: only a portion of the university innovation output, once compared with what can be transferred by other means to the market. As a matter of fact, a consistent share of academic research activities focus on basic research which outcomes cannot be patented \citep{Fritsch2007}; the main university focus, in fact, is  on publishing articles in scientific journal, a transfer mechanism recognized for its relative inefficiency \citep{Rogers2001}. 

\citet{Balderi2007} reported instead a relative increase of the effort directed toward spin-offs, with respect to the resources devoted to licensing activities; they are preferable, due to their ability to generate more revenue than licensing and visibility than grants \citep{Rasmussen2006}. However, the extent and the scope of their impact on the society is intimately connected to their business models and products: they may represent an expensive choice in terms of time, human and financial resources, with a high opportunity cost if compared to other alternatives. Moreover, patent licenses and spin-offs do not represent a stable source of income. 

While codified output seems not so important, collaborative and contract research appear to be more important channels \citep{Bekkers2008}; along with consultancy, they account for the largest part of technology transfer activities \citep{Muscio2010}, more frequent and highly valued \citep{DEste2011}. As an example, \citet{Rasmussen2006} reported that a consistent part of the departmental R\&D is already \enquote{carried out in close co-operation with firms} who will later commercialize the technology. This perspective is linked to \citet{Link2007} who found that tenured faculty members spend about the 24\% of their time seeking for grants, which can be effectively substituted by private funding in the form of research collaboration and contracts. 

Similarly important are also employment, researcher mobility and personal contacts \citep{Bekkers2008}. In fact, interviews conducted by \citet{Siegel2003a} reported individual relationships more often than contractual ones; elsewhere, \citet{Sohn2012} considered university spillovers a \enquote{disproportionately large source of technology transfer} referring to informal knowledge exchanges typical of a Marshall's district perspective. Lastly, \citet{Link2007} suggested the potential prevalence of informal technology transfer, thus the need for its comprehension. 

\section{Barriers} 

Comparing the relative effectiveness of each technology transfer channel, a larger and more precise perspective can be obtained through the consideration of the various barriers which may pose obstacles, slow down or reduce the effectiveness of the process. These barriers can be categorized into three different types: cognitive, as in the distance between \enquote{open science} and entrepreneurial academic mindsets, as well as between the institution and firms; the lack of a supportive environment, including ad-hoc supporting mechanisms; the characteristics of the research outcome, the technology field and market needs, and the distance between them.

\subsection{Cognitive distance} 

The cognitive distance is a broad concept, which generically defines the difference that actors perceive between themselves. In the specific case of the technology transfer among different types of organizations, the cognitive distance refers to the difference in cultures, missions, objectives, activities, methodologies between firms and researchers. This difference may present itself in various forms and different context.

One of the largest differences is perceived between the nature of the university research and the industry needs and interests. \citet{Muscio2008} linked this distance to the diversity in \enquote{research methodologies and in the use and interpretation of knowledge}; in fact, his interviewees in the industrial environment indicated the academic research is \enquote{too advanced}, with little or no practical value. Similarly, \citet{Gilsing2011} described the problem as a \enquote{scientific knowledge being too general to be useful}, too theoretical and rooted into the science-based regime. \citet{Muscio2013} evaluated this distance as significant on the intensity of interactions (13.2\% for step, in a 5-point linkert scale) but apparently with no effect on the probability of the establishment of a relationship.

Apart from the different views on the role, type, and importance of knowledge, firms and universities also lack a proper mutual understanding when it comes to the respective norms, environments, expectations, objectives, priorities \citep{Siegel2003a, Link2005, Muscio2010}. Potentially leading to a conflict of interests, this misunderstanding represents a true cultural barrier, that limits the interest of industry in academic research and the university involvement in the technology transfer. A first example is the need for academics to publish research outcomes versus the fear of information leakages and spillovers for firms \citep{Gilsing2011}.

\citet{Siegel2007} approached this topic by analyzing the different objectives of the three main agents involved: scientists, technology transfer officers, firms and entrepreneurs. Academics seek technological and knowledge breakthroughs and a rapid dissemination of their results; TTOs and administrators try to valorize the intellectual property portfolio while safeguarding the university environment; firms seek to commercialize technologies to reach a competitive advantage, ultimately seeking a profit gain. Independently from being compatibles or not, these differences are perceived as an obstacle, thus impeding the establishment of new relationships and new flows of knowledge.

An example arises from the concept of innovation speed, defined as the elapsed time between an initial discovery and its commercialization. A shorter lag between the two events implies a higher ability to leverage research-related assets, to amortize the project costs and experiment with a greater number of ideas; in other words, it results in a greater innovativeness and efficiency, thus a sustainable competitive advantage \citep{Markman2005}. However, even if managers stress the importance of a faster time to market and the first mover advantage, the entire process of technology transfer is reported to require truly extended periods: \citet{Heher2006} evaluated the delay in the knowledge value chain in 6 to 10 years. \citet{Link2005} suggested as roots of this issue the low responsiveness of universities to the need of firms, even if exist a premium price for a faster transfer \citep{Markman2005}.

Considering the single research institution, a cognitive distance can be found between the open, science-based approach and the need for commercial activities, both for funding and a greater impact on the society. Examples are the issues found by \citet{Baldini2007}: lack of time, i.e.\ due to teaching and administrative role; information-related problems, i.e.\ a limited knowledge of the national and the university patent regulation and the lack of business experience; cultural issues, as the fear of opposition within the campus. Moreover, \citet{Jensen2003} reported that researchers might not disclose because they do not want to engage later transfer activities, otherwise necessary to the commercial success of the endeavor, a requirement for about the 71\% of inventions licensed.

An example of this issue is the problem of invention disclosure: as previously stated, various authors discussed how researchers fail to report new technologies and commercial ideas. According to \citet{Jensen2003}, TTO directors believe that only half of innovations with a potential market are disclosed, thus making the disclosure eliciting \enquote{one of the major problems}.

Similarly, a distance resides at the administration level of universities, as a tension between the various missions of the institutions. Specifically, \citet{Guerrero2014} cited the collegial structure and the typical decision-making pattern as common constraints. An example is the attitude toward surrogate entrepreneurship: many institutions show an aversion to externalizing the exploitation activity, due to lack of supporting evidence and positive cases, experience, information, uncertainties about the process and a reluctance to recognize the value of an external entrepreneur to the university activities. \citet{Franklin2001} suggested that surrogate entrepreneurs may be \enquote{at odds with the current way of thinking}, bounding the university to a more usual, but potentially obsolete, process design. 

\subsection{Supportive mechanisms}

The cognitive distance among universities and firms can be effectively bridged by an appropriate set of supporting policies and mechanisms. However, many authors have found issues also in this topic, distancing actors instead of getting them closer: the very organization, through its bureaucracy and inflexibility, act as a barrier to the technology transfer \citep{Siegel2003a}. Moreover, inefficiencies of administrations and organizational procedures \citep{Baldini2007} suggest that the very \enquote{key impediments tend to be organizational in nature} \citep{Siegel2007}. 

The center of this organizational issues can be identified in the TTO. Even if it was initially conceived as a facilitator, its position and the usually restricted availability of resources and personnel pose a threat to its activity. First of all, TTOs must balance the different objectives of the central administration and academic inventors \citep{Jensen2003}, which can be incompatibles rather than different, as previously stated. TTOs also have internal agendas, especially regarding the external context, which includes for example the need for the build of a solid reputation in the industry, eventually shelving some under-achieving inventions in contrast to the interest of administrators and academics \citep{Siegel2007}. 

Potential issues regarding the TTO participation ultimately refer to an underperformance due to a lack of human and financial resources. Examples are: lack of the time necessary to engage disclosure eliciting activities \citep{Siegel2007}; lack of officers that can effectively assess the commercial potential of projects, and a more generalized lack of business knowledge and experience \citep{Hertzfeld2006}; lack of employees specialized in the patent process \citep{Baldini2007}, its speed and costs \citep{Fini2009}; difficulties in identifying and securing contacts within industry, along with business representatives \citep{Markman2005} and suitable partners \citep{Muscio2010}.

It is also important to note that a low performance or improper behavior, i.e.\ inflating the commercial potential of a patent, affect both the firms' and academics' willingness to cooperate with TTOs, through a loss of credibility and reputation. Similar issues usually result in the attempt by companies to bypass the office and deal directly with the academic scientist \citep{Link2007}, leading to an unbalanced settlement. In fact, \citet{Baldini2006} referred to a low bargaining power that European universities might have in respect to industry. Lastly, the presence and active participation of TTOs is still considered fundamental to the engagement and the success of the transfer \citep{Muscio2010}.

Other issues on the supporting mechanisms topic may arise from various policies adopted by the university administration. The most cited are the insufficiency of reward for academics, both pecuniary and not \citep{Siegel2007}, especially for the royalty scheme of licensing revenues. Similarly, widely discussed have been other policies, as incentives, staffing and compensation practices \citep{Baldini2007}, the presence of a clear collaboration policy and a patent policy internal to the university \citep{Muscio2008,Muscio2010,Muscio2013}, for the resolution of dispute with firms \citep{Belenzon2007}, and more generally for knowledge and relation brokerage. Other areas of possible intervention are formal and informal activities of information among researchers: outside the US, national and internal patent policies are still fairly unknown to academics \citep{Baldini2006}.

A last important strand of support mechanisms refers to the financing issue: the lack of financial sources and resources has repeatedly been reported as a problem, in various shades. Examples are: low incomes due to small markets for new ventures \citep{Perez2003} seed and venture capital \citep{Rasmussen2006}, governmental programs \citep{Muscio2008}, for joint research projects \citep{Muscio2010}, and generic industry funding \citep{Muscio2013}.

\subsection{Compatibility and complementarity}

Other authors focused on the characteristics of the research outcome, the technology, and the market. As a starting point, \citet{Tijssen2006} found empirical support for significant differences among knowledge fields, while evaluating the university-industry cooperation intensity and outcomes. Indeed, the complementarity and compatibility between academic research application are of fundamental importance \citep{Geuna2009}.

The first element to analyze is the field of research which generate the new knowledge or technology. In the largest perspective, the main difference is between basic and applied sciences \citep{DEste2007}, especially when it comes to the selection of a channel for the technology transfer. As previously stated, some channels are more suitable than others for transmitting basic knowledge (publications, consulting, cooperative research) while others aim to the transfer of a more applied technology characterized by a higher readiness (licenses, spin-offs, contract research).

Similarly, another important difference separates codified and tacit knowledge, exemplified in two major fields: natural and social science \citep{Audretsch2004}. Again, the type of know-how involved affects the channel of choice, preferring more direct and participative mechanisms for tacit knowledge, i.e.\ spin-offs and cooperative research. Tacit knowledge may also limit the very geographical extent of potential university spillovers, by the need of oral communication and reciprocity, while influencing at the same time the location choices of firms.

Lastly, significant differences arise also between disciplines and faculties: many authors investigate the influence of an engineering department, rather than a medical school or biotech, information technologies and advanced materials \citep{DEste2007}, especially when evaluating the overall TTO and university performances. Eventually, the most significant disciplines are included in the so-called \enquote{science-based technologies}, in which the importance of the underlying scientific knowledge can be attested through the high frequency it is referenced \citep{Debackere2005}.

While the research field may determine the relative availability of technology to transfer, the technology characteristics and its industrial area of applicability will influence the level of interest of firms and the probability of a successful transfer. Among the many perspectives that authors have taken, three contributions have been considered particularly interesting.

Firstly, \citet{Shane2001} found four significant variables while determining the impact of different technology regimes, especially in new firm formation. First, the age of the technology field: younger ones favor new, flexible and adaptive organizations, while larger companies find the market too small to justify their investments. Secondly, the market segmentation: the more segmented and the more specific the market niche, the more favorable it will be for small firms. Third, the appropriability and the protectability of the new technology. Fourth, the need for complementary assets in order to build a competitive advantage, which may overcome a problem of low appropriability of the technology itself by making harder the imitation of new products. 

Apart from influencing the rate of new firm formation, as intended in the original research, these relevant technology characteristics also affect both the channel that will be used and its performance. As an example, younger fields with less scientific publications may prefer channels that allow the flow of tacit knowledge, directly from the academic to other human resources; a lower level of appropriability instead will push firms toward contractual forms, where their interests and activity can be covered by a non-disclosure agreement and alike. A greater market segmentation can favor the establishment of new academic spin-offs, while universities can provide little financial and industrial support, limiting the success probability of internal spin-offs in fields with a higher need for complementary assets.

On a similar topic, the second contribution of \citet{Bekkers2008} aimed at clustering technology attributes and linking them to different transfer mechanisms, eventually identifying six different groups. The first encompass truly scientific outputs, especially publications, students and other informal contacts, to be preferred for fields in which knowledge is fundamental, mainly codified and interdependent. The second group is centered around the labor mobility when knowledge is more tacit in nature. The third group include collaborative and contract research, for codified, systemic and interdependent knowledge to be transferred to large firms. The fourth refer to more formal and direct contacts, i.e.\ through alumni and professional organizations, particularly useful for social science. Similarly, the fifth group of specific, narrow activities organized for academics should be preferred for systemic and interdependent knowledge. Lastly, patents and licensing are more suitable for interdependent, applied knowledge. 

\citet{Nerkar2003} focused instead on the impact of the nature and level of industry concentration, specifically to the spin-off channel. They observed that while the ownership of broad patents on radical technology may positively affect the success of the spin-off, a high industry concentration may inhibit the survival of new firms. Thus, the increasing probability of success relies on a more fragmented market, where the concentration and the level of competitiveness are moderated by the size of the market. 

Other relevant technology and industry variables are: the rights of the licensee or contractor on the technology, the time to market, the presence of know-how associated with the patent, regulations, residual lifetime of the innovation, the presence of an established industry standard, ability to defend from imitation\citep{Balderi2010}. Lastly, \citet{Geuna2009} also considered fundamental the composition and structure of the local industry and the existence of a critical mass of firms.

\section{Recap}

The chapter begun with the seek for a framework for the classification and analysis of the various mechanisms, exposing the large number of variables that shape the form and inner processes of a technology transfer, among which has been chosen a formal-informal dual categorization.

Later, has been described the main channels available: patents and licensing, including a legal perspective; spin-offs, highlighting types and characteristics of entrepreneurs and teams; contract and cooperative research; and informal channels. The next section provided information on the relative importance and performance of channels, through a comparative analysis of the contribution and opinions of several authors. Notably, contract and cooperative research, as well as the informal exchange of knowledge, have gained a significant attention relatively to more common channels.

The last section instead took advantage of the previous exposition to analyze the various barriers that negatively influence the success probability of the technology transfer. The three main groups are: the cognitive distance, in culture, objectives and methodologies; the supportive mechanisms, their lack, misalignment or poor performance; the compatibility and complementarity between the technologies supplied by universities and the actual needs of the related market.

The channels here reported will constitute the basis for the later description of differences between approaches, especially regarding the relative importance and the barriers that may slack the transfer (\hyperref[Chapter6]{Chapter 6}). The same channels will also constitute a propedeutic framework in the description of the case study (\hyperref[Chapter7]{Chapter 7}), as well as its discussion (\hyperref[Chapter8]{Chapter 8}). 