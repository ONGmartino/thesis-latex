% Chapter 1

\Chapter{The Industry Perspective}{Firms, spin-offs and others} % Main chapter title

\label{Chapter3} % For referencing the chapter elsewhere, use \ref{Chapter1} 

The technology transfer process connects the university and its researchers, discussed in \hyperref[Chapter2]{Chapter 2}, to other organizations who will later exploit commercially the transfer object. While the industrial counterpart will be the same for both a public university and a private research organization, the major traits of the receivers may influence the transfer process, once coupled with different characteristics of these knowledge generators. Specifically, an arrangement between private realities may take advantage of their commonalities in culture, organization, and missions. To better understand the processes described in \hyperref[Chapter4]{Chapter 4} and the differences among paradigms proposed in \hyperref[Chapter6]{Chapter 6}, this chapter will describe the most common counterparts and their major characteristics.

\section{Firms}

Advanced knowledge has been recognized as fundamental for firms' competitiveness since the 80s, from the seminal work by \citet{Wernerfelt1984} on the Resource Based View. Investigating the ability of various types of resources to produce high returns in the long run, he hypothesized that the technological lead may be one of the strongest factors: it \enquote{will allow the firm higher returns, enabling it to keep better people and stimulating settings so that the organization can develop and calibrate more advanced ideas than market followers}. In fact, despite the threat of followers, the author observed that firms which invest in and acquire cutting-edge knowledge and technology are the best suited to further their capabilities and stay ahead the competition: like \enquote{a high tree in a low forest; since it will get more sun, it will grow faster and stay taller}.

The importance of being on the technological edge has been extensively studied and demonstrated; more specifically, many authors highlighted how innovation activities are necessary for competitiveness. \citet{Beath2000}, for example, stated that firms do depend on continual improvement, either in processes and products, whose source is the applied research. \citet{Yusuf2008} argued that firms need to foster their innovation processes, in order to sustain their competitiveness. \citet{Jimenez2011} referred to the knowledge base as the most strategic resource a firm can possess and deploy for obtaining a competitive advantage. \citet{Siegel2003a} observed the importance of a faster time to market for innovative products, which comes from their novelty and the collaboration with top notch organizations capable of transferring them in a timely fashion. Lastly, \citet{AzagraCaro2010} noted that cooperation, especially in R\&D, increases the firms' organizational learning capability, thus the innovation performance. 

Nowadays, the resource-based view is still one of the fundamental concepts in building an effective strategy, but the landscape has become more complex since the shift toward a knowledge economy. Innovations now tend to be more complex, interdisciplinary, systemic and depends both on the scientific knowledge and the market comprehension. \citet{Dahlander2010} recognized similar drivers, and made the strong observation that in the actual economic landscape \enquote{a single organization cannot innovate in isolation}. One of the most useful tools in this scenario is the open innovation.

The open innovation is a concept initially described by \citet{Chesbrough2003}, as a paradigm that sees organizations using both internal and external sources of knowledge and technology to make an innovative leap. In this case, firms can mix internal and external information, and decide to outsource or perform themselves the R\&D necessary to overcome the technical challenges, where outsourcing  means taking advantages from the increased division of labor and the new communication technologies. Different configurations are available to organizations: firstly, there is no clear division between open and closed innovation processes, therefore every setting represents a point in the continuum between these extremes; secondly, open innovation can be pecuniary or not, inbound or outbound. 

This particular system, which has gained a considerable attention over the last 10 years, allows firms to access knowledge and resources with greater flexibility, and to avoid sunk costs in specialized and larger R\&D units. However, it is well recognized that any organization, to gain an earlier and more detailed access to innovations and scientific knowledge, may need to \enquote{purchase a ticket of admission} in term of internal R\&D. That is, firms should not limit themselves to acquire the technology, but to collaborate in their development and absorb the underlying knowledge. This should be tightly linked to the adoption of a science approach internal to the organization which, in fact, pay itself: as found by \citet{Stern2004}, the open science system allows the firm to acquire better research staff, pay them less (notably, 27\%), employ more personnel, therefore raising the R\&D productivity and the rate of technological innovation. 

Regardless the cost of entering the innovation network, linking with other institutions has become a major factor in improving the innovativeness, thus the economic performance, of the firm. Relationships, cooperativeness, and blended social proximity give access to different, new knowledge that may be complementary to the previous technology portfolio, allowing the organization to develop innovations previously precluded. \citet{Bercovitz2006} found this approach to be particularly useful for exploration activities, highlighting the importance of new knowledge from partners and neighbors.

So, firms have a double benefit from engaging in cooperative research activities with other institutions: a reduction in R\&D costs and the access to more, high-quality knowledge. A further explanation of their benefits can be taken from \citet{Caloghirou2001}, who found that when collaborating with research institutions, specifically universities, firms mainly aims at \enquote{achieving research synergies, keeping up with major tech developments, sharing R\&D costs}. Therefore, what companies seek in research cooperation is a performance improvement, instrumental in gaining a competitive advantage over competitors. 

A slightly different perspective is the one of \citet{Bekkers2002}, who linked the domination of firms in a market characterized by an advanced technology (the GSM industry in the previous century) to their position in the alliance network and the ownership of essential intellectual property or patents. On the one hand, the cooperation with other organizations may lead the firm to a central role in the network, increasing its influence on the direction the market will take; on the other hand, this perspective introduces another angle for firms: patents and intellectual property. These tools help entrepreneurs and companies to gain control over the market and its future direction, other than the direct financial control over competitor and collaborators \citep{Siegel2003a}. 

To reach these advantages, firms of every size are collaborating with research institutions and universities. Also small businesses actively contribute to the innovation process, more than the extent that would be expected from their capabilities of investing resources \citep{Audretsch2005}. Larger companies, instead, balance their research agenda between short-term deliverables and long-term objectives \citep{Tijssen2006}. This trend led to a new \enquote{industrial ecology} of cooperation for R\&D and flows of knowledge.

However, the role of universities in this scenario is not clear. While \citet{Thursby2002} reported a positive change in the faculty orientation and its receptivity for commercial and industrial projects, the university involvement has been found somehow weak. Specifically, \citet{Yusuf2008} reported the results of an industry survey, which ranked universities least as innovation partners. These results seem even more negative when considering the importance of industry collaboration for these public institutions, i.e. the facilitating role of industrial linkages for new spin-offs \citep{OShea2005}.

In a glance, knowledge derived from scientific research proves itself a valuable asset for innovation-oriented companies. However, firms are required to possess already a proper knowledge base, developed through their own R\&D activities, in order to \enquote{absorb and appropriate} scientific know-how and new technologies: a concept named absorptive capacity.

\subsection{Absorptive capacity}

The absorptive capacity is a construct originally described in the seminal paper of \citet{Cohen1990}. They initially stated that is critical for firms to have the ability to recognize the value of, to assimilate and exploit external knowledge. More importantly, they hypothesized that this ability is mostly determined by the level of internal knowledge that the firm already possesses. At a basic level, a greater internal scientific knowledge includes skills and language capabilities, which allows companies to comprehend better and internalize external technology and relate with other innovative organizations and research institutions.

What firms need for increasing this ability is to perform their own R\&D activities and to be directly involved in the concept, design, and engineering of the product they will later commercialize. The inner rationale, according to the original authors, is that \enquote{from a cognitive and behavioral science perspective, accumulated prior knowledge increases both the ability to put new knowledge into memory and the ability to recall and use it} \citep{Cohen1990}. In other terms, by conducting its own R\&D, the firm may collect the prior knowledge, as in notions and experience, that will later be needed for understand and internalize other ground-breaking innovations and the science behind them. 

In this perspective, learning is cumulative: it is easier and faster to absorb knowledge that relates to the past background of the researcher and the firm. The same can be stated for the absorptive capacity itself: the greater the ability, the efficient the accumulation; the larger the knowledge base, the greater the ability. However, while a level of similarity is needed, suggesting a sort of path dependency, diversity in the internal background is strongly advised, for assimilating a variety of external innovations instead of being locked in a single market. 

Therefore, there is a trade-off between the focalization on a single topic, thus a narrow, better performance and the extensiveness of the disciplines the firm may internalize. A possible solution is what is known as transactive memory \citep{Wegner1987}, which states that nowadays, what is important to know is not the theory itself, but to know \enquote{who-know-what}. In this case, the solution might be to engage relationships with entities and organizations that have a complementary knowledge to the firm's pre-existing one. 

In a more operational perspective, the absorptive capacity can be divided into three different dimensions, according to the original authors. The first stage is the recognition of useful external knowledge, which depends on the prior related knowledge, on the characteristics of the counterpart and the relationship which link the institutions: cultural compatibility, trust, prior experiences and alike. The second phase is the assimilation: to bound external with previous internal knowledge; it depends on the organizational settings, such as flexibility, adaptability, specializations, objectives. Lastly the commercialization stage, which refers to the exploitation of both internal and the newly acquired knowledge, combined in a competitive configuration.

Later authors used a slightly different definition: \enquote{absorptive capacity can be defined as the organization's relative ability to develop a set of organizational routines and strategic process through which it acquires, assimilates, transforms and exploit knowledge from outside the organization in order to create value} \citep{Jimenez2011}. First, this perspective allows a better differentiation between assimilating the knowledge, as in a learning process, and the capacity to use and reconfigure it for other purposes. Secondly, this definition focus on the ability to create an organizational setting and an environment to better perform in the process, rather than the performance itself. 

\citet{Patterson2015} instead studied the absorptive capacity in 4 dimensions, as a process; notably, they argued that these phases are not consequential, but occur in an iterative fashion. The first stage, the acquisition, may be decomposed in \enquote{search and recognize} and the actual \enquote{acquisition}; the latter depends on a preliminary evaluation of the technology, which requires an initial, even if simplified, assimilation. Similarly, the \enquote{transformation} process may rely on the results of the commercialization phase, which itself may require additional development of the technology, thus its transformation. Lastly, the \enquote{exploitation} stage creates new knowledge that can initialize the process again. 

Some management practices can enhance the absorptive capacity of an organization. The first and most important is the human resource management: the selection and the organization of employees, the flows of communication between them, practices as rotating R\&D personnel etc. In particular, they must possess different but overlapping backgrounds, and be familiar with the firm's specific needs. Other authors referred to the outsourcing of similar activities, in order to increase the internal absorptive capacity through external tools such as technology transfer centers and consortia, or directly via corporate acquisition. Should be noted that these alternatives do not overcome the need for at least a minimal internal R\&D activity. A useful organizational tool is the creation of an internal technology transfer office similar to the university's one, which should act as a knowledge gatekeeper who actively seek for opportunities to exploit \citep{Alexander2013}.

Other authors provided empirical evaluations on the importance of this concept. An example comes from \citet{Nieto2005}: in their model, the absorptive capacity has, in fact, the greatest explanatory power for the firm innovative performance, and that a greater capacity will make irrelevant the presence and the extent of a technological opportunity. A second, more quantitative approach from \citet{Baba2009} suggests that the collaboration with Pasteur and Edison scientists enhance the R\&D productivity, evaluated in an increase of 1.13\% in patents.

\subsection{Social capital}

While the absorptive capacity refers to the internal ability to assimilate and exploit external knowledge, another concept describes the extent to which a firm can access and retrieve external resources: the social capital framework. The relevance of this concept arises from the findings of various authors, all pointing to the fact that the innovation performance mostly depends on the quality of relationships, measured as frequency, duration, emotional intensity or closeness, especially in the case of tacit and complex knowledge \citep{Perez-Luno2011}. 

The social capital is defined as \enquote{the sum of the actual and potential resources embedded within, available through and derived from the networks of relationships by an individual or social unit} \citep{Perez-Luno2011}. It was initially described by \citet{Coleman1988} as a particular kind of resource: social structures that facilitate actions by an agent within the structure. Like other types of resources, the social capital is productive and may be peculiar to the activity, the involved actors, and the context. In other words, it can be seen as a network of relationships that an individual or an organization can mobilize and use to a specific purpose, also individualistic; it is created by the participation to the network and the context itself in which the agent is located.

More specifically, the firm's embeddedness in a scientific network allows it to access a larger amount of knowledge and technologies, from an ensemble of different external actors, as in the mechanism of \enquote{localized learning}. A secondary, but relevant mechanism is the \enquote{social comparison}, through which an institution can use its role to influence the actions and behaviors of other actors \citep{Slavova2015}

\subsection{Networks and location}

The absorptive capacity and the social capital have their root in the idea of gaining new knowledge through a network of partners and other innovative institutions; nowadays, these networks, in which the firm is embedded in, are recognized as one of the most important resources it can access, manage and exploit. As stated before, in the case of an innovation network the main benefit from participating are the access to more and valuable resources and knowledge, lower uncertainty and other barriers to innovation, to gain flexibility through macro-level specialization and to adapt faster to the market needs. 

The most important, fundamental notion to consider is the Marshall's cluster theory \citep{Marshall1890} and the impact on knowledge diffusion. In his view, a district was a place where \enquote{mysteries of the trade become no mysteries; but are as it were in the air, and children learn many of them, unconsciously}, characterized by a distinct \enquote{industrial atmosphere}. These concepts led to the general recognition of the importance of localization to capture knowledge spillovers, mainly due to the availability of skilled labor and geographical and social proximity, which enable greater flows of information among institutions. These streams allow a social innovation process, incremental and collective, which can take full advantage of these spillovers.

These networks and districts may include firms, universities, and any organizational configuration of research institutions, i.e. public research organizations, private research centers and alike. Peculiar to universities, a relevant contribution to cite is the one by \citet{Audretsch2004} who investigated their on localization choices of companies. The rationale behind their research is that universities generate knowledge spillovers, that may be accessed by firms without a full compensation – thus lowering the research costs, increasing the expected profits. In this setting, the location can be considered a competitive advantage, even with differences in the field of knowledge. 

Other information on the impact of universities can be taken from \citet{Cantner2006} who applied social network analysis to the innovation network of Jena: in their perspective, social proximity has a stronger relevance than geographical proximity. They found that universities and other public research institutions, aided by government policies, are the core members, key actors in gather and distribute knowledge.

Following their approach, \citet{Giuliani2005} applied both social network analysis techniques and the concept of absorptive capacity in studying the Chilean wine cluster. Firstly, they defined the cluster absorptive capacity as its relative ability to absorb external knowledge, diffuse it among internal actors, and to exploit it externally. At the macro level, they found significant differences among the re-distribution of new information to internal firms, providing support for the hypothesis of the correlation between absorptive capacity and the relative importance of actors. In fact, their results suggest that firms with a greater capacity constitute the center of the network, acting as gatekeepers, surrounded by active and weak mutual exchangers, external stars and isolated firms.

This study provides an initial, quantitative understanding of the importance that absorptive capacity has in shaping a network and the firm position. \citet{Zeng2010} instead, provided an insight into the relation between the network activeness of actors and their overall economic performance: they found a significant positive relationship between the cooperativeness of the firm and its performance, especially for inter-firm cooperation and SME. However, they found weak support for the impact of cooperation with research institutions, universities, and governments, opening a quest for further investigations.


\section{Spin-offs}

Spin-offs are one of the most important and effective channels for technology transfer, but they also represent an external entity. In its industrial definition, a spin-off is a new entrepreneurial activity born from knowledge, technology or other activities of a parent organization. If the parent organization is a university, the new venture may be based on the licensing or acquisition of intellectual property from the originating institution \citep{Lockett2005a}. In other words, their business is the exploitation of research results developed within the academic environment \citep{Rizzo2015}.

Their relevance arises from an unusual ability to generate high economic returns for the local context, both in employment and financial terms \citep{OShea2004}, as well as their unique role in catalyzing knowledge and technology. They proved themselves fundamental also to the development and the performance of innovation networks \citep{Perez2003}, through their ability to connect research organizations and industry \citep{Rizzo2015} thus the natural impact on technological change and economic development. Similarly, \citet{Perez2003} suggested that these firms contribute to the dynamism of Regional innovation systems, by enforcing innovation and economic growth.

Many authors in fact recognized as one of the main drivers of their success their strong linkages with universities and research organizations. Assuming that universities' spillovers represent key resources in the knowledge economy, university spin-offs may have earlier access to them, and a greater absorptive capacity to exploit due to their origin. Therefore, a better economic performance. Even more, other actors may be located in the spatial proximity of the university, but the mere presence or availability of these spillovers represent only a necessary condition \citep{Colombo2010}, where the assimilation process need a social proximity.

For a comparative evaluation of these factors, thus the spin-off performance, a starting point is the \enquote{astonishing survival rate} \citep{Balderi2007}. In their empirical research, \citet{Leitch2005} found a failure rate of about 5\% over 20 years, compared with the average failure rate of venture capitalists in about 21\%. In fact, the survival rate is extremely high, especially if the firm aims to exploit a radical technology, possess broader patents and have linkages with investors \citep{OShea2004}. Studying the MIT case, \citet{Rogers2001} found that spin-offs created the 77\% of induced investment and 70\% of the employment, with respect to the overall MIT activities.

However, even if these new ventures seem a particularly effective channel for university-industry technology transfer, they require longer periods. As example, university spin-offs graduate later from their incubator, two years compared to one year from private incubators. This may be due to the embryonic and high risky projects they usually embrace \citep{Rothaermel2005}, which increase the lead time between the establishment and the actual generation of economic benefits \citep{Leitch2005}. Similarly, \citet{Perez2003} stated that rapid growth is \enquote{both rare and often even unwanted among spin-off}, due to the high uncertainties involved in the process.

At the base of the spin-off absorptive capacity, which allow them to gain such performance, there are two main factors: the employment of university personnel, as former scientists or active academic researchers, and the high proportion of turnover invested in R\&D, evaluated in about 10\% by \citet{Perez2003}. In fact, even if the common understanding of spin-off sees them as standalone technology transfer channels, they are separate entities: this will require a transfer between the original institution to the new venture, and later between the new venture and the market. 

This means that every channel of technology transfer may be applied to the spin-off firm itself. In example, \citet{Perez2003} cited training activities in the initial stage, consulting and product development in latter phase between the new venture and the market. Other examples may involve the access to laboratories and other resources, testing equipment, the usage of the university's own networks, formal and informal consulting with former colleagues, information services, etc.

Other common characteristics among spin-offs are the size of the customer base and its great flexibility. For the former, they usually have only a few customers in initial stages of technology development, which will serve as lead users and validators. For the latter, these firms usually are smaller, flexible organizations which can adapt faster and efficaciously to new technologies and market needs. 

Their peculiarities influence the types of activity they will perform. A generic framework differentiates among three strands of business activities \citep{Mustar2006}: consultancy; product oriented activities, i.e. product development; and assent oriented activities, i.e. R\&D aimed to license and development of infrastructures. \citet{Druilhe2004} further categorized spin-off activities among: consultancy and research, licensing, product related research activities, infrastructure development. 

These activities result in specific benefits for universities and parent organizations: firstly, they represent an effective instrument in bridging universities and the market, capable of solving complex and difficult contracting situation between the research institution and other firms \citep{Rizzo2015}. Secondly, these new ventures are the best suited for obtaining the best exploitation of research results and new technologies. Lastly, they represent an employment alternative for researchers and students, as well as an income source for tenured academics.

\subsection{Types and differences}

Spin-offs are only a small group in the landscape of new technology-based firms, and as \citet{Franklin2001} observed: \enquote{spinouts company scenarios in practices are highly variable and defy any formulaic approach}. In order to get a better understanding of the phenomenon, a comparative approach can be taken.

First, corporate spin-offs are far more frequent than universities'. One of the most significant difference relies on the difficulties they may encounter during their development path. Corporate spin-offs may lack the necessary linkages with R\&D institutions in early stages, slowing the development process; inversely, the ones arising from a university may lack channels for customers and suppliers in later stages, which a parent firm would have provided.

Widening the corporate spin-off concept, \citet{Colombo2010} considered the category of New-Technology Based Firms (henceforth NTBFs). The authors used this categorization to better uncover the advantages that Academic Start-Ups (ASUs) may take from the proximity to the university, in an absorptive capacity perspective. Their hypothesis was that these ASU, through social networks developed during the previous employment in the institution, would have privileged access to university resources, knowledge, innovations and networks; at the same time, the research orientation of these firms should enforce their absorptive capacity. Once combined, these factors should allow academic spin-offs to perform better than other NTBFs, as suggested from their empirical results.

\citet{Mustar2006} used a similar category, Research-Based Spin-Offs (RBSOs) when investigating the different challenges that academic and industrial spin-offs may encounter. They found that the university type, born in \enquote{what is historically a non-commercial environment} may experience specific problems as the lack of commercial resources and conflicts in objectives among stakeholders. Lastly, \citet{Leitch2005} analyzed the case of  second-generation spin-offs (a new venture generated by a spin-off) with mixed characteristics from both the industrial and the academic environment.

The same authors distinguished among university spin-offs, as new ventures based on the knowledge generated within the institution, and university founded companies, which refer to commercial opportunities exploited by academic personnel that may be unrelated to the organizational knowledge base and research activity. Within the university's category, however, a difference has gained far more attention from the literature: academic entrepreneurs versus surrogate entrepreneurs.

\citet{Radosevich1995} in particular analyzed and compared these different approaches to the spin-off entrepreneurial leadership. The first kind, the academic entrepreneur, refers to a university employee, i.e. researcher, lecturer or tenured professor, who takes the lead of the new venture both along with the previous occupation or by leaving it. The second type, the surrogate entrepreneur, is an external figure who is provided with the right to exploit a technology initially developed within the institution. Both alternatives have advantages and disadvantages either.

As an example, an academic entrepreneur may bring to the new venture a strong knowledge base, a wide network of personal contacts both in scientific and industrial research environments, and a strong commitment to the technology. However, their lack of business experience and knowledge may negatively affect the firm growth, especially if they refuse to leave the previous employment. Their downsides can be overcome by a strong support structure, which will require a significant financial effort from the institution. 

On the other hand, a surrogate entrepreneur may solve the problem of an inventor unwilling to leave his position, as well as bring business knowledge and expertise and useful industry linkages. Moreover, empirical findings showed a faster spin-off growth, when external figures take the lead. However, they may require a payment, especially up-front if they are not serial entrepreneurs, apart from the obvious lack of technical knowledge. 

\citet{Lacetera2006} instead approached the differences between university and industrial researchers in engaging the spin-off process, by modeling their activities as a game. Notably, he found that academic entrepreneurs may be both more reluctant to participate in the process or move faster, depending on the benefit they can derive from the pre-commercial research. On the other hand, industrial scientists focus on directly applicable research, lowering the cost of entering the market but possibly delaying the spun out.   

\subsection{Motives and factors}

Academic and industrial entrepreneurs may be driven by two general kind of forces: the opportunity, which arises from the individual level, and the necessity, mainly due to the external context. Therefore, the variety of influencing factors can be categorized in individual and environmental; other useful categories refer to the characteristics of the spinning university and the technology itself.

Individual motivation factors have already been discussed; however, specific variables have a have the greatest impact on the academic willingness to start a new venture. Apart from the generic entrepreneurial attitude, peer recognition, university culture and the need for research funding, \citet{Rizzo2015} found relevant the seek for independence and tax avoidance, tightly linked to the fundamental independence of a new organization. 

On this very topic, \citet{Perez2003} found the freedom to explore new ideas as fundamental, and more common among spin-off founders (against traditional new venture founders). \citet{Siegel2007} found determinant the involvement in local groups, especially the ones with a great entrepreneurial attitude, and suggested as an additional, possible motive the lack of academic recognition. The lack of prospect in the current employment may be significant \citep{Rizzo2015}; lastly, \citet{Ittelson2002} found that the academic's willingness to actively participate significantly, and positively influence the success of the venture.

Similarly,  some university characteristics, even if previously describe, should be cited for their peculiar impact. \citet{Lockett2005a} found, in their empirical research, a positive relationship between the university's stock of knowledge and the amount of spin-off generated, as well as for the availability and expenditure in the TTO and proper, professional advice. Other relevant variables are the linkages with other universities and the faculty involvement in the spin-off management, which however may delay, rather than facilitate, the spin-off graduation from the incubator \citep{Rothaermel2005}.

Environmental factors instead have been studied by many authors in a variety of contexts; should be noted, in fact, that context peculiarities expose different and sometimes unique elements. Transversal variables may include the access to complementary resources, funding constraints, availability of venture capital funds, presence of a supporting policy and relative tools, government expenditure in R\&D, patent regulation and effectiveness, legal assignment of inventions \citep{OShea2004, Fini2009, Rizzo2015}. Similar, but loosely coupled variables are the extent of knowledge spillovers from the university, the age of the spinning university, the number of local universities and the amount of educated human capital available \citep{Audretsch2005}.

Other factors seem to be peculiar to the local context, and do not appear as significant in every empirical research: these are rather linked to the geographical localization \citep{OShea2004}. Examples are the unemployment rate, a low demand for doctorate holders, a general dissatisfactory situation \citep{Rizzo2015}. Special attention has been devoted to the local and regional knowledge infrastructure: a developed infrastructure should increase the ability to access knowledge, competencies, expertise and relevant social networks \citep{OShea2004}, thus increasing the amount and extent of local entrepreneurial activities. Similarly, empirical findings from \citet{Audretsch2005} indicate a positive impact of the technology capacity of the region.

Lastly, the characteristics of the technology itself may influence the process, i.e. its ability to constitute a platform, the extent of potential applications, the presence of a market (different) standard, and incumbents. An important factor is the stage of development of the technology: early stages are usually characterized by higher transactional costs, making the spin-off channel a suitable alternative \citep{Rizzo2015}. 

\section{Other external agents}

\subsection{Research organizations}

Universities do not relate only with firms, nor the flow of knowledge and innovation is unidirectional. Some authors have studied various collaborative research organizations and in a minor extent other independent research institutions. For the former, examples are collective research centers, consortia research centers, university-based research centers and R\&D alliances. 

Firstly, \citet{Spithoven2011} studied the different approaches that SME and firms in traditional industries might take on the technology transfer; they hypothesized that these firms may need assistance in building their own absorptive capacity, uncovering a potential role for collective research centers. As previously stated, R\&D performances are the most common proxy for the absorptive capacity: empirical research shows that over a half (51.6\%) of SME do not have internal R\&D activities at all. In this scenario, collective research centers can perform three types of useful and relevant activities, from which firms can take advantages: identify and monitor relevant technology, as a proactive knowledge intelligence unit; assimilate and transform the incoming knowledge, as a knowledge agency on demand; disseminate information, as a knowledge repository for firms. Their ability to perform these activities comes from the effort they devote to R\&D activities, usually about half of their operations. Research related activities may involve collective and contract research and technology advisory services.

Collective research centers may be founded and financed in equity by the various organizations involved, while consortial research centers are the non-equity flavor of the same organization: all participants typically pay a membership fee or subscription and receive access to the knowledge output \citep{Hayton2013}. It is important to note that both these organizations may perform internal research activities, taking an active role and distinguishing themselves from a mere intermediary actor, such as a technology transfer center. In this perspective, they are a mechanism for sharing the risks of the development, especially for truly innovative projects. Other potential benefits include economies of scale and scope, reducing the required investment, and the convergence of new technologies toward a unique standard for the entire market. Therefore, their role as a source of competitive advantage.

R\&D alliances are similar in spirit, but have a very different purpose: instead of serving as a channel for the absorption and diffusion of new, external innovations, these organization rather aim to recombine previously available knowledge to actively create new technologies \citep{Lin2012}. In fact, these institutions are usually participated by an ensemble of actors with the explicit purpose of exploring and develop new technologies. Like the preceding examples, they also reduce required investments and risks, speed up the development cycle and reduce the lead time, give access to external knowledge and technology previously unavailable. Should be noted that of peculiar importance, in this case, is the cognitive distance among participants: the literature suggests a U-shaped relationship between the R\&D alliance success probability and the cognitive distance, exposing the need for different but overlapping knowledge bases.

\citet{Rogers2001} instead studied the university-based research center, which is a similar co-participated organization but characterized by a stronger orientation to the science approach, possibly due to the larger involvement of a participant university. They may be more effective than the single research university, due to the interdisciplinary nature of their research activities, but they do not have an intermediation approach, thus not very effective in bringing research outcomes to the market.

Other authors investigated the difference between universities and other public research organizations. In the most generic framework, \citet{Teirlinck2012} suggested that the latter may possess a more practical knowledge and a mission more focused on technology transfer, but they usually rely on more large-scale and complex research facilities. However, their business orientation, practices, experience, and linkages \citep{Debackere2005} unitedly to private management schemes, may help respond more effectively and quickly to specific industrial needs. Similarly, federal labs are characterized by a more interdisciplinary research and the ability to gather more resources \citep{Bozeman2000}.

\subsection{Technology transfer organizations}

As stated before, technology transfer offices and organizations may be external to the knowledge production institution. Some authors, in fact, found empirical proofs of the increasing relevance of external entities, being them private, public or government ventures.

A first example can be taken from \citet{Geuna2009}, who observed how European and national governments' approaches to the technology transfer had endorsed the formation and growth of TTOs associations and networks. Their main goals are to develop best practices and to diffuse them toward any associate, to provide training support and international connections, to collect data and influence national and European policies on the topic. Two examples are the Knowledge Integration Community, created by the Cambridge-MIT Institute (CMI) and financially backed by the UK government, and the NetVal, the Italian network for the valorization of the university research.

Similar entities are the technology transfer centers, which aim to promote cutting-edge research projects and help in transferring the results also to SME and firms in low and medium technology industries; moreover, they are significantly useful in bridging organizations with different backgrounds, in the case of multi-disciplinary projects, thus promoting cross-fertilization activities. \citet{Comacchio2012} found qualitative indications of the relevance of these actors based on their boundary-spanning activities: following a social network approach, they found a positive impact of such organizations on the network density, besides covering structural holes. Moreover, they act as an interface between institutions and private firms, translating not only needs and objectives but also the very knowledge that is transferred. These results should also apply to other intermediaries.

\subsection{Incubators and science parks}
Other external entities acquire a great relevance specifically in the spin-off process. The two main institutions considered here are incubators and science parks. 

Even if incubators can be found inside the university organization, they are usually external, independent entities. They provide several advantages, relatively to the new venture: to conserve cash, accelerate the commercialization process, give access to professional advice and laboratory facilities, help academic founders to shift from the university culture and perspective to an entrepreneurial one \citep{Ittelson2002}. Additionally, their involvement increase the confidence of investors, helping spin-off in accessing financial resources.

\citet{Clarysse2005} analyzed incubators' activities to uncover any difference in their approach; they found three different models. The first is the \enquote{low selective mode}, centered in advising and occasionally finance projects at a very early stage; they do no seek opportunities and rely on a natural selection process instead of evaluating and selecting the various proposal; due to the large number of spin-offs and firms they usually are incubating, the funding is typically seen as a mean of subsistence for founders, until later stages. 

The second is the \enquote{supportive model}: entry barriers to this type of incubators are higher, due to clear selection criteria and an active screening. They provide an extensive assistance up to the validation stage, through their professional staff; they may also provide financial support acting as a venture capital fund. Lastly, the \enquote{incubator model} have the clear, specific intent to create financially attractive spinouts: this needs an active opportunity seeking, a full and integrated support and the large use of external linkages with industry and venture capitalists.

In any case, rather than simply intermediate between industrial and financial partners and the spin-off, incubators provide access and integrate new ventures in their network, ultimately aiming at assisting firms in developing their own. In fact, both incubators and science parks are considered \enquote{intermediate organizations that provide the social environment, technological and organizational resources, and managerial expertise} \citep{Phan2005}. In a sociological perspective, as an example, incubators can be seen as micro-communities, where startuppers can develop and test business models under a protective umbrella. 

Science parks, in fact, are more oriented to provide this kind of asset. As stated by \citet{Siegel2003}, their main missions are: to foster the formation and growth of innovative firms; to provide an environment which promotes collaboration between large and new, innovative ventures; to facilitate the establishment of linkages with research centers and similar institutions. Similarly to incubators, the main mechanism is to provide the necessary business skills and knowledge, physical resources and financing.

What makes incubators and science parks differ, instead, are the objectives their mission is mapped to; science parks may be seen as an intermediary combining a technology transfer organization and a business incubator. According to \citet{Siegel2003}, their objectives are: to facilitate the university technology transfer; to foster the formation and growth of NTBFs; attracting external firms; promote the formation of strategic alliances. They may be functional also to the local and national institutions, by aiming at the economic development, job creation, and the enhancement of the local innovation system and entrepreneurial environment. 

\subsection{An endless list}

These are only examples of the ensemble of institutions and entities that may surround the university in its technology transfer activities. In fact, any actor can be involved, due to the pervasiveness the academic's third mission; however, many of them have been overlooked by the economic literature. As examples, some authors cited graduates and students as channels for technology transfer \citep{Segal1986, Audretsch2004,OShea2005,Guerrero2014}, but seems to lack a proper, dedicated and quantitative study on their impact on the innovation system. Another example, found cited only in \citet{Balderi2010} is the existence and usage of external firms specialized in very narrow activities, such as patent attorneys, firms specialized in patent valorization and patent infringement. These, however, are only examples, in the broad economic landscape of specialized businesses, and the more competitive will grow this phenomenon, the more specialized will be supporting firms. 

\section{Recap}

This chapter delineated the three main groups of receivers of academic knowledge.

Firstly, has been described the firms' motives in participate technology transfer processes and their impact on the competitiveness of the company. The section further articulated the major factors that may increase the probability of a successful transfer and its performance: the absorptive capacity, as in the capability to internalize and exploit new knowledge; and the social capital, networks, and localization choices, which determine the set of technologies accessible to the firm.

Secondly, has been presented the spin-off phenomenon. The section outlined their strengths and weakness in respect to common new venture; later, has been reported the various definition for a number of different types of spin-offs, each with different advantages and less favorable traits. Lastly, has been described the motives that drive the creation of a new venture, and the factors that shape their performance and affect their success probability. The third section provided information about the main other groups of knowledge receivers and their characteristics: research organizations, technology transfer organizations, incubators and science parks. 

Again, while public universities and private research organizations may share these groups of receivers, the modalities through which actors relates may be fundamental different. In example, firms may be more inclined to outsource their development activities to private entities, spinning a new venture may be easier from a private environment, while research and technology transfer organizations may be perceived more as competitors than as intermediaries. The \hyperref[Chapter6]{Chapter 6} will later expose the differences while discussing the shift in preferences and mechanisms for the technology transfer. 