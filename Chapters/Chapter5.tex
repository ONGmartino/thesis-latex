% Chapter 1

\Chapter{Policies and evaluation}{For universities and governments} % Main chapter title

\label{Chapter5} % For referencing the chapter elsewhere, use \ref{Chapter1} 

The previous chapters described both the ends of the transfer process, university (\hyperref[Chapter2]{Chapter 2}) and industry (\hyperref[Chapter3]{Chapter 3}), as well as the flow of information between them (\hyperref[Chapter4]{Chapter 4}). However, the technology transfer can also be represented as an organizational process, internal to every istitution that takes part in the transfer. This implicates the need to manage it, in order to gain a better performance: starting from its evaluation, up to the design of policies aimed to improve its efficiency or effectiveness. In fact, the monitoring and management of the organizational performance arise a greater significance in a private enviroment, as \hyperref[Chapter6]{Chapter 6} and \hyperref[Chapter7]{Chapter 7} will discuss later with the same approach as here.

\section{Evaluation}

Before the various policies implications and suggestions, it is useful to consider the evaluation of the technology transfer process, along with its issues. As a starting point, the evaluation can be conducted in two different levels: efficiency and effectiveness. The former refers to the relation between the process costs and the general outcome, as in number of patents, spinoffs, and contracts, and the revenue stream they generate; the last describe the ability of the research organizations to modify the innovativeness and the overall economic performance of its context. 

While this simplified perspective may picture the evaluation as easy and straightforward, the availability of codified data and the identification of the true impact of research organizations makes \enquote{the measurement of the knowledge transferred from a public research institution towards other actors represent an impossible objective}, according to the European commission \citep{Balderi2010}. 

The data collection is the first issue. Quantitative and financial data are available only for formal technology transfer, but even these codified information may be incomplete: in example, data on patents usually refer to university-owned, rather than university-invented technologies \citep{Geuna2009}, while formal channels do not include direct transfers between academics and firms, where universities do not take part in the agreement. This issue uncovers the need for the usage of data at both the institutional and individual levels \citep{Wong2010}, including data on activities rather than the simple outcomes. 

The second issue refer to the seek for viable indicators of research organization impact on their context. A first problem is to identify the true effects of the technology transfer on the society, including as example the level of innovativeness, employment, economic performance, as well as numerous other socio-economic indicators. The second problem is to isolate the single organization's impact on these indicators from the presence and activity of other organizations in the context, i.e. other universities and public research institutions located in the same city or district. Again, a related issue is to gather data on the entirety of the phenomenon, including informal technology transfer. 

Moreover, there are divergences in the literature around the indicators, especially the ones used to evaluate TTOs performance \citet{Muscio2010}. Crucial indicators identified by the European commission include: numbers of identified inventions, patent applications, granted patents, licenses and their income, the number of spinoffs and of industry-funded research contracts \citep{Balderi2010}. Other relevant variables, used by \citet{Guerrero2014}, are: co-publication of scientific papers, co-invention of patents, new enterprises generated (which include, but is not limited to, spinoffs) and the increase in the rate of employment. 

It is to note, however, that similar indicators constitute only a proxy of the extent of the technology transfer activities carried on by universities \citep{Leydesdorff2010}: the measurement of formal or contractual channels do not assess the entirely technology transfer, nor the ratio of formal and informal mechanisms will be the same for every research organization. This is particularly true when considering the efficacy, rather than the efficiency of universities, due to the presence of significant indirect effects of the local system and the society more in general.

Lastly, the two dimensions of efficiency and effectiveness refer to two different levels of analysis: the organizational level, in refer to the single institution and its internal mechanisms, and the macro-economic perspective, to include its effects on the district, local system innovation or other unit of measurement of choice. Similarly, in the first case results can be used to improve the internal policies and processes; in the second case results may drive to suggestions for government and local policies.

Given the relative availability of shared, quantitative methods of analysis, a considerable part of the literature has focused on the evaluation of performances at a micro level, using production-function framework, especially the Stochastic Frontier Estimation and the Data Envelopment Analysis \citep{Siegel2007}. The macro-level analysis, instead, has long been restricted to more qualitative analysis, given the difficulties in comparing greatly different contexts and countries; however, the social network analysis nowadays constitutes a viable and effective alternative to the challenging evaluation of traditional perspective such as NIS, RIS and the triple helix frameworks.

\section{SFE, DEA and qualitative approaches}

Most analysis on the university efficiency in technology transfer activities uses the production-function approach, which relates the output of the process to its factor of production. Briefly, it prescribes the construction of a best practice frontier; the distance of the specific institution's combination and the frontier represent its level of inefficiency or inability to maximize its output. Two different methodologies can be used to build the production function: the parametric and the non-parametric approaches.

The parametric approach requires the researcher to specify the functional form of the production function; later, parameters will be estimated through a regression. Specifically, the manual specification of the functional form is at the same time an advantage and a disadvantage of this approach: it will enable the researcher to distinguish between the interesting variables and avoid the underlying noise in the dataset, but it assumes implicitly that the selected parameters will be effectively representative of the phenomenon and will be the same for every organization. A parametric methodology often used is the Stochastic frontier estimation (SFE); two key examples are the contribution of \citet{Siegel2003a} and \citet{Link2005}.

\citet{Siegel2003a} use eight different variables to describe the phenomenon: the number of inventions disclosed annually, TTO employees, annual expenditures in external legal support, the presence of a medical school, the status of public or private institution, the age of the TTO, the average industrial expenditure in R\&D and the average annual real output growth in the university's state. They found a positive correlation of licensing output and invention disclosures, while the number of employees influence the number of licenses but not the total income generated; the amount of legal expenditures instead is positively related to the total incomes, while slightly reducing the number of licenses. Lastly, older TTOs tend to perform better and local R\&D expenditures positively influence the university output. 

Notably, they conclude that deviations from the production frontier cannot be explained only through institutional and environmental factors: relative differences may also arise from organizational practices. Their contribution constitutes the very basis for any successive analysis of organizational factors previously cited discussing individual, TTOs, faculties and university level policies and practices. 

In fact, \citet{Link2005} further develop the model found in \citet{Siegel2003a}, including two organizational variables: the percentage of licensing royalties allocated to the academic inventor, and the type of structure for describing a centralized or decentralized licensing office. The first of the two new variables, used as a proxy for the various organizational practices that favor the technology transfer, has in fact the strongest impact on the relative performance: the higher the share, the higher the overall performance. They conclude suggesting a further development of the model to include non-pecuniary incentives, as promotion, tenure policies and alike. 

The non-parametric approach instead does not require to specify the functional form of the production function, allowing researchers to identify best practices. However, if on one hand it avoids similar assumptions, on the other hand cannot distinguish and avoid the noise embedded in the data. The most common non-parametric model is the Data envelopment analysis (DEA), originally developed by \citet{Fare1993}, usually used to compare the relative inefficiencies of universities. Three relevant examples are \citet{Thursby2002} and \citet{Anderson2007}.

\citet{Thursby2002} used the DEA framework to examine the growth in the Total factor productivity (TFP) between four different years. Specifically, they decomposed and calculated the movement toward or away from the frontier, therefore at the change in performance at individual level, and the frontier shift, as the difference on aggregate results. They found a substantial increase in the propensity to patent, while the propensity to disclose reported a modest increase. Notably, they found a negative growth for licenses, which they attributed to a bias in the analysis, specifically the time lag, or at universities factors, i.e. more demanding and aggressive TTOs. They also found a generalized growth in the frontier, but significant and increasing inefficiencies among universities.

Similarly, \citet{Anderson2007} used the DEA as a tool for productivity evaluation, but they focused on the direct comparison of universities. The analysis revealed a positive correlation between performance and the status of private universities, as well as the presence of a medical school; however, the regression shown a surprisingly little explanation power and statistical significant. Anyway, the assessment of inefficiency scores reported considerable high results ranging from 112.5\% to 619.3\%.

Lastly, \citet{Chapple2005} combined the usage of the SFE and DEA in assessing the relative performance of UK universities. For the parametric analysis, they expanded the model from \citet{Siegel2003a} including the total research income of the university, and substitute the average annual real output growth with the regional GDP per capita, with the first resulting significantly and positive influencing both the number and incomes of licensing agreements; they also found a positive effect for the number of employees. In all models, they found a substantial inefficiency: on average, universities were operating at a 18.7\% of efficiency. However, contrary to previous results based on US universities, they found decreasing returns of scale to in licensing activities.

\subsection{Other approaches}

The production-function framework is the most widely used quantitative method, but other authors took different approaches. \citet{Lee2000}, in example, provided an extensive qualitative assessment of the technology transfer phenomenon, investigating 425 different US R\&D projects. Notably, he reported that in 82\% of the projects is mentioned the company support, but the most frequent is the collaboration with the federal government. He seek the relative importance of benefits for both academics and firms: in the first case, 67\% of the interviewees mentioned as significant the funds necessary to support graduates, 66\% the insight that can be obtained in collaborating, 56\% the possibility to conduct filed tests. For firms, the main benefits are the access to new knowledge and research, the development of new products (76\%) and the contribution to new patents.

\citet{Resende2013} instead tried to develop a theoretical framework for assessing and guiding the technology transfer process, the \enquote{master plan for technology transfer}. The tool contains 271 rules and good practices, referring to 32 different facilitators in 7 groups. Eventually, they describe a tool called \enquote{Best Transfer Practices}, a qualitative tool that can be employed in assessing TTOs and institutions. At its root, the tool prescribes an initial documental analysis, to identify the key actors and the main operational issues; later, strategic objectives for the technology transfer should be formulated. In a second major phase, the goal is to map the various facilitators, to identify the most relevant ones, bottlenecks and critical facilitators, and measured their correlation. Lastly, facilitators should be remapped, then the reporting of both the analysis and recommendations. 

Among quantitative methods, traditional approaches based on the production-frontier framework are able to assess the individual efficiency in the usage of input resources. A major strength of these tools it the capability to incorporate both financial measures, i.e. revenues, attorney expenses and alike, and dummy variables for other major traits of the universities, which cannot be expressed by an economic measure: structure, presence of policies etc. However, these tools lack the ability to assess one of the most relevant factors that influences the technology transfer: the external linkages of the research organization, its context and the relative importance in it. 

As previously stated while discussing absorptive capacity, brokerage, social capital and other constructs, networks and the networking activity are essential to the technology transfer. To assess this aspect of the process, in the latest years the literature had largely employed a tool called social network analysis. This methodology uses both information about the individual actor and its relations, intended as presence and possibly intensity, to specifically assess the network position and performance of the institution and uncover the social structure in which it is embedded \citep{Pinheiro2015}. This methodology can be used also to assess the organization's efficiency, but its ability to compare the relative importance of actors among the network makes this tool best suited to evaluate the effectiveness of the technology transfer process, providing one of the few quantitative measurements available for the performance at a systemic level.

\section{Policies}

One of the goals of the aforementioned econometric analyses is to use results as guidelines for policies suggestions and modifications. In fact, many of the previously cited authors concluded with the policy implications of their work, but since they mainly treat the micro-level of the university itself, the largest part of quantitative-driven policies suggestions refers to the university-level policies, practices and alike. These will be the first too be presented here.

To investigate the institutional and government level policies, instead, a different approach must be taken: this level of policies mainly refers to the efficacy, rather than the efficiency, of the technology transfer. Since the previously stated issues in its assessment, the literature focused instead on qualitative and comparative approaches to evaluate the performance on a systemic level, mainly referring to frameworks as the triple helix and regional innovation systems instead of the pure technology transfer process. Few other authors used social network analysis tools and narrow quantitative analyses to uncover specific trait of the process. This second level of policies will be discussed later in this chapter.

A last level of analysis, proposed by \citet{Rasmussen2006} instead seems to be overlooked: the author differentiated between top-down policies, imposed by universities administrations and government, and bottom-up policies, emerging directly from individuals. This last type of policies mainly refers to informal expectations, procedures and instruments arising spontaneously from the academic social environment: due to the difficulties in gathering data on this phenomenon, no researches seem to have been performed in this perspective, at the best of my knowledge. However, institutional policies regarding the organizational climate can provide a useful insight on this topic.

\subsection{University-level policies}

As noted by \citet{Geuna2009}, some universities have made the strategic decision to institutionalize technology transfer activities, instead of letting them driven by individual researchers and their network. However, this decision must also involve the writing and enforcement of internal policies, in order to promote and manage these activities. In fact, \citet{Debackere2005} cited as responsible of the low levels of European university-industry linkages, among issues on both sides, institutional factors and incentive structures on the \enquote{science side}. In this perspective, universities must take an active role in fostering academics motivation and institutional supportiveness, and policies have been considered the most important tool.

Moreover, institution-wide policies are the most suited instruments for changing and shaping the university attitude toward technology transfer: as previously noted, the traditional open science approach can be a very restraining issue; procedures, mechanisms and specific policies can regulate the single aspect of the transfer and can provide supporting measures. Among other instruments, a powerful, illuminated leadership, context and networking activities can shape the attitude toward commercialization, but policies are in fact the most widely used, easier to manage and largely researched tools for this task. 

A first example of their relevance can be taken from \citet{Bercovitz2006}: policies can bridge the open science attitude and commercialization activities by seeking a \enquote{compromise that accommodates the public good nature of knowledge spillovers while providing the property rights that are required to guarantee returns for the additional private investment required to commercialize academic research}.  Considering the organizational mission as one of the many policy instruments available, \citet{Debackere2005} observes that the implementation of industry relations as a central component of the institutional mission can be an \enquote{especially successful} tool for improving the technology transfer capabilities and performance. 

A proper management of these policies, however, is fundamental: \citet{DEste2011} in example stated that undue policy emphasis on commercial activities, usually linked to the financial need of the university, may obscure other considerable benefits that industry engagement might have on research activities, especially non-pecuniary ones. Similarly, \citet{Debackere2005} reported the need to balance the portfolio of activities and financing sources, between government funds for long-term, fundamental research and industry financing for applied, short-term research. 

Another potential failure of these policies refers to tacit knowledge and informal transfer channels: according to \citet{Muscio2010} in this case management policies \enquote{can achieve little} in fostering informal transfer, especially considered the higher difficulties in assessing and evaluating the commercial value of this kind of knowledge. 

\subsubsection{General policies}

The first strand of policies has as its subject the entire organization, encompassing every office, faculty and researcher, through a range of mechanisms oriented toward two different objectives. The first is to foster the development of an entrepreneurial climate at the organizational level, i.e. through TTOs, procedures and policy for the IP management, engagement with industry partners etc.; the second refer to a set of human resource practices to intervein at a more individual level, like entrepreneurship education programs, mentoring and the enrolment of individuals with relevant expertise to act as role models \citep{Guerrero2014}. 

To foster the entrepreneurial climate, \citet{Klofsten2000} suggested the creation and maintenance of a proper organizational culture, along with the provision of separate courses for entrepreneurship and training programs. In fact, the main use for intervening at the organizational level is to modify the perception that researchers have of the context, its acceptance and supportiveness for entrepreneurial activities, thus their willingness to engage such activities.

\citet{Hunter2011} investigate this perspective further differentiating among commercialization-support climate and the boundary-spanning climate. The first case refers to the presence of a strong, oriented leadership, procedures on engaging in technology transfer and in managing the relation, the creation of normative expectations of entrepreneurial activities arising from researchers. The second case instead includes practices, procedures and organizational structures aimed at fostering the exchange of information and collaboration among different teams, units and faculties. In fact, empirical results from \citet{Hunter2011} evaluated the importance of this climate. He found that an increase in the supportiveness perception lead to a 4.44 times increasing of invention disclosures, 30.57 for boundary-spanning climate. 

Other critical organizational factors have been listed by \citet{Debackere2005}, i.e. the policy for royalty and equity distribution to researchers and the organizational role and position of the TTO, which specifically influence its ability to act as gatekeeper or boundary spanner. Among practical suggestion he made, are the combination of basic and applied research within teams, to promote direct transfer and day-by-day proximity, the build of relationships with venture capitalists, the usage of external technology transfer organizations and consultants. Further attention should be placed in the creation, management and monitoring of research contracts and knowledge policies.

Similarly, \citet{Guerrero2014} exposes the need for an entrepreneurial lead highly committed to transform the institution into entrepreneurial and innovative organizations. Other organizational tools may be: cross-functional research teams, the availability of positions for external and foreign researchers, alliances and networking activities with other organizations, the provision of financial capital to support the various transfer processes. Other themes form \citet{OwenSmith2001} are the publicity and widespread awareness of university successes in technology transfer and the status benefit ascribed to commercial accomplishment.

Building an entrepreneurial climate however does not influence only the individual attitudes, motives and incentives: it shapes also the organizational structure. \citet{Rasmussen2006} suggested to treat them not as separate entities, but to let culture and structure to mutually reinforce themselves, and to take advantages of the synergies. Also \citet{Guerrero2014} expressed the need for entrepreneurial organizational structures, specifically linking them to the university ability to transmit and communicate its willingness to engage in technology transfer and entrepreneurial activities to both internal researchers and officers, and to external entities.  

Lastly, \citet{OShea2005} suggested a four-point policy roadmap for developing an organizational context that can positively influence the technology transfer: to build and maintain a commercially supportive culture; to engage active partnerships with financial support from industry and government; to recruit and develop academic stars; to develop a commercial infrastructure in support to commercial and valorization activities.

Elsewhere, the need for proper human resource practices arises from the fundamentality of the human capital in technology transfer processes. Many universities have already formal policies for encourage researchers to seek industry collaborations and assignments \citep{DEste2011}; basic examples are the preferential treatment for inventors, the provision of specific contractual arrangements and access to R\&D laboratories and other university resources for any further research required by the transfer \citep{Fini2009}.

Among individual policies, a considerable part of the literature agrees on the primary importance of personal evaluation and tenure promotion policies, and their link to technology transfer activities \citep{Debackere2005}. One of the most important contribution on this topic is \citet{Genshaft2016}, who made 5 recommendations on the topic that can be summarized as follow:
\begin{itemize}
\item Policy statements should acknowledge the importance of the technology transfer, but they should also regulate the relation between research, teaching activities and the third mission, including specific safeguards against conflicts of interests among activities.
\item technology transfer and commercialization activities should be included explicitly among the relevant criteria for the personal evaluation and tenure promotion; they should be assessed with enough flexibility to include many relevant characteristic of the transfer, i.e. the relative importance in various disciplines, innovativeness and effort required, using the intellectual contribution and the expected social benefit as guidelines. 
\item While it is important to acknowledge and evaluate such activities, it is fundamental to consider them as an optional component of the researcher performance, not among mandatory objective; \citet{Rasmussen2006} highlight the importance of a voluntary contribution, including the freedom to publish results and the possibility to choose between commercialization and traditional research activities. 
\end{itemize}

While organizational practices and policies may enable and support technology transfer, these policies sustain the individual motivation to engage these activities: the opportunity for additional incomes may be insufficient, apart from its need to be managed. Similar in spirit are pecuniary and non-pecuniary incentives for academics \citep{Link2007}, i.e. attractive individual remuneration packages \citep{Debackere2005}. However, non-pecuniary incentives as tenure and promotion policies should be preferred for policy intervention, since \citet{Friedman2003} found that greater pecuniary rewards are not significantly associated to higher commercialization activities.

Among pecuniary incentives, \citet{Beath2000} specifically investigate the two major alternatives that universities have to use technology transfer activities as a source of incomes. They consider two major alternatives, specific to the case of consulting, collaboration and contract researches: overhead charges and to hold down academic pay. Four factors shape the optimal \enquote{taxation} level: relative productivity of researchers in fundamental and applied research; relative amount of time required by keeping up with the literature; the intrinsic preference for fundamental research. However, the authors individuated a rule-of-thumb for establishing the tax rate: the ratio between university wage and industry wage.

Alongside with researchers, some of these policies may be extended to students themselves: access to laboratories, financial incentives, positive impact on their academic evaluation and career and alike. After all, as stated by \citet{Rasmussen2006}, one of the university's objectives is to \enquote{educate and support students in their commercialization efforts}, by integrating commercialization into education activities and academic curricula. 

Lastly, it is important to use a strategic approach in managing these different policies \citep{Siegel2007}. They must match the overall institutional goals and priorities, research fields of choices, effort in the regional development, resources allocation and similar institutional factors. However, \citet{Rasmussen2006} empirically found that university initiatives are \enquote{mainly set up to support individuals and projects already in process, while few measures are taken to motivate and stimulate the creation of new projects}; this lack may be linked to an insufficient usage of technology transfer and related policies as strategic lever, thus an undervaluation of the entire phenomenon.

---Has been demonstrated that characteristics of individual researchers have a stronger impact than departments' or universities' ones. (Guena, 2007)

\subsubsection{Patenting}

Patents are considered fundamental as means of technology transfer, because of their influence in the individuals' ability to appropriate the economic value of the newly generated knowledge and technologies \citep{Bercovitz2006} thus their ability in bridging commercial and academics reward structures \citep{OwenSmith2001}. Following the results from \citet{Tijssen2006}, university patenting output is determined by university endogenous factors, and unitedly to their relevance, these two factors eventually lead to the need for patent strategies \citep{Siegel2007} and patent regulation.

According to \citet{Baldini2007} a university patent regulation describes: which steps the inventor have to take to patent their inventions; actors and mechanisms for relative decision making; duties and benefit for both the academic and the university; the royalty distribution scheme; which part bears the costs of patenting and control the licensing process. The presence of a similar policy significantly reduces the perceived obstacles, with an impact evaluated in a triplication of patent filled \citep{Baldini2006,Baldini2007}. Contrary to the previous forms of policies, centered on the royalty distribution formula, a more comprehensive regulation in fact can be useful in determining organizational changes and supporting inventors: among possible improvement for technology transfer activities, it ranked 3rd thanks to the effectiveness due to its large scope. 

Another important area for policy intervention is the modalities through which the TTO starts and manages the patenting and licensing process; as previously stated, an appropriate behavior of the TTO is a necessary condition for the researchers' attitude to disclose inventions internally instead of contacting firms directly. Specifically, \citet{Panagopoulos2013} observes that in the worst case scenario, in which academics do not take advantage of the TTO's professionalized services, the difference in bargaining power between researchers and firms may expose a considerable loss of efficiency in the overall technology transfer process.

The author in fact proposed as solution the institution of an \enquote{insurance} for proceeding through institutional channels, specifically the TTO: this insurance, that can be both pecuniary and not, should be granted on disclosure, and guaranty the researcher some return in case the licensing fails to take place. To establish its amount, administrators and TTO officers should compare the expected payoff, for both the university and the researcher, to the probability of licensing, possibly adjusting it iteratively through learning by doing.

The obvious necessary condition for these two, as well as any similar policies, is the diffusion of information among researchers about their existence. In fact, the same can be stated for any of the potential benefit of technology transfer activities, given the previously cited empirical demonstration of the misinformation surrounding this phenomenon. 

While there is a need for the establishment of relationships with industrial actors \citep{Baldini2007}, there seems to be a lack or misalignment of strategic choices in technology transfer policies. In example, \citet{Phan2005} reported that the most attractive combination of technology development stage and licensing strategies for industry is at the same time the least likely to be favored by universities. Lastly it must be reported that, according to \citet{Leydesdorff2010}, patenting activities in the most advanced economies are decreasing since the 2000s; the author suggests that the motivation may be the disappearing of institutional incentives to patent, due to the changes in the evaluation and ranking procedures for universities.

\subsubsection{Spinoffs}

In respect to the patenting and licensing process, spinoff activities require a larger involvement of researchers and inventors, which in fact are the ultimate responsible for the success of the process. Therefore, a different set of policies should be deployed, specifically focused on the training and supporting of these academic entrepreneurs.

A first area for policy intervention is the so-called \enquote{knowledge gap} for new ventures, defined by \citet{Lockett2005a} as \enquote{the difference between the knowledge endowment of the start-up and the knowledge it requires to succeed}. In other words, spinoffs may lack certain knowledge resources, as in business and market knowledge, networks and relationships with industry partners and alike, that may jeopardize the probability of success. Therefore, the first objective for policies should be to bridge this gap, through the development of an appropriate culture and infrastructure, active partnerships with industry and government, the recruiting and development of star scientists. 

These activities might be confused with those previously cited as general policies for shaping the attitude of researchers toward industry and commercialization activities; however, in this case the intensity and scope of the intervention must be greater, to urge academics to embrace entrepreneurships and start their own ventures.

More specific policies may regulate the spinoff process, i.e. steps to take, rules for decision making, financial contribution, surrogate entrepreneurs \citet{Franklin2001} and alike, and the relation between the new venture and the university, as in the access to university infrastructures and laboratories, and the licensing of university-owned technologies \citep{Fini2009}. Other policies may institute business plan competitions, university incubators, science parks and a venture funds. \citet{Siegel2007} specifically cited policies for supporting the building and maintenance of network of peer scientists, which may significantly influence the performance of the spinoff. Lastly, these policies should include also students as possible entrepreneurs. 

\subsubsection{TTOs}

TTOs and their management are another major strand of policy intervention of universities. Their position in brokering relations and technologies make them essential to the technology transfer process, but they have to be managed properly in order to achieve a good performance both in efficiency and effectiveness. In fact, they can undertake their central position as facilitators, as originally intended, or as bottlenecks in the process \citep{Siegel2003a, Geuna2009}. 

\citet{Litan2008} studied this issue using an historical perspective. He linked the creation of the firsts TTOs to the Bayh-Dole act; they were not originally intended by the act: they rather arisen to cope with the increasing demand for specialized services in commercializing universities' technologies, thus acting as facilitators. Over time, however, two series of university policies encouraged their shifting to bottlenecks: the most relevant example is the type of policies that concentrated on maximizing revenue streams, rather than maximizing the volume of innovations transferred, giving them the direction toward bottlenecks. Secondly, the institution of TTOs as university monopolies on the commercialization activities gave them the ability to assume the role of gatekeeping, thus enabling them to become bottlenecks.

Nowadays, this issue can be found in the kind of incentives TTOs' officers are subjected to: their rewards are usually linked to the gross income of the office, implicitly suggesting to prefer patentable and patented technologies, with known potential licensees to with grant exclusive rights. While this kind of management might seems flawless in a competitive, open market, the profitability of the technology transfer is not the only objective and mission of both the office and the administration: these policy do not consider the impact extent of innovations and open science incentives, i.e. publication and diffusion objectives, failing in motivating academics to disclose and transfer. 

\citet{Litan2008} in fact advocates for a decentralized and more specialized organizational structure for TTOs, which should further the institutional entrepreneurial culture, increase the involvement of academics, and higher performance due to the rising specialization. Moreover, he suggested other management models as alternatives: the concurrent usage of external agencies, among which the researcher can choose; the externalization of TTOs' functions to regional alliances, if properly managed; to use internet-based matchmaking tools; to automatically assign any intellectual property right to the faculty, making them more responsible for commercial activities. 

Another area for policy intervention is the role and participation of the TTO: its supportiveness is determined by the stock of financial and human resources at disposal \citep{Siegel2007}. The demand for skilled and competent resources can be undertake by proper policies intended to attract and remunerate appropriate human resources and capital \citep{Rasmussen2006}.

Apart from skilled, TTO's staff must also be specialized in different areas \citep{Guerrero2014}: lawyers and managers with previous experience in both industry and academy, preferably former researchers \citep{Debackere2005}, with satisfactory marketing and negotiation skills \citep{Siegel2003a} and generally wide commercial skills \citep{Siegel2007}, who are also able to significantly contribute to the university networks with their own social capital \citep{Geuna2009}.

\section{Institutional-level policies}

As previously stated, the technology transfer phenomenon has a deep geographical connotation, both as proximity to other actors as well as the institutional and economic context the institution is embedded in. As examples, the geographical proximity enable a more direct communication among organizations, which allow an informal flow of knowledge and further the performance of any channel; location choices also largely influence the pool of potential partners and the probability of establishing a successful relationship, let alone incentives from the local government, taxation etc.

In a general perspective, the existence and the effect of a functional relation between organizations and their context have been studied since the Marshall's cluster theory, published within Principles of Economics \citep{Marshall1890}. Every school of economic thought have added its contribution on the phenomenon, considering different angles and perspectives, especially on the role of national government, local institution and their legislative activities.

However, it is not the aim of this work to present and discuss a literature review on this specific but wide theme. In what follows, will be presented some of the most important concepts on the topic, in the perspective of understanding what matters the most for the technology transfer processes.

\subsection{Geography does matter}

Apart from its value for the technology transfer, geography is fundamental to the very innovation process itself \citep{Asheim2009}. Proximal partners, which belong to the same context, should share a common language, code of communication, conventions and norms, possibly intra- and interorganizational routines. Similar factors facilitate the exchange of knowledge, especially if tacit, and socially organized learning processes, which constitute the most important basis for innovation-based value creation. This is the source for the importance of innovative clusters: only a firm embedded in this context can take advantage from the region's unique endowment. 

Moreover, clusters tend to specialize by industrial sectors, therefore by the knowledge base, which should be similar among embedded actors, further enhancing their mutual absorptive capacity. Agglomeration of actors and knowledge is even more important in the case of tacit knowledge, which require mechanisms as the learning by interacting, exposing the relevance of a common context. In fact, characteristics required for an effective technology transfer are highly time and space-specific \citep{Asheim2009}.

\citet{Florida2002} made a relevant example of how the context and its attributes deeply influence the kind and the performance of knowledge-related activities. He started from the concept of the creative class: a working class of creative professionals and knowledge workers, which demonstrated themselves as increasingly fundamental for both corporate profits, economic growth and social development. In his book, the author shown that while firms' location choices are influenced by the presence and extent of this working class in the local context, the creative class's location choices are determined primarily by the openness, the diversity and other social factors of the region. Thus, the significant impact of the context on economic and knowledge activities, plus the need for local and national institution, as well as the individual firm, to actively and effectively manage the cultural trait of the context. 

\citet{Yokura2013} instead used the social network analysis as a different approach to uncover the relevance of the geographic specialization; specifically, he analyzed the japan innovation system, along nine regional blocks of technology specialization in six different technology groups. Notably, he found that the spatial reach of R\&D networks is different for each field, but over the 50\% of them is limited at 100kms, regardless the domain. He also suggested that intraregional and interregional R\&D relationships are both important, but for new and smaller firms shorter distances are crucial, thus the location choice of establishment. In this perspective, also peripheral universities may have a prime role in technology transfer, not as producers but as provider of skilled labor and as a knowledge provider, an antenna for new technologies developed elsewhere.

Similarly, \citet{AzagraCaro2010} focused on intra- and extra-regional university-industry cooperation, by investigating the relation between firms' absorptive capacity and universities' R\&D activities. He found that an increase in the first will increase firms' participation in the regional university-industry interaction, while an increment of universities activities will strength the extra-regional collaboration. The author suggested that maximizing university-industry interaction, along with technology transfer activities, might be in contrast with local development policies.

Relevant is also the regional organization, which has been found by \citet{Agrawal2014} to have a significant impact on innovative performance. He suggested that the contemporary presence of at least one large research laboratory and a \enquote{sizeable} population of smaller laboratories and satellite firms may in fact increase the regional innovation productivity. In fact, the author empirically evaluated this effect in a 17\% increase of citation-weighted patent count per inventor, and a 28\% increase of spinoff formation. Specifically, a larger research center should expose a greater number of \enquote{misfit ideas} in respect to the main activities, thus producing more spillover and spinoffs, competitively fitted for a context constituted by a large population of small firms.

Lastly, policy attempts to artificially establish a cluster, a district or a high-tech region require a significant effort, both in financial resources and in time. As an example, the North Carolina's Research Triangle Park required 50 years to realize valuable economic benefits \citep{Bercovitz2006}. However, there are different policy objectives that institutions can aim to in shorter spans, to improve the performance and competitiveness of regions: human capital, organizations and institutions, investments in R\&D, industrial structure, and ‘sequences', meaning a chain of events ignited by a unique public investment (Niosi and Bellon, 2002). Moreover, \citet{Heher2006} observed that similarities in performance among different countries and cultures suggest that the innovation process might be similar whatever the environment; this should imply that also governmental policies among countries might have strategies and tools in common.  

\subsection{Short-term policiy objectives}

In fact, there are many narrow and selective policy interventions that institution can make to improve local systems in the short term, actions that are considered essential nevertheless. \citet{Tijssen2006} found in fact that the patent intensity of universities, the ratio between patents and publication productivity, representative of the overall commercial performance, are indeed determined by exogenous factors: domestic policies, regulatory frameworks, support systems. A relevant example is the case of SMEs in university-industry cooperation.

SMEs mostly present a passive attitude toward technology transfer, especially the acquisition of external technologies \citep{Yusuf2008}. However, \citet{Zeng2010} noted that government policies \enquote{can be effective only when they focus on the need to promote cooperation between SMEs and innovative partners}: it is necessary, anyway, for policies to create or promote a favorable context for cooperative projects and technology transfer in general. An alternative is to establish or enforce different forms of intermediaries, i.e. consortia and technology transfer centers. 

Similar in spirit are policies aiming to increase the amount of R\&D expenditure for local firms. As previously stated, internal R\&D activities are fundamental in fostering the firm's absorptive capacity, as well as generating spillovers in favor of the local economy and community. \citet{Fritsch2007} investigated the extent of the impact of local R\&D activities on the innovative performance of the region; using the number of employees, he found a strong correlation coefficient with the number of patents (0.73). Moreover, his empirical results strongly suggest that the geographical proximity to the university significantly influence the R\&D employment, with an elasticity between 0.22 and 0.17 in a radius of 50kms, falling to 0.07 between 50 and 75kms, a distribution that has a \enquote{remarkable degree of correspondence} with the university's external funding. Policies suggestion in this cases include the fostering of private R\&D, urging universities to seek for external funds, to favor the spatial concentration of firms. 

Another issue frequently reported by universities is the funding of university's commercial endeavor. The financial support from national governments is important, but there is a widespread believe in the economic literature that a more competitive and diversified funding might in fact further the university's willingness to engage in technology transfer and commercial activities \citep{Rasmussen2006}. On the other hand, policies should also consider the relative importance of financing intermediaries, i.e. venture capital funds, which have proven themselves as \enquote{particularly helpful} for both funding and business knowledge they can provide to spinoffs \citep{Yusuf2008}.

On the topic of business knowledge, \citet{Chapple2005} also suggests that policymakers should also intervein in guarantying the availability of appropriate human resources. This purpose that can be partially fulfilled also by public policies fostering the employee mobility \citep{Franco2000}, which eventually lead to a greater technological diffusion and knowledge exchange, both regarding scientific discoveries and business experiences and methods. 

\citet{Fini2009} summarized the different mechanisms that local and national institutions have to support the innovative context: financial development (VCs, institutional funds etc.); specific entrepreneurial support services (training, loans, physical infrastructures and direct services); and the local industrial composition (providing networks and related services). He also pointed at Bayh-Dole type legislations and similar reforms as fundamental. 

\subsection{RIS, NIS and triple helix}

For more long-term policies for innovation-based economic activities, the literature based its suggestion on the concepts of regional innovation systems and the triple helix. The first refer to a more static perspective of the phenomenon, while the triple helix focus more on a dynamic view.

The regional innovation system can be defined as the \enquote{institutional infrastructure supporting innovation within the production structure of a region} \citep{Asheim2009}. Starting point is the previously stated tendency of regions to specialize their economic activities on a relatively narrowed knowledge base, exhibiting a path dependency that bonds local organizations and institutions. Therefore, the regional competitiveness could be enhanced by promoting stronger relationships at a systemic level, both between organizations and between them and the region's knowledge and institutional infrastructure.

The literature proposed different definitions for the RIS; the simplest incorporates universities, public and private research institutions, private organizations and firms, local government. A broader definition instead includes \enquote{all parts and aspects of the economic structure and the institutional set-up affecting learning as well as searching and exploring} \citep{Etzkowitz2000}. Moreover, \citet{Asheim2009} distinguish between three different types of RIS: the first is the territorially embedded regional innovation systems, in which innovation activities take place mainly in firms with low cooperation with knowledge producer. Secondly, the regionally networked innovation systems, where the emergence of the system met stronger policy intervention and institutional planning. Lastly, regionalized national innovation systems are driven by and integrated in the entire national innovation system, with the highest degree of policy intervention.

More dynamic is the triple helix framework, which expand the RIS concept to its evolution over time. The model is based on same three different institutional spheres: university and research organizations, industry and government. Again, actors are characterized by their function, as in their set of competencies and roles in the system, and are connected by various kind of relationships: technology transfer, collaboration, networking activities. However, the triple helix suggests that the equilibrium and the connections among actors are dynamic: in the evolutionary perspective, institutions are a \enquote{co-evolving sub-sets of social systems that interact through an overlay of recursive networks and organizations, that reshape their institutional arrangements through reflexive sub-dynamics} \citep{Ranga2013}. In other words, the social dynamics underlying the innovation systems autonomously adapt to the circumstances, constituting an evolutionary mechanism that lead to institutional transformation \citep{Etzkowitz2000}.

The RIS and the triple helix frameworks can be used to picture the objective that governmental and local policies should aim to: a system in which the driving force for collaborative activities is indeed the expectation of profits, but moderated by utility functions and proper opportunity structures \citep{Etzkowitz2000}. Building blocks should be the technology innovation, which provides the variation necessary for the evolution, as its engine; markets, which should be the prevailing selector for survival; and institutional structures to provide the control system, as guidance. The resulting system should favor the collaboration and conflict moderation, especially through collaborative leadership and substitution between institutional spheres, and leave room for uncertainties and the change process, without necessarily seeking to resolve the tension among different roles \citep{Ranga2013}. 
