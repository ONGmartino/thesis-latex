\subsection{Internal perspective}

The inner processes behind these channels have been initially structured by the statue and other policies, but the actual design has also bene refined by the experience of the involved human resources, who helped in further shaping the processes' flows in an emergent fashion. Eventually, it appear as unique and unified, where different offices and employees will be activated on the basis of the ongoing needs. 

From a comprehensive perspective, the most of the technology transfer activities are performed by the Innovation and Territorial Relationship Area (AIRT). Apart from the advisory role for the various economic aspects of the research, AIRT holds five main processes: patenting and licensing, support for spin-offs, grants, and contracts, relationships management.

\subsubsection{Patenting process}

The sole Research Centers responsible for the origination of patentable technologies are the Center for Materials and Microsystems (CMM) and the Center for Information and Communication Technologies (ICT). The process of patenting, in its beginning, differ among the Center of origin, due to the presence of a specific organizational position in staff to the CMM Director, which act as facilitator through informal preparatory activities

Within the CMM, the process usually begins with researchers delivering a patentable technology, or a commercial idea, to the facilitator employee, through informal meetings or presentations. Specifically, researchers may directly suggest the patent channel; however, the decision is usually postponed to the examination and assessment of the potential commercial value of the technology, while comparing concurrent exploitation alternatives. The researcher is expected to expose key facts of the new technology, how it differs from the previous state-of-the-art, and how to extract economic and social value from it. 

After a first evaluation from the facilitator employee, especially for the anteriority research, a basic template will be written to clarify the most common and important contractual aspects. It contains information and clauses about the proprietary structure, financial incentives for the inventor, the legal process, and alike. Lately, the patent attorney will be contacted to better assess the feasibility and potential value of the endeavor. 

Otherwise, if the invention originates within the ICT, researchers should directly contact AIRT, to ask for an initial evaluation of the project, replacing the support of the CMM facilitator. In both cases, after different initial phases, the process flows will converge in a more formal procedure that begins with the so-called \enquote{invention notice}. 

It consists in a notice, similar in spirit to a dossier, which notifies AIRT and the Center Director of the reaching of a potentially patentable research product; it should include information about the discovery, its applicability to industry, identified market segment, novelty, and alike. The invention notice will be therefore evaluated, to be eventually accepted or refused. A necessary condition for a positive evaluation is the recognition of a clear potential customer or customer segment, already identified.

If the notice is positively evaluated, the patent attorney will be involved to support researchers in developing it into an inventor declaration, which will be deposited initially as an Italian patent. After twelve months, the application will be converted in a PCT or EU patent, in order to gain additional time to invest in commercialization activities before the nationalization process and its relative costs.

Valorization activities will begin immediately after the first deposit. Since the characteristic early stage of the patenting technology, AIRT and the CMM facilitator will start seeking for a potential investor, specifically interested in further developing the technology (1), apart from the usual organizations interested in licensing (2) or acquire (3) the pending patent. In the nearest future, a specific partner will be engaged in this phase: a firm specialized in patent valorization, who will perform activities like market analysis, partner scouting and customer seeking. 

While the described process matches a typical academic case, it represents only a fraction of the potential patents: the largest part of patent proposals seems to arise from EU-granted projects and direct assignments. In the former case, EU funds (especially H2020 funds) are reported to be assigned preferentially to consortia of organizations which include at lease one firm; according to the interviewee, the industrial partners usually participates EU projects because interested in any child patent that can be originated. Therefore, apart from the possible direct request of patenting of participant firms, even their sole presence will influence FBK and other research organizations to patent the results, simplifying the decision process and the seek for a licensee. 

For the latter case, instead, the direct research contract can anticipate the need for the patenting of research results: FBK already establishes and agrees, at the signing, to patent the research output. However, the property of the patent may be shared or entirely of the firm, therefore influencing the licensing process.

In both these last cases, the patenting process eventually differs for the absence of the invention disclosure and the initial evaluation of feasibility and profitability, resulting in a simplified seek for a licensee.

\subsubsection{Relationships development and management}

One of the most important tasks for AIRT is to manage the relationships with firms: the source for contract and cooperative research, the primary technology transfer channel for FBK. The inner activities behind the task can be decomposed in three different processes, due to the involvement of various employees with different objectives, which eventually constitute several points of contact between the organization and the market.

In a chronological perspective, the first is the business developer employed by AIRT. Along with marketing activities, the resource actively elicits information from researchers on their research projects, to later seek for a match between these and potential markets and customers. Differently from other employees involved in this process, his approach usually points at new, unexplored markets, in a disruptive rather than incremental perspective. 

More specifically, once gained information about a project, the employee will model any business opportunity that may originate from internal research, seeking any market niche or segment in which the technology may be deployed. After the identification of the best competitors in these markets, the employee will contact them and try to gain an initial interest, in order to establish a relationship. After this first contact, the potential customer will be redirected to other specialized AIRT's employees. 

The main process of relationship development, in fact, is held by a different resource, whose sets of activities includes: the identification and contact of potential customers, focusing on regional promotion based on FBK competencies; the reception of interested firms; the maintenance of relationships. 

The identification and contact of potential customers and partners begin with the seek for technological issues in entire industrial sectors related to FBK internal research areas. A necessary condition, which will be discussed later, is the knowledge of internal strengths and the applicability fields of any internal research activity. Technological issues mainly refer to a limitation in current processes, being qualitative, quantitative or in terms of efficiency, which can be overcome by a technological leap. Once spotted a similar issue, if an FBK's technology can achieve, or help to achieve this leap, will be compiled a list of potential customers or partners, along with a brief documental analysis for every actor. 

Then, a first contact will be made. Even in initial stages, the approach is relational, direct and informal: usually, a meeting which involves a presentation on the entire organization and its projects. The main objective of this stage is the understanding of the firm, its activities and industrial context, to better adapt the technological offer to its peculiarities: the perspective shift from a resource-based view to a market-based approach.
 
In this phase, the tendency is to rely on personal, local and previous networks of collaboration, rather than different contexts and industries, which can be seen as an effect of the local development mission of the organization. In example, the first contact will be made through a common acquaintance, if available. However, this approach may have an adverse impact, such as a self-limitation in the market scope and auto-financing opportunities.

Direct firms’ requests, instead, can be made directly to the office, to a researcher, or to external partner organizations, i.e.\ Hub Innovazione Trentino and Trentino Sviluppo; in any case, the request will be forwarded to the office, for its analysis and development. The necessary condition for the acceptance is the innovativeness of the project and the final objective of an advanced demonstration of the newly developed technologies. Otherwise, the interviewee suggests that extended projects could compromise the FBK's role of research institute, shifting from a knowledge generator to a \enquote{supervisor}, distracting resources from core activities of research.

In this scenario, the relationship evolves as follow:

\begin{enumerate}

\item A preparatory phase dedicated to the assessment of the counterpart;
\item The exchange of specific technical information about the potential project, under a non-disclosure agreement;
\item Contract drafting and negotiation;
\item Additional bureaucratic procedures;\footnote{The most significant example is for local firms: the \enquote{Legge 6}, a law promulged by the Autonomous Province of Trento, act as a financial aid for research collaborations; if the contract is eligible, the process will include an additional precontractual agreement and the submission of the proposal.}
\item Sign of the final contract and its technical annexes;
\item The actual research phase, based on stages defined by contract;
\item Continuous follow-up;
\item Final demonstration;

\end{enumerate}

A secondary share of this process is held directly by the CMM Director’s staff. In this case, external requests usually involve the development and production of specific hardware, which constitutes the core competency of the CMM and its researchers. Secondly, the seek for customers is performed as a secondary and residual activity, mainly through the participation in various exhibitions and fairs.

\subsubsection{Spin-off process}

In respect to the patenting and licensing process, the spin-off process is less structured and standardized. Even if a specific policy does exist, and it anticipate differently, the process usually begins in an informal fashion. In fact, the standard exploitation channels for FBK are the research contracts and licenses: the opportunity to found a new venture is more of an exception, with the chance being usually considered during informal meetings and alike.

The initial proposal may come directly from researchers, or driven by AIRT; which explicitly suggest the opportunity to the scientist. It may take the form of a communication of intent or more informal, in a simple inter-office meeting; after a first analysis and discussion between researchers and the TTO, the proposal will be articulated in a business plan, which will be officially submitted to the Entrepreneurship Evaluation Committee. 

The committee consists of 6 members, aiming at gaining a complete evaluation by including perspectives from different professional backgrounds and areas of expertise. Half of the committee is composed of FBK members, representing the mission and scientific knowledge of FBK; the other three members are representatives of institutional partners, involved in different industries: startups and spin-offs, venture capitalists, patent attorneys. Together, they should be able to provide a complete assessment, including the mere market value and the strategic value the proposal may yield to FBK and the local context.

After this preparatory phase, a more complete report will be redacted by the proponent, including a business plan and the draft of the new venture's statue. Eventually, the report will be presented to the Board of Directors for a more formal and complete assessment. The main objective of the entire process is clearly to extensively assess the robustness of the proposal and to ensure its alignment with the FBK mission. If the project receives a positive evaluation, an investment can be made, both in equity, grant or loan.

A final note on this process must be made: the process is currently being redesigned at its root. No startup has been spun off in the last years, to describe the recent form of the process flow, and no final design of the process has been implemented yet. This analysis is based on previous projects and different spin-off opportunities arisen in past years.	

\subsubsection{Grant support}

The Research Funding Office can provide a separate support for EU grants, especially in the form of H2020 calls, and local, less known and participated calls.

In the first case, the support for H2020 projects is mostly limited to the administrative and bureaucratic assistance for the application. This restriction is due to the scientific specificity embedded in each project: the researcher himself, over years of experience and activity in a specific field and its relative network, has a deeper knowledge regarding available calls, their financial entity, and feasibility, as well as other external individuals or firms to involve. The single office can not outperform the researcher in such activities. The focus of its activity, in fact, is sharper: to relieve researchers from any bureaucratic, organizational and legal affair, while exploiting their personal networks and scientific capabilities, enabling them to focus on the scientific aspects of the application.

The office performs activities such as budgeting, documental management, control and package of the proposal, submitting the application. If many actors are involved, the employees will coordinate them, at least on the overall administrative affairs instead of scientific matters. Later, if the application is granted, the office will manage the successive negotiation phase, especially the contractual phase for any Consortium agreement and the introduction of amendments. Lastly, no scouting activity will be performed. 

In the second case, for local, smaller and less known calls, the office will provide a more extensive support. As a preparatory activity, the scouting for potentially interesting calls will be made through institutional communications, alerts, and channel such as official EU publications. After an initial evaluation on the eligibility of FBK researchers, these last will be contacted for any call related to their research area. 

More specifically, the documentation will be sent to any scientist possibly interested: the assumption here, as for many other activities in the Area, is to know the actual interests of every researcher. In this particular case, the main tool is the experience of the employee involved, gathered through direct interviews and the development of personal contacts in almost any research unit. An example of the importance of these relationships is the continuous and spontaneous follow-up that researchers send to the office: in this scenario, no eliciting is needed.

Later, the willing researcher will be supported in every aspect of the proposal procedure. As usual,  the office may also be activated by a direct request from scientists. 

An interesting insight gained from interviewees is about the characteristics of researchers this support activity is aimed to, both on the organizational and individual level. Specifically, smaller units tend to be less competitive, negatively influencing their probability to successfully apply for major grants, i.e.\ H2020 calls. At the individual level instead, apart from the individual ability and talent, great emphasis has been placed on the personal network of the researcher. 

\subsubsection{Contractual support}

The legal support office performs activities of contract drafting and negotiation; it can be activated by the individual researcher, for single collaboration with firms and grants, and by other FBK positions, i.e.\ Center Directors and the Secretary General, for institutional agreements with a variety of actors. Types of contracts include:

\begin{itemize}

\item Collaboration Agreements; 
\item R\&D Agreements; 
\item Patent, know-how and software licenses; 
\item Services, for the prototypation and production of hardware;
\item Feasibility Studies; 
\item Program Agreement, frameworks for the development of relationships with public institutions; 
\item Grant agreements for European projects. 
\item Non-Disclosure Agreements; 

\end{itemize}

The contractual support typically begins with a formal internal request. The initial phase requires the employees to discuss with the researchers the basis and the evolutionary pattern of the contractual relationships. Later, a basic contractual model will be customized to the specific circumstance, which will be sent to the counterpart to begin the negotiation of inner clauses. 
 
This process can easily require several weeks and requires a continuous strategic evaluation of the economic, financial and strategic means of every change. Eventually, the process ends with the reaching of a satisfactory combination of FBK’s and the counterpart’s interests, as the compromise the contract represent. This process is performed alongside with researchers, to ensure the desirability of the technical contents of the contract.

Among others, a standard contractual model includes three significant elements: intellectual property, publication, and the clauses of licence-back and best-effort.

The intellectual property describes the property structure of knowledge and technologies involved in the process. It differentiates between: the background, known at the beginning of the process: the sideground, generated during the project, but not directly linked to the research objective; and the foreground, the actual results. The first two are usually property of the partner which developed it, while the third can be conjoint or individually assigned to one organization. 

The publication is another important element for any research contract. The typical clause states that any potential publication will be sent to the counterpart for a formal control prior to the public release; the organization may deny it, presenting a report that clearly defines the motivation. In any case, publications must not jeopardize patenting activities, i.e.\ by postponing every external communication to the patent application, as well as no confidential information disclosed.

Two minor but relevant are the licence-back and the best-effort clauses. Firstly, FBK usually requires a clause of license-back for the non-exclusive, unlimited use of the foreground generated, in the case in which the property is entirely assigned to the counterpart. The best-effort clause instead protects the Foundation from the potential absence or incompleteness of research results, due to the intrinsic uncertainty and risks of these activities.

In any case, researchers have the control over, and responsibility for the resulting contract: they lead the negotiation process and decide which form will be the accepted for final; the office act only as a support, providing skills and competencies and having little control over the economic and scientific content of the contract.

\subsubsection{Additional insights}

Interviewees involved in this process highlight several interesting topics.

Firstly, the need for a correct identification of internal competencies and research areas. In similar, highly-structured organizations with clear boundaries between research teams, inter-unit and inter-area communications become harder: the TTO may encounter difficulties in gathering this knowledge, especially considering the esteem and trust needed for a continuous and spontaneous communication. In such scenario, it might be fundamental to have a relevant experience in research, in order to gain a reputation before researchers, to be seen as a peer, therefore improving the communication performance. Formal and institutional moments of exchange may be useful but also potentially insufficient.

Moreover, disclosure eliciting is harder in the case of a research unit led by a researcher whose vision for its group is a \enquote{strongly independent, autonomous isle}. These units usually have a low auto-financing ratio, due to few cooperation contracts and little cooperation with the TTO and external firms. On the other hand, more proactive and propositional units show a positive attitude in collaborating with other entities, leading to a higher auto-financing ratio. This difference can be due both to the leadership, in a top-down approach, but also the individual behavior, as in a bottom-up approach. 

Secondly, interviews highlighted the importance of the perceived credibility, both for FBK as institution, its representatives, the researchers, and the patent attorney (if the contract anticipates the patent application). Specifically, the interviewees consider helpful to appears, as FBK, more application oriented, lean and flexible than a fully public institution; the representative credibility instead seems to be directly linked to the experience in industrial contexts and previous relationships with other firms. 

Another topic is the researchers’ perspective in the technology transfer activities. According to the interviewees, one of the main reasons for them to participate in these activities is actually the financial incentive, both as potential royalties and generic rewards. However, these incentives may be for the individual benefit, but also resources for consecutive research projects.  

Elsewhere, interviews have highlighted the need for researchers to participate in every phase of the relationship development, in an active and propositive fashion. Specifically, the initial stages can be used to build a direct channel of communication between the researcher and the firm's technical office, improving the flexibility and efficiency of exchanges in the prototypation phase.

Regarding the contract objectives, instead, has been highlighted the strong preference of firms for incremental R\&D instead of a more disruptive innovation, which influences the innovativeness of projects. In fact, interviewees reported that at the initial request, during the very same meeting, the researcher is usually capable of immediately assess the feasibility of the project, indicating that the scientific complexity of the proposal might be less challenging than expected.

Lastly, a perspective on the origin and localization of FBK relationships. According to the interviewees, about 70\% of the external relationships arise from the personal network of employees, reflecting the importance of the involved social capital. Moreover, direct external requests seem to origin prevalently from firms in the Trento local context. Apart from the local development mission of FBK, the phenomenon can be directly linked to the Autonomous Province of Trento and its legislation, especially for financial incentives, i.e.\ the previously cited \enquote{Legge 6}.

