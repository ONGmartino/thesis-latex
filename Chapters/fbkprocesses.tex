\subsection{Internal perspective}

The inner processes behind these channels have arisen both from policies and the original organizational plan, and as an emergent design refined by the experience of the involved human resources, whom helped in further shaping the processes' flow. Eventually, the process appears as unique and unified, where different offices and employees will be activated on the basis of the ongoing need. 

From a comprehensive perspective, the most part of technology transfer activities - for the exception of the research itself - are performed by the Innovation and Territorial Relationship Area (AIRT). Apart from the advisor role for economic aspects of the research, AIRT holds five main processes: patenting and licensing, support for spinoffs, grants, and contracts, relationships management.

\subsubsection{Patenting process}

The Reseach Centers responsible for the origination of patentable technologies are the Center for Materials and Microsystems (CMM) and the Center for Information and Communication Technologies (ICT). The process of patenting, in its beginning, differ for the Center of origin, due to the presence of a specific organizational position, in staff to the CMM Director, which act as faciliator through informal preparatory activities

Within the CMM, the process usually begins with researchers delivering a patentable technology, or a commercial idea, to the facilitator employee, through informal meetings or presentations. Specifically, the researcher may directly suggests the patent channel; however, the decision is usually postponed to the examination and assessment of the technology for its commercial potential, while comparing concurrent exploitation alternatives. The researcher is expected to expose key facts of the new technology, how it differs from the previous state-of-the-art alternatives, and how to extract economic and social value from it. 

After a first evaluation from the facilitator employee, especially for the anteriority research, a basic template will be written to clarify the most common, and important, contractual aspects. It contains information and clauses about the proprietary asset, financial incentives for the inventor, the legal process, and alike. Lately, the patent attorney will be contacted to better assess the feasibility and potential value of the endeavor. 

Otherwise, if the invention originates in the ICT Center, researchers should directly contact AIRT, to get an initial evaluation of the project, replacing the support of the CMM facilitator. In both cases, anyway, after a difference in the initial phases, the process flows will converge in a more formal procedure that begin with the so-called "invention notice". 

It consists in a notice, similar in spirit to a dossier, which notify AIRT and the Center Director of the reaching of a potentially patentable research product; it should include information about the discover, its applicability to industry, identified market segment, novelty, and alike. The invention notice will be thererfore evaluated, to be eventually accepted or refused. A necessary condition for a positive evaluation is the recognition of a clear potential customer, or customer segment, already identify.

If the notice is positively evaluated, the patent attorney will be involved to support researchers in converting the invention notice into an inventor declaration, which will be deposited initially as an Italian patent. After twelve months, the application will be converted in a PCT or EU patent, in order to gain additional time to invest in commercialization activities before the nationalization process and relative costs.

Then, the valorization activity will begin. Since the early stages of development of the patenting technology, AIRT and the CMM facilitator will start seeking specifically for a potential investors, interested in further developing the technology (1), apart from the usual organization interested in licensing (2) or acquire (3) the pending patent. In the nearest future, a specific partner will be engaged in this phase: a firm specialized in patent valorization, who will perform activities like market analysis, partner scouting and customer seeking. 

While the described process match the typical academic case, it represents only a small fraction of the potential patent applications. The great part of the proposal actually comes from EU financed projects and direct assignments. For the former, EU financed projects and specifically H2020 funds are said to be assigned to consortiums that at least include one firm; this, according to the interview, usually states its interest in any patent that can originate from the project. For the latter, instead, the signaling of interest in resulting patent comes from the very contents of the underlying contract.

Comparing those different paths, it is easy to spot the consequentiality between supply and demand. In the first, minoritary case the idea precedes any declared needs, like the stereotype of the scientist’s drawer. This case however highlight a financing issue for both any further research needed for the patent application and the application itself, that can easily reach high costs. 

The latter case however represents a favorable scenario for FBK and generically any research center. Customers present themselves with their necessities and requirements, even if not frequently clears, prior to the actual technology development. With a project grounded in a research contract, firms show a greater willingness to finance private research, especially for hardware projects that need tests on materials, which requires higher financing due to personnel, materials and machinery.

Lastly, one more problem has been exposed in regards to the trust bonding all actors involved. Firstly, the firm must be persuaded of the potentiality of the technical and market potential of the invention; secondly, she must trust the research center, its competence, ability and commitment. Thirdly, not obviously, the firm must have confidence also in the preferred patent attorney chosen by the research center. In fact, the protection that constitute the basis for a patent application, largely depends on how the application is written, especially claims.

To conclude, the licensing process involve many employee in all the organization: researchers, heads of unit, Center director and his staff, the Innovation Area in many of its office, eventually the purchasing office. Externally, this process will involve patent attorney, valorization firm and a firm. The main task for the Area is to keep the relationships between all involved actors, sorting the information flow and make the process keeping up with schedules.

\subsubsection{Spinoff process}

The spin off process begin in an informal fashion, even if a specific policy exists. The proposal can arise directly from researchers, or driven by the TTO that explicitly suggest the opportunity to scientists. The presentation of this proposal can take the form of a letter of intent or be informal, in a simple inter-office meeting; after a first opinion, the proposal will become a business plan that will be officially submitted to the Entrepreneurship Evaluation Committee. 

This board comprise about 6 prominent members with different professional background. To get a better evaluation, only half of the committee is composed by FBK members, embodying representatives from external partners involved in different industries, such as startup and spinoff, venture capitalists, patent attorneys. Together, they provide a complete assessment concerning both the mere market value and the strategic value the proposal will yield to FBK and local context.

After this first phase, a more complete report/dossier will be redacted by the proponent, and eventually it will be presented to the board of directors for a more formal and complete assessment. The main objective of the entire process is clearly to extensively assess the robustness of the proposal and ensure that it is aligned with FBK mission and the market before any expense other than employees time. Eventually, if the proposal pass the evaluation, an investment can be made; this can take various types of forms.

Differently form the patenting one, this process involves mainly only the head of the Area and an employee, a business developer that will perform activities such as counseling in business modelling, networking, advise for business plans and so on. 

A final note on this process: it is currently being redesign at its root. No startup has been spin off in the last year, no final design of the process has been decided yet, and no policy has been written yet. Thus, it cannot be well described. 

\subsubsection{Grant support}

The support the Research Funding Office can provide for grants must be treat differently according to the different human resources involved and the differences between tasks.

Half of the office activity is basically limited to administrative assistance in H2020 call for proposal and applications. This restriction comes from the scientific specificity embedded in the various call: the researcher himself, over years of experience in the field and activity in the specific network, has a far more in-depth and updated knowledge regarding available call, financial entity and feasibility, other involved researchers or potential consortium. That is, the single office will never outperform the researcher in such activities. In fact, the focus is sharper: to relieve researchers from any bureaucratic, organizational and legal inconvenience, exploiting at the same time their networking activities and scientific capabilities and enabling them to focus on the scientific part of the application.

More precisely, the Office perform activities such as budgeting, retrieve all necessary document, control and package of the proposal, submitting the application. If many actors are involved, the office will perform also coordination among them, still on the overall administrative affairs instead of scientific matters. Later, if the application is granted, the Office is in charge for the later negotiation phase (now reduced in the H2020 program), especially the contractual phase if a Consortium agreement is needed. Moreover, the Office will be activated by any amendment introduced. Indeed, for such large and important grants no scouting for call will be performed. 

On the other hand, the Office also provide a far more extensive support on local, smaller or less known calls. In this case, many other activities will be performed in a complete fundraising approach. Firstly, the scouting for potentially interesting calls will be made through institutional communications, alerts and channel such as official EU publications. After an initial evaluation about the eligibility of FBK researchers, these will be informed about calls inherent to their research area. 

More specifically, the call documentation will be sent to any possible scientist whom work relates to the object of the call: the point here, as for many other activities in the Area, is to know the actual interests of researchers and who knows what. In this specific case, the answer resides in the experience of the employee involved, that has been gathered early in her career in FBK, through many direct interviews to any researcher focused on their competencies. In this interview process, personal bond has been established, from institutional moments to personal relationships, making the employee irreplaceable: a positive example is the continuous and spontaneous follow up that researchers send to the office about their current interest. In this scenario, no eliciting is needed.

Later the willing researcher will be guided in every aspect of the proposal procedure. As usual, this half of the office will activate itself also on direct request from scientists. 

An interesting point is the population of researchers this specific activity is dedicated to. In this prospective, they can be divided into two main groups, characterized both by individual and organizational variables. According to the experience of the interviewed personnel, at the organizational level smaller Units tend to be less competitive, precluding them chances to win major grants as H2020 ones. At the individual level instead, apart from the obvious personal ability and talent, great emphasis has been placed on the personal network of the researcher. 

The impact of the office can be counted in at least a few days of researchers’ time for each proposal, allowing them to focus elsewhere. This evaluation however suffers from a very high variability, apart from the obvious stage reached: starting from a couple hours for small proposal, where the scientist want to control every aspect, up to a week for a consortia proposal with a consortium agreement. This is due to the experience and the knowledge accumulated by human resources, mostly tacit and hardly transferable, that includes information about the internal organization and resources, external linked institutions, apart from the obvious know-how related to the structure and procedures behind the specific grant. 

Moreover, other support offices are involved in such activities: project management groups have been established in staff to the Center directors, with the specific purpose of providing support in related technical and scientific matters. However, there is a clear lack of coordination between offices, turning down any change for synergies exploitation.

\subsubsection{Contractual support}

Under the Research Funding Office, the legal office performs activities of contract drafting and negotiation. It employs 2 different resources, encompassing both a long experience and a specific educational background. They act on requests of researchers, for single collaboration with firms and grants, and from other FBK major personnel (i.e. Center Directors, the Secretary General) for other institutional agreements with a variety of actors.

Types of contracts includes: Non-Disclosure Agreements, that makes possible a free exchange of information between counterpart; Collaboration Agreements, for a research project which require knowledge from both counterpart; R\&D Agreements, in which FBK develop a technology for a financial compensation; Licensing, for patents, knowhow and software developed in-house; Services, for the prototypation and production of hardware (mainly sensors and similar from the Clean Room); Feasibility Study, for a research about the technical feasibility of a request (which can be configured both as a Collaboration agreement or as an R\&D agreement); Program Agreement, with other (public) institution for further agreements about a specific research area; Grant agreements for European projects. 

The contractual support typically starts form the request of a researcher, who has been contacted by – or is developing a relationship with – a firm. The first relevant objective is to fully understand how the relationship should be based on and should evolve under the researcher conception, and to translate it into legal terms. After this initial phase, a base contractual model will be customized to the specific circumstance, and sent to the counterpart. Here starts an indirect negotiation of clause with the track changes mechanism, that stops when a satisfactory combination of FBK’s and the counterpart’s contractual model and interests is accepted by both. This process can take several weeks, and requires a continuous (strategic) evaluation of the compromise the contract represent. Furthermore, this process is performed alongside with involved researchers, to be sure the final form match his interests.

A standard contract primarily include directive about the Intellectual Property. Specifically, it is considered in respect of the background (previously generated relevant knowledge, patents etc), the sideground (knowledge generated during the project, but not directly linked to its expected result), and the foreground (the actual project results). For the first two, both are usually property of the part who developed it, while the third that can be conjoint or individually owned. Furthermore, FBK usually try to insert in the contract a clause of license back, for the non-exclusive unlimited use of the complete foreground generated, even if property of the counterpart.

The publication clause is another important part of the contract. Generically, any potential publication will be sent to the counterpart for a formal control; if they reject the document, they must present clearly defined motivations. In any case, any publication must respect possible patenting activities, which usually means to postpone everything to the patent publication. Obviously, no confidential information must be disclosed, ant the counterpart can decide whether to insert a citation obligation. 

Another specific relevant clause is inserted in every research contract: the best effort clause. Being research activity a risky one, and being results uncertain, the Foundation protect itself from the potential absence or incompleteness of results with this clause.  

Anyway, researchers have the complete control and responsibility for the (quality of) the resulting contract: they lead the negotiation process, deciding what will be the accepted final form. In this perspective, the contract can be divided into: a scientific part, where background and objectives will be stated; a legal part, as an ensemble of protective clause and similar; an economic part, where economic interests, payment and other agreements should be included. Being the legal part somewhat a support for the other two, and having the legal office clear competencies limited only to it (except for a general counseling), the decisional authority driven by the economic and scientific part is held by researchers.

\subsubsection{Relationships management}

The process that encompass relationships with firms involve various employees, establishing many points of contact between the organization and the market.

The first that should be mentioned under a chronological perspective is the business developer employed in the Area. Along with marketing activities, he actively elicits information from researchers about their activities and try to find a match between these and potential new markets. Oppositely to other employees involved in this process, his approach seems to point at new, unexplored or still completely uncovered markets, in a disruptive rather than incremental perspective. 

So, once gained information about a project, the employee will try to model any business opportunity that can be made from it, then seek any market niche regarding the technical potentiality of the project; later, he will identify the best competitors in this market, then try to contact them and establish a relationship. After this first contact, he should transfer the whole project and the potential client to A. Bozzoli.

In fact, the most part of this process is held by Alessandro Bozzoli, who focus on three main sets of activities: the identification and contact of potential customers, focusing on territorial promotion based on FBK competencies; the reception of interested firms; the maintenance of relationships. 

Every set has its root in the correct identification of internal competencies and research areas in which every unit is involved. In such organizations, with clear boundaries between research teams, inter-unit and inter-area communication becomes harder, up to a level in which even TTOs encounter difficulties getting information about the research area of scientists. Moreover, a matter of trust is involved. In such scenario, it is fundamental to have a relevant experience in research, in order to gain a reputation before researchers and therefore to be seen as a peer. In this case, TTO employees can build better relationships with researchers, eliciting disclosure.

In regard to this activity, the involved employee shows a specific trait that support the gathering of such information: a decennial experience as FBK researcher and Unit Head, in which he created a network of personal relationships based on reciprocal esteem and trust. In this specific case, the elicitation of information from researchers is composed of both formal/institutional and informal moments of exchange, which helped over time in picturing a complete portrait/mind-map of internal competencies.

Interviews has uncovered an important fact regarding this matter. Eliciting disclosure is harder for units lead by a researcher who views its group as a strongly independent, autonomous isle with respect to FBK. These units often have low auto-financing ratios, due to few cooperation contracts and low probabilities in getting external research funding through grants; generally speaking, they have low-to-null incomes and, in the long run, no bright future. On the other hand, more proactive and propositional units show positive attitude in collaborating with other actors, both internal and external, leading to higher external financing, won grants and, generally, a higher auto-financing ratio. This difference can be due to the leader, in a bottom-up approach, but organizational context has a clear impact in this different behavior and thus the performance, as in a top-down approach. Specifically, the latter will help avoiding any governance issue.

Regarding the first process, the identification and contact of potential customers/partners will starts with the seek for technological problems in entire industrial sectors related to FBK internal research areas. Again, starting point is the knowledge of internal strengths and the applicability fields of any internal research product. The technological problem mainly refers to a limitation in current processes, being qualitative, quantitative, cost effective and so on, that can be overcome with a technological leap. Once spotted, if this leap can be made with an FBK technology, a list of potential customers/partners will be compiled together with a brief documental analysis for every actor. 

Later, a first contact will be made. Even in initial stages the aim is to identify the counterpart as a partner rather than a customer. In fact, the very first contact will be made to ask for a meeting or a visit, to directly establish a direct and more structured relationship; if available, this contact will be made through a common acquaintance, to help propose FBK as a local embedded actor. The main effort at this stage will be the complete understanding of the firm, its activities and its industrial context, to better understand how to adapt FBK’s technological offer to firm’s peculiarities. 

An interesting point here is the attempt to balance (A) an extreme focus on a resource/knowledge-based view as starting point of the entire process, followed later by (B) the effort put into the understanding of firm’s needs, then the elaboration of a specific offer. DA APPROFONDIRE 

Another fact that should be highlighted is the tendency to relies on personal and previous network of collaboration, rather to seek partners in different contexts and industries. Apart from a positive increased efficiency that can be expected from this narrowness of aims, a negative impact can be seen in regard to the organizational mission, such as a restricted impact on local context and the self-limitation in scope and auto-financing opportunities.

Furthermore, in initial stages seems to be quite important the perceived credibility of: FBK as a research institute, its representative (the AIRT employee), the researcher. Firstly, FBK has seems to have already distinguish itself as a more reliable actor than universities and similar entities: being a private, non-profit institution, it appears more application oriented, lean and flexible than a full public institution. Secondly, the representative credibility can be seen (again) as directly linked to his experience in the industrial context, previous relationships with other firms, his ability to interpret industrial needs into researchers specific "language". For the latter, researcher will be informally trained, with various information about firm and industry, to better perform at meetings.

In respect to the second process, namely the reception of firms’ requests, the initial contact can be made to an entire set of internal employee and external actors. Examples, specifically to the latter, are Hub Innovazione Trentino, Trentino Sviluppo, other personal contacts. In this case, the firm has already a specific idea about their needs and the shape that the relationship should evolve into. Anyway, necessary condition for the acceptance is the innovativeness of the project and a reduced aim to an advanced demonstration of the newly developed technology. Otherwise, the interview suggests that extended projects could compromise the research institute role in the partnerships, shifting from a knowledge generator and supplier a to a supervisor, therefore distracting resources from core activities of research.

Researchers can also request AIRT for advice and support in relationship-building activities. Should be mention that every single unit has its own channels: personal contacts, international visibility of researchers, publications, grant agreements, websites. However, a unit that has been directly contacted by a firm is strongly recommended to use AIRT services, due to the professionalisms involved. Under this concept, AIRT has grown as the main interface between internal employees and external agents, whenever a financial flow can be involved, as in the intermediary tradition of TTOs.
In this scenario, the relationship evolves as follow

\begin{itemize}

\item An exploratory, initial phase regarding competencies, knowledge, interests and research areas, dedicated to the assessment of the counterpart;
\item The NDA: a non-disclosure agreement will be signed to allow the exchange of specific and technical information about the specific potential project;
\item Contract negotiation: documental preparation for and negotiation of the future underlying collaboration/R\&D contract;
\item Specific to local firms: the Trento Province has promulgated the "Legge 6" (sixth law) that act as a financial aid for research collaboration; if the contract is eligible, the process will additionally include a precontractual agreement, the proposal submission, potentially the reception of regional funding;
\item Stipulation of a definitive contract, which include a technical annex;
\item Finally, the actual research will be performed, based on stages defined by contract;
\item Final demonstration of the research outcomes;
\item Conclusion of the productive part of the relationships; tentative for up/cross-selling.

\end{itemize}

The interview relative to this process also exposed some interesting issues about the technology transfer concept in general, and the scientist involvement in particular.

The first is the researchers’ perspective in the technology transfer activities. One of the main reasons highlighted in the interview for researchers to participate in these activities is still the pecuniary one. Even before the inventor acknowledgement in potential patents, that should be linked to the publication incentive and therefore the scientists’ reputation, many researchers are driven by potential royalties and generic rewards. Has to be determined if these are for individual (as in personal) benefit or for consecutive deploy in later research projects. 

Moreover, a fundamentals question is whether this ensemble of activity can be defined as technology transfer, being the object a technology that the firm commission FBK to develop. In fact, from such perspective, the research process seems more like the delivery of products or the provision of services, rather than a flow of knowledge between organizations. Therefore, the question is whether and how the knowledge, surrounding and supporting the technology, will be transferred. How firms will be able to modify, adapt, further develop a tool that FBK once provided?

Answers partially came from the type of underlying contract. In the case of R\&D contracts, in which FBK is required to individually develop a technology, the industrial counterpart usually has little interest in participating the process: it only need a technical solution to integrate in its products, clearly delimited in technical documents attached to the contract, that can be understand by the /*reparto tecnico*/ - who can directly ask researchers for more information if needed. In this scenario, object of transfer is not the actual knowledge (as sometime pointed by the literature) but the innovative product itself, as long as it is the only requirement from firms.

Another perspective come from the definition of technology transfer itself. Although a clear and shared definition does not exist, one of the better structured and well-known is the tech transfer as (shorter version) a process that involve the transfer of scientific advancement and ends with the commercialization of a product that include it. In such perspective, no "pure" knowledge has to be transferred, as long as a new product will be developed. In fact, exacerbating this perspective, a model can be theorized where all innovative and newly developed knowledge belongs and resides only in research centers and universities, apart from very little technical knowledge that firms need to initialize absorptive capacity. 

However, this perspective must be compared with FBK mission, along with the path intended to pursue it: can a positive impact on the context be achieved by transferring only products instead of knowledge? How can it match the nonprofit semi-public status of the Foundation? 

Finally, the initial question can be redefined as: to what extent such activity can be defined as tech transfer instead of a peculiar service outsourcing? The answer should differ greatly according to the status of the knowledge keeper. The easier case seems to be private research centers, which can be seen as the free market answer that match a technological demand. However, considering universities, the triple mission must be respected, along with the entire debate that surrounded the topic for at least 20 years. Although this is not a place to discuss it, a point should be highlighted for the peculiar FBK position /* (public born and administrate, privately managed) */: to what extent third mission activities of commercialization can be carried out with no synergy with teaching and research mission? How well can an actor perform, without caring for mission balancing and synergy exploitation?

A secondary share of this process is held directly by the Center Director’s staff. Also in this case, relational activities have been divided between in the handle of external direct manifestation of interest, and the seek for new customers. In this circumstance, however, the latter is performed more likes a task for spare time, through the participation, therefore the search, to various exhibitions and fairs.

For the prior, the first interesting point is that these requests mainly originate, according to the interviewee, from his personal network. In about 70\% of cases, the interlocutor comes not from the institutional, but the personal network of the employee, reflecting the importance of the social capital involved in research organization and the previous reported fact. 

The geographical origin of these requests exposes another issue: they come prevalently from the local context of Trento, with a little presence at the national and international level. This reflect as expected an organizational mission that focus on the Trentino context but, accordingly to the interview, it largely reflects the Trentino government and its financial incentives, namely the ‘Legge 6’. 
 APPROFONDIRE 

This peculiarity of the Trentino context also reflects a scarce propensity of firms to innovate in more risky fields, preferring incremental innovation APPROFONDIRE
These types of requests, made directly to the Center Director’s staff, mainly regard service activities such as the development of specific hardware (MEMS and other sensors). In this case, every request follows a path that can be divided into 6 different but standard phases, with great similarities with the previous process (the one held by A. Bozzoli): 

\begin{itemize}

\item in the initial phase, the firm exposes its idea (or, at least, a draft) that includes product features needed and various other specifications. At this stage the idea has already been recognized internally by the firm, and its potential has already been attested by internal R\&D and/or external consultants. At the meeting the involved researchers should already assess the feasibility of the project, indicating that the scientific complexity of the proposal is not quite challenging.
\item The second stage of the relationship development refer to an internal evaluation of the proposal, its feasibility, that result in drafting a technical/scientific document on realization phases and potential applications. This includes a generic evaluation of the development costs.
\item the third refer to the contract drafting and stipulation. This contract usually includes the ability for firms to unilaterally interrupt the collaboration in any stage of development, which is financed/payed step by step.
\item the fourth stage finally includes the technology prototypation and test; if satisfy, the firm can choses to continue the relationship with the develop of a commercial prototype, which fit the industrial production requirements.
\item finally, for specific products, the firm can ask the Foundation for serial production.

\end{itemize}

In respect to the relationship between this organizational role and researchers, credibility of the intermediary is (again) fundamental, along with the professionalization needed for the relation with firms to take off. Moreover, has been highlighted that is fundamental for the researcher to participate even in early meetings, to be an active counterpart to firms and avoid being silent workforce: in the prototypation phase, they must communicate without intermediaries to speed up the process and avoid misunderstanding. 

The participation of this employee (the one that held the intermediary position in the director’s staff) to the research climate is therefore fundamental: this will offer him tacit information about researchers interests, willingness to participate in tech transfer, the availability of innovations and research with high potential, and so on.

\subsubsection{Technological scouting}

The technological scouting is an additional activity performed by the CMM Director’s staff. This activity aims to identify new trends in both scientific domain and markets. For any new movement spotted, should be understand the impact that it will cause on the Foundation, in a SWOT analysis fashion: will this increase our strengths? Can we use it to be more competitive? Or will it erode our position in the market or in the European network of research? 

To fully understand this impact, a resource has been appointed to perform analysis of the state-of-the-art and the market. If any interesting trend is identified, scientists from contiguous areas will be contacted to determine if their interests in research match new developments, thus if they can apply themselves in these new areas. Anyway, must be noted that no support staff has the authority to coerce scientists to follow a specific line of research; however, in case of clear and major necessity, Heads of Unit and the Center Director should force scientists. 

Another relevant activity can be the management and coordination of subsidiary companies in which FBK own a stake – mainly on the behalf of the Autonomous Province of Trento. This activity is currently being down-sizing, according to the new policy of ‘no-subsidiaries’ that should be complete in 2017.
Other activities are mainly bureaucratic. 
