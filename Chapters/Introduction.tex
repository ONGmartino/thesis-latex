
\Chapter{Introduction}{Definitions, perspectives, issues} % Main chapter title

\label{Chapter1} % Change X to a consecutive number; for referencing this chapter elsewhere, use \ref{ChapterX}

%----------------------------------------------------------------------------------------
%	INTRODUCTION - CHAPTER 1
%----------------------------------------------------------------------------------------

In layman's terms, Technology Transfer can be broadly defined as the process that takes research results to final customers in the market. However, behind this simple definition, the topic has risen considerable attention in the last 25 years from many academics, institutions and governments, leading to a substantial literature encompassing almost any possible perspective.

One side effect of such large literature is that even the technology transfer itself has no shared definition, and no unique notion can possibly comprehend every characteristic, angle and shade. Anyway, a simple and recognized definition can be taken from the Association of University Technology Managers: \enquote{the process of transferring scientific findings form one organization to another for the purpose of further development and commercialization} \citep{Genshaft2016}. Even \citet{Bozeman2000}, in his well-known review, recognize the definition issue, eventually citing Roessner (2000) for \enquote{the movement of know-how, technical knowledge or technology from one organizational setting to another}. 

Another definition, that better explain the aim of this process, is the one given by \citet{Rogers2001}: \enquote{the technology transfer process usually involves moving a technological innovation from an R\&D organization to a receptor organization; it is fully transferred when it is commercialized into a product that is sold in the marketplace}. 

Often the object of transfer is not the technology itself, as a product, but the entire knowledge surrounding its conception, use and application. In recognition of these cases, a broader definition can be taken from \citet{Argote2000}: \enquote{knowledge transfer is the process through which one unit is affected by the experience of another}. It can be considered at the systemic, organizational, and the individual level, either explicit or implicit. Similarly, \citet{Zhao1992} observe that the definition acquires greatly different shades according to the discipline it is considered in, i.e. among economists, sociologist, anthropologists. 

Elsewhere, has been noted that the seek for a canonical and universal definition is futile, but it gains importance as long as it promotes a better understanding of the phenomena by the comparison of differences \citep{Bozeman2000}. This definition issue is not limited to the technology transfer, but it extends to the very fundamental topics this process is built upon: even technology has not a clear definition, but it is commonly seen as a \enquote{tool}, further opening the issue for the qualification of this term. 

Innovation, instead, is broadly recognized as the process of generating and recombining ideas: \enquote{the process of doing something new or adding value to old things by changing the way they're done} \citep{Baskaran2016}. Its concept, however, greatly varies across cultures and contexts, ranging from a more radical and disruptive nuance in advanced economies to traditiovations and lateral thinking in developing countries.

Moreover, in a traditional perspective on innovation, firms acquire consulting service and built relationships with technology centers to foster internal R\&D and to mitigate uncertainties embedded in the research activity. Nowadays instead, in a technology transfer setting, the benefit is reciprocal: \citet{Siegel2003a} states that 65\% of scientists have experienced positive effects on their experimental work, either in quantity and quality, even in basic research. It is safe to say that in a real-world scenario the technology transfer appears to allow information to flow in both direction.

Similar meaning have the concepts of University-Industry relationships, defined as \enquote{trusting, committed and interactive relationships between University and Industry entities enabling the diffusion of creativity, ideas, skills and people with the aim of creating mutual value over time}, and R\&D collaboration, as in \enquote{the cooperation within a group or teamwork both in organizational and individual levels with an objective to create a useful and valuable innovation to achieve the common goal set collectively} \citep{Frasquet2012}. 

Rather than an evolutionary approach, \citet{Bozeman2000} uses an historical perspective. He divides policies regarding generic technology development and exchange in three consequential paradigms. The first was the market failure paradigm, which identifies an historical moment with clear negative externalities, extremely high transaction costs, unavailable or distorted information, creating the opportunity for a government intervention. His role was limited to removing barriers to the free market through appropriate policies (i.e. IP policies), leaving to Universities the role of source and gatekeeper of basic research.

Secondly, the mission technology paradigm recognize that private actors involved in \enquote{national interest} related R\&D could not easily and effectively reach their objectives; this phase witnessed the redefinition and enlargement of (federal) government as R\&D performer, due to its unique ability to gather resources and exercise influence. The third stage refer to a turning point in which each actor in the national innovation system stops working in isolation, and starts to act cooperatively, as part of a network of specialized entities. As stated, \enquote{the logic is simple: universities and government labs make, industry takes}, sustained by an ensemble of policies to sustain interaction, exchange and collaboration. 

For a more European-centered historical perspective of the phenomenon, a brief introduction can be taken from \citet{Geuna2009}. They focused on the university perspective, observing the role they had in the shift to the knowledge-based economy. They state that the main change resides in the new institutionalization of University-Industry linkages, aimed to increase the direct involvement of academic staff through a change in their activities.

In fact, in the past the traditional interpretation of technology transfer heavily relied on the effort and the initiative of individuals. \citet{Balderi2007} use five stages to illustrate the shift in the University approach to technology transfer. Firstly, the discovery of the phenomenon, with sporadic and localized initiatives to inspire researchers. Later, an acceptance phase and the appearance of spontaneous actions, proving the raising awareness and acknowledgement of this process. Thirdly, enthusiasm and expectations lead to a radical change in attitudes and the establishment of dedicated offices and policies. Fourth, a learning process takes places, with the experimentation of models and settings, exposing the need for a rationalization of the process and a change in national legislation. Lastly, this lead to the seek for a positive discontinuity, with a new comprehensive model.

These needs have been developed alongside more comprehensive economic theoretical frameworks; representative examples are the National Innovation System and the Triple Helix, which emphasize the role of the University in the new economy landscape \citep{Balderi2007}. This trend found a match even in the legislator will: most of the more developed countries, through their various institutional authorities and agencies, are actively rethinking the role of local research organizations \citep{Geuna2009}.

In fact, as part of their strategy for the development of a knowledge-based society, governments are soliciting a more active role from universities in the national economic development, specifically through the demand for more industry-funded research \citep{Geuna2009} and university-based entrepreneurship \citep{OShea2004}. Needless to say, the greater the university research funded by firms, the smaller the government funds required \citep{Yusuf2008}. Elsewhere, government policies aim to increase economic returns form publicly funded research \citep{Bercovitz2006}.

Questions have been raised on the appropriateness of these aggressive policies. Specifically, budgetary stringency policies and the demand for more applied and contractual research (as opposed to basic, open science) has been criticize also by the public opinion. The answer resides in the evaluation of the impact that scientific knowledge in general and academic research in particular can have on the national economic performance.

A first, quantitative indicator comes from the Community Innovation survey, in which universities has been found to represent the 9\% of partners collaborating in any innovative activity \citep{Muscio2008}, even if elsewhere other surveys in industry ranked universities least as innovation partners \citep{Yusuf2008}. Other authors showed how about 10\% of the new products and processes commercialized by firms will never be introduced without the university contribution \citep{Bekkers2008}. 

In a qualitative perspective, instead, a relevant contribution comes from \citet{Bercovitz2006}. He investigated the driver for an increase in university-industry collaboration, highlighting the growing scientific and technical content of all types of industrial production and the new, high opportunities offered by technological platforms. \citet{Bozeman2000}, in earlier days, stated the rising importance of this trend through four different indicators: major policy initiatives, dedicated academic journals, technology transfer inserted into organizational mission statements, specific job titles and thousands of articles. 

Another macroeconomic perspective can be taken from \citet{Markman2005}, who observes that an increase in R\&D expenditure and activity yield more inventions, thus producing a larger number of inventions; these have a positive impact on productivity and growth, leading to economic development and well-being. A microeconomic perspective instead has its root in the resource-based view \citep{Wernerfelt1984} and the strategic value of organizational knowledge. In fact, as clarified by \citet{Argote2000}, this should be the principal source of competitive advantage, pushing organizations to invest into internal knowledge difficult to imitate.

In this scenario, relevant contributes comes from \citet{OShea2004} and \citet{Yusuf2008} who refer to various forces that should enforce the university's role into the knowledge economy. Some of these come from the firms' demand of innovation, which is increasingly used as a tool for sustaining their competitiveness; in fact, more and more firms are taking advantage from new opportunities generated from scientific advantage. 

Moreover, if knowledge and technology remain key factors in firm competitiveness, and if a significant part of this innovation continues to be generated into universities, three trends can be foreseen \citep{Yusuf2008}: a further increase in the demand for high skilled labor force; greater investments in either basic and applied science, form both public and private entities; a growing importance of (technological and scientific) entrepreneurship and of intermediating entities. 

This perspective match he one of \citet{OShea2004} who recognize the growing importance of knowledge creation and exploitation, especially the one linked to new technological-based entrepreneurship. These firms assumed a fundamental role in linking science to market, demonstrating themselves as the best suited entities for converting new scientific discovery into market opportunities. 

Empirical support for this trend has already been found. The dual benefit of collaboration, both for firms and universities, the desire of the latter to differentiate funding sources, the interest of government to developing contexts whose R\&D capabilities attract multinational corporations' investments, they all drive to an intensification of university-industry interaction over time, sustained by a change in the economic and institutional environment \citep{Debackere2005}.

What is usually neglect by the literature, instead, is that even if this trend has its root in the university-industry cooperation, it refers to a larger perspective: the transfer of knowledge and technologies from research institution to firms as market gatekeepers. That is, while researches on technology transfer has been largely devoted to the university as the source, little attention has been paid to other types of institution, i.e. private research centers, research foundations, development facilities etc.

One of the reasons behind this narrowed focus may be an easily acknowledgeable bias, the actual association of many authors to universities; this association could have led to an over-study of the nearest and better known context, leaving overlooked less accessible institutions and contexts.

A very similar bias is recognized inside the very technology transfer literature: as reported by \citet{Muscio2008} the great part of empirical researches focus mainly on mechanisms that can be easily codified, but have their limits in the weakness and limited information they provide in respect to the magnitude of the phenomena. In other words, the easier are to gather data and information about a phenomenon, the better understood it will be. 

The same author observes that this process \enquote{seems to have not dominant actor or mechanisms, clearly and directly identifiable}, reporting a difficult in finding an overall, general model that could guide institutions, a problem that arise from the difficult in the typicization of the phenomena.

So, academics has a deep understanding of the technology transfer between university and, ultimately, the market. But the same literature has quite overlook the technology transfer from other types of institutions, especially private entities. Contemporary, the changing economic environment is leading us to the knowledge economy, where all sources of newly developed knowledge and technology will acquire greater and greater importance. 

Therefore, the main rationale behind this thesis is to understand whether or not the technology transfer, as in a perspective of mechanisms and processes, could be considered the same whether the source is a (public) university or a private institution. The main question is: does all we know about the technology transfer process, between university and industry, still apply to a private-market scenario, in which both the source and the receiver of knowledge are private institution? Do we need to simply substitute the word \enquote{university} with the expression \enquote{research institution} in the literature, or do exists some major differences that will make the current knowledge of this process obsolete, when referring to private entities?

To answer this question, firstly will be constructed an organic review of the overall literature. Will be considered the main authors and publications in the last decade and a half, in an attempt to cover the most important contributions. There is no ambition in cover all the knowledge related to the process, given the extremely large corpus of scientific publications related; however, an attempt will be made to draw for the reader an overview of the entire process, covering actors, channels and evaluation perspectives. Later, will be discussed all the theoretical differences between university-to-industry and business-to-business technology transfer. Finally, these differences will be confronted with a real-world experience in the field, to gain an empirical insight of the topic.
